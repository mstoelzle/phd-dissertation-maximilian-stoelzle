\chapter*{Summary}
\addcontentsline{toc}{chapter}{Summary}
\setheader{Summary}
As we increasingly strive to integrate robots into human-centric environments, safety is a top priority. Traditionally, rigid collaborative robots have relied on safety-aware computational control policies, which are susceptible to perception errors and often lead to overly cautious behavior that limits performance. In contrast, soft robotics offers a promising alternative by ensuring passive compliance throughout the robot’s structure via material softness. This mechanical compliance inherently mitigates safety issues arising from perception or control errors, although this has been paid for with a substantial drop in precision. In recent years, significant advances have been seen in soft robotics, with exciting new developments in design, smart materials, actuators, sensors, models, and control strategies. However, the modeling and control of continuum soft robots continue to pose major challenges due to their infinite degrees of freedom, complex nonlinear dynamics, and time-dependent behaviors like hysteresis. As a result, soft robots often lack the necessary capability and motion precision, leading to a tradeoff where performance is sacrificed for safety. With this thesis, we argue that this tradeoff can be overcome by developing more advanced algorithms that can reason on the physics of the soft robot. More specifically, we propose combining powerful learned models with efficient and effective model-based control approaches that allow for interpretability into the actions and admit stability guarantees.

Currently, two main approaches exist for controlling soft robots. The first employs model-based control using approximated physics-based models derived from first principles. The second directly learns control policies, primarily through reinforcement learning. Both strategies face notable limitations. Existing model-based controllers are unable to fully manage and eventually exploit the dynamics of soft robots because their underlying models inadequately capture complex behaviors, particularly how actuation and external interactions affect the robot’s deformation. Moreover, deriving these models requires extensive expert knowledge. Additionally, the combined complexity and uncertainty of the dynamics between a soft robot and its environment make it currently infeasible to develop comprehensive world models from first principles alone, thereby motivating the integration of machine learning approaches that can effectively leverage data-driven insights. Conversely, directly learning the controller — such as via reinforcement learning — lacks interpretability and stability guarantees while being highly sample inefficient, a significant drawback given the time-dependent material properties and limited lifespan of current soft robots.

In this thesis, we contend that combining learned models with model-based controllers presents a promising alternative that brings together the advantages of both approaches: expressive, data-driven models that require less expert knowledge paired with controllers that are both interpretable and provably stable. Although recent years have seen increased interest in leveraging learned models for control, most work in this area depends on computationally intensive optimal control methods, such as MPC, to optimize the actuation sequence with the learned model. However, the high computational cost of solving these optimal control problems limits the maximum control frequency during deployment, preventing us from fully exploiting the dynamic capabilities of soft robots. Instead, this thesis pursues closed-form controllers that utilize the physical structure of learned models within an energy-shaping framework. The main challenge here is to develop approaches that integrate such physical structures—specifically, kinetic and potential energy terms—into the learning of dynamical models for soft robots.
Before addressing this main challenge, we first had to advance physics-based models derived from first principles and identify novel techniques to leverage them for control. On one hand, this clarified which physical priors were available for learning, while on the other hand, it inspired new ways to integrate model-based controllers with learned models.
The thesis addresses this topic through several interconnected key contributions. 

First, we argue that quantifying the safety of soft robots is crucial for designing and controlling them to ensure that the closed-loop system meets the specific safety requirements of their intended applications. To this end, we present the first safety metric for continuum soft robots, which assesses the safety of an integrated soft robot design by accounting for both its embodied and computational intelligence.

Secondly, this thesis enhances shape sensing for soft robots by leveraging insights from kinematic models. We accomplish this by formulating and solving nonlinear optimization problems that align sensor measurements with the backbone shapes predicted by the kinematic model. We present two distinct approaches that integrate commercial sensors—namely, visual and magnetic—with SLAM algorithms and a learned sensor measurement model, respectively, to accurately estimate the soft robot’s state, a key requirement for effective feedback control.

Thirdly, this thesis introduces advanced physics-based actuation models, including those for robots actuated by auxetic metamaterials - referred to as HSA robots—and models that capture the actuation dynamics of piston-driven pneumatic soft robots. We then leverage the acquired model insights to design provably stable nonlinear controllers—specifically, an integral-saturated PID combined with potential shaping and Cartesian-space impedance control for planar HSA robots, as well as a backstepping controller for pneumatic piston-driven soft robots. This contribution deepens our understanding of actuation, a critical aspect of soft robot behavior, and demonstrates how such insights can be incorporated into model-based control strategies. Moreover, experiments with HSA robots have highlighted the limitations of purely physics-based models in capturing complex phenomena like hysteresis, thereby motivating the exploration of learning-based approaches. In the future, the developed actuation models can serve as valuable physical priors for learned models.

Fourthly, the thesis presents techniques for learning soft robot models that incorporate physical structures while ensuring stability. We accomplish this by embedding physics-based dynamical models into the learning algorithm, which determines the free parameters of the dynamics and optionally optimizes a coordinate transformation—such as encoding into latent space. Two notable approaches are introduced: (1) an algorithm that extracts low-dimensional soft robot strain models from samples of the robot backbone’s shape evolution, and (2) a network of coupled harmonic oscillators for learning latent dynamics from high-dimensional observations like images. The explicit inclusion of kinematic and potential energy terms in these models allows for stability analysis using standard nonlinear system theory tools, such as Lyapunov methods. For instance, we prove that the coupled oscillator network is both globally asymptotically stable and input-to-state stable.

Fifthly, we exploit the physical structure of the learned models from contribution four to design closed-form setpoint regulators. The controller contains two key components: (1) a potential shaping feedforward term that positions the local/global minimum of the closed-loop potential energy at the setpoint by leveraging the learned model knowledge, and (2) an integral-saturated PID feedback term that rejects disturbances and compensates for modeling errors to prevent steady-state errors. The stability of the closed-loop system can then be analyzed using Lyapunov arguments.

Finally, the thesis explores methods for generating compliant motion behaviors in soft robots beyond low-level control. One approach focuses on assisting users, particularly elderly individuals, with activities of daily living by guiding the low-level controller with brain signals. This is achieved by combining motor imagery classification from wearable EEG devices with compliant impedance control in operational space. The second approach combines an orbitally stable dynamical system in latent space with a bijective neural network parametrized encoder to learn periodic motions from demonstrations. By avoiding reliance on time references, this learned motion policy enables natural and compliant tracking of demonstrated periodic motions. This contribution ensures that not just the robot structure and low-level controller are compliant, but also the high-level motion strategy.

In summary, this thesis aims to work towards safe and precise soft robot behavior in human-centric environments. In this context, safety is achieved through the soft body’s mechanical compliance. To enhance precision while maintaining insight into decision-making, ensuring stability, and preserving computational efficiency, the core contribution of this work is the development of closed-form soft robot controller architectures and their connection to learned models.
Several supporting contributions include a metric for quantifying the safety of soft robots, strategies for integrating kinematic models into shape sensing methods, advanced physics-based actuation models that could serve as future priors for learned models, and moving beyond low-level controllers by devising compliant motion strategies. All proposed models, sensing methods, control strategies, and motion approaches have been thoroughly verified through simulation or real-world testing on soft robots. The code and data from most chapters have been made publicly available on GitHub, thereby contributing to the broader research community. 


\chapter*{Samenvatting}
\addcontentsline{toc}{chapter}{Samenvatting}
\setheader{Samenvatting}

{\selectlanguage{dutch}
Aangezien we er in toenemende mate naar streven robots te integreren in omgevingen die op mensen gericht zijn, staat veiligheid voorop. Doorgaans baseren rigide collaboratieve robots zich op veiligheidsbewuste computationele regelalgorithmen, die gevoelig zijn voor waarnemingsfouten en vaak leiden tot een te voorzichtige werking die de prestaties beperkt. Zachte flexibele robotica (Soft Robotic) daarentegen biedt een veelbelovend alternatief door passieve compliantie in de gehele robotstructuur te waarborgen via materiaaleigenschappen. Deze mechanische compliantie beperkt inherent veiligheidsproblemen die voortkomen uit fouten in waarneming of regeling, al gaat dit ten koste van een aanzienlijke daling in precisie. In de afgelopen jaren zijn grote vooruitgangen geboekt in de zachte robotica, met veelbelovend nieuwe ontwikkelingen in ontwerp, slimme materialen, actuatoren, sensoren, modellen en regelstrategieën. Het modelleren en regelen van continuümzachte robots brengt echter grote uitdagingen met zich mee vanwege hun oneindige vrijheidsgraden, complexe niet-lineaire dynamica en tijdsafhankelijke verschijnselen zoals hysterese. Daardoor missen zachte robots vaak de benodigde capaciteit en bewegingsprecisie, wat leidt tot een compromis waarbij prestaties worden opgeofferd voor veiligheid. Met dit proefschrift stellen wij dat dit compromis kan worden overwonnen door geavanceerdere algoritmen te ontwikkelen die de fysica van de zachte robot in acht nemen. Meer specifiek stellen wij voor om krachtige geleerde modellen te combineren met efficiënte en effectieve modelgebaseerde regelmethoden die inzicht in de handelingen bieden en stabiliteitsgaranties toelaten.

Momenteel bestaan er twee primaire strategie voor het regelen van zachte robots. De eerste maakt gebruik van modelgebaseerde regeling met benaderde fysische modellen die zijn afgeleid uit eerste principes. De tweede leert regeltechnieken rechtstreeks, voornamelijk via reinforcement learning. Beide strategieën kennen aanzienlijke beperkingen. Bestaande modelgebaseerde regelaars zijn niet in staat de dynamica van zachte robots volledig te beheersen en uiteindelijk te benutten, omdat hun onderliggende modellen complexe gedragingen — met name hoe actuatie en externe interacties de vervorming van de robot beïnvloeden — ontoereikend vastleggen. Bovendien vergt het afleiden van deze modellen uitgebreide expertise. Daarnaast maken de gecombineerde complexiteit en onzekerheid van de dynamica tussen een zachte robot en haar omgeving het momenteel onhaalbaar om louter vanuit eerste principes volledige wereldmodellen te ontwikkelen, wat de integratie motiveert van machine-learning-methodes die datagedreven inzichten effectief kunnen benutten. Daarentegen biedt het rechtstreeks leren van de regelstrategie, bijvoorbeeld via reinforcement learning, geen interpretatie en stabiliteitsgaranties en gaat het gepaard met een hoge monsterinefficiëntie, wat problematisch is door de tijdsafhankelijke materiaaleigenschappen en beperkte levensduur van huidige zachte robots.

In dit proefschrift betogen wij dat het combineren van geleerde modellen met modelgebaseerde regelaars een veelbelovend alternatief vormt dat de voordelen van beide strategien samenbrengt: expressieve, datagedreven modellen die minder expertkennis vereisen, gekoppeld aan regelaars die zowel interpreteerbaar als aantoonbaar stabiel zijn. Hoewel de laatste jaren toenemende interesse is ontstaan in het benutten van geleerde modellen voor de regeltechniek, is het merendeel van dit werk afhankelijk van computationeel intensieve optimale-regelmethoden, zoals MPC, om de actuatiesequentie te optimaliseren met behulp van het geleerde model. De hoge rekenkosten van het oplossen van deze optimale-regelproblemen beperken echter de maximale regelfrequentie tijdens inzet, waardoor we de dynamische mogelijkheden van zachte robots niet volledig kunnen benutten. Dit proefschrift richt zich daarom op closed-form-regelaars die de fysische structuur van geleerde modellen benutten binnen een energy-shaping-raamwerk. De belangrijkste uitdaging hierbij is het ontwikkelen van methoden die dergelijke fysische structuren — specifiek kinetische en potentiële energie-termen — integreren in het leren van dynamische modellen voor zachte robots.

Voordat we deze uitdaging aanpakten, moesten we eerst fysica-gebaseerde modellen afleiden uit eerste principes en nieuwe technieken identificeren om ze voor regeling te benutten. Enerzijds maakte dit duidelijk welke fysische voorkennis beschikbaar waren voor het leren, terwijl het anderzijds nieuwe manieren inspireerde om modelgebaseerde regelaars met geleerde modellen te integreren. Het proefschrift behandelt dit onderwerp via verschillende onderling verbonden kernbijdragen.

Ten eerste stellen wij dat het kwantificeren van de veiligheid van zachte robots essentieel is voor het ontwerpen en regelen ervan, zodat het gesloten-lus regelsysteem voldoet aan de specifieke veiligheidseisen van de beoogde toepassingen. Daartoe presenteren wij de eerste veiligheidsmaatstaf voor continuümzachte robots, die de veiligheid van een geïntegreerd ontwerp van een zachte robot beoordeelt door zowel de belichaamde als de computationele intelligentie in rekening te brengen.

Ten tweede verbetert dit proefschrift vormwaarneming voor zachte robots door inzichten uit kinematische modellen te benutten. Dit bereiken we door niet-lineaire optimalisatieproblemen te formuleren en op te lossen die sensormetingen uitlijnen met de ruggraadvormen die door het kinematische model worden voorspeld. We presenteren twee verschillende benaderingen die commerciële visuele en magnetische sensoren te integretren met integreren met respectievelijk SLAM-algoritmen en een geleerd sensormeetmodel om de toestand van de zachte robot nauwkeurig te schatten, een sleutelfactor voor doeltreffende feedbackregeling.

Ten derde introduceert dit proefschrift geavanceerde fysica-gebaseerde actuatiemodellen, waaronder modellen voor robots die worden aangedreven door auxetische metamaterialen — aangeduid als HSA-robots — en modellen die de actuatiedynamica van zuiger-gedreven pneumatische zachte robots vastleggen. Vervolgens benutten we de verkregen modelinzichten om aantoonbaar stabiele niet-lineaire regelaars te ontwerpen — met name een integraal-gesatureerde PID gecombineerd met potentiaalvormende en impedantieregeling in Cartesiaanse ruimte voor planaire HSA-robots, alsook een backstepping-regelaar voor pneumatische zuiger-gedreven zachte robots. Deze bijdrage verdiept ons begrip van actuatie, een cruciaal aspect van het gedrag van zachte robots, en toont aan hoe dergelijke inzichten kunnen worden geïntegreerd in modelgebaseerde regelstrategieën. Bovendien hebben experimenten met HSA-robots de beperkingen blootgelegd van puur fysica-gebaseerde modellen bij het vastleggen van complexe fenomenen zoals hysterese, waarmee de noodzaak van op leren gebaseerd benaderingen wordt onderstreept. In de toekomst kunnen de ontwikkelde actuatiemodellen dienen als waardevolle fysische voorkennis voor geleerde modellen.

Ten vierde presenteert het proefschrift technieken voor het leren van modellen van zachte robots die fysische structuren integreren en tegelijk stabiliteit waarborgen. Dit realiseren we door fysica-gebaseerde dynamische modellen op te nemen in het leeralgorithme, dat de vrije parameters van de dynamica bepaalt en optioneel een coördinatentransformatie — zoals een latente-ruimte-encodering — optimaliseert. Twee noemenswaardige benaderingen worden geïntroduceerd: (1) een algoritme dat laagdimensionale vervormingsmodellen van zachte robots afleidt uit voorbeelden van de vorm-evolutie van de ruggengraat van de robot, en (2) een netwerk van gekoppelde harmonische oscillatoren voor het leren van latente dynamica uit hoogdimensionale observaties zoals beelden. De expliciete opname van kinematische en potentiële-energie-termen in deze modellen maakt stabiliteitsanalyse mogelijk met tools uit de niet-lineare systeemtheorie, zoals Lyapunov-methoden. Zo bewijzen wij bijvoorbeeld dat het netwerk van gekoppelde oscillatoren zowel globaal asymptotisch stabiel als input-to-state-stabiel is.

Ten vijfde benutten wij de fysische structuur van de geleerde modellen uit bijdrage vier om closed-form-setpoint-regelaars te ontwerpen. De regelaar bevat twee kernelementen: (1) een potentiaalvormende-feedforwardterm die het lokale of globale minimum van de gesloten-lus-potentiële energie op het setpoint plaatst door de kennis uit het geleerde model te benutten, en (2) een integraal-gesatureerde PID-feedbackterm die verstoringen onderdrukt en modelleringsfouten compenseert om stationaire fouten te voorkomen. De stabiliteit van het gesloten-lus regelsysteem kan vervolgens met Lyapunov-argumenten worden geanalyseerd.

Ten slotte verkent het proefschrift methoden om conform methode in zachte robots te genereren, voorbij de laag-niveau-regeling. Eén bewegingen richt zich op het ondersteunen van gebruikers, met name ouderen, bij activiteiten van het dagelijks leven door de laag-niveau-regelaar te sturen met hersensignalen. Dit wordt bereikt door motor-beeld-classificatie van draagbare EEG-apparaten te combineren met compliant impedantieregeling in de operationele ruimte. De tweede methode combineert een orbitale-stabiel dynamisch systeem in latente ruimte met een bijectieve neurale-netwerk-geparametriseerde encoder om periodieke bewegingen uit demonstraties te leren. Doordat het bewegingsbeleid geen gebruikmaakt van tijdreferenties, kan het op een natuurlijke en meegevende manier periodieke gedemonstreerde bewegingen volgen. Deze bijdrage waarborgt dat niet alleen de robotstructuur en de laag-niveau-regelaar compliant zijn, maar ook de hoog-niveau-bewegingsstrategie.

Samengevat beoogt dit proefschrift bij te dragen aan veilig en nauwkeurig gedrag van zachte robots in mensgerichte omgevingen. Veiligheid wordt bereikt via de mechanische compliantie van het zachte lichaam. Om de precisie te verbeteren terwijl inzicht in besluitvorming, stabiliteit en computationele efficiëntie behouden blijven, is de kernbijdrage van dit werk de ontwikkeling van closed-form-regelarchitecturen voor zachte robots en hun koppeling aan geleerde modellen. Verscheidene ondersteunende bijdragen omvatten een maatstaf voor het kwantificeren van de veiligheid van zachte robots, strategieën voor het integreren van kinematische modellen in vormwaarnemingsmethoden, geavanceerde fysica-gebaseerde actuatiemodellen die als toekomstige voorkennis voor geleerde modellen kunnen dienen, en het overstijgen van laag-niveau-regelaars door compliant bewegingsstrategieën te ontwikkelen. Alle voorgestelde modellen, waarnemingsmethoden, regelstrategieën en bewegingsbenaderingen zijn grondig gevalideerd via simulaties of experimenten in de echte wereld op zachte robots. De code en data van de meeste hoofdstukken zijn publiekelijk beschikbaar gesteld op GitHub en dragen zo bij aan de bredere onderzoeksgemeenschap.
}

\chapter*{Zusammenfassung}
\addcontentsline{toc}{chapter}{Zusammenfassung}
\setheader{Zusammenfassung}

{\selectlanguage{german}
Da wir zunehmend bestrebt sind, Roboter in menschenzentrierte Umgebungen zu integrieren, hat Sicherheit höchste Priorität. Traditionell haben starre kollaborative Roboter auf sicherheitsbewusste, rechnergestützte Regelstrategien zurückgegriffen, die anfällig für Wahrnehmungsfehler sind und häufig zu übervorsichtigem Verhalten führen, das die Leistung begrenzt. Im Gegensatz dazu bietet die weiche Robotik (Soft Robotic) eine vielversprechende Alternative, indem sie durch Materialweichheit eine passive Nachgiebigkeit im gesamten Roboteraufbau sicherstellt. Diese mechanische Nachgiebigkeit mindert Sicherheitsprobleme, die aus Wahrnehmungs- oder Regelungsfehlern resultieren, geht jedoch mit einem erheblichen Präzisionsverlust einher. In den letzten Jahren wurden in der weichen Robotik bedeutende Fortschritte erzielt, mit interessanten neuen Entwicklungen in Design, intelligenten Materialien, Aktoren, Sensoren, Modellen und Regelungsstrategien. Dennoch stellen die Modellierung und Regelung von kontinuierlichen weichen Robotern nach wie vor große Herausforderungen dar, da sie unendlich viele Freiheitsgrade, komplexe nichtlineare Dynamiken und zeitabhängige Verhaltensweisen wie Hysterese aufweisen. Folglich mangelt es weichen Robotern häufig an der erforderlichen Leistungsfähigkeit und Bewegungspräzision, was zu einem Kompromiss führt, bei dem Leistung der Sicherheit geopfert wird. In dieser Arbeit argumentieren wir, dass dieser Kompromiss überwunden werden kann, indem fortschrittlichere Algorithmen entwickelt werden, die auf der Physik des weichen Roboters basieren. Genauer gesagt schlagen wir vor, leistungsstarke, gelernte Modelle mit effizienten und effektiven modellbasierten Regelungsansätzen zu kombinieren, die Interpretierbarkeit des Handelns ermöglichen und Stabilitätsgarantien zulassen.

Derzeit existieren zwei Hauptansätze zur Regelung weicher Roboter. Der erste verwendet modellbasierte Regelungen, die auf approximierten, physikbasierten Modellen beruhen, die aus ersten Prinzipien abgeleitet sind. Der zweite Ansatz erlernt Steuerungsrichtlinien direkt, überwiegend mittels Reinforcement Learning. Beide Strategien weisen erhebliche Einschränkungen auf. Bestehende modellbasierte Regler können die Dynamik weicher Roboter nicht vollständig beherrschen und schließlich ausnutzen, da ihre zugrunde liegenden Modelle komplexe Verhaltensweisen, insbesondere wie Aktuierung und externe Interaktionen die Verformung des Roboters beeinflussen, unzureichend erfassen. Darüber hinaus erfordert die Ableitung dieser Modelle umfangreiches Expertenwissen. Ferner machen die kombinierte Komplexität und Unsicherheit der Dynamik zwischen einem weichen Roboter und seiner Umgebung es derzeit unmöglich, umfassende Weltmodelle ausschließlich aus ersten Prinzipien zu entwickeln, was die Integration von Machine-Learning-Ansätzen motiviert, die datengetriebene Erkenntnisse effektiv nutzen können. Umgekehrt mangelt es beim direkten Erlernen des Reglers – etwa durch Reinforcement Learning – an Interpretierbarkeit und Stabilitätsgarantien, während es zudem hochgradig sample-ineffizient ist, ein signifikanter Nachteil angesichts der zeitabhängigen Materialeigenschaften und der begrenzten Lebensdauer heutiger weicher Roboter.

In dieser Arbeit vertreten wir die Auffassung, dass die Kombination gelernter Modelle mit modellbasierten Reglern eine vielversprechende Alternative darstellt, welche die Vorteile beider Ansätze vereint: ausdrucksstarke, datengetriebene Modelle, die weniger Expertenwissen erfordern, gepaart mit Reglern, die sowohl interpretierbar als auch nachweislich stabil sind. Obwohl in den letzten Jahren das Interesse gewachsen ist, gelernte Modelle für die Regelung zu nutzen, basiert der Großteil der Arbeiten in diesem Bereich auf rechenintensiven optimalen Regelungsmethoden wie MPC, um mit dem gelernten Modell die Antriebssequenz zu optimieren. Die hohen Rechenkosten bei der Lösung dieser Optimalsteuerungsprobleme begrenzen jedoch die maximale Regelungsfrequenz während des Einsatzes und verhindern so die volle Ausschöpfung der dynamischen Fähigkeiten weicher Roboter. Stattdessen verfolgt diese Arbeit geschlossene Regler in geschlossener Form, die die physikalische Struktur gelernter Modelle ausnutzen um deren Energieprofil durch Regelung zu verformen. Die Hauptschwierigkeit besteht darin, Ansätze zu entwickeln, die solche physikalischen Strukturen – insbesondere kinetische und potenzielle Energieanteile – in das Erlernen dynamischer Modelle für weiche Roboter integrieren.

Bevor wir uns dieser Hauptaufgabe widmeten, mussten wir zunächst physikbasierte Modelle aus ersten Prinzipien weiterentwickeln und neue Techniken identifizieren, um sie für die Regelung zu nutzen. Einerseits wurde dadurch geklärt, welche physikalischen Priors für das Lernen verfügbar sind, andererseits inspirierte es neue Wege, modellbasierte Regler mit gelernten Modellen zu integrieren. Die Arbeit behandelt dieses Thema durch mehrere miteinander verbundene Hauptbeiträge.

Erstens argumentieren wir, dass die Quantifizierung der Sicherheit weicher Roboter entscheidend ist, um sie so zu entwerfen und zu regeln, dass das geschlossene System die spezifischen Sicherheitsanforderungen seiner vorgesehenen Anwendungen erfüllt. Zu diesem Zweck präsentieren wir die erste Sicherheitsmetrik für kontinuierliche weiche Roboter, die die Sicherheit eines integrierten weichen Roboters bewertet, indem sie sowohl seine verkörperte als auch seine rechnerische Intelligenz berücksichtigt.

Zweitens verbessert diese Arbeit die Formwahrnehmung (Shape Sensing) für weiche Roboter, indem Erkenntnisse aus kinematischen Modellen genutzt werden. Dies erreichen wir, indem wir nichtlineare Optimierungsprobleme formulieren und lösen, die Sensormessungen mit den durch das kinematische Modell vorhergesagten Rückgratsformen abgleichen. Wir präsentieren zwei unterschiedliche Ansätze, die kommerzielle Sensoren – nämlich visuelle und magnetische – mit SLAM-Algorithmen bzw. mit einem gelernten Sensor-Messmodell kombinieren, um den Zustand des weichen Roboters genau zu schätzen, eine wichtige Voraussetzung für eine effektive Regelung mit Rückkopplung.

Drittens führt diese Arbeit fortschrittliche physikbasierte Aktuierungsmodelle ein, darunter solche für Roboter, die durch auxetische Metamaterialien – sogenannte HSA-Roboter – angetrieben werden, sowie Modelle, die die Antriebsdynamik von kolbengesteuerten pneumatischen weichen Robotern erfassen. Anschließend nutzen wir die gewonnenen Modellkenntnisse, um nachweislich stabile nichtlineare Regler zu entwerfen – konkret einen integralgesättigten PID-Regler kombiniert mit Umformung der potentiellen Energie und Impedanzregelung im kartesischen Raum für planare HSA-Roboter sowie einen Backstepping-Regler für pneumatische, kolbengesteuerte weiche Roboter. Dieser Beitrag vertieft unser Verständnis des Antriebs, eines entscheidenden Aspekts des Verhaltens weicher Roboter, und zeigt, wie solche Erkenntnisse in modellbasierte Regelungsstrategien eingebunden werden können. Darüber hinaus haben Experimente mit HSA-Robotern die Grenzen rein physikbasierter Modelle bei der Erfassung komplexer Phänomene wie Hysterese aufgezeigt und motivieren so die Erforschung lernbasierter Ansätze. Zukünftig können die entwickelten Antriebsmodelle als wertvolle physikalische Vorwissen für gelernte Modelle dienen.

Viertens stellt die Arbeit Techniken zum Erlernen von Modellen weicher Roboter vor, die physikalische Strukturen integrieren und gleichzeitig Stabilität gewährleisten. Dies erreichen wir, indem wir physikbasierte dynamische Modelle in den Lernalgorithmus einbetten, der die freien Parameter der Dynamik bestimmt und optional eine Koordinatentransformation – etwa eine Kodierung in einen latenten Raum – optimiert. Zwei hervorzuhebende Ansätze werden vorgestellt: (1) ein Algorithmus, der aus Stichproben der Entwicklung der Rückgratform des Roboters niedrigdimensionale Deformationsmodelle extrahiert, und (2) ein Netzwerk gekoppelter harmonischer Oszillatoren zum Erlernen latenter Dynamik aus hochdimensionalen Beobachtungen wie Bildern. Die explizite Einbeziehung von kinematischen und potenziellen Energieanteilen in diese Modelle ermöglicht eine Stabilitätsanalyse mit Standardwerkzeugen der nichtlinearen Systemtheorie, wie z. B. Lyapunov-Methoden. So beweisen wir beispielsweise, dass das Netzwerk gekoppelter Oszillatoren sowohl global asymptotisch stabil als auch Eingangs-Zustands (Input-to-State) stabil ist.

Fünftens nutzen wir die physikalische Struktur der aus Beitrag vier gelernten Modelle, um Regler in geschlossener Form für Sollwertregelung zu entwerfen. Der Regler enthält zwei zentrale Komponenten: (1) einem Vorwärtsregelung-Teil basierend auf Umformung der potenziellen Energie, der das lokale/globale Minimum der geschlossenen Potenzialenergie mithilfe des gelernten Modellwissens am Sollwert positioniert, und (2) einen integralgesättigten PID-Feedback-Term, der Störungen unterdrückt und Modellierungsfehler kompensiert, um stationäre Fehler zu vermeiden. Die Stabilität des geschlossenen Systems kann anschließend mithilfe von Lyapunov-Argumenten analysiert werden.

Schließlich untersucht die Arbeit Methoden zur Erzeugung nachgiebiger Bewegungsverhalten in weichen Robotern über die Regelung auf niedriger Ebene (Low Level) hinaus. Ein Ansatz konzentriert sich darauf, Benutzer, insbesondere ältere Menschen, bei alltäglichen Aktivitäten zu unterstützen, indem der Low-Level-Regler durch Gehirnsignale gesteuert wird. Dies wird erreicht, indem die Klassifizierung der Bewegungsimagination von tragbaren EEG-Geräten mit einer nachgiebigen Impedanzregelung im Arbeitsraum kombiniert wird. Der zweite Ansatz verbindet ein orbital stabiles dynamisches System im latenten Raum mit einem bijektiven, durch ein neuronales Netzwerk parametrisierten Encoder, um periodische Bewegungen aus Demonstrationen zu erlernen. Durch den Verzicht auf Zeitreferenzen ermöglicht diese gelernte Bewegungsstrategie eine natürliche und nachgiebige Verfolgung der demonstrierten periodischen Bewegungen. Dieser Beitrag stellt sicher, dass nicht nur der Roboteraufbau und der Low-Level-Regler nachgiebig sind, sondern auch die Bewegungsstrategie auf hoher Ebene (High-Level).

Zusammenfassend zielt diese Arbeit darauf ab, sicheres und präzises Verhalten weicher Roboter in menschenzentrierten Umgebungen zu ermöglichen. In diesem Kontext wird Sicherheit durch die mechanische Nachgiebigkeit des weichen Körpers erreicht. Um die Präzision zu erhöhen und gleichzeitig Einblick in die Entscheidungsfindung zu behalten, Stabilität sicherzustellen und die Rechenaufwände gering zu halten, besteht der Kernbeitrag dieser Arbeit in der Entwicklung von Reglerarchitekturen in geschlossener Form für weiche Roboter und ihrer Verknüpfung mit gelernten Modellen. Mehrere unterstützende Beiträge umfassen eine Metrik zur Quantifizierung der Sicherheit weicher Roboter, Strategien zur Integration kinematischer Modelle in Formwahrnehmungsmethoden, fortschrittliche physikbasierte Aktuierungsmodelle, die als zukünftiges Vorwissen für gelernte Modelle dienen können, sowie das Überschreiten von Low-Level-Reglern durch die Entwicklung nachgiebiger Bewegungsstrategien. Alle vorgeschlagenen Modelle, Sensormethoden, Regelungsstrategien und Bewegungsansätze wurden umfassend durch Simulation oder reale Tests an weichen Robotern verifiziert. Der Code und die Daten aus den meisten Kapiteln wurden auf GitHub öffentlich zugänglich gemacht und tragen so zur breiteren Forschungsgemeinschaft bei.
}



