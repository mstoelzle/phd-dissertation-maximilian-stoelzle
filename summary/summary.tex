\chapter*{Summary}
\addcontentsline{toc}{chapter}{Summary}
\setheader{Summary}

% As we work towards bringing robotics into human-centric environments, safety is paramount. However, today, safety is mostly ensured through computational control policies, which makes it susceptible to perception error and often leads to overly cautious behavior, limiting the robot's performance. Instead, soft robotics promises to improve safety by establishing passive compliance of the entire body through material softness. This \emph{embodied intelligence} is not negatively influenced by perception or control errors. There has been tremendous progress in the last few years in the domain of soft robotics, with new designs, smart materials, actuators, sensors, models, and control approaches being proposed by the research community. In particular, the modeling and control of continuum soft robots is very challenging as they theoretically have infinite degrees of freedom, exhibit complex nonlinear dynamics, and even time-dependent behavior such as hysteresis. Currently, two approaches for control are dominant: model-based control based on physics-based models derived from first principles under significant approximations and learned control policies, mainly by means of reinforcement learning.
% However, there exist major issues with both approaches: i) existing model-based controllers are unable to exploit the full dynamic of soft robots as their underlying models do not sufficiently capture the complex dynamic behavior of soft robots, specifically concerning the influence of actuation and external interaction on the soft robot's deformation, and require expert knowledge in order to meet suitable and appropriate model design choice. On the other side, ii) reinforcement learning does not offer interoperability and stability guarantees and is additionally very sample inefficient, which is an issue when considering the time-dependent material properties and currently limited lifetime of soft robots. 
% Therefore, this thesis argues that a promising alternative approach would be to combine learned models with model-based controllers, which would combine benefits from both worlds: expressive models that require less expert knowledge derived from data combined with controllers whose behavior is interpretable and provably stable.
% The primary challenge for achieving this solution is to identify the necessary characteristics and structures that a learned model needs to exhibit in order to allow us to apply the existing model-based control strategies, such as PID-like feedback+feedforward while making sure that the closed-loop robot system exhibits compliant and safe behavior.
% We tackle this complex challenge in this doctoral thesis through multiple interconnected key contributions:
% 1) leveraging kinematic models for soft robot shape sensing, for example, based on the readout of visual or magnetic sensors, allows us to receive more accurate estimates of the soft robot's state, which is a requirement for effective feedback control; 2) we develop advanced physics-based actuation models, such as modeling the behavior of robots that are actuated through auxetic metamaterials or considering the actuation dynamics of piston-driven pneumatic soft robots;
% 3) we identify techniques that allow us to learn soft robot models with physical structures and stability guarantees. Two of the approaches that we propose are a) an algorithm that identifies low-dimensional, physics-based strain models based on samples of the shape evolution of the robot's backbone and b) a network consisting of coupled harmonic oscillators for learning the latent dynamics of a physical system from high-dimensional observations such as images. The physical structure that is exposed by both approaches allows us to integrate the model into a PID-like feedback controller with potential shaping serving as the feedforward term;
% 4) finally, as a means of moving beyond low-level control, we propose two approaches for generating compliant motion behaviors for soft robots. The first approach, i), is tailored to assist the user, for example, elderly people, with activities of daily living and allows the user to guide the goal of the low-level controller via their thoughts. To achieve a compliant control behavior, we combine motor imagery classified based on the data measured by wearable EEG devices with compliant impedance control in Cartesian space. The second approach, ii) combines an orbitally stable dynamical system in latent space with a bijective neural-network parametrized encoder for learning periodic motions from demonstrations. As the learned motion policy does not rely on a time reference, it allows for natural and compliant tracking of the demonstrated motion.
% The behaviors of some models and controllers have been analyzed theoretically in terms of stability characteristics.
% The models, sensing, control, and motion strategies developed in this thesis have been extensively validated both in simulation and on real-world (soft) robots. The code and data of most chapters have been open-sourced on GitHub.

As we work towards integrating robotics into human-centric environments, safety remains a paramount concern. While safety is traditionally ensured through computational control policies, this approach is vulnerable to perception errors and often results in overly cautious behavior that limits robot performance. Soft robotics presents a promising alternative by establishing passive compliance throughout the entire robot body through material softness. This embodied intelligence is inherently resistant to perception or control errors. Recent years have witnessed remarkable progress in soft robotics, with researchers developing new designs, smart materials, actuators, sensors, models, and control approaches. However, the modeling and control of continuum soft robots presents significant challenges due to their infinite degrees of freedom, complex nonlinear dynamics, and time-dependent behaviors such as hysteresis.

Currently, two dominant approaches exist for controlling soft robots. The first relies on model-based control using physics-based models derived from first principles under significant approximations. The second employs learned control policies, primarily through reinforcement learning. Both approaches face substantial limitations. Existing model-based controllers cannot fully exploit the dynamics of soft robots as their underlying models inadequately capture complex dynamic behavior, particularly regarding how actuation and external interaction influence the robot's deformation. These models also require considerable expert knowledge to make appropriate design choices. Conversely, reinforcement learning lacks interoperability and stability guarantees while being highly sample inefficient—a significant drawback given the time-dependent material properties and limited lifetime of current soft robots.

This thesis proposes that combining learned models with model-based controllers offers a promising alternative approach, uniting the benefits of both methods: expressive data-derived models requiring less expert knowledge coupled with controllers whose behavior is both interpretable and provably stable. The primary challenge lies in identifying the necessary characteristics and structures that a learned model must possess to enable the application of existing model-based control strategies, such as PID-like feedback with feedforward control while ensuring the closed-loop robot system maintains compliant and safe behavior.
The thesis addresses this complex challenge through several interconnected key contributions. 

First, we argue that quantifying the safety of soft robots is crucial for designing and controlling them to ensure that the closed-loop system meets the specific safety requirements of their intended applications. To this end, we present the first safety metric for continuum soft robots, which assesses the safety of an integrated soft robot design by accounting for both its embodied and computational intelligence.

Secondly, this thesis advances kinematic models for soft robot shape sensing, utilizing visual or magnetic sensor readings to obtain more accurate estimates of the soft robot's state—a crucial requirement for effective feedback control. 

Thirdly, the thesis develops sophisticated physics-based actuation models, including those for robots actuated through auxetic metamaterials and models accounting for the actuation dynamics of piston-driven pneumatic soft robots. Subsequently, we exploit the gained model knowledge with provably stable nonlinear controllers.
This contribution highlights the control-oriented structure of physics-based models, which will be utilized in the subsequent contribution. Additionally, the research underpinning this contribution reveals the current limitations regarding the complexity and computational demands of physics-based models, thereby motivating the exploration of potentially more computationally efficient neural network-parametrized models.

Fourthly, the thesis identifies techniques for learning soft robot models with physical structures and stability guarantees. Two notable approaches are presented: an algorithm that identifies low-dimensional, physics-based strain models using samples of the robot backbone's shape evolution and a network of coupled harmonic oscillators for learning latent dynamics of physical systems from high-dimensional observations such as images. The physical structure exposed by both approaches enables their integration into PID-like feedback controllers with potential shaping as the feedforward term.

Finally, the thesis explores methods for generating compliant motion behaviors in soft robots beyond low-level control. One approach focuses on assisting users, particularly elderly individuals, with activities of daily living by guiding the low-level controller with brain signals. This is achieved by combining motor imagery classification from wearable EEG devices with compliant impedance control in task space. The second approach combines an orbitally stable dynamical system in latent space with a bijective neural network parametrized encoder to learn periodic motions from demonstrations. By avoiding reliance on time references, this learned motion policy enables natural and compliant tracking of demonstrated periodic motions. This contribution ensures that not just the robot structure and low-level controller are compliant but also the high-level motion strategy.

The thesis includes a theoretical analysis of the stability characteristics of several models and controllers. All proposed models, sensing methods, control strategies, and motion approaches have undergone extensive validation through either simulation or real-world testing on soft robots. Additionally, the code and data from most chapters have been made publicly available through GitHub, contributing to the broader research community's ongoing work in this field.


\chapter*{Samenvatting}
\addcontentsline{toc}{chapter}{Samenvatting}
\setheader{Samenvatting}

{\selectlanguage{dutch}

  Samenvatting in het Nederlands.
}

\chapter*{Zusammenfassung}
\addcontentsline{toc}{chapter}{Zusammenfassung}
\setheader{Zusammenfassung}

{\selectlanguage{german}

  Zusammenfassung in Deutsch.
}



