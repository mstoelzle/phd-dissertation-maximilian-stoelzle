\section{Conclusion}
In this chapter, we proposed to combine motor imagery-based \gls{BMI} systems with continuum soft robots. This symbiosis promises the safe and compliant operation of robots that can assist people with limb impairments in their daily lives.
While the binary motor imagery classifier achieves an accuracy of only $\approx \SI{70}{\percent}$, we demonstrated experimentally its effectiveness through assistance with an activity of daily living and safe operation.
As demonstrated in the \gls{ADL} experiment, the physical intelligence of the soft robot can compensate for errors and deviations in the output of the \gls{BMI} classifier.
% Furthermore, we introduced a Cartesian impedance controller for planar \gls{HSA} robots that can deal with the peculiar characteristics of these robots (e.g., underactuation, non-affinity in control, etc.), and allows for model-based control without interfering with the structural compliance of the system.

% For future work, we suggest running a user study that includes a diverse group of non-expert subjects, and a more diverse set of soft robots.
% Furthermore, we it would be interesting to deploy the methodology on soft robots with more \glspl{DOF}, such as, for example, the Helix soft robot~\citep{guan2023trimmed}.
% Finally, relying on \gls{SOTA} \gls{EEG} classifiers, such as CTNet~\citep{zhao2024ctnet}, might increase the classification accuracy and ultimately allow for full 3D spatial brain control of the soft robot's task space motion - including the addition of a \emph{rest} mode which would allow the end-effector to remain at a pose.
For future work, we recommend conducting a user study with a diverse group of non-expert participants and incorporating a broader range of soft robots. Connected, it would be interesting to apply this methodology to soft robots with more \glspl{DOF}, such as the Helix soft robot~\citep{guan2023trimmed}. Finally, utilizing \gls{SOTA} \gls{EEG} classifiers like CTNet~\citep{zhao2024ctnet} could improve classification accuracy and ultimately enable full 3D spatial brain control of the soft robot’s task space motion—including the addition of a \emph{rest} mode to keep the end-effector at a fixed pose.