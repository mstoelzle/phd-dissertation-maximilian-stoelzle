\chapter{Conclusions and Future Work}
\label{chp:conclusion}
% Write a separate chapter called ‘Conclusion’ (or, for example, Conclusion & Discussion)
% - Present a clear answer to your main question / relate conclusions clearly to your main aim
% - Evaluate the validity of that answer in light of possible limitations
% - State the scientific and societal importance of your study
% - Reflect on possible applications and suggest avenues for future research

The conclusions of your thesis.

\section{Conclusions}\label{sec:conclusions}

Finally, we compare the differences of the two proposed methods. A major benefit of the strain-model-based approach presented in Chapter~\ref{chp:pcsregression} lies in the fully \emph{white-box} nature as the final derived dynamical model is fully physics-based and respectable. Furthermore, it exhibits a strong inductive bias for the particular dynamics of soft robots, which can be both a benefit and a potential weakness: the inductive bias improves data efficiency, reduces the danger of overfitting, and leads to exceptional extrapolation capabilities. However, when the soft robot exhibits dynamics, such as hysteresis, nonlinear elasticity, or deforming cross-sections, they could likely not be captured by the 

\section{Future Work}\label{sec:future_work}
