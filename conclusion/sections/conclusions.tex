\section{Conclusions}\label{sec:conclusion:conclusions}
% Write a separate chapter called ‘Conclusion’ (or, for example, Conclusion & Discussion)
% - Present a clear answer to your main question / relate conclusions clearly to your main aim
% - Evaluate the validity of that answer in light of possible limitations
% - State the scientific and societal importance of your study
% - Reflect on possible applications and suggest avenues for future research

% In the following, we will first draw conclusions for each of the key contributions that we made in this thesis. Subsequently, we will in Sec.~\ref{sub:conclusion:conclusions:core_contribution} deduce overarching conclusions of this thesis, specifically relating to the core contribution of this thesis.
In the following, we will begin by summarizing the conclusions associated with each key contribution of this thesis. Next, as detailed in Sec.~\ref{sub:conclusion:conclusions:core_contribution}, we will synthesize overarching insights, with a particular focus on the core contribution.

\subsection{Contribution I - Safety Metric}
% \begin{itemize}
%     \item \textbf{Answer to RQ:} We propose a quantitative safety metric that is tailored to soft robots and that can be computed in a computationally tractable fashion.
%     \item \textbf{Limitations:} Missing experimental verification, constrained to slender structures, assumes linear contact, conservative as it relies on an approximation of the kinematic model behavior
% \end{itemize}
% In Chapter~\ref{chp:safetymetric}, we derived a quantitative safety metric for soft robots that accounts for the peculiar characteristics and dynamics of soft robots by deriving the collision dynamics between the soft robot and a human from first principles.
% Specifically, we take analog to the ISO design standard for \glspl{Cobot}~\citep{Isots_15066_2016}, the maximum contact pressure experienced during the collision as a proxy for the possible injury severity.
% To reduce the computational burden for evaluating this metric, we identify a conservative upper bound on the injury severity that allows us to state the maximum contact pressure in closed form.
% Crucially, the proposed \glsxtrfull{SRISC} allows for collisions along the entire continuum soft robot body and is built based on existing and verified soft robotic dynamic models which take into account the inertia, elasticity, and actuation of the soft robot.
% We extend this metric to assess the safety of a soft robotic design by evaluating the maximum \gls{SRISC} within the operating conditions of the soft robots.
Addressing RQ~\ref{rq:soft_robotic_safety}, we introduced in Chapter~\ref{chp:safetymetric} a quantitative safety metric for soft robots by modeling the collision dynamics between a soft robot and a human from first principles. Following the ISO standards for cobots~\citep{Isots_15066_2016}, we use the maximum contact pressure during a collision as a proxy for injury severity. To keep the evaluation computationally efficient, we derived a conservative upper bound that allows for a closed-form expression of the maximum contact pressure. Notably, the proposed \gls{SRISC} accommodates collisions along the entire continuum of the soft robot and is built on validated dynamic models that incorporate its inertia, elasticity, and actuation. We then apply this metric to assess a soft robotic design by evaluating the maximum \gls{SRISC} under its operating conditions.

% We concede that the proposed safety metric has not yet been experimentally verified and validated. 
% % Furthermore, as a consequence of the underlying dynamic model, it only
% Additionally, we take several assumptions (e.g., no dissipation, a linear spring contact force model, the human as a constrained point mass, and a finite number of backbone segments) to make the evaluation of the metric more computationally tractable, leading to a conservative estimate of the injury severity.
% This underestimation of the safety could possibly lead to less performant and too cautious robot behavior.
% Future work might explore how the safety could be computed more accurately while preserving a computationally tractable setting.
We acknowledge that our safety metric has not yet been experimentally validated. To ensure computational tractability, we made several simplifying assumptions—such as neglecting dissipation, using a linear spring contact model, modeling the human as a constrained point mass, and limiting the number of backbone segments—which result in a conservative injury severity estimate. This conservative bias might lead, for example when leveraging the metric for safety-aware control, to overly cautious robot behavior. Future work should focus on refining the metric to improve accuracy while maintaining computational efficiency.

In Section~\ref{sec:conclusion:future_work}, we give recommendations for future research and interesting applications for the proposed safety metric.

\subsection{Contribution II - Leveraging Kinematic Models for Soft Robot Shape Sensing}
% \begin{itemize}
%     \item \textbf{Answer to RQ:} We can leverage kinematic models for shape sensing by solving optimization problems that aim to find an agreement between sensor measurements and the attainable backbone shapes allowed by the kinematic model.
%     \item \textbf{Limitations:} No real-world experiment with multi-segment soft robots and more complex kinematic models. Many sensing methodologies that we did not investigate. Knowledge about the dynamics should be made more use of, and observers should be embraced.
% \end{itemize}
% We have shown in Chapters~\ref{chp:srslam} \& \ref{chp:promasens} how kinematic models (e.g. \gls{PCC}~\citep{webster2010design}) can be leveraged within shape sensing by solving a, likely nonlinear, optimization problem which aims to find an agreement between the sensor measurements and the backbone shapes attainable by the chosen kinematic model.
% We proposed (1) a approach relying on cameras attached to the soft robot's body perceiving the environment combined with \gls{vSLAM} algorithms, and (2) embedded magnets and magnetic sensors.
% We have verified both approaches experimentally with a pneumatically-actuated soft segment. Future work should investigate the reliability and performance of such methods on multi-segment soft robots.
% Furthermore, we realize that current proprioception approaches~\cite{wang2018toward, hegde2023sensing} rarely make use of dynamic model knowledge to filter the state estimates. Therefore, we suggest to further explore and embrace the usage of nonlinear observers~\citet{shao2023model} for improving the state estimates.
Answering \gls{RQ}~\ref{rq:shape_sensing}, in Chapters~\ref{chp:srslam} and \ref{chp:promasens}, we demonstrated that kinematic models (e.g., \gls{PCC}~\citep{webster2010design}) can be employed for shape sensing by solving a nonlinear optimization problem that aligns sensor measurements with the backbone shapes attainable by the chosen kinematic model. We proposed two approaches: one using cameras, the robot perceiving and tracking its environment combined with \gls{vSLAM} algorithms, and another utilizing embedded magnets with magnetic sensors. Both methods were validated experimentally on a pneumatically-actuated soft segment, though future work should assess their reliability and performance on multi-segment robots.
% Furthermore, we noticed exciting recent work~\citep{caroleo2025soft} that (also) integrates commercial sensors, in this case, miniature \gls{TOF} sensors, for the proprioception of soft robots by analyzing the movement of the environment in the local frame of the soft body, that could benefit of leveraging kinematic model knowledge for shape sensing.
Furthermore, we observed exciting recent work~\citep{caroleo2025soft} that also integrates commercial sensors—specifically, miniature \gls{TOF} sensors—for soft robot proprioception by analyzing environmental motion in the soft body’s local frame, similar to the \gls{vSLAM} method presented in Chapter~\ref{chp:srslam}, an approach that could benefit from incorporating kinematic model knowledge for improving shape sensing.
Finally, as current proprioception methods rarely incorporate dynamic model information for filtering state estimates, we advocate exploring nonlinear observers~\citep{shao2023model} to enhance state estimation.

\subsection{Contribution III - Control with Advanced Physics-based Actuation Models for Soft Robots}
% \begin{itemize}
%     \item \textbf{Answer to RQ:} We can develop advanced actuation models for soft robots by deriving the actuation behavior from first principle and can exploit the new model knowledge using existing nonlinear control techniques such as backstepping for control.
%     \item \textbf{Limitations:} Very soft robot specific. As a large variety of actuation methods exist for soft robots, a separate model needs to be developed for every case.
% \end{itemize}
% In Chapters~\ref{chp:hsamodel} \& \ref{chp:backstepping}, we have demonstrated the derivation of physics-based actuation models for soft robots from first principles and have subsequently demonstrated how the newly obtained actuation model knowledge can be exploited for control using existing techniques, for example via mapping into actuation coordinates~\citep{pustina2024input} or backstepping~\citep{khalil2002nonlinear}.
% In the case of planar \gls{HSA} robots, we also have conducted extensive experimental studies verifying both the model and the derived controller.
% Beyond the limitations of the individual models and controllers, we note that the large variety of actuation modalities for soft robots spanning from pneumatic, tendon-driven actuation via dielectric elastomers and metamaterials~\citep{zaidi2021actuation} to shape memory alloys poses a major challenge for the community as each actuation concept needs to be individually modeled and specialized controllers need to be derived~\citep{copaci2020flexible, soleti2024nonlinear, soleti2025model}.
Examining \gls{RQ}~\ref{rq:physical_structure_learned_models}, in Chapters~\ref{chp:hsamodel} and \ref{chp:backstepping}, we derived physics-based actuation models for soft robots from first principles and demonstrated how to leverage these models for control—using methods like mapping into actuation coordinates~\citep{pustina2024input} and backstepping~\citep{khalil2002nonlinear}. For planar \gls{HSA} robots, we further validated both the model and controller through extensive experiments. However, the wide range of soft robot actuation modalities—from pneumatic and tendon-driven systems to dielectric elastomers, metamaterials~\citep{zaidi2021actuation}, and shape memory alloys—poses a significant challenge for the community, as each actuation method requires its own tailored modeling approach and specialized controller~\citep{copaci2020flexible, soleti2024nonlinear, soleti2025model}.

\subsection{Contribution IV - Integrating Physical Structure and Stability Guarantees into Learned Models}
% \begin{itemize}
%     \item \textbf{Answer to RQ:} Learn models while incorporating physical priors and stability guarantees by directly using the dynamics of a defined mechanical/physical system and learning the free dynamic parameters and optionally a change of coordinates/encoding.
%     \item \textbf{Limitations:} No real-world experiments, strong assumptions on attractor type (i.e., \gls{GAS})/soft robot kinematic model, no terms to capture time-dependent effects, hybrid or discontinuous dynamics, etc.
% \end{itemize}
% In this thesis, we presented two approaches for integrating physical structure and stability guarantees into learned models for soft robots.
% With physical structure, we specifically refer to a formulation of the dynamics that allow us to define potential and kinetic energy terms.
% Both approaches build on existing dynamic models of physical systems, and instead of capturing the entire dynamics using a \emph{black-box} model (e.g., a neural network), we identify the free parameters of the physics-based dynamic model. This can be optionally combined with a coordinate change/encoding into a latent space to increase expressiveness and reduce the dimensionality of the learned dynamics.
In this thesis, to tackle \gls{RQ}~\ref{rq:physical_structure_learned_models}, we proposed two methods to integrate physical structure and stability guarantees into learned models for soft robots. By physical structure, we mean a formulation of the dynamics that explicitly defines potential and kinetic energy terms. Rather than modeling the entire dynamics with a black-box approach (e.g., a neural network), both methods leverage existing physics-based dynamic models by identifying their free parameters. This process can optionally include a coordinate transformation or encoding into a latent space to enhance expressiveness and reduce dimensionality.

% The first approach included in Chapter~\ref{chp:pcsregression} proposes a data-driven strategy for the derivation of the \gls{PCS}-a based kinematic model. In particular, our approach is able to make the trade-off between the dimensionality of the configuration and the accuracy of the body shape reconstruction explicit. Subsequently, a dynamic model in Euler-Lagrange form is derived, and its parameters can be identified in closed form.
% The stability of both open-loop and closed-loop systems can subsequently analyzed using Lyapunov arguments~\citep{khalil2002nonlinear}, as done previously for similar physics-based soft robotic models~\citep{della2023model}.
The first method, presented in Chapter~\ref{chp:pcsregression}, introduces a data-driven strategy for deriving a \gls{PCS}-based kinematic model. Our approach explicitly balances the trade-off between configuration dimensionality and backbone shape reconstruction accuracy. An Euler-Lagrange dynamic model is then derived, with its parameters identified in closed form. Stability for both open-loop and closed-loop systems can subsequently be analyzed using Lyapunov arguments~\citep{khalil2002nonlinear}, similar to previous work on physics-based soft robotic models~\citep{della2023model}.

% The second approach presented in Chapter~\ref{chp:con} proposes to learn latent dynamics of physical systems, and specifically of soft robots, using a network of nonlinearly coupled harmonic oscillators. This strategy provides a mechanical interpretation to the latent space as now each latent variable corresponds to the position of an oscillator. Furthermore, the kinetic and potential energy terms of this mechanical system allow us to prove its open-loop stability, specifically global asymptotic and input-to-state stability, using established Lyapunov arguments. 
The second method, described in Chapter~\ref{chp:con}, focuses on learning the latent dynamics of soft robots through a network of nonlinearly coupled harmonic oscillators. This strategy provides a clear mechanical interpretation of the latent space, as each latent variable corresponds to the position of an oscillator. Moreover, the kinetic and potential energy terms inherent to this system enable us to prove its open-loop stability—specifically global asymptotic and input-to-state stability—using established Lyapunov techniques.
% A current limitation of the approach for learning latent dynamics presented in Chapter~\ref{chp:con} is that the input encoder-decoder, which is essential for mapping the physical actuation into a forcing on the oscillators and vice-versa, is currently not guaranteed to be a diffeomorphism which could possibly lead to a mismatched mapping of the control input into an actuation on the system. Instead, future work could explore the use of bijective neural-network encoders, such as the ones based on normalizing flows~\citep{kobyzev2020normalizing}, for learning a mapping into latent space that is analytically invertible.
A current limitation of the latent dynamics approach presented in Chapter~\ref{chp:con} is that the input encoder-decoder—crucial for converting physical actuation into a forcing on the oscillators and vice versa—is not guaranteed to be a diffeomorphism. This may result in an incorrect mapping of the control input to the system’s actuation. Future work could, therefore, explore bijective neural-network encoders, such as those based on normalizing flows~\citep{kobyzev2020normalizing}, to learn an analytically invertible mapping of the inputs into latent space.

% Finally, we compare the differences of the two proposed methods. A major benefit of the strain-model-based approach presented in Chapter~\ref{chp:pcsregression} lies in the fully \emph{white-box} nature as the final derived dynamical model is fully physics-based and inspectable. Furthermore, it exhibits a strong inductive bias for the particular dynamics of soft robots, which can be both a benefit and a potential weakness: the inductive bias improves data efficiency, reduces the danger of overfitting, and leads to exceptional extrapolation capabilities. However, when the soft robot exhibits dynamics, such as hysteresis, nonlinear elasticity, deforming cross-sections, or significant interactions with the environment, they could likely not be captured by this \gls{PCS}-based model.
% This is the major advantage of the second approach, where the coordinate transformation/encoding into latent space increases the ability to learn a diverse set of different dynamics, as demonstrated by the simulation results.
% Where this is still not sufficient, the network could be, in the future, easily augmented with additional terms, such as hysteresis or contact models, which we go into more detail on in Sec.~\ref{sec:conclusion:future_work}.
% This added expressiveness, however, comes at the loss of the direct interpretation of the latent variables as parameters of a kinematic model parametrizing the soft robot's backbone shape.
Finally, we compare the two methods. A major advantage of the strain-model-based approach from Chapter~\ref{chp:pcsregression} is its fully white-box nature, resulting in a fully interpretable, physics-based dynamical model with a strong inductive bias for soft robot dynamics. This bias improves data efficiency, reduces overfitting, and enhances extrapolation, but it may struggle with dynamics such as hysteresis, nonlinear elasticity, deforming cross-sections, or significant environmental interactions. In contrast, the second approach, through its latent space encoding, is better equipped to learn a wider variety of dynamics, as demonstrated in the presented results. Where needed in the future, the network can be easily augmented with additional components (e.g., hysteresis or contact models, discussed in Sec.~\ref{sec:conclusion:future_work}).
The expressiveness added by the encoder sacrifices the direct interpretability of the latent variables as parameters of a kinematic model for the soft robot’s backbone shape.

\subsection{Contribution V - Exploiting Learned Models for Closed-Form Model-Based Control}
% \begin{itemize}
%     \item \textbf{Answer to RQ:} We can leverage learned models with a physical structure, particularly kinetic and potential energy terms, for closed-form model-based control via energy-shaping.
%     \item \textbf{Limitations:} Only setpoint regulation - no trajectory tracking, no real-world experiments, not clear how/if it would work in the underactuated setting.
% \end{itemize}

% This thesis has found that we can leverage existing PID+energy shaping controllers~\citep{kelly1995tuning, kelly1996class, kelly1998global, sciavicco2012modelling, della2023model} for closed-form control with learned models that exhibit a physical structure as introduced in Contribution~\ref{contrib:learned_models}.
% Specifically, the regulation controller that we studied in this thesis consists of an integral-saturated PID (P-satI-D)~\citep{pustina2022p} as the feedback term and a potential shaping feedforward term. The intention here is for the proportional term to reject disturbances, the saturated integral term to compensate for modeling error in a stable fashion, and the derivative term to add additional damping to the closed-loop system if necessary. 
% The feedforward term modulates the potential energy of the learned dynamics such that its local/global minimum is at the desired state.
This thesis, while exploring \gls{RQ}~\ref{rq:physical_structure_learned_models}, has shown that existing PID+energy shaping controllers~\citep{kelly1995tuning, kelly1996class, kelly1998global, sciavicco2012modelling, della2023model} can be effectively combined with learned models that incorporate a physical structure. In particular, the regulation controller we examined employs an integral-saturated PID (P-satI-D)~\citep{pustina2022p} as the feedback component alongside a potential shaping feedforward term. Here, the proportional term rejects disturbances, the saturated integral term compensates for modeling errors in a stable manner, and the derivative term adds extra damping as needed. Meanwhile, the feedforward term adjusts the potential energy of the learned dynamics so that its local or global minimum coincides with the desired state.

% We applied this control strategy to both the identified \gls{PCS}-based soft robot dynamics in Chapter~\ref{chp:pcsregression} and the learned latent dynamics based on the \gls{CON} in Chapter~\ref{chp:con} and found in simulations that the P-satI-D+potential energy controller is very effective at regulating the system towards a given setpoint. In particular, the characteristics of the (learned) potential field are interesting: In the case of the \gls{ISS} \gls{CON}, the learned potential field is guaranteed to be convex, which makes the (learned) closed-loop latent space system as established by the potential shaping feedforward term also globally asymptotically stable, even if that might not be the case when controlling the real soft robot.
% On the other hand, the learned \gls{PCS}-based soft robot dynamics can capture a non-convex potential field as a function of the learned gravitational and elastic effects. In turn, this might render the closed-loop system of the learned open-loop dynamics + potential shaping feedforward only locally asymptotically stable.
We applied this control strategy to both the identified \gls{PCS}-based soft robot dynamics in Chapter~\ref{chp:pcsregression} and the learned latent dynamics from the \gls{CON} in Chapter~\ref{chp:con}. Simulation results indicate that the P-satI-D plus potential energy controller effectively steers the system toward a given setpoint. Notably, the characteristics of the learned potential field are quite interesting: for the \gls{ISS} \gls{CON}, the potential field is guaranteed to be convex, ensuring that the closed-loop latent space system—shaped by the feedforward term—is globally asymptotically stable, even if this might not hold for the actual soft robot. In contrast, the learned \gls{PCS}-based dynamics can produce a non-convex potential field influenced by gravitational and elastic effects, which may result in only local asymptotic stability for the closed-loop system.

% There remain several open avenues for future research: (1) experimental validation of the control with learned models has not yet been done ; (2) in this thesis, we have only considered setpoint regulation with learned models, and an extension to trajectory tracking controllers would be both interesting and valuable; (3) in this thesis, we have only considered configuration where the dimensionalities of the learned model and the actuation agrees (i.e., the fully actuated setting). However, in particular in the case of learning latent dynamics, also the underactuated setting~\citep{della2023model}, with the dimensionality of the dynamics being larger than the number of actuators, would be very interesting as it would allow to increase the expressiveness of the learned dynamical model.
% A valuable tool for tackling the underactuated case might be the mapping into actuation coordinates proposed by \citet{pustina2024input}, which we already leveraged for the physics-based \gls{HSA} model in Chapter~\ref{chp:hsacontrol}.
Several avenues for future research remain open: (1) experimental validation of the control strategy with learned models is yet to be performed; (2) while this thesis focused on setpoint regulation, extending the approach to trajectory tracking controllers would be both interesting and valuable; (3) we have so far considered only configurations where the learned model’s dimensionality matches that of the actuation (i.e., the fully actuated case). In particular, for learned latent dynamics, exploring underactuated settings~\citep{della2023model}—where the dynamics have a higher dimensionality than the available actuators—could enhance the model’s expressiveness. A promising tool for addressing underactuation may be the mapping into actuation coordinates proposed by \citet{pustina2024input}, which we have already utilized for the physics-based \gls{HSA} model in Chapter~\ref{chp:hsacontrol}.

\subsection{Contribution VI - Beyond Low-Level Control: Generating Compliant Motion Behaviors for Soft Robots}
% \begin{itemize}
%     \item \textbf{Answer to RQ:} We found it essential that not only the low-level controller but also the high-level motion policy is compliant. However, an approach exists in the realm of rigid robotics, which we can re-purpose for soft robots with (minor) modifications.
%     \item \textbf{Limitations:} High-level motion planning/policies for soft robots are still underexplored. Strategies developed for rigid manipulators often implicitly aim to avoid (unnecessary) contact and only interact with the environment through the end-effector/gripper. Instead, soft robots should embrace contact as this allows them to exploit their embodied intelligence. Future motion planners/policies for soft robots should embrace contact throughout the entire body.
% \end{itemize}
% In this thesis, we proposed two approaches to instilling compliant motion behavior in soft robots and guiding low-level controllers by passing setpoints: In Chapter~\ref{chp:braincontrol}, we present a \gls{BMI} approach for allowing users to guide soft robots via motor imagery. Specifically, we measure brain signals with a wearable \gls{EEG} device and subsequently propose an effective protocol for moving the soft robot's end-effector based on binary classification of the motor imagination.
Finally, exploring \gls{RQ}~\ref{rq:compliant_motion_behaviors}, in this thesis, we introduced two methods for achieving compliant motion in soft robots that allow guidance of low-level controllers with setpoints. In Chapter~\ref{chp:braincontrol}, we developed a \glsxtrfull{BMI} that enables users to control soft robots through motor imagery. By binary classifying brain signals recorded with a wearable \gls{EEG} device, this approach effectively and safely directs the robot’s end-effector.

% The second approach included in Chapter~\ref{chp:osmp} devised a \gls{SMP}-based approach for parametrizing a periodic motion policy with a dynamical system while guaranteeing orbital stability. 
% This allows for learning complex periodic motion from demonstration without giving up stability guarantees and insight into the motion behavior.
% The velocity commanded by the \gls{SMP}, for example, defined in task or configuration space, can then serve as a reference for low-level controllers.
% We demonstrated how such an approach leads to more compliant and natural motions under perturbations by the environment compared to traditional approaches, such as a time-parametrized trajectory tracking controller.
In Chapter~\ref{chp:osmp}, we presented a \gls{SMP}-based method to parameterize periodic motion policies using a dynamical system that guarantees orbital stability. This technique allows for the learning of complex periodic motions from demonstrations without sacrificing stability or insight into the motion behavior, with the generated velocity commands—whether defined in task or configuration space—serving as references for low-level controllers. Our results show that this method produces more compliant and natural motions under environmental perturbations compared to traditional time-parametrized trajectory tracking.

% Both approaches have been experimentally verified on \gls{HSA} and helicoid~\citep{guan2023trimmed} soft robots, respectively.
% Furthermore, both approaches how we can adapt approaches originally developed for rigid robots/manipulators for soft robots under minor modifications.
% Sill, we find, while there has been some initial research~\citep{goldman2014compliant, greer2020robust, selvaggio2020obstacle, rao2024towards, rao2024tendon}, high-level motion planning/policies for soft robots are still underexplored.
% In particular, approaches developed for rigid manipulators inherently aim to avoid contact with the environment outside of planned interactions of the end-effector/gripper when its unavoidable, for example, for picking up objects.
% Instead, we argue that soft robots should embrace contact with the environment throughout its entire body to allow full exploitation of its physical intelligence.
% This will require a shift in mindset and in the future adapted motion planning approaches and motion policies that optimally select and exploit such contacts with the environment.
Both approaches were experimentally validated on \gls{HSA} and helicoid soft robots~\citep{guan2023trimmed}, respectively, illustrating that techniques originally developed for rigid manipulators can be adapted with minimal modifications for soft robots. However, despite some initial research~\citep{goldman2014compliant, greer2020robust, selvaggio2020obstacle, rao2024towards, rao2024tendon}, high-level motion planning and policy design tailored to soft robots remain underexplored. Unlike rigid manipulators that avoid unintended environmental contacts, soft robots should capitalize on contact along their entire bodies to fully leverage their embodied intelligence. This shift calls for new motion planning strategies that optimally exploit these interactions.

\subsection{Core Contribution - Safe and Precise Soft Robots via Closed-form Control with Learned Models}\label{sub:conclusion:conclusions:core_contribution}
% \begin{itemize}
%     \item \textbf{Answer to RQ:} Safety is ensured by soft material; our learned models promise to capture the soft robot dynamics more accurately, allowing for precise control. The key here, however, was to reduce the integration of physical structure into the learned models, enabling analysis of the system characteristics using nonlinear systems theory (e.g., stability analysis using Lyapunov arguments) and the derivation of closed-form control policies leveraging the kinetic and potential energy knowledge of the learned model, implemented as P-satI-D+energy-shaping.
%     \item \textbf{Limitations:} No experimental verification of learned models, preferably with various soft robot embodiments, no trajectory tracking, no underactuation, environment interactions, etc.,. No safety-aware control.
% \end{itemize}

% In the following, we will summarize how we achieved safe and precise motion behavior for soft robots via control with learned models.
In the following, we summarize our approach for achieving safe and precise motion in soft robots by leveraging control strategies based on learned models.

% As discussed in Chapter~\ref{chp:safetymetric}, the elasticity of both the structure and the soft robot surface improves the safety of continuum soft robots with respect to traditional rigid manipulators, even though further research is necessary to experimentally validate the proposed safety metric and quantify the added safety with respect to rigid robots.
As discussed in Chapter~\ref{chp:safetymetric}, the inherent elasticity of both the structure and the surface of soft robots enhances their safety compared to traditional rigid manipulators. Nevertheless, further research is needed to experimentally validate the proposed safety metric and quantify the additional safety benefits relative to rigid robots.

% Existing control approaches were not able to generate precise motion behavior without compromising safety as (1) controllers based on entirely physics-based models derived from first principles are not able to fully capture the complex dynamical behavior of soft robots (e.g., nonlinear elasticity, hysteresis, etc.) and the environment around them, and (2) fully learning the controllers, for example, using \gls{RL}, is sample inefficient, we don't have any insight into the decision making and lose all stability guarantees, therefore potentially compromising safety.
Existing control methods struggle to deliver precise motion behavior without sacrificing safety. This is because (1) controllers based solely on physics-based models derived from first principles cannot fully capture the complex dynamics of soft robots—such as nonlinear elasticity and hysteresis—as well as their interactions with the environment, and (2) fully learned controllers (e.g., those based on \gls{RL}) are sample inefficient, provide limited insight into their decision-making processes, and lack stability guarantees, potentially compromising safety.

% Instead, we propose in this thesis to control soft robots with learned models, which enables the combination of the power and expressiveness of modern \gls{ML} approaches with improved data efficiency and insight into decision-making.
% While this approach is not entirely new (for example, there is significant literature on world models and learning reduced-order models), existing approaches based on \gls{MPC}, optimal control, or \gls{MBRL} are computationally very demanding and, therefore, the control frequency they can run at is limited preventing us from fully exploiting the dynamic behavior of soft robots.
% By contrast, we establish in this thesis approach that it leverages physical structure in the learned model for control in closed form, which significantly increases the computational efficiency. Specifically, we devised in this thesis a model-based controller that combines an integral-saturated PID (P-satI-D)~\citep{pustina2022p} with a potential-shaping~\citep{della2023model} feedforward term for regulation.
% The key innovation that enabled the application of this control strategy was to infuse a physical structure into the learned models. Specifically, we need to be able to determine the kinetic and potential energy of the learned systems, which is not possible for most existing dynamic model learning strategies, such as \glspl{RNN} or \glspl{NODE}.
% In this thesis, we proposed two approaches (Chapter~\ref{chp:pcsregression} \& Chapter~\ref{chp:con}) that both rely on leveraging physics-based models for the dynamics and the learning algorithm to identify the free parameters of the physical model in addition to an optional change of coordinates (e.g., an encoding into latent space).
% For both cases, we are able to express the potential and kinetic energy of the learned system, which allows us to devise the potential shaping feedforward term in closed-form.
% Additionally, this enables us to analyze the open- and closed-loop stability of the system using standard tools from nonlinear systems theory (e.g., Lyapunov arguments)~\citep{khalil2002nonlinear}.
Instead, this thesis, in its second part, proposes controlling soft robots with learned models, a strategy that combines the expressive power of modern \gls{ML} techniques with enhanced data efficiency and improved transparency in decision-making. Although similar ideas have been explored in the literature on world models and reduced-order models, existing methods leveraging learned models for control through \gls{MPC}, optimal control, or \gls{MBRL} is computationally intensive. This computational burden restricts their control frequency and prevents full exploitation of soft robots’ dynamic capabilities. In contrast, our approach embeds a physical structure into the learned model, enabling closed-form control and, therefore, significantly boosting computational efficiency. Specifically, we developed a model-based controller that integrates an integral-saturated PID (P-satI-D)~\citep{pustina2022p} with a potential-shaping feedforward term~\citep{della2023model} for regulation. The key innovation was incorporating physical structure into the learned models, which allows us to compute the kinetic and potential energy of the system—something most existing dynamic model learning strategies, such as \glspl{RNN} or \glspl{NODE}, cannot do. In this thesis, we introduce two approaches (detailed in Chapter~\ref{chp:pcsregression} and Chapter~\ref{chp:con}) that both leverage physics-based dynamic models. Our learning algorithm identifies the free parameters of these models, along with an optional coordinate transformation (e.g., an encoding into latent space). In both cases, we can express the system’s kinetic and potential energy, allowing us to derive the potential shaping feedforward term in closed form. This framework also enables us to analyze the open- and closed-loop stability of the system using standard nonlinear systems theory tools, such as Lyapunov methods~\citep{khalil2002nonlinear}.

% There exist limitations regarding the content. For one, both the model learning and the control with learned models have not yet been experimentally verified.
% Therefore, we advise for future work an extensive experimental verification of both parts - preferably with different soft robot embodiments.
% Furthermore, we have so far only considered setpoint regulation problem settings with learned models. For future work, we suggest also investigating how such strategies could be generalized to the trajectory tracking context.
% Additionally, in Chapter~\ref{chp:con}, we export control only in the case where the dimensionality of the latent dynamics matches the dimensionality of the actuators (i.e., the fully actuated case). For future work, we advise exploring an underactuated case where increasing the latent space could allow for learning the dynamics more accurately.
% Additionally, the learned models currently do not consider interactions with the environment, which is particularly important for soft robots which we envision to fully leverage their embodied intelligence by frequently being in contact with the environment.
% Finally, while the softness of the robot improves the safety of both open and closed-loop system, we find it interesting for future to explicitly take into account safety when controlling the robot. One promising avenue, for example, would be to use the safety metric proposed in Chapter~\ref{chp:safetymetric} as a constraint within \gls{MPC} or \gls{CBF} frameworks. 
There are, however, some limitations to this work. 
While we have been able to extensively experimentally validate Part~\ref{part:physicsmodels} of this thesis, the model learning nor the control strategies based on learned models have been experimentally validated yet. We therefore recommend extensive experimental testing of both components in future work, preferably using various soft robot embodiments. Additionally, our current study has focused solely on setpoint regulation scenarios; future research should explore extending these strategies to trajectory tracking. In Chapter~\ref{chp:con}, we address control only for the case where the dimensionality of the latent dynamics matches that of the actuators (i.e., the fully actuated case). Future investigations might consider underactuated scenarios, where expanding the latent space could lead to more accurate dynamic modeling. Moreover, the current learned models do not account for environmental interactions—a crucial factor for soft robots intended to exploit their embodied intelligence through frequent contact with their surroundings. Finally, while the inherent softness of the robot enhances the safety of both open- and closed-loop systems, it would be worthwhile in future work to explicitly incorporate safety into the control strategy. One promising direction could be to use the safety metric proposed in Chapter~\ref{chp:safetymetric} as a constraint within \gls{MPC} or \gls{CBF}~\cite{ames2016control} frameworks.
