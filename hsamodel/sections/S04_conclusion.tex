\section{Conclusion}\label{sec:hsamodel:conclusion}
%
This work provided for the first time solutions for modeling the kinematics and the dynamics of electrically-actuated continuum soft robots based on Handed Shearing Auxetics.
%
We have shown that coupling the twist strains to rest lengths can allow simulators based on the discrete Cosserat rod theory. %to capture the behaviour of both single \glspl{HSA} and entire \glspl{HSA} robots.
While the proposed linear approximation of the auxetic trajectory works well for closed \glspl{HSA} within a bounded motion range, future work shall derive a more general model also applicable for semi-closed and open \glspl{HSA}~\cite{good2022expanding}.
% Namely, we were able to invoke all motion primitives: elongation, bending, and twisting.
Furthermore, we have proposed the \gls{SPCS} kinematic model that can express the shape of \glspl{HSA} with $11$ DoF.
Fitting this kinematic model to the experimental results showed a very good match for representing the shape of the \glspl{HSA}. In particular for large actuation magnitudes within the twisting motion primitive, the \glspl{HSA} leave the auxetic trajectory and seem to experience buckling behaviour. For this case, the \gls{SPCS} model is not accurate anymore.
% Our simulations and the experimental campaign showed that the twist \& stretch strains can be assumed to be constant along the entire \gls{HSA} while at least $2$ \gls{CS} segments are necessary to capture the bend \& shear strains during the twisting motion primitive. Furthermore, the magnitude of shear strains is significant during bending and twisting and cannot be neglected. %, as it was done in previous work~\cite{garg2022kinematic}.
Future work will focus on utilizing the kinematic model proposed in this work for model-based control of \gls{HSA} robots.