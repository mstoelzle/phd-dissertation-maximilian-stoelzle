\section{Conclusion}\label{sec:hsamodel:conclusion}
%
This chapter provided solutions for the first time for modeling the kinematics and the dynamics of electrically-actuated continuum soft robots based on Handed Shearing Auxetics.
%
We have shown that coupling the twist strains to rest lengths can allow simulators based on the discrete Cosserat rod theory. %to capture the behaviour of both single \glspl{HSA} and entire \glspl{HSA} robots.
While the proposed linear approximation of the auxetic trajectory works well for closed \glspl{HSA} within a bounded motion range, future work shall derive a more general model also applicable for semi-closed and open \glspl{HSA}~\cite{good2022expanding}.
% Namely, we were able to invoke all motion primitives: elongation, bending, and twisting.
Furthermore, we have proposed the \gls{SPCS} kinematic model that can express the shape of \glspl{HSA} with $11$ DoF.
Fitting this kinematic model to the experimental results showed a very good match for representing the shape of the \glspl{HSA}. In particular, for large actuation magnitudes within the twisting motion primitive, the \glspl{HSA} leave the auxetic trajectory and seem to experience buckling behavior. For this case, the \gls{SPCS} model is not accurate anymore.
% Our simulations and the experimental campaign showed that the twist \& stretch strains can be assumed to be constant along the entire \gls{HSA} while at least $2$ \gls{CS} segments are necessary to capture the bend \& shear strains during the twisting motion primitive. Furthermore, the magnitude of shear strains is significant during bending and twisting and cannot be neglected. %, as it was done in previous work~\cite{garg2022kinematic}.
Finally, we presented a control-oriented dynamic model for planar \gls{HSA} robots and verified it experimentally.
The conducted experiments gave us deep insights into the particular characteristics of \glspl{HSA} and how well our model is able to capture them. We see excellent agreement for predicting the dynamical behavior of \gls{HSA} robots made of FPU material.
We observed the time lag of the model to be larger for EPU-based robots, as seen in Fig.~\ref{fig:hsamodel:planar_hsa_robot_model:model_verification:epu:chiee}. This probably is the result of the hysteresis characteristics of \glspl{HSA}~\cite{good2022expanding}.
For EPU-based \glspl{HSA} robots, we observe that the model does not fully capture the shear dynamics.

\section{Afterword}
This chapter presented approaches for representing the kinematics and dynamics of \gls{HSA} rods in the general case. Additionally, we introduced the first low-dimensional kinematic parameterizations and control-oriented dynamical models for planar \gls{HSA} robots.
Next, Chapter~\ref{chp:hsacontrol} will focus on leveraging the models proposed in this chapter for model-based control of planar \gls{HSA} robots.
Namely, this includes the development of stable and effective configuration-space PID+feedforward and task-space impedance controllers.
Subsequently, Chapter~\ref{chp:braincontrol} will develop a \gls{BCI} strategy for guiding soft robots, and specifically \gls{HSA} robots, using motor imagery in a safe and compliant fashion.