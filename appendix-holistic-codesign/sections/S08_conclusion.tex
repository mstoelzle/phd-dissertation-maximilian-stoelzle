\section{Conclusion and Open Challenges}

\subsection{Conclusion}
% In this chapter, we presented a way how to achieve soft robots that are both safe and high-performing via co-design.
In this chapter, we introduced an approach that leverages holistic co-design to develop soft robots that are both safe and high-performing.
First, we reviewed existing soft robot design approaches, covering both traditional design processes and early research on co-design. This review revealed shortcomings in current co-design methods, such as computational inefficiencies and a tendency to focus solely on aspects that can be fully simulated while neglecting critical downstream factors like prototyping, real-world evaluation, manufacturability, safety, and regulatory issues. We subsequently proposed a framework for holistic co-design that incorporates several essential modifications and additions. This framework emphasizes the optimization of broader design values—such as manufacturability, safety, and cost considerations in both manufacturing and operation—rather than relying solely on a single, simulation-based performance metric.
Furthermore, we identified several avenues to boost the computational efficiency of co-design, including the use of reduced-order design spaces, the co-learning of reduced-order models, and auxiliary metrics that provide early, cost-effective feedback on controllability, observability, and open-loop compliance, as well as the model-based derivation of controllers. We also presented a probabilistic perspective on co-design metrics, explaining how high-fidelity simulations and a strategic selection of prototypes can build confidence and reduce uncertainty in evaluation metrics. Looking ahead, exploiting the trade-off between design refinement and realization may lead to solutions that are not only optimal in simulation but also effective in the real world. Finally, we stressed the importance of synergistic cross-disciplinary collaboration, noting that a holistic co-design approach can help preserve design knowledge and enhance reproducibility within soft robotics.

This holistic co-design approach would also be instrumental in formalizing the trade-off between safety and performance, allowing us to pinpoint effective soft robotic designs—integrating both morphology and control—that excel at the task while satisfying a broad range of values and requirements without compromising safety.

\subsection{Next Steps and Open Challenges}
As the next steps, we suggest taking initial steps to enable its realization. This includes but is not limited to: (1) deriving and verifying metrics such as manufacturability, fabrication and operation costs, controllability, and observability in a model-based fashion so they can be computationally evaluated and integrated into co-design methodologies—with an emphasis on making these metrics openly accessible and easily combinable; (2) investigating effective methods to incorporate design priors, such as existing soft robotic designs~\citep{navez2024design} or biological inspirations from invertebrates~\citep{laschi2012soft, krieg2015design, chen2020design, laschi2024bioinspiration}, into the design sampling process. Generative and foundation models—such as \glspl{VLM} trained on internet-scale data~\citep{grattafiori2024llama, hurst2024gpt}—offer promising avenues for accelerating the development of novel soft robotic architectures. Initial studies should explore how these models can serve as design samplers by proposing bioinspired or unconventional morphologies that might not arise from traditional optimization routines. For instance, a \gls{VLM} could be prompted to suggest soft robot designs for specific tasks and constraints~\citep{stella2023can, ghasemi2025vision}, potentially conditioned on renderings or descriptions of previously evaluated designs - i.e., through in-context learning~\citep{brown2020language}, which could then be mapped into a reduced-order design space $x$ via a learned neural network. Connected to this, (3) \glspl{VLM} could also be effective as design critics~\citep{ghasemi2025vision}—assessing factors such as perceived human safety, environmental impact, feasibility, ergonomics, and functional versatility, aspects that are challenging to model from first principles. 
(4) As discussed in Sec.~\ref{sec:apx:holisticcodesign:probabilistic_co_design_metrics}, it is crucial to investigate how to effectively exploit the trade-off between design refinement and realization. The ideal approach should account for the resources (e.g., effort, cost) required for each refinement and realization cycle, allocating them optimally to enhance design improvement efficiency.
(5) As emphasized in Sec.~\ref{sec:apx:holisticcodesign:codesigning_physical_intelligence}, a major challenge in the co-design of soft robots is optimizing effective control policies in a computationally tractable manner. One potential alternative to the model-based control approach presented here is to draw inspiration from emerging X-Embodiment policies~\citep{o2024open} developed for rigid manipulators and to develop controllers that work across a variety of soft robotic designs, thereby avoiding the need to train a specialized controller for each design from scratch. Moreover, (6) characterizing the Pareto front between safety and performance will enable a structured trade-off analysis, particularly in common tasks such as pick-and-place, where payload capacity, speed, and compliance must be optimized simultaneously.