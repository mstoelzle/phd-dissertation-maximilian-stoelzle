\section{Exploiting Safety vs. Performance Trade-off via Multi-Objective Optimization}\label{sec:apx:holisticcodesign:safety_vs_performance_tradeoff}

Safety and performance are fundamental components in multi-objective co-design, often in conflict. Specifically, safety can be treated as either a maximization objective—continuously optimizing for compliance—or a design/control constraint, ensuring a minimum threshold is met.
% Furthermore, compared to the rigid robotic domain which mostly ensures safety via the computational approaches and the existing literature in soft robots which guarantees compliance via material softness, we argue in this perspective that the safety of soft robots needs to be looked at by jointly considering the characteristics of \emph{body} and \emph{brain}.
Furthermore, unlike rigid robotics, which predominantly ensures safety through computational methods, or conventional soft robotics, which relies primarily on material compliance, we argue in this chapter that the safety of soft robots should be addressed by jointly considering both the characteristics of the robot’s \emph{body} and its \emph{brain}.

In our proposed holistic approach, we treat the trade-off between safety and performance as a multi-objective optimization problem during the design of the soft robot’s \emph{body} (i.e., morphology) and \emph{brain} (i.e., controller). Specifically, we simultaneously maximize both objectives while enforcing a constraint to ensure that the worst-case safety of the closed-loop system always meets or exceeds the specified task requirements. During deployment (i.e., the control phase), adaptive strategies dynamically balance safety and performance according to real-time conditions, again ensuring safety always remains above the required task threshold.
This layered approach broadens the design space by reducing constraints traditionally imposed on structural components. It also increases resilience, as the robot remains safe even in scenarios where control performance is compromised.