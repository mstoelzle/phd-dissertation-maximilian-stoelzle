\section{Reduced-Order Design Parametrizations}\label{sec:apx:holisticcodesign:reduced_order_design_parametrizations}
Since direct optimization of soft robots requires parameterization, we represent the design via $n_\mathrm{x}$ variables $x \in \mathcal{X}$, where $\mathcal{X} = \mathbb{R}^{n_\mathrm{x}}$ defines an $n_\mathrm{x}$-dimensional continuous design space.\footnote{While a continuous space is assumed here, discrete design choices can be naturally embedded into this framework.} In soft robot co-design, these parameters typically encode essential characteristics, including the spatial geometry of the robot’s body, actuator~\citep{wang2024diffusebot} and sensor~\citep{spielberg2021co, junge2022leveraging} placement, material selection, and other structural attributes. Traditionally, two methods have been employed to parameterize soft robot designs: (1) size optimization~\citep{chen2020design}, where (morphological) design optimization parameters such as the number of segments, radii, segment lengths, and materials are chosen explicitly~\citep{guan2023trimmed, calisti2011octopus, pagliarani2024variable, polygerinos2015modeling, navez2024design, junge2022leveraging}, or (2) discretization-based shape/topology\footnote{Shape optimization refers to adjusting the boundaries or surfaces of an existing design while maintaining the overall connectivity (or topology) of the structure. On the other hand, topology optimization rethinks the structure from the ground up while potentially adjusting the connectivity (e.g., creating or removing elements) of the structure~\citep{chen2020design}.} optimization~\citep{chen2020design}, which partition the continuous design into two- or three-dimensional voxels or particles~\citep{caasenbrood2020computational, pinskier2024diversity, bhatia2021evolution, medvet2022impact, wang2022curriculum, nadizar2022schedule}, each categorized as empty (removed), passive (unactuated), active (actuated), or sensorized.

However, these traditional approaches have distinct drawbacks. Manually selected parameters are often suboptimal, can introduce redundancies, or might include parameters that have minimal or no impact on design objectives, unnecessarily complicating the co-design process. Discretization-based parameterizations, on the other hand, create high-dimensional optimization spaces that are computationally expensive to explore effectively. Furthermore, these discretizations typically oversimplify practical constraints, such as interactions between actuators and structural components, and frequently yield designs difficult to realize in practice~\citep{legrand2023reconfigurable}.


To overcome these challenges, recent studies~\citep{song2024morphvae, wang2024diffusebot, navez2024contributions} have proposed optimizing within a reduced-order design space and then using a (potentially learned) generative decoder to reconstruct the complete design—such as generating a 3D mesh of the soft robot that includes sensor and actuator placements. In this context, we incorporate a design sampler in the reduced-order space along with a design decoder into our comprehensive co-design framework. Specifically, we define a probabilistic design policy $\vartheta(x): \mathcal{X} \to [0,1]$ that maps the design parameters $x$ to a probability $\mathrm{Pr}(x)$. During co-design, the generator samples parameters from this distribution, $x \sim \vartheta(\cdot)$, which are then expanded into a full design description $d \in \mathcal{D}$ by the decoder $d = \psi(x)$. The decoder may be either model-based - e.g., utilizing parametric \gls{CAD} - or implemented as a learned generative model, such as \glspl{VAE}~\citep{song2024morphvae, navez2024contributions}, \glspl{GAN}~\citep{hu2022modular}, or Diffusion models~\citep{wang2024diffusebot}. 
% Design priors, such as existing effective soft robot designs or biological priors that have been encoded into the reduced-order design space can be used to shape the initial guess (i.e., the prior) of the distribution $\vartheta(x)$.
% During the co-design process, the optimizer iterative shapes the posterior belief of $\vartheta(x)$ via an update step.
Design priors—for example, proven soft robot designs~\citep{navez2024contributions}, or biological inspiration~\citep{mazzolai2020vision, chen2020design, laschi2024bioinspiration}—can inform the initial guess (i.e., the prior) for the distribution $\vartheta(x)$. Throughout the co-design process, the optimizer iteratively refines the posterior belief of $\vartheta(x)$ via update steps~\citep{song2024morphvae, sutton1998reinforcement}.

This flexible framework supports both model-based and learning-based approaches while enabling optimization in a reduced-order space, thereby making the co-design process more computationally tractable without sacrificing the detailed design information required for accurate evaluation and subsequent fabrication.

Sensitivity analysis can play a crucial role within the aforementioned “design generator” by reducing the dimensionality of the design parameters we sample and improving the conditioning of the design decoder~\citep{chen2020design, guan2023trimmed, navez2024contributions}. When choosing geometrical parameters, manufacturing constraints, and actuation variables, sensitivity analysis is vital for pinpointing and eliminating parameters that exert minimal influence on performance. This involves systematically varying parameters—such as wall thickness, segment lengths, material stiffness, and actuator placement—to evaluate their impact on key performance metrics like gripping force, range of motion, and energy consumption. Formally, for a given metric $m_j(x): \mathbb{R}^{n_\mathrm{x}} \to \mathbb{R}^{n_{\mathrm{c}_j}}$ that maps design parameters to an optimization objective $c_i$, the local sensitivity of the $i$th design parameter $x_i$ at a design $x$ is given by $s_i = \left| \frac{\partial m_j(x)}{\partial x_i} \right| \in \mathbb{R}^{n_{\mathrm{c}_j}}$, which measures the magnitude of its influence. For instance, in a soft robotic gripper designed for fruit harvesting, sensitivity analysis may show that finger curvature and material elasticity have a significant effect on grasping delicate produce (yielding a high $s_i$), whereas small variations in actuator placement may have a negligible impact (a low $s_i$). Consequently, the optimization process can focus computational resources on refining the most critical parameters—either by (a) employing a curriculum approach that defers optimization of less sensitive parameters until later stages or (b) completely fixing these parameters to reduce the search space.

Encoding deformations in a low-dimensional subspace improves computational efficiency for both design optimization and control. Rather than tackling high-dimensional, resource-intensive simulations, designers can employ reduced-order models that encapsulate the primary deformation modes of soft robotic morphology.