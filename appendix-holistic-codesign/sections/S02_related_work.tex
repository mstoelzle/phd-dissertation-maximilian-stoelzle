\section{The Past and Present Soft Robot Design Process}\label{sec:apx:holisticcodesign:related_work}

In this section, we review established design processes in the field of soft robotics. The discussion is organized into two parts: first, we examine the fundamental design cycle typically employed in soft robotics, covering the progression from initial task specifications to experimental prototyping and final design approval. Second, we explore recent work on the co-design of soft robots that showcases some initial ideas on how we could jointly design the body and brain.
% highlighting approaches aimed at integrating various design aspects more effectively in the development of soft robotic systems.

\subsubsection{Traditional Design Process of Soft Robots}
The basic design cycle typically described in the literature involves several key steps~\citep{roozenburg1995product, van2020delft}: (1) The cycle begins with defining criteria, where designers explicitly specify the values, requirements, and functions that the design must satisfy; (2) Next is the provisional design stage, in which solution ideas, concepts, and implementations are synthesized; (3) Following this, simulations can be used to anticipate the potential properties and performance of the design. These predictions are validated in step (4) by manufacturing prototypes at various \glspl{TRL}~\citep{NASA_TRL}; (5) Finally, if the prototypes meet the established criteria, the design is approved.

This traditional design process is also widely applied within soft robotics. As illustrated in Fig.~\ref{fig:apx:holisticcodesign:current_design_vs_holistic_codesign}~(Left), it begins with task specifications and proceeds to a preliminary conceptual design, followed by detailed mechatronic development. The open-loop behavior of the resulting design is verified through simulation, fabrication, and testing on breadboard prototypes. Typically, control design and reduced-order modeling occur only after validating a viable soft robotic prototype, enabling verification of closed-loop behavior within laboratory settings. Furthermore, steps (2)-(4) often involve iterative cycles of diverging and converging phases~\citep{feldhusen2013pahl}. During the \emph{diverging} phase, new design ideas, concepts, and implementations are generated. Conversely, the \emph{converging} phase evaluates these concepts through simulations or prototype tests to identify, for example, via selection matrices~\citep{ulrich2016product}, promising designs worthy of further development.

However, the traditional design cycle has several notable deficiencies:
(i) Although multiple iterations can theoretically "close the loop" of the design cycle, practical constraints—such as costs and complexity—often render iterative processes prohibitively expensive. Additionally, no formal or automated method typically exists to systematically incorporate insights from evaluated performance metrics back into improving the original design.
(ii) Similarly, divergence and convergence cycles frequently lack effective feedback mechanisms, preventing valuable knowledge acquired during detailed design or high-fidelity prototyping from informing the generation of new design solutions.
Indeed, for example, Suh's first axiom on design theory for systems argues that cycles should be avoided and functional requirements should be orthogonal to each other~\citep{suh1998axiomatic}.
% However, many examples in practice in product development have demonstrated how an iterative approach can significantly improve products over time.
However, numerous practical examples in product development have shown that iterative approaches can substantially enhance product quality and performance and decrease cost over time. Therefore, a design process for soft robots should embed such feedback mechanisms from the beginning.
(iii) Constraints and metrics critical in later stages, such as manufacturability and compliance with regulatory standards, such as \gls{ISO} norms, are often inadequately considered during initial stages.
(iv) The sequential workflow between mechatronic design, perception, modeling, and control teams creates informational silos. For instance, modeling engineers typically transfer completed models to control engineers without sufficient feedback channels, thus restricting iterative enhancements.
(v) A significant drawback of conventional design processes is the potential loss of "design history." Critical insights, rationale for decisions, and trade-offs are frequently undocumented, particularly when team members transition roles or leave the project. Consequently, revisiting earlier stages or building on previous knowledge becomes increasingly challenging. For example, a research team seeking to modify an earlier actuator design after identifying field performance issues may find the original design rationale inaccessible.

\subsubsection{Co-Design of Soft Robots}
Given the complexity of soft robots and the limitations of traditional design processes—especially the absence of structured and effective feedback cycles and the isolated design of components—the community has recently begun exploring how co-design algorithms could support soft robot development~\citep{spielberg2019learning, cianchetti2021embodied, bhatia2021evolution, van2022co, wang2022curriculum, wang2023preco, wang2024diffusebot, junge2022leveraging}.
% The term "\textit{co}-design" can emphasize several difference compared to traditional design approaches, including a \textit{co}mpositional and hierarchical nature (i.e., designing all components of a system together), a \textit{co}llaborative approach involving all teams and stakeholders, an a \textit{co}ntinuous improvement of the design~\citep{zardini2023co}.
The term “\textit{co}-design” highlights several distinctions from traditional design approaches, including a \textit{co}mpositional and hierarchical nature - i.e., designing all system components, such as body and brain, together~\citep{junge2022leveraging}; a \textit{co}llaborative approach involving all teams and stakeholders, and a \textit{co}ntinuous process of design improvement~\citep{zardini2023co}.
Co-design strategies have proven highly effective at solving complex, possibly multi-objective, optimization problems with a well-defined design space, cost function, and equality/inequality constraints. 
Notable examples stem from the fields of chemistry~\citep{norskov2009towards,vaissier2018computational}, construction engineering~\citep{knippers2021integrative}, mobility systems~\citep{zardini2020co, zardini2022co}, autonomous vehicles~\citep{zardini2023co,zardini2021co}, articulated robotics~\citep{ha2018computational,zhao2020robogrammar}.
% Progress has been made in developing co-design algorithms for soft robots~\citep{van2018spatial, legrand2023reconfigurable, wang2024diffusebot}, but current approaches lack computational efficiency (specifically in modeling and control), offer limited optimality guarantees, and often cover only selected components of the design (e.g., exclude sensing, actuation or control).
% Furthermore, the generated designs can rarely be fabricated.\\

Initial attempts have applied co-design methods to soft robots by simultaneously optimizing the body/morphology (i.e., structural shape, actuation, and sensing) and the brain (e.g., controller)~\citep{spielberg2019learning, cianchetti2021embodied, bhatia2021evolution, van2022co, wang2022curriculum, wang2023preco, navez2024contributions, wang2024diffusebot, junge2022leveraging}. Drawing on extensive research in mechanical and morphological design—particularly geometric design—these approaches can be classified~\citep{chen2020design} into three categories: (a) size optimization, which focuses on regular shapes defined by (hand-picked) geometric parameters (e.g., radii, segment lengths, pneumatic chamber dimensions)~\citep{dammer2018design, wang2018programmable, guan2023trimmed, calisti2011octopus, pagliarani2024variable, polygerinos2015modeling, navez2024design, junge2022leveraging}; (b) shape optimization, which incrementally modifies the parametric surfaces of an initial design while preserving its connectivity or topology~\citep{siefert2019bio}; and (c) full topology optimization, which rethinks the structure entirely by creating, repurposing, or removing elements from the topology~\citep{sigmund2013topology, jewett2019topology, zhang2018topology, caasenbrood2020computational, spielberg2019learning, wang2022curriculum, legrand2023reconfigurable, wang2023softzoo, wang2023preco, wang2024diffusebot, pinskier2024diversity}.

However, the existing co-design approaches exhibit several shortcomings:
(i) The current topology optimization approaches often oversimplify the design process by discretizing the morphology spatially into passive, actuated, or sensorized voxels or particles~\citep{spielberg2019learning, medvet2021biodiversity, medvet2022impact, wang2022curriculum, legrand2023reconfigurable, wang2023softzoo, wang2023preco, wang2024diffusebot}, resulting in designs that rarely translate effectively to practical applications~\citep{legrand2023reconfigurable, wang2024diffusebot}. Additionally, (ii) most existing methods train learning-based controllers (e.g., using \gls{RL}) from scratch for each iteration, creating a significant computational bottleneck in the overall co-design cycle~\citep{bhatia2021evolution, wang2022curriculum, wang2023preco}. Recent progress in differentiable physics-based simulation~\citep{coevoet2017software, hu2019chainqueen, fang2020kinematics} suggests that integrating real-time gradient-based optimization of neural-network parametrized controllers could significantly reduce the computational overhead associated with iterative design-control loops, improving sample efficiency and convergence speed~\citep{spielberg2019learning, bacher2021design, wang2024diffusebot}. However, these methods assume smooth, end-to-end differentiability—which can fail under contact or discontinuities—and can be prone to local minima. 
Relatedly, (iii) these approaches typically require completing an entire cycle—from design generation, control policy training, and high-fidelity simulation through task-specific performance evaluation—before updating the design. This sequence can be highly computationally demanding and results in unnecessary resource use for designs that could have been earlier identified as suboptimal through intermediate metrics. 
% While “learning-in-the-loop” co-design~\citep{spielberg2019learning} reduces redundant evaluations by jointly optimizing morphology and control, a key drawback is its heavy reliance on simulation accuracy. 
Finally, (iv) most studies predominantly focus on computationally implementable aspects of the design process (i.e., \emph{computational co-design}), neglecting critical downstream considerations~\citep{junge2022leveraging} such as breadboard testing, validation with prototypes, or regulatory compliance. Although topology optimization and multi-material design techniques~\citep{chen2020design} are gaining traction, their integration with fabrication constraints remains an open challenge. Considering manufacturability from the early stages of co-design—particularly in the context of additive manufacturing- could enhance the feasibility of computationally optimized soft robots~\citep{navez2024contributions}.
However, we conclude that the high computational demands of co-design caused by computationally expensive simulations and inefficient optimization routines currently restrict its practical usefulness, limiting the scope of the design space that can be explored within available computing budgets~\citep{chen2020design}.