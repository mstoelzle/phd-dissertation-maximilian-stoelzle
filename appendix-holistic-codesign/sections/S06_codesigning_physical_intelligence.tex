\section{Co-Design of Physical Intelligence}\label{sec:apx:holisticcodesign:codesigning_physical_intelligence}
For rigid robotic manipulators, the joint and link topology is well-defined, which simplifies deriving a kinematic model that describes the robot’s state using a limited number of configuration variables (e.g., joint angles)~\citep{siciliano2010robotics, zhao2020robogrammar}. Such models are crucial for downstream tasks like model-based control, model-based RL, and motion planning. In contrast, the continuous deformability of continuum soft robots makes it challenging to establish such reduced-order models. Traditionally, the soft robotics community has relied on expert intuition and trial-and-error to determine finite-dimensional descriptions of the robot’s backbone shape, such as \gls{PCC}~\citep{webster2010design}, \gls{PCS}~\citep{renda2018discrete}, \gls{GVS}~\citep{renda2020geometric}. However, the appropriateness of these choices depends heavily on the robot’s topology, actuation, and dynamic modes. Consequently, it is vital to co-design the reduced-order model alongside the soft robot hardware. In this context, our focus is on applying algorithms capable of automatically identifying a suitable, potentially control-oriented model without requiring user intervention. One promising direction involves modern machine learning methods to derive reduced-order models in a data-driven fashion~\citep{thuruthel2017learning, bern2020soft, alora2023data, chen2024data, menager2023direct}, with recent applications emerging in co-design/design optimization~\citep{navez2024contributions}. Alternatively, to address the sample inefficiency and limited extrapolation of fully learned models, the community has recently proposed methods to automatically adapt existing finite-dimensional strain approximations to a given soft robot design using data-driven approaches~\citep{alkayas2025soft, valadas2025learning, navez2025modeling}.

As noted in Sec.~\ref{sec:apx:holisticcodesign:related_work}, most current co-design approaches indirectly assess the system’s controllability and observability by running simulations or real-world experiments with a learned controller. However, this process provides a weak signal because it is difficult to determine whether poor control performance arises from an inadequate controller or from inherent challenges in controlling or observing the system. We contend that integrating direct metrics for controllability and observability into the co-design process could accelerate convergence and reduce the reliance on computationally expensive simulations and experiments. For example, the controllability and observability of a nonlinear soft robot dynamic model can be evaluated using established techniques~\citep{griffith1971observability, zheng2019controllability}. Alternatively, one may linearize the system around an equilibrium (e.g., the straight configuration) and then use the resulting state-space description to assess the well-known linear properties. Moreover, it is important to consider the stability of the closed-loop system dynamics. When guiding the soft robot towards a desired configuration $q^\mathrm{d}$, previous work has shown that common control strategies, such as PD+Feedforward, are locally stable within a region where the potential field is convex (e.g., when $\frac{\partial^2 \mathcal{U}}{\partial q^2} + K_\mathrm{p} \succ 0$). Given that we typically aim to keep proportional feedback gains as low as possible to enhance phase margins and reduce control effort, it follows that achieving closed-loop stability over a larger region—or even global asymptotic stability—requires optimizing the soft robot design so that $\frac{\partial^2 \mathcal{U}}{\partial q^2} \succ 0$ throughout the entire desired workspace.
