\chapter*{Acknowledgments}
\addcontentsline{toc}{chapter}{Acknowledgments}
\setheader{Acknowledgments}

% First of all and most importantly, thank you to my advisor and co-promotor Dr. Cosimo Della Santina for taking a chance on me and giving me this opportunity to pursue a Ph.D., for at start giving me a very well defined and well guided research project that allowed me to get acquinted with soft robotics and advanced nonlinear control theory in a rapid, step-by-step fashion with a significantly flattened learning curve, for always spot-on technical guidance whenever I had questions or encountered technical difficulties, the always candid but highly constructive and feedback on research and paper drafts - serving as the ''\emph{feared}`` \emph{\nth{2}}-reviewer already long before paper submission, for all the advice in structuring slides which allowed me to significantly elevate my presentation skills, for already early on trusting me in contributing to student mentoring, for giving me the connections and platforms to pursue impactful collaborations on a European and international level, for gradually allowing me to work more independently and propose research projects and manage projects myself, for gradually giving me more ''senior``reponsibilities such as leading the development of the ICS PA and EU EMERGE deliverables, and most importantly for consistently and effectively supporting me through the entire Ph.D. journey.
First and foremost, I want to thank my advisor and co-promotor, Dr. Cosimo Della Santina, for believing in me and giving me the chance to pursue a Ph.D. Initially, you entrusted me with a well-scaffolded project that let me dive into soft robotics and advanced nonlinear control step by step, smoothing the learning curve. Over time, you progressively empowered me to design and manage research projects on my own. Your technical guidance was always spot-on whenever I hit a roadblock, and your candid yet constructive feedback on my research and draft papers—serving as the ''feared second reviewer`` long before paper submission—pushed the work to a higher standard. Your coaching on presentation structure and slide style elevated my presentation skills, and your early confidence in my ability to mentor students broadened my professional growth. Thank you for connecting me with leading collaborators across Europe and beyond, and for gradually entrusting me with \emph{senior} responsibilities like leading the ICS practical assignment and EU EMERGE deliverables. Above all, thank you for your steadfast and effective support throughout this entire Ph.D. journey.

I am deeply grateful to my advisor, Professor Robert Babuška, for always being available and for responding so quickly to every request. His critical assessments of my progress, streamlined coordination, fast feedback on thesis drafts, and our engaging discussions about the propositions were invaluable.

Professors Daniela Rus and Gioele Zardini, the many impactful collaborations with you fundamentally shaped this Ph.D. research, and my visiting period in your groups significantly contributed to my personal development—thank you.

I appreciate the independent committee members—Junior Professor Gianluca Rizzello, Dr. Moritz Bächer, Professors Tom Ooomen, Guido de Croon, and Gioele Zardini—for graciously accepting the invitation, accommodating our scheduling constraints, and dedicating significant time to reviewing the thesis and attending the defense.

I am indebted to all scientific collaborators whose contributions enliven the chapters of this thesis. In particular, Ryan Truby and Lillian Chin, thank you for inviting me to work on the HSA robot platform—a project that was both enjoyable and pivotal to this thesis.
Thank you, Sonal Baberwal, for a highly interdisciplinary and almost daring collaboration on brain-controlled soft robots—bold, harmonious, and ultimately very successful. Working on this project with you was a joy, and I’m grateful for the friendship we’ve built along the way.

Thank you to all my students—whether you joined for a research internship, a Bachelor End Project (BEP), or a master’s thesis—for trusting my project ideas and guidance. Your motivation, dedication, and creativity made our work both productive and inspiring, and many of your projects became publications that form the backbone of this dissertation.
I had the privilege of co-mentoring several M.S. students—Emanuele Rosi, Ricardo Valadas, Gabriele Di Marzo, Riccardo Sepe, Gioele Buriani, and Iván López Brocéno—as well as the 2023 BEP team of Bendert de Roij van Zuidewijn, Hannah Gielen, Quirijn Bos, and Floris Cuperus.
A special shout-out goes to the 2021 BEP team—Thomas Baaij, Marn Klein Holkenborg, Daan van der Tuin, and Jonathan Naaktgeboren. From my first month as a Ph.D. student, I had the privilege to be involved in your BEP project, which subsequently inspired an ambitious research project that you embraced, even though it was (initially) well beyond the techniques you knew at the time. Your perseverance over the 1.5 years turned that idea into a joint publication, and our collaboration was filled with unforgettable moments—not least your ''Lord of the Rings``-inspired animation of our journey together.

My master’s mentors at ETH Zürich—Julian Zilly (IDSC), Takahiro Miki (RSL), Martin Azkarate and Levin Gerdes (PRL @ ESA), Florian Achermann, Nicholas Lawrance, and Jen Jen Chung (ASL)—ignited my passion for robotics and helped me build the foundation that made this Ph.D. possible.

Beginning in the CoR department during COVID, when on-site work was limited to one day a week, was unusual. I am especially grateful to department veterans Bruno Brito and Padmaja Kulkarni, and at the time senior master student Francesco Stella, for their warm welcome.
Shortly after I began my Ph.D., postdocs Pablo Borja and Sagar Joshi joined the group. Thank you both for fostering a friendly lab culture and for the many enjoyable dinners and drinks we shared.
To the PhI-Lab teammates who joined later—Pietro Pustina, Tomás Coleman, Anton Bredenbeck, Jiatao Ding, Chuhan Zhang, Jingyue Liu, Ebrahim Shahabi, Giovanni Franzeso, Rodrigo Pérez-Dattari, Zhaoting Li, Mariano Ramirez Montero, Daniel Feliu Talegon, Kirsten Lussenburg, and Semanur Küçük—thank you for the great times and for our successful collaborations on research, the ICS course, and student mentoring.

I also appreciate our visiting researchers—Kyle Walker, Xiang-Yu Shao, Maja Trumić, Francesco Piqué, Michele Pierallini (a fantastic Pisa tour guide), Sonal Baberwal (an exceptionally kind collaborator), Domenico Donà, Michele Martini, and Lorenzo Paiola—for bringing fresh perspectives and for the memorable moments we shared.

Special thanks to Pietro Pustina for always being ready to help whenever I (yet again) had a question about control theory or actuation coordinates.

To my TUD-EMERGE teammates Jingyue Liu, Mariano Ramirez Montero, and Ebrahim Shahabi: thank you for your effective teamwork on our many time-consuming reports and deliverables. I also thank the entire EMERGE consortium for our productive collaboration on cutting-edge interdisciplinary topics—especially our Pisa partners Andrea Ceni, Andrea Cossu, Claudio Gallichio, and Davide Bacciu.

I’m grateful to all past and present CoR colleagues for making the department such a supportive place, cultivating a spirit of collaboration, and always being willing to lend expertise or equipment. Bas van der Heijden, Giovanni Franzeso, Rodrigo Pérez-Dattari, and Lasse Peters—thank you for the many insightful research discussions.
My sincere thanks go as well to the CoR secretaries for their kindness, assistance, and for shielding us from excessive university bureaucracy.

My time at MIT was truly transformative. I thank Daniela Rus and Gioele Zardini for making the visit possible, and my many peers—Zach Patterson, Konstantin Rusch, Annan Zhang, Erfan Aasi, Emily Sologuren, Pascal Spino, Joseph DelPreto, Alaa Maalouf, Shiva Sreeram, Wei Xiao, Kiwan Wong, Xinling Li, Yujun Huang, Jiarui Li, Marius Furter, and Riccardo Fiorista—for their warm welcome and for including me in every group activity. I am especially grateful to Konstantin Rusch and Zach Patterson for their impactful collaboration and effective mentorship, and to Kiwan Wong for trusting my guidance—despite initial reluctance to revisit soft-robot projects.
Although my Boston schedule was hectic, the experiences and time spent with friends were essential to maintaining a healthy work-life balance. Among many others, thank you, Niccolò Pagliarani, Francesco Stella, Laurence Willemet, Luzia Knödler, Alberto Comoretto, Viola Del Bono, and Deniz Albayrak for the memorable gatherings, dinners, parties, and good times.

Organizing the Priors4Robots workshop at RSS 2024 was both insightful and fun. I thank all co-organizers—especially John Alora, Luis Pabon, and Roshan Kaundinya—for the great camaraderie (and I’m sorry I couldn’t guide you around Delft in person). I also appreciate every speaker who accepted our invitation, and Michael Lutter for the fascinating private tour of Boston Dynamics.

Conferences and Ph.D. schools were among the most enjoyable parts of my doctoral journey, offering the chance to meet researchers in related fields and build lasting friendships. Niccolò Pagliarani, Burcu Seyidoglu, Francesco Stella, Brandon Caasenbrood, Nana Obayashi, Kai Junge, Zach Patterson, Annan Zhang, Lillian Chin, Ian Good, Davide Calzolari, Mariano Ramirez Montero, Daniel Feliu Talegon, and many others: thank you for the unforgettable RoboSoft adventures and the connected trips to Bali, Joshua Tree National Park, Los Angeles, Portofino, and beyond.
My thanks go to the Dutch soft-robotics community for organizing outstanding symposia and Ph.D. schools, and especially to Brandon Caasenbrood, Philip Mitterbach, Benn Proper, Krishna Kommuri, Mostafa Atalla, Nick Willemstein, and Vera Kortman for the great experiences. It was a pleasure meeting Enrico Donato, Niccolò Pagliarani, Burcu Seyidoglu, Elisa Setti, and many others at the Dutch Soft-Robotics Summer School—thank you all.
Enrico Donato, Elisa Setti, Philip Mitterbach, Benn Proper, Michele Martini, Pietro Pustina, and Daniele Caradonna—thank you for the exceptional culinary adventures in Rome during the soft-robotics modeling and control graduate course.
Despite occasionally feeling like an outsider in the control-systems community, the mostly DLR crew—Davide Calzolari, George Pollayil, and Xuming Meng—made ACC 2022 in Atlanta a blast. ISER 2023 in Chiang Mai was equally enjoyable, thanks to Christopher Bradley, Adrian Piedra, Rafael Papallas, and Nathaniel Simon. 
Mónika Farsang, thank you for teaming up at DRL and for navigating the kilometer-long row of posters with me at NeurIPS 2024 in Vancouver.

To my colleagues and friends in Delft: you made this period special and joyful. Many of my original office mates—Luzia Knödler, Elia Trevision, Yujie Tang, Anna Mészáros, Saray Baker, Khaled Mustafa, and Nils Wilde—started around the same time I did, and together we navigated the highs and lows of Ph.D. and research life. I cherished our lunch-time conversations, tea breaks, and social outings—dinners, karting, skiing, bouldering, tennis, and more. Julian Schumann, thank you for supplying so many delicious cakes for our ''cake breaks`` and for the fun tennis matches.
Tomás Coleman, Anton Bredenbeck, and Italo Belli—thank you for hosting countless barbecues and dinners that united the department. I also appreciate the many other wonderful people at CoR—Corrado Pezzato, Giovanni Franzese, Rodrigo Pérez-Dattari, Mariano Ramirez Montero, Gustavo Rezende, Alvaro Serra Gómez, Jelle Luijkx, Julian Schumann, Max Lodel, Oscar de Groot, Andreu Matoses Gimenez, Max Spahn, Lorenzo Lyons, Dennis Benders, Anastasios (Tasos) Tsolakis, Linda van der Spaan, Bas van der Heijden, Bruno Brito, Alex Ratschat, Ekaterina (Katy) Karmanova, and Fiorella Sibona—for cultivating such a positive department atmosphere.

My social support system has been vital—lifting my spirits, providing welcome distractions from research stress, and accepting the time demands of a Ph.D. I thank my Swiss friends Loris, Philip, Laurens, Daniele, and Claudio for always making me feel at home when I visited. I’m sorry the Netherlands and its beach bars didn’t fully win you over a few years ago; I hope your visit for my defense changes that.

Above all, I am deeply grateful to my family—especially my mom, Sabine, and my brother, Johannes—for being the rock I can always lean on and for inspiring me to become the best version of myself.

To my life partner over this period, Léa: thank you for sharing these years with me and for the many sacrifices you’ve made—most notably moving to Delft / the Netherlands, to join me on this journey. I appreciate your understanding of the long workdays, weekend hours, deadline-driven vacation schedules, and the period abroad in the US that were part of this Ph.D. Your unwavering love and belief in me have been an incredible source of strength and support.

I’m sure I have inadvertently overlooked others who made meaningful contributions to this Ph.D. journey—please know you have my deepest thanks as well.


\section*{Acknowledgements Relating to Specific Chapters}
\vspace{0.3cm}

\begin{itemize}
    \item \textbf{Chapter~\ref{chp:hsamodel}.} The authors would like to thank Ian Good and Jeffrey Lipton from the University of Washington, U.S., for sharing the mechanical characterization published in \citep{good2022expanding}. We also acknowledge Sagar Joshi from the Delft University of Technology, the Netherlands, for their valuable guidance on attaching reflective markers to the HSA.
    \item \textbf{Chapter~\ref{chp:hsacontrol}.} We would like to acknowledge Pietro Pustina for the valuable insights into a coordinate transformation into collocated variables, control of underactuated soft robots, and his help in revising Chapter~\ref{chp:hsacontrol}.
    \item \textbf{Chapter~\ref{chp:braincontrol}.} The authors thank Dr. Fabien Lotte for his suggestions concerning the protocol, Dr. Tomas Ward and the Neuroconcise team for their support with the FlexEEG device, and J.K. Balasubramanian for his assistance with the EEG setup.
    \item \textbf{Chapter~\ref{chp:promasens}.} The authors acknowledge Ehsan Hoseini, Jasper Insinger, and Tom Salden from Delft University of Technology, the Netherlands, for their guidance in designing the \gls{PCB}.
    \item \textbf{Appendix~\ref{chp:apx:holisticcodesign}.} The authors would like to acknowledge Yujun Huang and Marius Furter from the Zardini Lab at MIT and Andrew Fletcher from UC San Diego for reviewing the manuscript draft.
\end{itemize}

\begin{flushright}
{\makeatletter\itshape
    Maximilian Stölzle \\
    Delft, August 2025
\makeatother}
\end{flushright}


