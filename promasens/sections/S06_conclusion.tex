\section{Conclusion}\label{sec:promasens:conclusion}

This chapter proposed a sensing strategy for soft-bodied robots that relies on multiple magnetic sensors embedded directly in the robot. Thanks to a novel kinematics-aware neural architecture, we can simultaneously use information coming from all the sensors to reliably reconstruct the full robot shape. 
The decoupling of the kinematics from the learned sensor measurement predictor allows modifications to the placement of the sensors without requiring a re-training of the neural network.
The proposed method is agnostic to the used kinematic state parametrization, which we verified in simulations using either \gls{PCC} or affine curvature models.
% The method was extensively tested with experiments on a soft segment.
Extensive experiments with a soft segment showed that a model can be trained on one trajectory type and then be used for inference on a variety of other trajectories in the same workspace.
In future work, we will validate the proposed proprioception methodology to execute closed-loop control. % on a multi-segment robot. 
We also invite future research studying the optimal placement of sensors in continuum soft robots, where the optimization procedure might take advantage of the gradients provided by our algorithm.
% Additionally, we would like to validate the proprioception methodology, which we have developed for the general case of $n_\mathrm{b}$ segments, experimentally on a multi-segment robot. 
