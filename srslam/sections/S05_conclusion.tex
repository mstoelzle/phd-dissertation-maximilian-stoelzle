\section{Conclusion}
\label{sec:srslam:conclusions}

This chapter investigated using a monocular camera in shape sensing of continuum soft robots, with the ultimate goal of implementing precise and reliable estimations at the cost of introducing small rigid parts into the hardware design. The contribution of this chapter has been twofold. First, we proposed to use monocular \gls{SLAM} with a soft robot. Second, we propose a regularization of the estimation based on a nonlinear projection in the manifold of the admitted configuration. A nonlinear optimization implements the latter based on the kinematic model of the robot. We have performed extensive simulations with rendered images in Blender and lab experiments with a single-segment soft robot. The nonlinear optimization based on the robot's kinematic model led to a significant improvement in translations and a marginal improvement in rotations. 
Future work will focus on extending the experimental validation of the method to multiple segments and cameras, bettering the SLAM by feeding back the kinematic projection in its state and using this estimation to implement closed-loop control.
While we conducted our experiments in a lab environment under ideal conditions with the camera pointed at checkerboard patterns, thus resulting in plenty of image features for \gls{SLAM} to track, future work should investigate whether deployment environments for soft robots would be sufficiently feature-rich for the use of our proposed method.
