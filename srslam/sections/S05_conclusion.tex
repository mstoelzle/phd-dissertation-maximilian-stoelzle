\section{Conclusion}
\label{sec:srslam:conclusions}

This chapter investigated using a monocular camera in shape sensing of continuum soft robots, with the ultimate goal of implementing precise and reliable estimations at the cost of introducing small rigid parts into the hardware design. The contribution of this chapter has been twofold. First, we proposed to use monocular \gls{SLAM} with a soft robot. Second, we propose a regularization of the estimation based on a nonlinear projection in the manifold of the admitted configuration. A nonlinear optimization implements the latter based on the kinematic model of the robot. We have performed extensive simulations with rendered images in Blender and lab experiments with a single-segment soft robot. The nonlinear optimization based on the robot's kinematic model led to a significant improvement in translations and a marginal improvement in rotations. 
Future work will focus on extending the experimental validation of the method to multiple segments and cameras, bettering the SLAM by feeding back the kinematic projection in its state, and using this estimation to implement closed-loop control.
While we conducted our experiments in a lab environment under ideal conditions with the camera pointed at checkerboard patterns, thus resulting in plenty of image features for \gls{SLAM} to track, future work should investigate whether deployment environments for soft robots would be sufficiently feature-rich for the use of our proposed method.

\section{Afterword}
This chapter demonstrated how inexpensive commercialized monocular cameras can be effectively used together with established \gls{SLAM} algorithms and kinematic models (e.g., \gls{PCC}) to achieve shape sensing for soft robots. One of the main advantages of this solution is that all necessary components are readily available - either commercially or even via open source. However, the presented approach also has some drawbacks: 
i) Firstly, while cameras have been significantly miniaturized in recent years, the requirement for them to have a clear, unobstructed view of the environment necessitates the inclusion of a rigid component on the surface of the soft robot, which we generally prefer to avoid for safety reasons. 
Secondly, ii), the performance of \gls{SLAM} algorithms is affected in environments with limited visually distinguishable features.
Finally, iii) high-dimensional perceptive data such as images are generally computationally expensive to process, leading to higher computational requirements and/or relatively low sampling rates of the shape-sensing information.
Therefore, we present in Chapter~\ref{chp:promasens} an alternative shape-sensing approach based on magnetic sensors. The necessary magnets and magnetic sensors can be deeply embedded into the soft robot body, thus allowing us to keep the robot surface entirely soft and compliant. Furthermore, the sensory data is several orders of magnitude lower dimensional, and thus, its processing is potentially much less computationally demanding.