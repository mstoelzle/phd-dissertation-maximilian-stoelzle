\chapter{Background on Modeling and Control of Soft Robots}\label{chp:background}
\section{Kinematics}
\section{Dynamics}
Irrespective of the specific chosen kinematic parameterization, the Euler-Lagrange dynamics of a soft can be usually stated in the following form~\citep{della2023model}:
\begin{equation}
    B(q) \, \ddot{q} + C(q, \dot{q}) \, \dot{q} + K(q) + G(q) + D \, \dot{q} = A(q) \, \tau,
\end{equation}
where the mass matrix $B(q) \in \mathbb{R}^{n \times n}$\footnote{Note that in some chapters of this thesis, we also use the symbol $M(q)$ to refer to the mass matrix $B(q)$.} captures the inertia of the robot, $C(q,\dot{q}) \in \mathbb{R}^{n \times n}$ integrates the Coriolis and centrifugal effects, $K(q)$ .
We assume that the soft robot is actuated through the term $A(q) \, \tau$, where $A(q) $.
\footnote{We note while this control-affine formulation with a configuration-dependent actuation matrix $A(q)$ can capture the characteristics of most currently existing soft robots actuation methods, such as pneumatic or tendon-driven actuation, the actuation }




\subsection{Operational-space Dynamics}

\subsection{Actuation-space Dynamics}

\section{Model-Based Controllers}
For simplicity, we refer in the following to the fully actuated case and specifically the case of an identified actuation matrix $A = \mathbb{I}_n$.
For an extended discussion on the control in the underactuated setting (i.e., $m < n$), we refer the interested reader to \citet{pustina2025analysis}.