\section{Euler-Lagrangian Dynamics}\label{sec:background:dynamics}
In the following, we will introduce the form and most important characteristics of the dynamics of continuum soft robots formulated in Euler-Lagrangian form.
For a detailed discussion on the derivation of the dynamics, we point the interested reader to several (survey) papers in literature~\citep{gazzola2018forward, armanini2023soft, alessi2024rod, della2023model}.

Irrespective of the specific chosen kinematic parameterization, the \gls{EOM} of a soft can, in most cases, be stated in Euler-Lagrangian form~\citep{della2023model}:
\begin{equation}\label{eq:background:dynamics:eom}
    M(q) \, \ddot{q} + C(q, \dot{q}) \, \dot{q} + K(q) + G(q) + D \, \dot{q} = \alpha(q,\tau) = A(q) \, \tau,
\end{equation}
where the mass matrix $M(q) \succeq 0 \in \mathbb{R}^{n \times n}$
% \footnote{Note that in some chapters of this thesis, we also use the symbol $M(q)$ to refer to the mass matrix $B(q)$.} 
captures the inertia of the robot, $C(q,\dot{q}) \in \mathbb{R}^{n \times n}$ integrates the Coriolis and centrifugal effects, $K(q), G(q) \in \mathbb{R}^{n}$ contribute elastic and gravitational forces, respectively, and $D \succeq 0$ is the damping matrix.
The model verifies similar properties as traditional rigid robots, such that the expression $\dot{M}(q) - 2 \, C(q,\dot{q})$ is skew-symmetric, as long as $C(q,\dot{q})$ is defined with Christoffel symbols~\citep{della2020model}.
We assume that the soft robot is driven by $m$ actuators through the term $\alpha: q \times \tau \mapsto A(q) \, \tau$, where $A(q) \in \mathbb{R}^{n \times m}$. \footnote{We note while this control-affine formulation with a configuration-dependent actuation matrix $A(q)$ can capture the characteristics of most currently existing soft robots actuation methods, such as pneumatic or tendon-driven actuation, the actuation can also become a more general nonlinear function $\alpha(q,\tau)$, such as it is the case for the \gls{HSA} robot actuation model presented in Chapter~\ref{chp:hsamodel}. Furthermore, when additionally considering the actuator dynamics, as in Chapter~\ref{chp:backstepping} for pneumatic pistons, the generalized actuation torque is possibly also a function of the actuator's state $(\mu,\dot{\mu})$.}
If $n=m$, the robot model is considered to be fully actuated; if $n < m$ (rather rare), the soft robot is called overactuated~\citep{pustina2024input}; and if $n > m$, the soft robot model is referred to as underactuated~\citep{pustina2025analysis}.

\subsection{Formulation as an Ordinary Differential Equation (ODE)}
After defining the state of the soft robot as $y = \begin{bmatrix}
    q\\ \dot{q}
\end{bmatrix} \in \mathbb{R}^{2n}$, Eq.~\ref{eq:background:dynamics:eom} can be expressed as a control-affine \gls{ODE} in nonlinear state-space representation:
\begin{equation}
    \dot{y} = f(y,\tau) = \begin{bmatrix}
        \frac{\mathrm{d} q}{\mathrm{d}t}\\
        \frac{\mathrm{d} \dot{q}}{\mathrm{d}t}\\
    \end{bmatrix} = \begin{bmatrix}
        \dot{q}\\
        -M^{-1}(q) \left ( C(q,\dot{q}) \, \dot{q} + K(q) + G(q) + D \, \dot{q} \right )
    \end{bmatrix} + \underbrace{\begin{bmatrix}
        0_{n \times m}\\ M^{-1}(q) \, A(q)
    \end{bmatrix}}_{B(q)} \, \tau.
\end{equation}

\subsection{Kinetic and Potential Energy}
Usually, we can derive analytically the kinetic and potential energy of the system
\begin{equation}
    \mathcal{T}(q,\dot{q}) = \frac{1}{2} \dot{q}^\top M(q) \, \dot{q},
    \qquad
    \mathcal{U}(q) = \mathcal{U}_\mathrm{K}(q) + \mathcal{U}_\mathrm{G}(q) = \int K(q) + G(q) \, \mathrm{d}q,
\end{equation}
where $\mathcal{T}(q,\dot{q}) \geq 0$ is the kinetic and $\mathcal{U}(q)$ the potential energy consisting of elastic and gravitational terms $\mathcal{U}_\mathrm{K}(q)$, $\mathcal{U}_\mathrm{G}(q)$.
which results in $K(q) + G(q) = \frac{\partial \mathcal{U}}{\partial q}$. 
In the case of linear elasticity $K(q) = S \, q$, where $S \succ 0 \in \mathbb{R}^{n \times n}$ is the stiffness of the generalized coordinates, the elastic potential energy becomes $\mathcal{U}_\mathrm{K}(q) = \frac{1}{2} q^\top S \, q$.
The Lagrangian then becomes $\mathcal{L}(q,\dot{q}) = \mathcal{T}(q,\dot{q}) - \mathcal{U}(q)$.
% 
As seen in \citet{della2020model, della2023model} and Chapters~\ref{chp:backstepping} \& \ref{chp:con}, this energy field can serve as a starting point for analyzing the stability of the open- and closed-loop system using Lyapunov arguments~\citep{khalil2002nonlinear}.

\subsection{Task Space Dynamics}\label{sub:background:dynamics:task_space}
As initially proposed in the seminal work by \citet{khatib1987unified} and later adopted to soft robots by \citet{della2020model}, the dynamics of the system can be directly expressed in task space\footnote{The task space is often also referred to as operational space. As oftentimes end-effectors (e.g., grippers) are attached to the distal end of the robot, the task space is usually defined as the pose $\chi \in SE(3)$ of the end-effector.}
Irrespective of the exact definition of the task space (e.g., end-effector pose), for the given forward kinematics $\pi: q \mapsto x$, where $x \in \mathbb{R}^o$ is the task-space coordinate with $o \leq n$, and the associated Jacobian $J(q) = \frac{\partial \pi}{\partial q} \in \mathbb{R}^{o \times n}$, Eq.~\ref{eq:background:dynamics:eom} can be equivalently expressed in task space as
\begin{equation}\label{eq:background:dynamics:eom_task_space}
    \Lambda(q) \, \ddot{x} + \eta(q, \dot{q}) \, \dot{q} + J_\mathrm{M}^{+\top}(q) \left ( G(q) + K(q) + D \, \dot{q} \right ) = J_\mathrm{M}^{+^\top}(q) \, A(q) \, \tau,
\end{equation}
where 
\begin{equation}
    J_\mathrm{M}^{+\top}(q) = M^{-1}(q) \, J^\top(q) \left ( J(q) \, M^{-1}(q) \, J^\top(q) \right )^{-1} \in \mathbb{R}^{n \times o},
\end{equation}
is the dynamically consistent pseudo-inverse~\citep{chang1995manipulator}, which can be used to map task space velocities into configuration space and generalized torques into task space,
\begin{equation}
    \Lambda(q) = \left ( J(q) \, M^{-1}(q) \, J^\top(q) \right )^{-1} \in \mathbb{R}^{o \times o},
\end{equation}
is the task space inertia matrix, and the task space Coriolis and centrifugal matrix $\eta(q, \dot{q})$\footnote{Please note that in Chapter~\ref{chp:hsacontrol} we refer to this task space as $\mu(q,\dot{q})$.} is defined as~\citep{khatib1987unified, della2020model}
\begin{equation}
    \eta(q, \dot{q}) = \Lambda(q) \, \left ( J(q) \, M^{-1}(q) \, C(q,\dot{q}) - \dot{J}(q,\dot{q})) \right ) \in \mathbb{R}^{o \times o}.
\end{equation}


\subsection{Actuation Space Dynamics}\label{sub:background:dynamics:actuation_space}
\citet{pustina2024input} showed that it is often possible to directly formulate the dynamics in actuation coordinates $\varphi \in \mathbb{R}^n$ where the actuation force $\tau$ directly acts on the first $m$ actuation coordinates through an identity matrix. This is particularly interesting for control applications as it allows us to very easily map a control input devised in the generalized (collocated) coordinates into an actuation $\tau$, but it can, in the future, also find use in other applications such as inverse kinematics~\citep{della2025pushing} or workspace estimation. As this mapping into collocated form is particularly useful in the underactuated setting, as later demonstrated in Chapter~\ref{chp:hsacontrol}, we will in the following focus on the setting $m \leq n$.

According to \citet{pustina2024input}, there exists a transformation into collocated form when there is a function $g(q): \mathbb{R}^n \mapsto \mathbb{R}^m$ such that $\frac{\partial g}{\partial q} = A^\top (q),$
which is referred to as \emph{integrability assumption}. In practice, as shown in Sec.~\ref{sec:hsacontrol:configuration_space_regulation}), this mapping $g(q)$ is found by solving the integral $g(q) = \int_0^t \dot{q}(t')^\top \, A(q') \, \mathrm{d} t'$.
If $m < n$, we stress that the coordinates given by $g(q)$ would not be sufficient to fully describe the dynamics of the system as there would exist a null space of dimensionality $n - m$. Therefore, we can choose which  $n-m$ coordinates from the original configuration space to keep. For example, the following transformation into actuation coordinates would be valid for a given $g(q)$~\citep{pustina2024input}
\begin{equation}
    \varphi = \begin{bmatrix}
        \varphi_\mathrm{a}\\ \varphi_\mathrm{u}
    \end{bmatrix} = h(q) \begin{bmatrix}
        g(q)\\ 0_{n-m}
    \end{bmatrix} + \begin{bmatrix}
        0_{m \times m} & 0_{m \times (n-m)}\\
        0_{(n-m) \times m} & \mathbb{I}_{n-m}
    \end{bmatrix} \, q,
\end{equation}
where $\varphi_\mathrm{a} \in \mathbb{R}^m$ and $\varphi_\mathrm{u} \in \mathbb{R}^{n-m}$ refer to the actuated and unactuated actuation coordinates, respectively, and the function $h(q): \mathbb{R}^{n} \to \mathbb{R}^n$ maps from configuration space into collocated form. $J_\mathrm{h}(q)$ is its associated Jacobian defined as
\begin{equation}
    J_\mathrm{h}(q) = \frac{\partial h(q)}{\partial q} = \begin{bmatrix}
        A_\mathrm{a}^\top(q) & A_\mathrm{u}^\top(q)\\
        0_{(n-m) \times m} & \mathbb{I}_{n-m}
    \end{bmatrix},
\end{equation}
and $A_\mathrm{a}(q) \in \mathbb{R}^{m \times m}$, $A_\mathrm{u}(q) \in \mathbb{R}^{(n-m) \times m}$ stem from the following partitioning: $A(q) = \begin{bmatrix}
    A_\mathrm{a}(q)\\ A_\mathrm{u}(q)
\end{bmatrix}$.
The dynamics in actuation coordinates are then given by
\begin{equation}
    M_\varphi(q) \, \ddot{\varphi} + \eta_\varphi(q,\dot{q}) \, \dot{q} + J_{\mathrm{h},\mathrm{M}}^{+\top}(q) \left ( G(q) + K(q) + D \, \dot{q} \right ) = \begin{bmatrix}
        \tau\\ 0_{n-m}
    \end{bmatrix}
\end{equation}
where analog to the task space dynamics, the actuation space inertia is defined as~\citep{khatib1987unified}
\begin{equation}\label{eq:background:dynamics:eom_actuation_coordinates}
    M_\varphi(q) = \left ( J_\mathrm{h}(q) \, M^{-1}(q) \, J_\mathrm{h}^\top(q) \right )^{-1} \in \mathbb{R}^{n \times n},
\end{equation}
$J_{\mathrm{h},\mathrm{M}}^{+\top}(q)$ is the dynamically-consistent pseudo-inverse~\citep{chang1995manipulator}
\begin{equation}
    J_{\mathrm{h},\mathrm{M}}^{+\top}(q) = M^{-1}(q) \, J_\mathrm{h}^\top(q) \left ( J_\mathrm{h}(q) \, M^{-1}(q) \, J_\mathrm{h}^\top(q) \right )^{-1} \in \mathbb{R}^{n \times n},
\end{equation}
and $\eta_\varphi(q,\dot{q})$ captures the Coriolis and centrifugal effects
\begin{equation}
    \eta_\varphi(q, \dot{q}) = M_\varphi(q) \, \left ( J_\mathrm{h}(q) \, M^{-1}(q) \, C(q,\dot{q}) - \dot{J}_\mathrm{h}(q,\dot{q})) \right ) \in \mathbb{R}^{n \times n}.
\end{equation}
Crucially, we now notice in Eq.~\ref{eq:background:dynamics:eom_actuation_coordinates} how $\tau$ directly acts via an identity actuation matrix on the first $m$ actuation coordinates.
%
We can partition the dynamics of Eq.~\ref{eq:background:dynamics:eom_actuation_coordinates} into the first $m$ (actuated) rows, and the last $n-m$ unactuated rows:
\begin{equation}
    \begin{bmatrix}
        M_{\varphi,\mathrm{a}}(q)\\ M_{\varphi,\mathrm{u}}(q)
    \end{bmatrix} \, \ddot{\varphi} + \begin{bmatrix}
        \eta_{\varphi,\mathrm{a}}(q,\dot{q})\\ \eta_{\varphi,\mathrm{u}}(q,\dot{q})
    \end{bmatrix} \, \dot{q} + \begin{bmatrix}
        J_{\mathrm{h},\mathrm{M},\mathrm{a}}^{+\top}(q)\\
        J_{\mathrm{h},\mathrm{M},\mathrm{u}}^{+\top}(q)
    \end{bmatrix} \left ( G(q) + K(q) + D \, \dot{q} \right ) = \begin{bmatrix}
        \tau\\ 0_{n-m}
    \end{bmatrix},
\end{equation}
where $\eta_{\varphi,\mathrm{a}}(q,\dot{q}), J_{\mathrm{h},\mathrm{M},\mathrm{a}}^{+\top}(q) \in \mathbb{R}^{m \times n}$, $\eta_{\varphi,\mathrm{u}}(q,\dot{q}), J_{\mathrm{h},\mathrm{M},\mathrm{u}}^{+\top}(q) \in \mathbb{R}^{(n-m) \times n}$, $M_{\varphi,\mathrm{aa}}(q) \in \mathbb{R}^{m \times m}$, $M_{\varphi,\mathrm{uu}}(q) \in \mathbb{R}^{(n-m) \times (n-m)}$, and $M_{\varphi,\mathrm{ua}}(q) \in \mathbb{R}^{(n-m) \times m}$, and $M_{\varphi,\mathrm{au}}(q) \in \mathbb{R}^{m \times (n-m)}$.

Assuming access to the inverse map $q = h^{-1}(\varphi)$, we can also directly map the potential and kinetic energy of the system into the new coordinates:
\begin{equation}
    \mathcal{T}_\varphi(\varphi, \dot{\varphi}) = \mathcal{T}\left ( q, \dot{q} \right ) \big |_{q=h^{-1}(\varphi), \, \dot{q}=J_\mathrm{h}^{-1} \, \varphi},
    \qquad
    \mathcal{U}_\varphi(\varphi) = \mathcal{U}\left ( q \right ) \big |_{q=h^{-1}(\varphi)}.
\end{equation}
Therefore, we can write the dynamics equivalently as
\begin{equation}
    \begin{bmatrix}
        M_{\varphi,\mathrm{aa}}(q) & M_{\varphi,\mathrm{au}}(q)\\ 
        M_{\varphi,\mathrm{ua}}(q) & M_{\varphi,\mathrm{uu}}(q)
    \end{bmatrix} \, \ddot{\varphi} + \begin{bmatrix}
        \eta_{\varphi,\mathrm{a}}(q,\dot{q})\\ \eta_{\varphi,\mathrm{u}}(q,\dot{q})
    \end{bmatrix} \, \dot{q} + \begin{bmatrix}
        \partial_{\varphi_\mathrm{a}} \mathcal{U}_\varphi(q)\\
        \partial_{\varphi_\mathrm{u}} \mathcal{U}_\varphi(q)
    \end{bmatrix} + \begin{bmatrix}
        D_{\varphi,\mathrm{a}}(q)\\
        D_{\varphi,\mathrm{u}}(q)
    \end{bmatrix} \dot{q} = \begin{bmatrix}
        \tau\\ 0_{n-m}
    \end{bmatrix},
\end{equation}
with $\partial_\varphi \mathcal{U}(\varphi) = \frac{\partial }{\partial \varphi}  \mathcal{U}_\varphi(\varphi) \in \mathbb{R}^n$ the potential forces and $D_{\varphi}(q) \in \mathbb{R}^{n \times n}$ the damping matrix in actuation coordinates, respectively.