% \chapter{ISS Coupled Oscillator Networks for Model-based Control in Latent Space}
\chapter{Learning Latent Dynamics with ISS Coupled Oscillator Networks}
\label{chp:con}

\begin{foreword}
    In Chapter~\ref{chp:pcsregression}, we presented a data-driven approach for learning strain models for continuum soft robots that exhibits full \emph{white-box} interpretability of the identified kinematic parametrization and the physics-based dynamics.
    However, the kinematic and dynamical behavior of certain more complex robots, such as non-slender soft robots or soft robots exhibiting nonlinear stiffness, varying cross-sections, and hysteresis, cannot be captured by the proposed strain-based approach that builds on the Cosserat rod theory.
    Furthermore, we sometimes cannot easily extract low-dimensional state information and instead only have access to high-dimensional observation in the form of images or depth measurements.
    As a solution, we leverage in this chapter \glsxtrfull{CON} for learning latent dynamics of physical systems directly from image observations. The \gls{CON} network, as introduced in Section~\ref{sec:con:con}, consists of harmonic oscillators that are coupled by a potential energy. As described in Section~\ref{sec:con:learning_latent_space_dynamics}, we utilize a \gls{VAE}-based autoencoder to encode the pixels into latent space. Interestingly, this setup now allows us to assign a mechanical interpretation to the latent encoding of the image, as each latent variable corresponds to the position of a harmonic oscillator. Furthermore, we can learn the potential energy of the system in such a way that strong stability guarantees for the latent dynamics are automatically incorporated. In particular, we can, for example, guarantee \glsxtrfull{GAS} for the unactuated system and \glsxtrfull{ISS} for the actuated system.
    Finally, we exploit in Section~\ref{sec:con:latent_space_control} the physical structure of the \gls{CON} dynamics for closed-form model-based control with P-satI-D+Feedforward controllers, as already seen previously in Chapters~\ref{chp:hsacontrol} \& \ref{chp:pcsregression}.
\end{foreword}

\blfootnote{This chapter is partly based on \faFileTextO \, \faTrophy~\emph{\textbf{M. Stölzle}, and C. Della Santina (2024). Input-to-State Stable Coupled Oscillator Networks for Closed-form Model-based Control in Latent Space. In Proceedings of Advances in Neural Information Processing Systems (NeurIPS) 37, Accepted as a \textbf{Spotlight} paper~\citep{stolzle2024input}.}}

\pagebreak

\begin{abstract}
    Even though a variety of methods have been proposed in the literature, efficient and effective latent-space control (i.e., control in a learned low-dimensional space) of physical systems remains an open challenge.
    We argue that a promising avenue is to leverage powerful and well-understood closed-form strategies from control theory literature in combination with learned dynamics, such as potential-energy shaping.
    We identify three fundamental shortcomings in existing latent-space models that have so far prevented this powerful combination: (i) they lack the mathematical structure of a physical system, (ii) they do not inherently conserve the stability properties of the real systems, (iii) these methods do not have an invertible mapping between input and latent-space forcing.
    This work proposes a novel Coupled Oscillator Network (CON) model that simultaneously tackles all these issues. 
    More specifically, (i) we show analytically that CON is a Lagrangian system - i.e., it possesses well-defined potential and kinetic energy terms. Then, (ii) we provide formal proof of global Input-to-State stability using Lyapunov arguments.
    Moving to the experimental side, we demonstrate that \gls{CON} reaches \gls{SOTA} performance when learning complex nonlinear dynamics of mechanical systems directly from images.
    An additional methodological innovation contributing to achieving this third goal is an approximated closed-form solution for efficient integration of network dynamics, which eases efficient training.
    We tackle (iii) by approximating the forcing-to-input mapping with a decoder that is trained to reconstruct the input based on the encoded latent space force.
    Finally, we leverage these three properties and show that they enable latent-space control. We use an integral-saturated PID with potential force compensation and demonstrate high-quality performance on a soft robot using raw pixels as the only feedback information.
\end{abstract}


%% Start the actual chapter on a new page.
\newpage

\section{Introduction}
% Learning latent dynamics models from high-dimensional observations such as images is a promising avenue toward letting our artificial and physical agents better and efficiently understand the world around them and is essential for achieving both artificial and physical intelligence~\cite{}
Learning how the environment evolves around us from high-dimensional observations (i.e., world models~\cite{ha2018world}) is essential for achieving both artificial and physical intelligence~\cite{hafner2023mastering}.
For example, world models are required for effectively planning an artificial/robotic agent's actions in complex and unstructured environments~\cite{matsuo2022deep}. 
However, learning such dynamics directly in high-dimensional observation space is usually intractable. Seminal works have shown that we can leverage autoencoders to compress the state information into a low-dimensional latent space~\cite{liou2014autoencoder, kingma2014auto} in which it is much more feasible to learn the dynamics~\cite{watter2015embed, lenz2015deepmpc, wahlstrom2015learning, champion2019data, zhong2020unsupervised}.
%
However, strong limitations still persist when it comes to using these learned models to generate low-level intelligence.

One outstanding challenge is how to perform closed-loop control in the learned latent space - i.e., how to generate control inputs based on a high dimensional sensory input such that a desired movement is generated. Prior works have explored, among other approaches, \gls{RL}~\cite{van2016stable, gelada2019deepmdp, hafner2019dream, schwarting2021deep}, \gls{MPC}~\cite{lenz2015deepmpc, hafner2019learning, hewing2020learning, alora2023robust}, \glspl{LQR}~\cite{brunton2016koopman, mamakoukas2019local, haggerty2023control} and gradient-based optimization~\cite{yamada2023leveraging} for planning and control towards a target evolution that is given in observation space.
However, all existing latent-space control strategies have shortcomings, such as a limited planning horizon and slow control rates (\gls{MPC} and gradient-based approaches), sample inefficiency (RL), or they pose a requirement for learning linear dynamics~\cite{watter2015embed} (\gls{LQR}), which is not even possible for systems that are inherently non-linearizable~\cite{cenedese2022data}.
One interesting avenue is to leverage model-based control approaches, such as potential shaping~\cite{bloch2001controlled, ortega2021pid, zhong2020unsupervised}, for effective and computationally efficient control in latent space~\cite{lepri2023neural}. 
For these techniques to be feasible, the dynamical model needs to fulfill four characteristics: (i) the dynamics need to have the mathematical structure of physical systems, (ii) conserve the stability properties of real systems, (iii) the latent state needs to be relatively low-dimensional, and (iv) there needs to exist a well-defined, invertible mapping between the input and the forcing in latent space.
% the latent state needs to be relatively low-dimensional, (3) there needs to exist a well-defined, invertible mapping between the input and the forcing in latent space, and (4) the latent dynamics need to be easily stabilizable.
However, existing model structures that are used for learning latent dynamics~\cite{botev2021priors} do not meet all of these criteria. Relevant examples are \glspl{MLP}, \glspl{NODE}~\cite{chen2018neural, sholokhov2023physics}, many variants of \glspl{RNN} (e.g., LSTMs~\cite{hochreiter1997long}, \glspl{GRU}~\cite{cho2014learning}, etc.), and physics-informed neural networks (e.g., \glspl{LNN}~\cite{lutter2018deep, cranmer2020lagrangian, zhong2020unsupervised}, \glspl{HNN}~\cite{greydanus2019hamiltonian}). For example, \glspl{MLP} do not have a physical interpretation and do not provide an invertible mapping of the forcing generated by the input, \glspl{NODE} are usually not easily stabilizable~\cite{white2023stabilized}, most \glspl{RNN} require a relatively high-dimensional latent space (i.e., many hidden states), and energy-shaping control approaches based on \glspl{LNN}~\cite{zhong2020unsupervised} do not come with any formal stability guarantees. %Interestingly, \glspl{LNN} and HNNs could be a valid venue, are empirically %\glspl{LNN} are often not stabilizable when trained in latent-space.

In recent years, oscillatory networks~\cite{rusch2020coupled, rusch2021unicornn, ceni2024random, lanthaler2024neural, rusch2024oscillatory} have been shown to exhibit state-of-the-art performance on time sequence modeling tasks while being parameter-efficient, thus fulfilling our requirement (iii).
% As we have a good understanding of the behavior of harmonic oscillators, they are also interpretable.
%
Consequently, we believe that they are a promising option for control-oriented dynamics learning in latent space.
%
Still, these models do not fulfill the remaining requirements that we have listed above. % open gaps (i),(ii), and (iv). 
%
Despite being an interpretable combination of harmonic oscillators, they do not have the structure of a physical system - i.e., they do not possess a well-defined energy function.
%
Moreover, only local stability~\cite{rusch2020coupled, ceni2024random} has been shown, with sufficient conditions that appear to be very stringent. %, as the energy dissipation of these networks cannot be modeled.
%
Finally, in addition to training an encoder that maps inputs to latent-space forcing, we propose also training a decoder that learns to reconstruct inputs based on latent-space forcing. This enables us to easily switch between inputs and forcing, which is essential when implementing control strategies.

We resolve all the above-mentioned challenges by proposing \glspl{CON}, a new formulation of a coupled oscillator network that is inherently \gls{ISS} stable, for learning the dynamics of physical systems and subsequently exploiting its structure for model-based control in latent space.
The network consists of damped, harmonic oscillators connected through elastic springs, damping elements, and a neuron-like coupling force and can be excited by a nonlinear actuation term.
We identify a transformation into a set of coordinates from which we can derive the networks' kinetic and potential energy.
This allows us to leverage Lyapunov arguments~\cite{khalil2002nonlinear} for proving the global asymptotic stability of the unforced system and \gls{ISS} stability for the forced system under relatively mild assumptions on the network parameters.
Even though we constrain the dynamics to a very specific structure, we demonstrate (a) the CON network achieves similar performance as \glspl{NODE} when learning the dynamics of unactuated, mechanical systems with two orders of magnitude fewer parameters and (b) that the proposed model achieves, for the complex task of learning the actuated, highly nonlinear dynamics of continuum soft robots~\cite{alora2023data, stolzle2021piston} directly from pixels, a \SI{60}{\percent} lower prediction error than \gls{coRNN}~\cite{rusch2020coupled} and reaches the SoA performance across all techniques that we tested.
Finally, we show some initial results that the proposed \gls{CON} model is also able to learn the latent dynamics of \glspl{PDE}, in this case containing reaction-diffusion~\cite{champion2019data, epstein2016reaction} dynamics.

\begin{figure}[t]
    \centering
    \subfigure[Coupled Oscillator Network (CON)]{\includegraphics[width=0.45\columnwidth]{con/figures/con/blockdiagram_coupled_oscillator_network_v2_cropped.pdf}\label{fig:con:con}}
    \subfigure[Learning latent-space dynamics with CON]{\includegraphics[width=0.53\columnwidth]{con/figures/autoencoder/blockdiagram_autoencoder_v1_cropped.pdf}\label{fig:con:blockdiagram_autoencoder}}
    \caption{\textbf{Panel (a)}: The proposed CON network consists of $n$ damped harmonic oscillators that are coupled through the neuron-like connection $\tanh(Wx+b)$ and the non-diagonal stiffness $K-k$ and damping coefficients $D-d$, respectively. The state of the network is captured by the positions $x(t)$ and velocities $\dot{x}(t)$ of the oscillators. The time-dependent input is mapped through the (possibly nonlinear) function $g(u)$ to a forcing $\tau$ acting on the oscillators. 
    \textbf{Panel (b):} Exploiting \glspl{CON} for learning latent dynamics from pixels: We encode the initial observation $o(t_0)$ and the input $u(t)$ into latent space where we leverage the \gls{CON} to predict future latent states. Finally, we decode both the latent-space torques $\tau(t)$ and the predicted latent states $z(t)$.
    }
\end{figure}

Subsequently, we derive an approximate closed-form solution, that is, in parameter regimes in which the linear, decoupled dynamics dominate transient, more accurate than numerical integrators with comparable computational requirements and which increases training speed by 2x with a small decrease in prediction accuracy.
Finally, as we can derive the system's potential energy, we can leverage potential shaping~\cite{bloch2001controlled, ortega2021pid} to derive a controller that combines an integral-saturated PID controller with a feedforward term compensating potential forces.
As the feedback acts on a well-shaped potential field, tuning the feedback gains becomes very simple and out-of-the-box, and the controller exhibits a faster response time and a \SI{26}{\percent} lower trajectory tracking \gls{RMSE} than a pure feedback controller based on a latent \gls{NODE}~\cite{chen2018neural} model.

The proposed methodology is particularly well-suited for learning the latent dynamics of mechanical systems with continuous dynamics, dissipation, and a single, attractive equilibrium point. Examples of such systems include many soft robots, deformable objects with dominant elastic behavior, Lagrangian systems immersed in a dominant potential field, or locally other mechanical systems such as robotic manipulators, legged robots, etc. For these systems, we can fully leverage the structural prior of the proposed latent dynamics, including the integrated stability guarantees. If the system is actuated, the learned dynamics can be subsequently exploited for model-based control, as demonstrated in Sec.~\ref{sec:con:latent_space_control}.

The code associated with this chapter is available on GitHub\footnote{\url{https://edu.nl/yu9vw}}.
Furthermore, we provide more details about the experimental setup and present additional results in Apx.~\ref{chp:apx:con}.

% In summary, our work makes the following contributions:
% \begin{itemize}
%     \item We propose a coupled oscillator network formulation with strong stability guarantees and a physical interpretation of its dynamics in terms of kinetic and potential energy that exhibits performance on par with the state-of-the-art, non-structured methods for learning dynamics in the tested situations.
%     \item We leverage for the first time oscillator networks for learning latent dynamics. This provides structure to the learned dynamics while preserving expressiveness.
%     \item The special characteristics of the network allow us to apply partial feedback linearization-based control techniques in a learned latent space.
%     \item An approximated closed-form solution for integrating the dynamics of the coupled oscillator network is provided, which is more accurate compared to numerical \gls{ODE} integrators with the same computational demand when the dynamics of underdamped, decoupled, and linear oscillators are dominant.
% \end{itemize}
\section{Input-to-State Stable (ISS) Coupled Oscillator Networks (CONs)}\label{sec:con:con}

\begin{figure}[t]
    \centering
    \includegraphics[width=0.8\linewidth]{con/figures/con/blockdiagram_coupled_oscillator_network_v2_cropped.pdf}
    \caption{The proposed Coupled Oscillator Network (CON) network consists of $n$ damped harmonic oscillators that are coupled through the neuron-like connection $\tanh(Wx+b)$ and the non-diagonal stiffness $K-k$ and damping coefficients $D-d$, respectively. The state of the network is captured by the positions $x(t)$ and velocities $\dot{x}(t)$ of the oscillators. The time-dependent input is mapped through the (possibly nonlinear) function $g(u)$ to a forcing $\tau$ acting on the oscillators.}
    \label{fig:con:con}
\end{figure}

\subsection{Formulation} 
The integral component to (coupled) oscillatory \glspl{RNN}~\cite{rusch2020coupled, rusch2021unicornn, ceni2024random, lanthaler2024neural} are one-dimensional, potentially damped, harmonic oscillators, which are described by their state $y_i = \begin{bmatrix}
    x_i & \dot{x}_i
\end{bmatrix}^\mathrm{T} \in \mathbb{R}^2$, where $x_i$ and $\dot{x}_i$ are the position and velocity of the oscillator, respectively. Then, the oscillator's dynamics are defined by the following \gls{EOM}
\begin{equation}\label{eq:con:harmonic_oscillator}
    m_i \, \ddot{x}_i(t) + d_i \, \dot{x}_i(t) + \kappa_i \, x_i(t) = F_i(t),
    \qquad
    \text{with } m_i, \kappa_i, d_i \in \mathbb{R}^+.
\end{equation}
Here, $m_i$ is the mass, $\kappa_i$ is the stiffness, and $d_i$ is the damping coefficient of the damped harmonic oscillator. $F_i(t) \in \mathbb{R}$ is a (possibly time-dependent) external forcing term acting on the mass.

Even though the state is extremely low dimensional and the number of parameters is small, this single, damped harmonic oscillator can already exhibit a variety of (designable) behaviors:
The expressions $\omega_{\mathrm{n},i} = \sqrt{\frac{\kappa_i}{m_i}}$ and $\zeta_i = \frac{d_i}{2 \, \sqrt{\kappa_i \, m_i}}$ let us determine the natural frequency and the damping factor, respectively and allow us to design the transient behavior. For example, $\omega_{\mathrm{n},i}$ lets us isolate a spectrum of the input signal $F_i(t)$~\cite{ceni2024random} and $\zeta_i$ determines the damping regime: underdamped ($\omega_{\mathrm{n},i} < 1$), critically damped ($\omega_{\mathrm{n},i} = 1$), overdamped ($\omega_{\mathrm{n},i} > 1$). 
Furthermore, as (damped) harmonic oscillators are omnipresent in nature (and especially in physical systems), they have been intensively studied and are well understood (e.g., characteristics, closed-form solutions, etc.). 
In this work, we will exploit some of these properties and knowledge to learn stable (latent) dynamics efficiently.

By intercoupling damped harmonic oscillators, we can drastically increase the expressiveness of the dynamical system~\cite{rusch2020coupled, ceni2024random, lanthaler2024neural} while preserving some of the intuition and understanding we have for these systems. In this work, we propose a \gls{ISS}-stable \gls{CON} consisting of $n$ damped harmonic oscillators that are coupled through both linear and nonlinear terms. The networks' state is defined as $y = \begin{bmatrix}
    x^\mathrm{T} & \dot{x}^\mathrm{T}
\end{bmatrix}^\mathrm{T} \in \mathbb{R}^{2n}$ and its dynamics can be formulated as a \nth{2}-order \gls{ODE}
\begin{equation}\label{eq:con:con_dynamics}
    \dot{y}(t) = \begin{bmatrix}
        \frac{\mathrm{d}x}{\mathrm{d}t}\\
        \frac{\mathrm{d}\dot{x}}{\mathrm{d}t}
    \end{bmatrix} = f(y(t), u(t)) = \begin{bmatrix}
        \dot{x}(t)\\
        g(u(t)) -K x(t) - D \, \dot{x}(t) - \tanh(W \, x(t) + b)
    \end{bmatrix},
\end{equation}
where $K, D \in \mathbb{R}^{n \times n}$ are the linear stiffness and damping matrices, respectively. The neuron-inspired term $\tanh(W \, x(t) + b)$ with $W \in \mathbb{R}^{n \times n}$, $b \in \mathbb{R}^n$ provides nonlinear coupling between the harmonic oscillators.
The network is excited by the time-dependent input $u(t) \in \mathbb{R}^m$ through the possibly nonlinear mapping $g: \mathbb{R}^m \to \mathbb{R}^n$.
Specifically, we consider in this work a formulation where an input-dependent matrix $B(u) \in \mathbb{R}^{n \times m}$ projects the input $u(t)$ to a time-dependent forcing on the oscillators: $\tau = g(u) = B(u) \, u$. Here $B(u)$ could, for example, be parametrized by a \gls{MLP}.

We specifically designed the network architecture such that (i) the system exhibits a unique and isolated equilibrium and (ii) we can derive expressions for the kinetic and potential energies. These two features allow us to (a) prove \gls{GAS} and \gls{ISS} stability using an established procedure based on strict Lyapunov arguments~\cite{calzolari2020exponential, wu2022passive}, and (b) implement model-based controller based on potential shaping.

\subsection{Potential energy and determination of equilibria}

One key insight of this work is that in the coordinates $x(t), \dot{x}(t)$, we cannot derive a potential 
% and subsequently use Lyapunov arguments to prove stability 
as the hyperbolic force $\tanh(W x(t) + b)$ is not symmetric. Therefore, we propose a coordinate transformation into $\mathcal{W}$-coordinates: $y_\mathrm{w}(t) = \begin{bmatrix}
    x_\mathrm{w}(t)\\
    \dot{x}_\mathrm{w}(t)\\
\end{bmatrix} = \begin{bmatrix}
    W \, x(t)\\
    W \, \dot{x}(t)
\end{bmatrix} \in \mathbb{R}^{2n}$. The coordinate transformation is valid if its Jacobian is full-rank, which is the case if $\mathrm{rank}(W) = n$.
In $\mathcal{W}$-coordinates, the dynamics can be rewritten as
\begin{equation}\label{eq:con:conw_dynamics}
    \dot{y}_\mathrm{w}(t) = \begin{bmatrix}
        \frac{\mathrm{d}x_\mathrm{w}}{\mathrm{d}t}\\
        \frac{\mathrm{d}\dot{x}_\mathrm{w}}{\mathrm{d}t}
    \end{bmatrix} = f_\mathrm{w}(y(t), u(t)) = \begin{bmatrix}
        \dot{x}_\mathrm{w}(t)\\
        M_\mathrm{w}^{-1} \, \left (g(u(t)) -K_\mathrm{w} x_\mathrm{w}(t) - D_\mathrm{w} \, \dot{x}_\mathrm{w}(t) - \tanh(x_\mathrm{w}(t) + b) \right )
    \end{bmatrix}
\end{equation}
with $K_\mathrm{w} = K \, W$, $D_\mathrm{w} = D \, W$ and $M_\mathrm{w} = W^{-1}$. 

A difference of this formulation compared to prior work~\cite{rusch2020coupled, rusch2021unicornn, ceni2024random, lanthaler2024neural} is that (i) the forcing produced by the input term $\tau = g(u)$ is fully separated from the forcing produced by the elastic coupling terms $K_\mathrm{w}$, and (ii) the generalized force is symmetric as stated in Lemma~\ref{lemma:con:conw_symmetric_potential}, allowing us to define a potential energy expression, which we can later on leverage for stability analysis and control.

\begin{lemma}\label{lemma:con:conw_symmetric_potential}
    Let $\tilde{x}_\mathrm{w} \in \mathbb{R}^n$ be generalized coordinates and $\bar{x}_\mathrm{w}, b \in \mathbb{R}^n$ constants.
    Then, the potential force of system \eqref{eq:con:conw_residual_dynamics}
    \begin{equation}
        \tilde{f}_{\mathcal{U}_\mathrm{w}}(\tilde{x}_\mathrm{w}) = K_\mathrm{w} (\bar{x}_\mathrm{w} + \tilde{x}_\mathrm{w}) + \tanh(\bar{x}_\mathrm{w} + \tilde{x}_\mathrm{w} + b),
    \end{equation}
    stems from the potential
    \begin{equation}
        \mathcal{U}_\mathrm{w}(\tilde{x}_\mathrm{w}) = \sum_{i=1}^n \int_{0}^{\tilde{x}_{\mathrm{w},i}} \tanh(\bar{x}_{\mathrm{w},i}+\sigma+b_i) \, \mathrm{d} \sigma - \sum_{i=1}^n \int_{0}^{\tilde{x}_{\mathrm{w},i}} \tanh(\bar{x}_{\mathrm{w},i}+b_i) \, \mathrm{d} \sigma \in \mathbb{R}.
    \end{equation}
\end{lemma}
\begin{proof}
    First, we take the derivative of $\mathcal{U}(\tilde{x}_\mathrm{w})$:
    \begin{equation}
        \frac{\partial \mathcal{U}_\mathrm{w}}{\partial \tilde{x}_\mathrm{w}} = K_\mathrm{w} (\bar{x}_\mathrm{w} + \tilde{x}_\mathrm{w}) + \tanh(\bar{x}_\mathrm{w} + \tilde{x}_\mathrm{w} + b) = \tilde{f}_{\mathcal{U}_\mathrm{w}}.
    \end{equation}
    The Hessian of the potential is given by
    \begin{equation}
        H_{\mathcal{U}_\mathrm{w}}(\tilde{x}_\mathrm{w}) = \frac{\partial^2 \mathcal{U}_\mathrm{w}}{\partial \tilde{x}_\mathrm{w}^2} = \frac{\partial \tilde{f}_{\mathcal{U}_\mathrm{w}}}{\partial \tilde{x}_\mathrm{w}}  = K_\mathrm{w} + S_\mathrm{sech}^{2}(\tilde{x}_\mathrm{w}) \in \mathbb{R}^{n \times n}.
    \end{equation}
    As $K_\mathrm{w} \succ 0 \Rightarrow K_\mathrm{w} = K_\mathrm{w}^\mathrm{T}$, we can easily show that the potential force is symmetric:
    \begin{equation}
        H_{\mathcal{U}_\mathrm{w}}^\mathrm{T} = K_\mathrm{w}^\mathrm{T}  + S_\mathrm{sech}^{2}(\tilde{x}_\mathrm{w})^\mathrm{T} = K_\mathrm{w} + S_\mathrm{sech}^{2}(\tilde{x}_\mathrm{w}) = H_{\mathcal{U}_\mathrm{w}}.
    \end{equation}
\end{proof}

The equilibria $\bar{y}_\mathrm{w} = \begin{bmatrix}
    \bar{x}_\mathrm{w}^\mathrm{T} & 0^\mathrm{T}
\end{bmatrix}^\mathrm{T} \in \mathbb{R}^{2n}$ of the unforced network are given by the roots of the characteristic equation $\tanh(\bar{x}_\mathrm{w} + b) + K_\mathrm{w} \, \bar{x}_\mathrm{w} = 0$.%  The following Lemma states that the network exhibits a single, isolated equilibrium when $K_\mathrm{w}$ is positive-definite.
\begin{lemma}\label{lemma:con:conw_single_equilbrium}
    Let $K_\mathrm{w} \succ 0$. Then, the dynamics defined in \eqref{eq:con:conw_dynamics} have a single, isolated equilibrium $\bar{y}_\mathrm{w} = \begin{bmatrix}
        \bar{x}_\mathrm{w}^\mathrm{T} & 0^\mathrm{T}
    \end{bmatrix}^\mathrm{T}$.
\end{lemma}
\begin{proof}
    We regard the characteristic equation as a function: $h_\mathrm{eq}(x_\mathrm{w}) = \tanh(x_\mathrm{w} + b) + K_\mathrm{w} \, x_\mathrm{w}$. For there to exist multiple equilibria, $h_\mathrm{eq}(\bar{x}_\mathrm{w}) = 0$ would need to be true for multiple $\bar{x}$. However, we take the partial derivative of $h_\mathrm{eq}(x_\mathrm{w})$ w.r.t. $x_\mathrm{w}$ and see that
    \begin{equation}\label{eq:con:lemma_conw_single_equilbrium}
        \frac{\partial h_\mathrm{eq}}{\partial x_\mathrm{w}} =  K_\mathrm{w} + S_\mathrm{sech}^2(x_\mathrm{w}) \succ 0, \quad \forall x_\mathrm{w} \in \, \mathbb{R}^n
        \qquad \text{with }
        S_\mathrm{sech}(x_\mathrm{w}) = \mathrm{diag}(\mathrm{sech}(x_\mathrm{w} + b)) \in \mathbb{R}^{n \times n}
    \end{equation}
    as $S_\mathrm{sech}(x_\mathrm{w}) \succ 0 \quad \forall x_\mathrm{w} \in \, \mathbb{R}^n$ and $K_\mathrm{w} \succ 0$. Therefore, $h_\mathrm{eq}(x_\mathrm{w})$ is continuously increasing and can only cross the zero line once.
\end{proof}
Next, we introduce a mapping into the $\tilde{\text{tilde}}$ coordinates $\tilde{y}_\mathrm{w} = y_\mathrm{w} - \bar{y}_\mathrm{w}$. The residual dynamics (w.r.t. the equilibrium $\bar{y}_\mathrm{w}$) can now be stated as
\begin{equation}\label{eq:con:conw_residual_dynamics}
    \dot{\tilde{y}}_\mathrm{w}(t) = \tilde{f}_\mathrm{w}(y, u) = \begin{bmatrix}
        \dot{\tilde{x}}_\mathrm{w}(t)\\
        M_\mathrm{w}^{-1} \, \left (g(u(t)) -K_\mathrm{w} \left (\bar{x}_\mathrm{w} + \tilde{x}_\mathrm{w}(t) \right ) - D_\mathrm{w} \, \dot{\tilde{x}}_\mathrm{w}(t) - \tanh(\bar{x}_\mathrm{w} + \tilde{x}_\mathrm{w}(t) + b) \right )
    \end{bmatrix}
\end{equation}

% In the following, we will write $\lVert A \rVert$ to denote the induced norm of matrix $A$ and $\lambda_\mathrm{m}(A)$, $\lambda_\mathrm{M}(A)$ to refer to its minimum and maximum Eigenvalue respectively.

\begin{figure}[t]
    \centering
    \subfigure[GAS: $g(u)=0$]{\includegraphics[width=0.46\columnwidth]{con/figures/iss/gas_illustration_cropped.pdf}}
    \subfigure[ISS]{\includegraphics[width=0.49\columnwidth]{con/figures/iss/iss_illustration_cropped.pdf}}
    \caption{Illustration of global asymptotic stability for the unforced system with $g(u)=0$ and input-to-state stability for the forced system, where the black dashed line denotes the input $u(t)$.}
    \label{fig:con:gas_iss_illustration}
\end{figure}

\subsection{Global Asymptotic Stability (GAS) for the unforced system} 
We first consider the unforced system with $\tau = g(u) = 0, \: \forall t \in [t_0, t_\infty)$ and strive to prove global asymptotic stability~\cite{khalil2002nonlinear} for the attractor $\bar{x}$.
For this, we need to first identify a valid, (strict) Lyapunov function and subsequently demonstrate that, expect at the global equilibrium, the time derivative of the Lypapunov function is negative definite.

\subsubsection{Strict Lyapunov candidate}
We propose a strict Lyapunov candidate with skewed level sets~\cite{wu2022passive}
\begin{equation}\label{eq:con:conw_lyapunov_function}
\begin{split}
    V_\mu(\tilde{y}_\mathrm{w}) =& \: \frac{1}{2} \, \tilde{y}_\mathrm{w}^\mathrm{T} \, P_\mathrm{V} \, \tilde{y}_\mathrm{w} + \sum_{i=1}^n \int_{0}^{\tilde{x}_{\mathrm{w},i}} \tanh(\bar{x}_{\mathrm{w},i}+\sigma+b_i) \, \mathrm{d} \sigma - \sum_{i=1}^n \int_{0}^{\tilde{x}_{\mathrm{w},i}} \tanh(\bar{x}_{\mathrm{w},i}+b_i) \, \mathrm{d} \sigma,\\
    =& \: \frac{1}{2} \, \tilde{y}_\mathrm{w}^\mathrm{T} \, P_\mathrm{V} \, \tilde{y}_\mathrm{w} + \sum_{i=1}^n \left ( \mathrm{lcosh}(\bar{x}_{\mathrm{w},i} + \tilde{x}_{\mathrm{w},i}+b_i)-\mathrm{lcosh}(\bar{x}_{\mathrm{w},i}+b_i)-\tanh(\bar{x}_{\mathrm{w},i}+b_i) \, \tilde{x}_{\mathrm{w},i} \right ),\\
    \text{with } P_\mathrm{V} =& \: \begin{bmatrix}
        K_\mathrm{w} & \mu \, M_\mathrm{w}\\
        \mu \, M_\mathrm{w}^\mathrm{T} & M_\mathrm{w}
    \end{bmatrix} \in \mathbb{R}^{2n \times 2n}, \qquad \mathrm{lcosh}(\cdot) = \log(\cosh(\cdot)), \qquad \text{and } \mu > 0.
\end{split}
\end{equation}

In the following, we will write $\lVert A \rVert$ to denote the induced norm of matrix $A$ and $\lambda_\mathrm{m}(A)$, $\lambda_\mathrm{M}(A)$ to refer to its minimum and maximum Eigenvalue respectively.
% 
The gradient of $V_\mu(\tilde{y}_\mathrm{w})$ w.r.t. the residual coordinate $\tilde{y}_\mathrm{w}$ is given by
\begin{equation}\label{eq:con:V_mu_gradient}
    \frac{\partial V_\mu}{\partial \tilde{y}_\mathrm{w}}(\tilde{y}_\mathrm{w}) = P_\mathrm{V} \, \tilde{y}_\mathrm{w} + 
    \begin{bmatrix}
        \tanh(\bar{x}_\mathrm{w} + \tilde{x}_\mathrm{w} + b)\\ 
        0^{n}
    \end{bmatrix} - \begin{bmatrix}
        \tanh(\bar{x}_\mathrm{w} + b)\\ 
        0^{n}
    \end{bmatrix}.
\end{equation}
Next, the Hessian of the Lyapunov candidate can be derived as
\begin{equation}\label{eq:con:V_mu_hessian}
    H_\mathrm{V}(\tilde{x}_\mathrm{w}) = \frac{\partial^2 V_\mu}{\partial \tilde{y}_\mathrm{w}^2} = \begin{bmatrix}
        K_\mathrm{w} + S_\mathrm{sech}^{2}(\tilde{x}_\mathrm{w}) & \mu \, M_\mathrm{w}\\
        \mu \, M_\mathrm{w}^\mathrm{T} & M_\mathrm{w}
    \end{bmatrix} \in \mathbb{R}^{2n \times 2n},
\end{equation}
where
\begin{equation}\label{eq:con:S_sech_definition}
    S_\mathrm{sech}(\tilde{x}_\mathrm{w}) = \mathrm{diag}(\mathrm{sech}(\bar{x}_\mathrm{w} + \tilde{x}_\mathrm{w} + b)) \in \mathbb{R}^{n \times n} \succ 0 \quad \forall \tilde{x}_\mathrm{w} \in \mathbb{R}^n.
\end{equation}
Furthermore, the Schur complement of $P_\mathrm{V}$ is given by
\begin{equation}\label{eq:con:P_V_Schur_complement}
    S_{P_\mathrm{V}} = M_\mathrm{w} - \mu^2 M_\mathrm{w}^\mathrm{T} \, K_\mathrm{w} \, M_\mathrm{w}
\end{equation}

Below, we will first introduce three auxiliary Lemmas and subsequently proof Lemma~\ref{lemma:con:conw_lyapunov_candidate} that states that \eqref{eq:con:conw_lyapunov_function} is a valid Lyapunov candidate.

\begin{lemma}\label{lemma:con:P_V_Schur_complement_positive_definite}
    % Given $M_\mathrm{w} \succ 0, K_\mathrm{w} \succ 0$, we can always choose a positive constant $\mu > 0$ in \eqref{eq:con:conw_lyapunov_function} such that $S_\mathrm{w}(\tilde{y}_\mathrm{w})$ is positive definite.
    Suppose $M_\mathrm{w} \succ 0, K_\mathrm{w} \succ 0$ and $0 < \mu < \frac{\sqrt{\lambda_\mathrm{m}(M_\mathrm{w}) \, \lambda_\mathrm{m}\left(K_\mathrm{w}\right)}}{\lVert M_\mathrm{w} \rVert} \coloneqq \mu_\mathrm{V}$. Then $S_{P_\mathrm{V}}$, as defined in \eqref{eq:con:P_V_Schur_complement}, is positive definite.
\end{lemma}
\begin{proof}
    The minimum Eigenvalue of $S_{P_\mathrm{V}}$ is bounded by
    \begin{equation}
    \begin{split}
        \lambda_\mathrm{m}(S_{P_\mathrm{V}}) \geq& \: \lambda_\mathrm{m}(M_\mathrm{w}) - \mu^2 \, \lVert M_\mathrm{w}^\mathrm{T} \, K_\mathrm{w} \, M_\mathrm{w} \rVert,\\
        \geq& \: \lambda_\mathrm{m}(M_\mathrm{w}) - \mu^2 \, \frac{\lVert M_\mathrm{w} \rVert^2}{\lambda_\mathrm{w} \left (K_\mathrm{w} \right )}\\
    \end{split}
    \end{equation}
    Based on the assumption $M_\mathrm{w} \succ 0, K_\mathrm{w} \succ 0$, we can state $\frac{\lVert M_\mathrm{w} \rVert}{\lambda_\mathrm{w}(K_\mathrm{w})} > 0$. Therefore, the critical case for the lower bound on $\lambda_\mathrm{m}(S_{P_\mathrm{V}})$ is $
    \mu = \frac{\sqrt{\lambda_\mathrm{m}(M_\mathrm{w}) \, \lambda_\mathrm{m}\left(K_\mathrm{w}\right)}}{\lVert M_\mathrm{w} \rVert} \coloneqq \mu_\mathrm{V}$. Hence,
    \begin{equation}
    \begin{split}
        \lambda_\mathrm{m}(S_{P_\mathrm{V}}) >& \: \lambda_\mathrm{m}(M_\mathrm{w}) - \frac{\lambda_\mathrm{m}(M_\mathrm{w}) \, \lambda_\mathrm{m}\left(K_\mathrm{w}\right)}{\lVert M_\mathrm{w} \rVert^2} \, \frac{\lVert M_\mathrm{w} \rVert^2}{\lambda_\mathrm{w}(K_\mathrm{w})} \: = \: 0
    \end{split}
    \end{equation}
    Consequently, the Eigenvalue sensitivity theorem~\cite{golub2013matrix} demands that $S_{P_\mathrm{V}} \succ 0$.
\end{proof}

\begin{lemma}\label{lemma:con:P_V_positive_definite}
    Let $M_\mathrm{w} \succ 0, K_\mathrm{w} \succ 0$, and $0 < \mu < \frac{\sqrt{\lambda_\mathrm{m}(M_\mathrm{w}) \, \lambda_\mathrm{m}\left(K_\mathrm{w}\right)}}{\lVert M_\mathrm{w} \rVert} \coloneqq \mu_\mathrm{V}$. Then, it follows that $P_\mathrm{V} \succ 0$ and $H_\mathrm{V}(\tilde{x}_\mathrm{w}) \succ 0 \: \: \forall \: \tilde{x}_\mathrm{w}\in \mathbb{R}^n$.
\end{lemma}
\begin{proof}
    By inspecting the expressions for $P_\mathrm{V} \succ 0$ and $H_\mathrm{V}(\tilde{x}_\mathrm{w}) \succ 0 \: \forall \tilde{x}_\mathrm{w}\in \mathbb{R}^n$ in Equations \eqref{eq:con:conw_lyapunov_function} and \eqref{eq:con:V_mu_hessian}, respectively, it can be easily seen that $H_\mathrm{V}(\tilde{x}_\mathrm{w}) \succeq P_\mathrm{V} \: \forall \tilde{x}_\mathrm{w}\in \mathbb{R}$. As Lemma~\ref{lemma:con:P_V_Schur_complement_positive_definite} states that the Schur complement of $P_\mathrm{V}$ is positive definite, it follows that $H_\mathrm{V}(\tilde{x}_\mathrm{w}) \succeq P_\mathrm{V} \succ 0$.
\end{proof}

\begin{lemma}\label{lemma:con:V_tanh_term_lower_term}
    % The scalar function $h_\mathrm{V,th}(r) = \int_{0}^r \tanh(\sigma + a) \, \mathrm{d}\sigma - \int_{0}^r \tanh(a) \, \mathrm{d}\sigma$ with $r \in \mathbb{R}$ is positive semi-definite: $h_\mathrm{V,th}(r) \geq 0 \: \forall r \in \mathbb{R}$. 
    Suppose $\bar{x}_{\mathrm{w}}, \tilde{x}_{\mathrm{w}}, b \in \mathbb{R}^n$ and $n \in \mathbb{N}^+$. Then,
    \begin{equation}
        h_{\mathrm{V,th}}(\tilde{x}_{\mathrm{w}}) = \sum_{i=1}^n \int_{0}^{\tilde{x}_{\mathrm{w},i}} \tanh(\bar{x}_{\mathrm{w},i}+\sigma+b_i) \, \mathrm{d} \sigma - \sum_{i=1}^n \int_{0}^{\tilde{x}_{\mathrm{w},i}} \tanh(\bar{x}_{\mathrm{w},i}+b_i) \, \mathrm{d} \sigma
    \end{equation}
    is a positive semi-definite function.
\end{lemma}
\begin{proof}
    % $h_\mathrm{V,th}(r)$ can be rewritten as
    % \begin{equation}
    %     h_\mathrm{V,th}(r) = \log(\cosh(r+a))-\log(\cosh(a))-\tanh(a) \, r.
    % \end{equation}
    Proving $h_{\mathrm{V},\mathrm{th}}(\tilde{x}_\mathrm{w}) \geq 0$ is equivalent to showing that the scalar function $\breve{h}_{\mathrm{V},\mathrm{th}}(r) = \int_{0}^r \tanh(\sigma + a) \, \mathrm{d}\sigma - \int_{0}^r \tanh(a) \, \mathrm{d}\sigma \geq 0 \: \: \forall \: r, a \in \mathbb{R}$, where we set $r = \tilde{x}_{\mathrm{w},i}$ and $a = \bar{x}_{\mathrm{w},i} + b_i$.
    
    We strive to find the critical points (i.e., minimas and maximas) $\bar{r}$ of $\breve{h}_\mathrm{V,th}(r)$ and, for this, analyze where the first derivative of $\breve{h}_\mathrm{V,th}(r)$ is zero
    \begin{equation}
        \frac{\partial \breve{h}_\mathrm{V,th}}{\partial r}(\bar{r}) = \tanh(\bar{r}+a) - \tanh(a) = 0,
    \end{equation}
    which is the case only for $\bar{r}=0$. Next, we compute the second derivative at $\bar{r}$ as
    \begin{equation}
        \frac{\partial \breve{h}_\mathrm{V,th}}{\partial r}(\bar{r}) = \mathrm{sech}^2(\bar{r}) = 1.
    \end{equation}
    Thus, $\breve{h}_\mathrm{V,th}(r)$ is convex and its global minimum at $\bar{r}=0$ takes the value $h_\mathrm{V,th}(0) = 0$.
    As a result, $h_{\mathrm{V},\mathrm{th}}(\tilde{x}_\mathrm{w})$ is also positive semi-definite.
\end{proof}

\begin{lemma}\label{lemma:con:conw_lyapunov_candidate}
    The scalar function $V_\mu(\tilde{y}_\mathrm{w})$ defined in \eqref{eq:con:conw_lyapunov_function} is continuously differentiable and verifies the condition $V_\mu(0) = 0$.
    Furthermore, let $M_\mathrm{w}, K_\mathrm{w} \succ 0$.
    Now, if we choose $0 < \mu < \frac{\sqrt{\lambda_\mathrm{m}(M_\mathrm{w}) \, \lambda_\mathrm{m}\left(K_\mathrm{w}\right)}}{\lVert M_\mathrm{w} \rVert} \coloneqq \mu_\mathrm{V}$, then $V_\mu(\tilde{y}_\mathrm{w}) > 0 \: \forall \: \tilde{y}_\mathrm{w} \in \mathbb{R}^{2n} \setminus \{0 \}$. Additionally, then $V_\mu(\tilde{y}_\mathrm{w})$ is radially unbounded as $\lVert \tilde{y}_\mathrm{w} \rVert \rightarrow \infty \Rightarrow V_\mu(\tilde{y}_\mathrm{w}) \rightarrow \infty$.
    % ($V_\mu(\tilde{y}_\mathrm{w})$ is globally positive definite)
    % ($V_\mu(\tilde{y}_\mathrm{w})$ is radially unbounded)
\end{lemma}
\begin{proof}

    \textbf{Step 1:} 
    % Proof that the Lyapunov candidate is continuously differentiable. 
    It can be easily seen that $V_\mu(\tilde{y}_\mathrm{w})$ in \eqref{eq:con:conw_lyapunov_function} is smooth and continuously differentiable.
    
    \textbf{Step 2:} Proof that $V_\mu(0) = 0$.
    \begin{equation}
        V_\mu(0) = 0 + \sum_{i=1}^n \int_{0}^{0} \tanh(\bar{y}_{\mathrm{w},i}+\sigma+b_i) \, \mathrm{d} \sigma - \sum_{i=1}^n \int_{0}^{0} \tanh(\bar{y}_{\mathrm{w},i}+b_i) \, \mathrm{d} \sigma = 0.
    \end{equation}
    
    \textbf{Step 3:} Proof that the Lyapunov candidate is positive definite; i.e., $V_\mu(\tilde{y}_\mathrm{w}) > 0 \: \: \forall \tilde{y}_\mathrm{w} \in \mathbb{R}^n \setminus \{0 \}$.
    % If we can show that $V_\mu(\tilde{y}_\mathrm{w})$ is convex with the global minimum at $\tilde{y}_\mathrm{w} = 0$, we will have demonstrated this characteristic to be true.
    
    As the gradient of the Lyapunov candidate, as defined in \eqref{eq:con:V_mu_gradient}, is zero for $\tilde{y}_\mathrm{w} = 0$: \begin{equation}
        \frac{\partial V_\mu}{\partial \tilde{y}_\mathrm{w}}(0) = \begin{bmatrix}
            \tanh(\bar{x}_\mathrm{w} + b)\\ 
            0^{n}
        \end{bmatrix} - \begin{bmatrix}
            \tanh(\bar{x}_\mathrm{w} + b)\\ 
            0^{n}
        \end{bmatrix} = 0,
    \end{equation}
    $\tilde{y}_\mathrm{w} = 0$ is a critical point of $V_\mu(\tilde{y}_\mathrm{w})$. 
    According to Lemma~\ref{lemma:con:P_V_positive_definite}, the Hessian in \eqref{eq:con:V_mu_hessian} is positive-definite~\cite{boyd2004convex}: $H_\mathrm{V}(\tilde{y}_\mathrm{w}) \succ 0 \: \: \forall \tilde{y}_\mathrm{w} \in \mathbb{R}^{2n}$.
    With that, \eqref{eq:con:conw_lyapunov_function} is convex and its global minimum is at $\tilde{y}_\mathrm{w} = 0$, where $V_\mu(0) = 0$. In summary, we state $V_\mu(\tilde{y}_\mathrm{w}) > 0 \: \: \forall \tilde{y}_\mathrm{w} \in \mathbb{R}^n \setminus \{0 \}$.

    \textbf{Step 4:} Proof that the Lyapunov candidate is radially unbounded: i.e., $\lVert \tilde{y}_\mathrm{w} \rVert \rightarrow \infty \Rightarrow V_\mu(\tilde{y}_\mathrm{w}) \rightarrow \infty$. Lemma~\ref{lemma:con:V_tanh_term_lower_term} is exploited for identifying a lower bound on $V_\mu(\tilde{y}_\mathrm{w})$:
    \begin{equation}
    \begin{split}
        V_\mu(\tilde{y}_\mathrm{w}) =& \: \frac{1}{2} \, \tilde{y}_\mathrm{w}^\mathrm{T} \, P_\mathrm{V} \, \tilde{y}_\mathrm{w} + \sum_{i=1}^n \int_{0}^{\tilde{x}_{\mathrm{w},i}} \tanh(\bar{x}_{\mathrm{w},i}+\sigma+b_i) \, \mathrm{d} \sigma - \sum_{i=1}^n \int_{0}^{\tilde{x}_{\mathrm{w},i}} \tanh(\bar{x}_{\mathrm{w},i}+b_i) \, \mathrm{d} \sigma,\\
        % =& \: \frac{1}{2} \, \tilde{y}_\mathrm{w}^\mathrm{T} \, P_\mathrm{V} \, \tilde{y}_\mathrm{w} + \sum_{i=1}^n h_\mathrm{V,th,i}(\tilde{x}_{\mathrm{w}, i}), \qquad \text{with } a_i = \bar{y}_{\mathrm{w},i} + b_i,\\
        \geq& \: \frac{1}{2} \, \tilde{y}_\mathrm{w}^\mathrm{T} \, P_\mathrm{V} \, \tilde{y}_\mathrm{w} \: \geq \: \frac{1}{2} \, \lambda_\mathrm{m}(P_\mathrm{V}) \, \lVert \tilde{y}_\mathrm{w} \rVert^2.
    \end{split}
    \end{equation}
    Lemma~\ref{lemma:con:P_V_positive_definite} tells us that $P_\mathrm{V} \succ 0$ and with that $\lambda_\mathrm{m}(P_\mathrm{V}) > 0$. Therefore, if $\lVert \tilde{y}_\mathrm{w} \rVert \rightarrow \infty$, it also follows that $V_\mu(\tilde{y}_\mathrm{w}) \rightarrow \infty$.
\end{proof}

\subsubsection{Proof of global asymptotic stability}

In the following, we will demonstrate how the strict Lyapunov candidate of \eqref{eq:con:conw_lyapunov_function} allows us to prove global asymptotic stability of the unforced system. We will first introduce two auxiliary Lemmas before stating Theorem~\ref{theorem:con:global_asymptotic_stability}.

\begin{lemma}\label{lemma:con:V_d_tanh_term_lower_bound}
    Suppose $\bar{x}_\mathrm{w}, \tilde{x}_\mathrm{w}, b \in \mathbb{R}^n$ and $n \in \mathbb{N}^+$. Then, the function $h_{\dot{\mathrm{V}},\mathrm{th}}(\tilde{x}_\mathrm{w})$ defined as
    \begin{equation}
        h_{\dot{\mathrm{V}},\mathrm{th}}(\tilde{x}_\mathrm{w}) = \left ( \tanh(\bar{x}_\mathrm{w} + \tilde{x}_\mathrm{w} + b) - \tanh(\bar{x}_\mathrm{w} + b) \right )^\mathrm{T} \, \tilde{x}_\mathrm{w},
    \end{equation}
    is positive semi-definite.
\end{lemma}
\begin{proof}
    Proving $h_{\dot{\mathrm{V}},\mathrm{th}}(\tilde{x}_\mathrm{w}) \geq 0$ is equivalent to proving that the scalar function $\breve{h}_{\dot{\mathrm{V}},\mathrm{th}}(r) = \left ( \tanh(r + a) - \tanh(a) \right ) \, r \geq 0 \: \: \forall \: r, a \in \mathbb{R}$, where we set $r = \tilde{x}_{\mathrm{w},i}$ and $a = \bar{x}_{\mathrm{w},i} + b_i$.
    Now, we expand the hyperbolic tangent:
    \begin{equation}
    \begin{split}
        \breve{h}_{\dot{\mathrm{V}},\mathrm{th}}(r) =& \: \left ( \tanh(r + a) - \tanh(a) \right ) \, r = \left ( \frac{e^{2(r+a)}-1}{e^{2(r+a)}+1} - \frac{e^{2a}-1}{e^{2a}+1} \right ) \, r,\\
        =& \: \frac{2 \, e^{2a} \, \left ( e^{2r} - 1 \right )}{e^{2r+4a}+e^{2r + 2a}+e^{2a}+1} r \, \geq 0,
    \end{split}
    \end{equation}
    as the denominator $e^{2r+4a}+e^{2r + 2a}+e^{2a}+1 > 0 \:\: \forall \: r \in \mathbb{R}$ and as $\mathrm{sign} \left (2 \, e^{2a} \, \left ( e^{2r} - 1 \right ) \right) = \mathrm{sign}(r)$. For example, $e^{2r} - 1 \geq 0 \: \forall \: r \geq 0$. Analog, $e^{2r} - 1 < 0 \: \forall \: r < 0$. 
\end{proof}

\begin{lemma}\label{lemma:con:P_V_d_positive_definite}
    Let $M_\mathrm{w} \succ 0$, $K_\mathrm{w} \succ 0$, and $D_\mathrm{w} \succ 0$. Also, let $\mu \in \mathbb{R}^+$ be chosen such that $0 < \mu < \frac{\lambda_\mathrm{m}(D_\mathrm{w})}{\lambda_\mathrm{m}(M_\mathrm{w}) + \frac{\lVert D_\mathrm{w} \rVert^2}{4 \lambda_\mathrm{m} (K_\mathrm{w})}} \coloneqq \mu_{\dot{\mathrm{V}}}$.
    Then, the matrix $P_{\dot{V}} = \begin{bmatrix}
        \mu K_\mathrm{w} & \frac{1}{2} \mu D_\mathrm{w}\\
        \frac{1}{2} \mu D_\mathrm{w}^\mathrm{T} & D_\mathrm{w} - \mu M_\mathrm{w}
    \end{bmatrix} \in \mathbb{R}^{n}$ is positive definite.
\end{lemma}
\begin{proof}
    The Schur complement of $P_{\dot{V}}$ is given by
    \begin{equation}
        S_{P_{\dot{V}}} = D_\mathrm{w} - \mu M_\mathrm{w} - \frac{1}{4} \, \mu D_\mathrm{w}^\mathrm{T} \, K_\mathrm{w}^{-1} \, D_\mathrm{w}.
    \end{equation}
    The lower bound on the smallest Eigenvalue of $S_{P_{\dot{V}}}$ can be identified as
    \begin{equation}
        \lambda_\mathrm{m} \left (S_{P_{\dot{V}}} \right ) \geq \lambda_\mathrm{m}(D_\mathrm{w}) - \mu \, \lambda_\mathrm{m}(M_\mathrm{w}) - \mu \frac{\lVert D_\mathrm{w} \rVert^2}{4 \, \lambda_\mathrm{m}(K_\mathrm{w})}.
    \end{equation}
    As $K_\mathrm{w}, D_\mathrm{w} \succ 0$, we know that $\frac{\lVert D_\mathrm{w} \rVert^2}{\lambda_\mathrm{m}(K_\mathrm{w})} > 0$. Therefore, the case $\mu = \frac{\lambda_\mathrm{m}(D_\mathrm{w})}{\lambda_\mathrm{m}(M_\mathrm{w}) + \frac{\lVert D_\mathrm{w} \rVert^2}{4 \lambda_\mathrm{m} (K_\mathrm{w})}} \coloneqq \mu_{\dot{\mathrm{V}}}$ determines the lower bound on $\lambda_\mathrm{m}(S_{P_{\dot{V}}})$:
    \begin{equation}
    \begin{split}
        \lambda_\mathrm{m} \left (S_{P_{\dot{V}}} \right ) > \lambda_\mathrm{m}(D_\mathrm{w}) - \mu_{\dot{\mathrm{V}}} \left ( \lambda_\mathrm{m}(M_\mathrm{w}) - \frac{\lVert D_\mathrm{w} \rVert^2}{4 \, \lambda_\mathrm{m}(K_\mathrm{w})} \right ) = 0.
    \end{split}
    \end{equation}
    We conclude, based on the Eigenvalue sensitivity theorem of symmetric matrices~\cite{golub2013matrix}, that $S_{P_{\dot{V}}} \succ 0$ and with that $P_{\dot{V}} \succ 0$~\cite{boyd2004convex}.
\end{proof}

\begin{theorem}\label{theorem:con:global_asymptotic_stability}
    Let $M_\mathrm{w}, K_\mathrm{w}$ and $D_\mathrm{w}$ be positive definite and suppose the system be unforced: $g(u(t)) = 0$.
    Then, $\tilde{y}_\mathrm{w} = 0$ is globally asymptotically stable for the system dynamics defined \eqref{eq:con:conw_residual_dynamics} such that $\dot{V}_\mu (\tilde{y}_\mathrm{w}) < 0, \quad \forall \: \tilde{y}_\mathrm{w} \in \mathbb{R}^{2n} \setminus \{0 \}$.
    
    % around the equilibrium $\tilde{y}_\mathrm{w} = 0$ if $g(u(t)) = 0$, $M_\mathrm{w} \succ 0$, $K_\mathrm{w} \succ 0$, and $D_\mathrm{w} \succ 0$.
\end{theorem}
\begin{proof}
    First, we show that $\tilde{y}_\mathrm{w} = 0$ is an equilibrium of \eqref{eq:con:conw_residual_dynamics}:
    \begin{equation}
        \tilde{f}_\mathrm{w}(0, 0) = \begin{bmatrix}
            0\\
            M_\mathrm{w}^{-1} \, \left (-K_\mathrm{w} \, \bar{x}_\mathrm{w} - \tanh(\bar{x}_\mathrm{w} + b) \right )
        \end{bmatrix}
        = \begin{bmatrix}
            0\\
            M_\mathrm{w}^{-1} \, \left (-K_\mathrm{w} \, \bar{x}_\mathrm{w} + K_\mathrm{w} \, \bar{x}_\mathrm{w} \right )
        \end{bmatrix}
        = \begin{bmatrix}
            0\\
            0
        \end{bmatrix}
    \end{equation}

    Lemma~\ref{lemma:con:conw_lyapunov_candidate} states that we can always choose $\mu$ such that \eqref{eq:con:conw_lyapunov_function} is strict Lyapunov function (e.g., Lipschitz continuous, zero-valued at $\tilde{y}_\mathrm{w} = 0$, positive definite, and radially unbounded)~\cite{khalil2002nonlinear, wu2022passive}. We now evaluate the time-derivative of $\dot{V}_\mu(\tilde{y}_\mathrm{w})$ in the case of an unforced system (i.e., $g(u(t)) = 0$):
    \begin{equation}\label{eq:con:lyapunov_function_time_derivative}
    \begin{split}
        \dot{V}_\mu(\tilde{y}_\mathrm{w}) =& \: \frac{\partial V_\mu}{\partial \tilde{y}_\mathrm{w}} \, \dot{\tilde{y}}_\mathrm{w} = \frac{\partial V_\mu}{\partial \tilde{y}_\mathrm{w}} \, \tilde{f}_\mathrm{w}(\tilde{y}_\mathrm{w}) 
        % = \frac{\partial \dot{V}_\mu}{\partial \tilde{x}_\mathrm{w}} \, \tilde{\dot{x}}_\mathrm{w} + \frac{\partial \dot{V}_\mu}{\partial \tilde{\dot{x}}_\mathrm{w}} \, \tilde{f}_{\mathrm{w}, \ddot{\tilde{x}}_\mathrm{w}}(\tilde{y}_\mathrm{w}) 
        = \tilde{y}_\mathrm{w}^\mathrm{T} \, P_\mathrm{V} \, \dot{\tilde{y}}_\mathrm{w} + \left (\tanh(\bar{x}_\mathrm{w} + \tilde{x}_\mathrm{w} + b) - \tanh(\bar{x}_\mathrm{w} + b)\right )^\mathrm{T} \, \dot{\tilde{x}}_\mathrm{w}\\
        =& \: -\tilde{y}_\mathrm{w}^\mathrm{T} \, \underbrace{\begin{bmatrix}
            \mu \, K_\mathrm{w} & \frac{1}{2} \, \mu \, D_\mathrm{w}\\
            \frac{1}{2} \, \mu \, D_\mathrm{w}^\mathrm{T} & D_\mathrm{w} - \mu \, M_\mathrm{w}
        \end{bmatrix}}_{P_{\dot{\mathrm{V}}}} \, \tilde{y}_\mathrm{w} - \mu \left ( \tanh(\bar{x}_\mathrm{w} + \tilde{x}_\mathrm{w} + b)^\mathrm{T} + \tilde{x}_\mathrm{w}^\mathrm{T} \, K_\mathrm{w} \right ) \tilde{x}_\mathrm{w},\\
        =& \: -\tilde{y}_\mathrm{w}^\mathrm{T} \, P_{\dot{\mathrm{V}}} \, \tilde{y}_\mathrm{w} - \mu \, \left ( \tanh(\bar{x}_\mathrm{w} + \tilde{x}_\mathrm{w} + b) - \tanh(\bar{x}_\mathrm{w} + b) \right )^\mathrm{T} \tilde{x}_\mathrm{w},\\
        \leq& \: -\tilde{y}_\mathrm{w}^\mathrm{T} \, P_{\dot{\mathrm{V}}} \, \tilde{y}_\mathrm{w} 
        \: \leq \: -\lambda_\mathrm{m}\left(P_{\dot{\mathrm{V}}} \right) \, \lVert \tilde{y}_\mathrm{w} \rVert_2^2,
    \end{split}
    \end{equation}
    where we exploited the force balance at equilibrium $K_\mathrm{w} \, \bar{x}_\mathrm{w} = -\tanh(\bar{x}_\mathrm{w} + b)$ and Lemma~\ref{lemma:con:V_d_tanh_term_lower_bound} for defining the upper bound on $\dot{V}_\mu(\tilde{y}_\mathrm{w})$. Lemma~\ref{lemma:con:P_V_d_positive_definite} states that $P_{\dot{\mathrm{V}}} \succ 0$ for $0 < \mu < \mu_{\dot{\mathrm{V}}}$. Similarly, Lemma~\ref{lemma:con:conw_lyapunov_candidate} requests that $0 < \mu < \mu_{\mathrm{V}}$. Indeed, both conditions can always be fulfilled by choosing $\mu \in \left (0, \min \{ \mu_{\mathrm{V}}, \mu_{\dot{\mathrm{V}}} \} \right )$. 
    With $P_{\dot{\mathrm{V}}} \succ 0 \Leftrightarrow \lambda_\mathrm{m}\left(P_{\dot{\mathrm{V}}} \right) > 0$~\cite{golub2013matrix}, we can state that $\dot{V}_\mu(\tilde{y}_\mathrm{w}) < 0 \: \forall \: \tilde{y}_\mathrm{w} \in \mathbb{R}^{2n} \setminus \{0 \}$ and conclude that the unforced system is globally asymptotically stable around $\tilde{y}_\mathrm{w} = 0$.
\end{proof}

\subsection{Global Input-to-State Stability (ISS) for the forced system} 
We now take the forcing $g(u)$ into account again and demonstrate that the system states remain proportionally bounded to the initial conditions and as a function of the supremum of the input forcing.
In the following, we will first state two auxiliary Lemmas and subsequently prove Theorem~\ref{theorem:con:iss} to provides the conditions for \gls{ISS}.

\begin{lemma}\label{lemma:con:V_tanh_term_upper_bound}
    Let $\bar{x}_{\mathrm{w}}, \tilde{x}_{\mathrm{w}}, b \in \mathbb{R}^n$.
    Then,
    \begin{equation}
        h_{\mathrm{V},th}(\tilde{x}_{\mathrm{w}}) = \sum_{i=1}^n \int_{0}^{\tilde{x}_{\mathrm{w},i}} \tanh(\bar{x}_{\mathrm{w},i}+\sigma+b_i) \, \mathrm{d} \sigma - \sum_{i=1}^n \int_{0}^{\tilde{x}_{\mathrm{w},i}} \tanh(\bar{x}_{\mathrm{w},i}+b_i) \, \mathrm{d} \sigma \leq 2 \, \left | \tilde{x}_{\mathrm{w}} \right |.
    \end{equation}
\end{lemma}
\begin{proof}
    Proving $h_{\mathrm{V},th}(\tilde{x}_{\mathrm{w}}) \leq 2 \, \left | \tilde{x}_{\mathrm{w}} \right |$ is equivalent to proving that the scalar function $\breve{h}_{\mathrm{V},\mathrm{th}}(r) = \int_0^{r} \tanh(\sigma + a) \: \mathrm{d}\sigma - \int_0^{r} \tanh(a) \: \mathrm{d}\sigma  \leq 2 \, |r| \: \: \forall \: r, a \in \mathbb{R}$, where we set $r = \tilde{x}_{\mathrm{w},i}$ and $a = \bar{x}_{\mathrm{w},i} + b_i$.
    We perform the integration contained in $\breve{h}_{\mathrm{V},\mathrm{th}}(r)$:
    \begin{equation}
    \begin{split}
        \breve{h}_{\mathrm{V},\mathrm{th}}(r) =& \: \int_0^{r} \tanh(\sigma + a) \: \mathrm{d}\sigma - \int_0^{r} \tanh(a) \: \mathrm{d}\sigma,\\
        \breve{h}_{\mathrm{V},\mathrm{th}}(r) =& \: \log(\cosh(r+a)) - \log(\cosh(a)) - \tanh(a) \, r.
    \end{split}
    \end{equation}
    Next, we demonstrate that the slope of $2 \, |r|$ is always larger than the magnitude of the slope of $\breve{h}_{\mathrm{V},\mathrm{th}}(r)$:
    \begin{equation}
    \begin{split}
        \left | \frac{\partial \breve{h}_{\mathrm{V},\mathrm{th}}}{\partial r} \right | =& \: \left | \tanh(r+a) - \tanh(a) \right | < 2 = \frac{\partial}{\partial r} \, (2 \, |r|).\\
    \end{split}
    \end{equation}
    Additionally, $\breve{h}_{\mathrm{V},\mathrm{th}}(0) = 2 \, |0| = 0$. We conclude that $\breve{h}_{\mathrm{V},\mathrm{th}}(r) \leq 2 |r| \: \forall \: r \in \mathbb{R}$ and with that, $h_{\mathrm{V},th}(\tilde{x}_{\mathrm{w}}) \leq 2 \, \left | \tilde{x}_{\mathrm{w}} \right | \: \forall \: \tilde{x}_{\mathrm{w}} \in \mathbb{R}^n$.
\end{proof}

\begin{lemma}\label{lemma:con:iss_lyapunov_bounds}
    % Assuming that $M_\mathrm{w} \succ 0$ and $K_\mathrm{w} \succ 0$, we can state that the 
    Let $M_\mathrm{w} \succ 0$ and $K_\mathrm{w} \succ 0$. Then, \eqref{eq:con:conw_lyapunov_function} is bounded by the two scalar, class $\mathcal{K}_\infty$ functions $\alpha_1(r)=\frac{1}{2} \, \lambda_\mathrm{m}(P_\mathrm{V})  \, r^2$ and $\alpha_2(r)=\frac{1}{2} \, \lambda_\mathrm{M}(P_\mathrm{V}) \, r^2 + 2 \, \sqrt{n} \, r$: $ \alpha_1(\lVert \tilde{y}_\mathrm{w} \rVert_2^2) \leq V_\mu(\tilde{y}_\mathrm{w}) \leq \alpha_2(\lVert \tilde{y}_\mathrm{w} \rVert)_2^2$.
\end{lemma}
\begin{proof}
    With Lemma~\ref{lemma:con:conw_lyapunov_candidate}, we already showed that  $V_\mu(\tilde{y}_\mathrm{w})$ is a Lyapunov candidate. Now, we additionally also verify the conditions for ISS-Lyapunov candidates~\cite{khalil2002nonlinear}.

    \textbf{Step 1:} Establishing bounds on $V_\mu(\tilde{y}_\mathrm{w})$.\\
    We first identify the lower bound of $V_\mu(\tilde{y}_\mathrm{w})$ by leveraging Lemma~\ref{lemma:con:V_tanh_term_lower_term}:
    \begin{equation}
    \begin{split}
        V_\mu(\tilde{y}_\mathrm{w}) =& \: \frac{1}{2} \, \tilde{y}_\mathrm{w}^\mathrm{T} \, P_\mathrm{V} \, \tilde{y}_\mathrm{w} + \sum_{i=1}^n \int_{0}^{\tilde{x}_{\mathrm{w},i}} \tanh(\bar{x}_{\mathrm{w},i}+\sigma+b_i) \, \mathrm{d} \sigma - \sum_{i=1}^n \int_{0}^{\tilde{x}_{\mathrm{w},i}} \tanh(\bar{x}_{\mathrm{w},i}+b_i) \, \mathrm{d} \sigma,\\
        =& \: \frac{1}{2} \, \tilde{y}_\mathrm{w}^\mathrm{T} \, P_\mathrm{V} \, \tilde{y}_\mathrm{w} + h_{\mathrm{V},\mathrm{th}}(\tilde{x}_\mathrm{w}),\\
        \geq& \: \frac{1}{2} \, \tilde{y}_\mathrm{w}^\mathrm{T} \, P_\mathrm{V} \, \tilde{y}_\mathrm{w}
        \: \geq \: \frac{1}{2} \, \lambda_\mathrm{m}(P_\mathrm{V}) \, \lVert \tilde{y}_\mathrm{w} \rVert_2^2
        \: = \: \alpha_1(\lVert \tilde{y}_\mathrm{w} \rVert_2).
    \end{split}
    \end{equation}
    Similarly, we derive an upper bound for $V_\mu(\tilde{y}_\mathrm{w})$ exploiting Lemma~\ref{lemma:con:V_tanh_term_upper_bound}.
    \begin{equation}
    \begin{split}
        V_\mu(\tilde{y}_\mathrm{w}) =& \frac{1}{2} \, \tilde{y}_\mathrm{w}^\mathrm{T} \, P_\mathrm{V} \, \tilde{y}_\mathrm{w} + \sum_{i=1}^n \int_{0}^{\tilde{x}_{\mathrm{w},i}} \tanh(\bar{x}_{\mathrm{w},i}+\sigma+b_i) \, \mathrm{d} \sigma - \sum_{i=1}^n \int_{0}^{\tilde{x}_{\mathrm{w},i}} \tanh(\bar{x}_{\mathrm{w},i}+b_i) \, \mathrm{d} \sigma\\
        \leq& \: \frac{1}{2} \, \lambda_\mathrm{M}(P_\mathrm{V}) \, \lVert \tilde{y}_\mathrm{w} \rVert_2^2 + 2 \, \lVert\tilde{x}_\mathrm{w}\rVert_1
        \: \leq \: \frac{1}{2} \, \lambda_\mathrm{M}(P_\mathrm{V}) \, \lVert \tilde{y}_\mathrm{w} \rVert_2^2 + 2 \, \sqrt{n} \, \lVert\tilde{x}_\mathrm{w}\rVert_2\\
        \leq& \: \frac{1}{2} \, \lambda_\mathrm{M}(P_\mathrm{V}) \, \lVert \tilde{y}_\mathrm{w} \rVert_2^2 + 2 \, \sqrt{n} \, \lVert\tilde{y}_\mathrm{w}\rVert_2 = \alpha_2(\lVert\tilde{y}_\mathrm{w}\rVert_2).
    \end{split}
    \end{equation}

    \textbf{Step 2:} Proof that $\alpha_1(r), \alpha_2(r)$ belong to class $\mathcal{K}_\infty$.\\
    According to Lemma~\ref{lemma:con:P_V_positive_definite}, $P_\mathrm{V} \succ 0$ and with that $\lambda_\mathrm{m}(P_\mathrm{V}) > 0$.
    First, we analyze the behavior of $\alpha_1(r)$: as it is strictly increasing and $\alpha_1(0) = 0$, it belongs to class $\mathcal{K}$. Furthermore, we can evaluate $\lim_{r \rightarrow \infty} \alpha_1(r) = \infty$. Therefore, $\alpha_1(r) \in \mathcal{K}_\infty$~\cite{khalil2002nonlinear}.
    $\alpha_2(r)$ is also strictly increasing for $r \in [0, \infty)$, $\alpha_2(0) = 0$, and it is radially unbounded as $\lim_{r \rightarrow \infty} \alpha_2(r) = \infty$. For that reason, $\alpha_2(r) \in \mathcal{K}_\infty$ as well.
\end{proof}

\begin{theorem}\label{theorem:con:iss}
    Suppose $M_\mathrm{w}, K_\mathrm{w}, D_\mathrm{w} \succ 0$, $0 < \theta < 1$, and that we choose $0 < \mu < \min \{ \mu_{\mathrm{V}}, \mu_{\dot{\mathrm{V}}} \}\}$. Then,
    \eqref{eq:con:conw_residual_dynamics} is globally Input-to-State Stable (ISS)
    such that the solution $\tilde{y}_\mathrm{w}(t)$ verifies
    \begin{equation}\label{eq:con:input_to_state_stability}
        \lVert \tilde{y}_\mathrm{w} \rVert_2 \leq \beta \left (\lVert \tilde{y}_\mathrm{w}(t_0) \rVert_2, t-t_0 \right ) + \gamma \left ( \sup_{t_0 \leq t' \leq t} \lVert g(u(t')) \rVert_2 \right ), \qquad \forall \: t \geq t_0
    \end{equation}
    where $\beta(r, t) \in \mathcal{KL}$, $\gamma(r) = \sqrt{\frac{(1+\mu^2) \, \lambda_\mathrm{M}(P_\mathrm{V}) \, r^2 + 4 \, \theta \, \sqrt{n} \, \sqrt{1+\mu^2} \, \lambda_\mathrm{m}(P_{\dot{\mathrm{V}}}) \, r}{\theta^2 \, \lambda_\mathrm{m}(P_\mathrm{V}) \, \lambda_\mathrm{m}^2(P_{\dot{\mathrm{V}}})}} \in \mathcal{K}$.
\end{theorem}
\begin{proof}

    % \textbf{Step 1:} Proof that $\gamma(r) \in \mathcal{K}$.
    % From Theorem~\ref{theorem:con:global_asymptotic_stability} and the associated proof, we know that $\lambda_\mathrm{m}(P_\mathrm{V}), \lambda_\mathrm{M}(P_\mathrm{V}), \lambda_\mathrm{m}(P_{\dot{\mathrm{V}})} > 0$ for suitable choices $0 < \mu < \min \{ \mu_\mathrm{V}, \mu_{\dot{\mathrm{V}}}$. Furthermore, we define $0 < \theta < 0$. 
    \textbf{Step 1:} Bounds on ISS-Lyapunov candidate.\\
    Lemma~\ref{lemma:con:iss_lyapunov_bounds} provides the $\mathcal{K_\infty}$ functions 
    $\alpha_1(r)=\frac{1}{2} \, \lambda_\mathrm{m}(P_\mathrm{V})  \, r^2$ and $\alpha_2(r)=\frac{1}{2} \, \lambda_\mathrm{M}(P_\mathrm{V}) \, r^2 + 2 \, \sqrt{n} \, r$ such that $ \alpha_1(\lVert \tilde{y}_\mathrm{w} \rVert_2^2) \leq V_\mu(\tilde{y}_\mathrm{w}) \leq \alpha_2(\lVert \tilde{y}_\mathrm{w} \rVert)_2^2$.
    % that represent the lower and upper bound on $V_\mu(\tilde{y}_\mathrm{w})$, respectively. 
    
    \textbf{Step 2:} Minimum energy dissipation.\\
    % For ISS, it is required that $\dot{V}_\mu(\tilde{y}_\mathrm{w}, u(t)) \leq -\alpha_3(\tilde{y}_\mathrm{w}), \quad \forall \: \lVert \tilde{y}_\mathrm{w} \rVert_2 \geq \rho$.\
    Let $0 < \mu < \min \{ \mu_{\mathrm{V}}, \mu_{\dot{\mathrm{V}}} \}\}$ as in the proof of Theorem~\ref{theorem:con:global_asymptotic_stability}.
    We compute the input-dependent time-derivative of the ISS Lyapunov candidate. We do not repeat the derivations already made as part of \eqref{eq:con:lyapunov_function_time_derivative} (e.g., exploiting Lemmas~\ref{lemma:con:V_d_tanh_term_lower_bound} and \ref{lemma:con:P_V_d_positive_definite}).
    \begin{equation}
    \begin{split}
        \dot{V}_\mu(\tilde{y}_\mathrm{w}, u(t)) =& \: % \frac{\partial V_\mu}{\partial \tilde{y}_\mathrm{w}} \, \tilde{f}_\mathrm{w}(\tilde{y}_\mathrm{w}, u(t)) = 
        -\tilde{y}_\mathrm{w}^\mathrm{T} \, P_{\dot{\mathrm{V}}} \, \tilde{y}_\mathrm{w} - \mu \, \left ( \tanh(\bar{x}_\mathrm{w} + \tilde{x}_\mathrm{w} + b) - \tanh(\bar{x}_\mathrm{w} + b) \right )^\mathrm{T} \tilde{x}_\mathrm{w} + \tilde{y}_\mathrm{w}^\mathrm{T} \begin{bmatrix}
            \mu \, g(u(t))\\
            g(u(t))
        \end{bmatrix},\\
        \leq& \: -\lambda_\mathrm{m}\left(P_{\dot{\mathrm{V}}} \right) \, \lVert \tilde{y}_\mathrm{w} \rVert_2^2 + \left \lVert \tilde{y}_\mathrm{w}^\mathrm{T}\begin{bmatrix}\mu \, g(u(t))\\ g(u(t)) \end{bmatrix} \right \rVert_1,\\
        \leq& \: -\lambda_\mathrm{m}\left(P_{\dot{\mathrm{V}}} \right) \, \lVert \tilde{y}_\mathrm{w} \rVert_2^2 +  \lVert \tilde{y}_\mathrm{w} \rVert_2 \left \lVert \begin{bmatrix}\mu \, g(u(t))\\ g(u(t)) \end{bmatrix} \right \rVert_2,\\
        \leq& \: -\lambda_\mathrm{m}\left(P_{\dot{\mathrm{V}}} \right) \, \lVert \tilde{y}_\mathrm{w} \rVert_2^2 + \sqrt{1+\mu^2} \, \lVert \tilde{y}_\mathrm{w} \rVert_2 \lVert g(u(t)) \rVert_2,
    \end{split}
    \end{equation}
    where we leveraged Hölder's inequality. We choose $\theta$ such that $0 < \theta < 1$.
    As a consequence,
    \begin{equation}
        \dot{V}_\mu(\tilde{y}_\mathrm{w}, u(t)) \leq -(1-\theta) \, \lambda_\mathrm{m}\left(P_{\dot{\mathrm{V}}} \right) \, \lVert \tilde{y}_\mathrm{w} \rVert_2^2, % = \alpha_3(\lVert \tilde{y}_\mathrm{w} \rVert_2^2),
        \qquad
        \forall \: \lVert \tilde{y}_\mathrm{w} \rVert_2 \: \geq \: \frac{\sqrt{1+\mu^2}}{\theta \, \lambda_\mathrm{m}\left(P_{\dot{\mathrm{V}}} \right)} \, \lVert g(u(t)) \rVert_2  > 0.
    \end{equation}
    % Thus, if $\lVert \tilde{y}_\mathrm{w} \rVert_2 \geq $
    We define
    \begin{equation}
        \alpha_3(r) = (1-\theta) \, \lambda_\mathrm{m}\left(P_{\dot{\mathrm{V}}} \right) \, r^2,
        \qquad
        \text{and }
        \rho(r) = \frac{\sqrt{1+\mu^2}}{\theta \, \lambda_\mathrm{m}\left(P_{\dot{\mathrm{V}}} \right)} \, r.
    \end{equation}
    Lemma~\ref{lemma:con:P_V_d_positive_definite} shows that $\lambda_\mathrm{m}\left(P_{\dot{\mathrm{V}}} \right) > 0$. Therefore, $\alpha_3(r)$ is a continuous positive function. Furthermore, as $\mu > 0$, $\rho(r)$ is a strictly increasing for $r \in [0, \infty)$. Additionally with $\rho(0) = 0$ verified, it can be stated that $\rho(r)$ belongs to class $\mathcal{K}$~\cite{khalil2002nonlinear}.
    We conclude that
    \begin{equation}
        \dot{V}_\mu(\tilde{y}_\mathrm{w}, u(t)) \leq - \alpha_3 \left (\lVert \tilde{y}_\mathrm{w} \rVert_2^2 \right ),
        \qquad
        \forall \: \lVert \tilde{y}_\mathrm{w} \rVert_2 \: \geq \: \rho \left ( \lVert g(u(t)) \rVert_2 \right )  > 0.
    \end{equation}

    \textbf{Step 3:} Conclusions.

    As a result of Steps 1 and 2, the system is input-to-state stable, and with that, the solution $\tilde{y}_\mathrm{t}$ satisfies~\cite{khalil2002nonlinear}
    \begin{equation}
        \lVert \tilde{y}_\mathrm{w} \rVert_2 \leq \beta \left (\lVert \tilde{y}_\mathrm{w}(t_0) \rVert_2, t-t_0 \right ) + \gamma \left ( \sup_{t_0 \leq t' \leq t} \lVert g(u(t')) \rVert_2 \right ),
    \end{equation}
    with
    \begin{equation}
        \gamma(r) = \alpha_1^{-1} \circ \alpha_2 \circ \rho(r) = \sqrt{\frac{(1+\mu^2) \, \lambda_\mathrm{M}(P_\mathrm{V}) \, r^2 + 4 \, \theta \, \sqrt{n} \, \sqrt{1+\mu^2} \, \lambda_\mathrm{m}(P_{\dot{\mathrm{V}}}) \, r}{\theta^2 \, \lambda_\mathrm{m}(P_\mathrm{V}) \, \lambda_\mathrm{m}^2(P_{\dot{\mathrm{V}}})}}.
    \end{equation}
    Indeed, based on Theorem~\ref{theorem:con:global_asymptotic_stability} and the associated proof, we can easily verify that $\gamma(r)$ is strictly increasing for $r \in [0, \infty)$ and that $\gamma(0) = 0$. As a consequence, $\gamma(r) \in \mathcal{K}$~\cite{khalil2002nonlinear}.
\end{proof}
\section{An Approximate Closed-Form Solution to the CON Dynamics}

\subsection{Approach}\label{sub:con:cfa:approach}
To predict future system states, we need to integrate the ODE in Eq.~\eqref{eq:con:con_dynamics}, with the solution given by $y(t_{k+1}) = y_{t_k} + \int_{t_k}^{t_{k+1}} f(y(t'), u(t')) \, \mathrm{d}t'$.
Unfortunately, a closed-form solution for the nonlinear dynamics $f(y, u)$ does not (yet) exist. Therefore, we traditionally need to revert to (high-order) numerical \gls{ODE} solvers that are computationally very expensive and introduce additional memory overhead~\citep{kidger2021neural}. This considerably increases the training time of models involving such continuous-time dynamics.
While the computational time can be reduced by increasing the (minimum) time step of the integrator, this comes at the expense of an integration error, and we lose (part of) the theoretical guarantees and practical characteristics that the nominal \gls{ODE} provides.
In this work, we take an alternative approach by splitting the problem into (i) decoupled linear dynamics that can be cheaply and precisely integrated using a closed-form solution and (ii) the residual, coupled nonlinear dynamics, which we integrate numerically at a slower time scale:

\begin{equation}\label{eq:con:con_split_linear_terms}
\begin{split}
    \ddot{x}(t) = \underbrace{F -\kappa x(t) - d \, \dot{x}(t)}_{f_{\ddot{x}, \mathrm{ld}}(y): \, \text{decoupled, linear dynamics}} + \underbrace{g(u(t)-(K-\kappa) x(t) - (D-d) \, \dot{x}(t) - \tanh(W x(t) + b)}_{f_{\ddot{x}, \mathrm{nld}}(y, u): \: \text{coupled, nonlinear dynamics}}
\end{split}
\end{equation}
where $\kappa = \mathrm{diag}(K_{11} \dots K_{nn})$, $d = \mathrm(D_{11} \dots D_{nn})$ are the diagonal components of the stiffness and damping matrices, respectively, and $F \in \mathbb{R}^{n}$ is a constant, external forcing term on the oscillators.

For a short-time-interval $\delta t$, we now approximate \eqref{eq:con:con_split_linear_terms} as
\begin{equation}\label{eq:con:con_approx}
    \ddot{x}(t_k+\delta t) \approx f_{\ddot{x}, \mathrm{ld}}(y(t_k+\delta t), F(t_k)) 
    \qquad \text{with  }
    F(t_k) =  -f_{\ddot{x}, \mathrm{nld}}(y(t_k), u(t_k)).
    % f_\mathrm{ext}(t_k) = -g(u(t_k))-(K-\kappa) x(t_k) - (D-d) \, \dot{x}(t_k) - \tanh(W \, x(t_k) + b)
\end{equation}
For a scalar \nth{2}-order, linear \gls{ODE} of form $\dot{y}_i = f_{\mathrm{ld},i}(x(t'), F(t_k))$, a well-known, closed-form solution~\citep{Pas2023damped} exists.
We exploit this characteristic by formulating the approximate solution as 
\begin{equation}\label{eq:con:cfa_con}
    y(t_k + \delta t) \approx \, f_\mathrm{CFA-CON}(y(t_k), u(t_k)) =  y(t_k) + \int_{t_k}^{t_k + \delta t} f_{\mathrm{ld}}(y(t'), F(t_k)) \, \mathrm{d}t'
\end{equation}
and denote $f_\mathrm{CFA-CON}: \mathbb{R}^n \times \mathbb{R}^m \to \mathbb{R}^n$ as the \gls{CFA-CON} model.
The implicit assumption behind \eqref{eq:con:con_approx} is that $f_{\ddot{x}, \mathrm{ld}}(y) \succ f_{\ddot{x}, \mathrm{nld}}(y, u)$ (i.e., the linear, decoupled dynamics dominate the nonlinear, coupled, time-varying dynamics).

% \begin{equation}
%     y_i (t_{k+1}) 
%     = \begin{bmatrix}
%         x_i(t_{k+1})\\
%         \dot{x}_i(t_{k+1})
%     \end{bmatrix} 
%     = \begin{bmatrix}
%         \left ( c_{1,i} \, \cos(\beta_i \, \delta t) +  c_{2,i} \sin(\beta_i \delta t) \right ) \, e^{-\alpha_i \, \delta t} + \frac{f_{\mathrm{ext},i}}{\kappa_i}\\
%         -\left ( (c_{1,i} \alpha_i - c_{2,i} \beta_i) \, \cos(\beta_i \delta t) +  (c_{1,i} \beta_i + c_{2,i} \alpha_i) \, \sin(\beta_i \delta t) \right ) \, e^{-\alpha_i \, \delta t}
%     \end{bmatrix},
% \end{equation}
% exists, where $\delta t = t_{k+1} - t_k$, $c_{1,i}, c_{2,i} \in \mathbb{R}$ are scalar integration constants dependent on $y_i(t_k), F_{i}(t_k)$, and $\alpha_i, \beta_i \in \mathbb{R}^+$ are real positive constants. Please refer to Appendix~\ref{apx:sec:con_cfa} for more details on the determination of the constants and the implementation.


\subsection{Closed-Form Solution to a Forced Harmonic Oscillator}
As introduced in \eqref{eq:con:harmonic_oscillator}, we consider the linear dynamics of a 1D forced harmonic oscillator with state $y_{i} = \begin{bmatrix}
    x_i & \dot{x}_i
\end{bmatrix} \in \mathbb{R}^2$
\begin{equation}
    \dot{y}_i = \begin{bmatrix}
        \frac{\mathrm{d} x_i}{\mathrm{d} t}\\
        \frac{\mathrm{d} \dot{x}_i}{\mathrm{d} t}
    \end{bmatrix} = f_{\mathrm{ld},i}(y_{i}, F_{i}) = \begin{bmatrix}
        \dot{x}_i\\
        F_i(t) - \kappa_i \, x_i(t) - d_i \, \dot{x}_i(t)
    \end{bmatrix},
\end{equation}
where $F_i(t) \in \mathbb{R}$ is the externally applied force acting on the oscillator.

The characteristic equation for the unforced dynamics (i.e., $F_i(t) = 0)$ can be stated as~\citep{Pas2023damped}
\begin{equation}
    \lambda^2 + 2 \, \zeta_i \, \omega_{\mathrm{n},i}  \, \lambda + \omega_{\mathrm{n},i}^2 = 0,
    \qquad
    \text{with the solutions} \:
    \lambda_{1,2} = -\zeta_{i} \,  \omega_{\mathrm{n},i} \pm \omega_{\mathrm{n},i} \, \sqrt{\zeta_{i}^2 - 1},
\end{equation}
 where $\omega_{\mathrm{n},i} = \sqrt{\kappa_i}$ and $\zeta_i = \frac{d_i}{2 \, \sqrt{\kappa_i}}$ are the natural frequency and the damping factor of the $i$th homogeneous oscillator, respectively.
 This harmonic oscillator exhibits three regimes: underdamped ($\zeta_i < 1$), critically damped ($\zeta_i = 1$), and overdamped regime ($\zeta_i > 1$).

 We approximate the forcing using the Heavyside function $H(t)$: $F_i(t) = F_{i}(t_k) \, H(t)$, where $F_{i}(t_k)$ is the constant external forcing as computed by \eqref{eq:con:con_approx}. The solution for $\zeta_i \neq 1$ is given by~\citep{Pas2023damped}
\begin{equation}\label{eq:con:harmonic_oscillator_closed_form}
    y_i (t_{k+1}) 
    = \begin{bmatrix}
        x_i(t_{k+1})\\
        \dot{x}_i(t_{k+1})
    \end{bmatrix} 
    = \begin{bmatrix}
        \left ( c_{1,i} \, \cos(\beta_i \, \delta t) +  c_{2,i} \sin(\beta_i \, \delta t) \right ) \, e^{-\alpha_i \, \delta t} + \frac{F_{i}}{\kappa_i}\\
        -\left ( (c_{1,i} \alpha_i - c_{2,i} \beta_i) \, \cos(\beta_i \, \delta t) +  (c_{1,i} \beta_i + c_{2,i} \alpha_i) \, \sin(\beta_i \, \delta t) \right ) \, e^{-\alpha_i \, \delta t}
    \end{bmatrix},
\end{equation}
where $\delta t = t_{k+1} - t_k$, $\alpha_i = \zeta_i \, \omega_{\mathrm{n},i}$, and $\beta_i = \omega_{\mathrm{n},i} \, \sqrt{1-\zeta_i^2}$.
After enforcing the initial conditions $x_i(t_k), x_i(t_k)$, the integration constants
\begin{equation}
    c_{1,i} = x_i(t_k) - \frac{F_{i}(t_k)}{\kappa_i},
    \qquad
    c_{2,i} = - 2 \, j \, \frac{\dot{x}_i(t_k) + \alpha_i \, \left (x_i(t_k) - \frac{F_{i}}{k_i} \right )}{\Delta \lambda_i},
\end{equation}
can be identified with $\Delta \lambda_i = \lambda_{i,2} - \lambda_{i,1} = - 2 \, \beta_i \, j$, where $j$ is the imaginary value.
While we could derive the solution for the critically damped case $d_i < 2 \sqrt{\kappa_i}$ separately, we instead approximate it in our network dynamics with \eqref{eq:con:harmonic_oscillator_closed_form} by setting
$\Delta \lambda_i \cong \begin{cases}
    \mathrm{sign}(-2 \, \beta_i \, j) \, \epsilon   & |2 \, \beta_i \, j| < \epsilon \\
    - 2 \, \beta_i \, j & |2 \, \beta_i \, j| \geq \epsilon
\end{cases}$, where $\epsilon \in \mathbb{R}^+ \ll 1$ is a small, positive value.

\subsection{Algorithmic Implementation}
We can now leverage the closed-form solution to the evolution of a single, decoupled damped harmonic oscillator of \eqref{eq:con:harmonic_oscillator_closed_form} to solve the integral in \eqref{eq:con:cfa_con}
\begin{equation}\label{eq:con:cfa:full_con_cfa_dynamics}
\begin{split}
    y(t_{k+1}) \approx& \: f_\mathrm{CFA-CON}(y(t_k), u(t_k)),\\
    \begin{bmatrix}
        x(t_{k+1})\\ \dot{x}(t_{k+1})
    \end{bmatrix} \approx& \: \begin{bmatrix}
        \left ( c_{1} \odot \cos(\beta \, \delta t) +  c_{2} \odot \sin(\beta \, \delta t) \right ) \odot e^{-\alpha \, \delta t} + \frac{F(t_k)}{\kappa}\\
        -\left ( (c_{1} \odot \alpha - c_{2} \odot \beta) \, \cos(\beta \, \delta t) +  (c_{1} \odot \beta + c_{2} \odot \alpha) \, \sin(\beta \, \delta t) \right ) \odot e^{-\alpha \, \delta t}
    \end{bmatrix},
\end{split}
\end{equation}
with
\begin{equation}
\begin{split}
    \kappa = \mathrm{diag}(K_{11} \dots K_{nn}), \quad d = \mathrm(D_{11} \dots D_{nn}), \quad \omega_\mathrm{n} = \sqrt{\kappa}, \quad \zeta = \frac{d}{2 \sqrt{\kappa}},\\
    F(t_k) = g(u(t_k)) - (K - \kappa) \, x(t_k) - (D-d) \, \dot{x}(t_k) - \tanh \left (Wx(t_k)+b \right ),\\
    \delta t = t_{k+1} - t_k, \qquad \alpha = \zeta \odot \omega_\mathrm{n}, \qquad \beta = \omega_\mathrm{n} \: \sqrt{1-\zeta^2},\\
    c_{1} = x(t_k) - \frac{F(t_k)}{\kappa}, \qquad
    c_{2} = \frac{1}{\beta} \, \left ( \dot{x}(t_k) + \alpha \odot \left (x(t_k) - \frac{F(t_k)}{\kappa} \right ) \right ).
\end{split}
\end{equation}
We summarize the approach of integrating/rolling out the \gls{CFA-CON} dynamics in Algorithm~\ref{alg:con:con_cfa}.

\begin{algorithm}
\caption{Rollout of \gls{CFA-CON}.}\label{alg:con:con_cfa}
\hspace*{\algorithmicindent} \textbf{Inputs:} initial state $y(t_0)$, input sequence $\{u(t_0), \dots u(t_k), \dots u(t_N) \}$\\
\hspace*{\algorithmicindent} \textbf{Outputs:} state sequence $\{y(t_0), \dots y(t_k), \dots y(t_N) \}$
\begin{algorithmic}[1]
    \State $k \gets 0$
    % \For{\texttt{<some condition>}}
    %     \State \texttt{<do stuff>}
    % \EndFor
    \While{$k \leq N$}
        \State $(x(t_k), \dot{x}(t_k)) \gets y(t_k)$
        \State $F(t_k) \gets g(u(t_k)) - (K - \kappa) \, x(t_k) - (D-d) \, \dot{x}(t_k) - \tanh \left (Wx(t_k)+b \right )$
        \State $\omega_\mathrm{n}, \: \zeta \gets \sqrt{\kappa}, \: \frac{d}{2 \sqrt{\kappa}}$ \Comment{Compute the characteristics of the decoup. harm. oscillators.}
        \State $\alpha, \: \beta  \gets \zeta \odot \omega_\mathrm{n}, \: \omega_\mathrm{n} \: \sqrt{1-\zeta^2}$
        \State $c_{1} \gets x(t_k) - \frac{F(t_k)}{\kappa}$  \Comment{Compute integration constants using initial conditions.}
        \State $c_{2} \gets \frac{1}{\beta} \, \left ( \dot{x}(t_k) + \alpha \odot \left (x(t_k) - \frac{F(t_k)}{\kappa} \right ) \right )$
        \State $\delta t = t_{k+1} - t_k$  \Comment{Set time step.}\\
        \Comment{Update state with approximated closed-form solution.}
        \State $x(t_{k+1}) \gets  \left ( c_{1} \odot \cos(\beta \, \delta t) +  c_{2} \odot \sin(\beta \, \delta t) \right ) \odot e^{-\alpha \, \delta t} + \frac{F}{\kappa}$
        \State $\dot{x}(t_{k+1}) \gets -\left ( (c_{1} \odot \alpha - c_{2} \odot \beta) \, \cos(\beta \, \delta t) +  (c_{1} \odot \beta + c_{2} \odot \alpha) \, \sin(\beta \, \delta t) \right ) \odot e^{-\alpha \, \delta t}$
        \State $k \gets k+1$  \Comment{Update time index.}
    \EndWhile
\end{algorithmic}
\end{algorithm}

\subsection{Approximation Bounds for CFA-CON}
Lemma~\ref{lemma:con:con_cfa_bounds_lemma} demonstrates how, for the particular case of no external input and linearly decoupled oscillators (which we are always free to choose), we can establish bounds on the approximation error when using the closed-form solution instead of the ground-truth coupled oscillator dynamics.

\begin{lemma}\label{lemma:con:con_cfa_bounds_lemma}
    Suppose that the network is unforced with $g(u(t)) = 0$ and that $K = \mathrm{diag}(\kappa_1, \dots, \kappa_n)$, $D = \mathrm{diag}(d_1, \dots, d_n)$ such that the oscillators are not linearly coupled. Then, given any $t \geq 0$, and the initial state $y(0) \in \mathbb{R}^{2n}$, the error between the continuous dynamics $\ddot{x}(t)$ of \eqref{eq:con:con_dynamics} and the approximated dynamics $\hat{\dot{x}}(t)$ in \eqref{eq:con:con_approx}, is bounded by $\lVert \ddot{x}(t) - \hat{\ddot{x}}(t) \rVert \leq 2$. 
\end{lemma}
\begin{proof}
    \eqref{eq:con:con_dynamics}, \eqref{eq:con:con_approx} and $F =  -f_{\ddot{x}, \mathrm{nld}}(y(0), 0)$ give us
    \begin{equation}
    \begin{split}
        \lVert \ddot{x}(t) - \hat{\ddot{x}}(t) \rVert =& \left (f_{\ddot{x},\mathrm{ld}}(y(t), 0) + f_{\ddot{x},\mathrm{nld}}(y(t), 0) \right ) - f_{\ddot{x},\mathrm{ld}}(y(t), F),\\
        =& \: -K x(t) -D \dot{x} -\tanh(W x(t) + b) + K x(t) + D \dot{x} + \tanh(W x(0) + b),\\
        =& \: \lVert -\tanh(W x(t) + b) + \tanh(W x(0) + b) \rVert
        \: \leq \: 2.
    \end{split}
    \end{equation}
    
\end{proof}

\subsection{Empirical Evaluation of Approximation Error}
In Table~\ref{tab:con:cfa_evaluation}, we present a comparison of \gls{CFA-CON} with several other strategies for integrating nonlinear dynamics, such as \gls{CON}.
Following the implicit assumption made in the concept for the closed-form approximation (i.e., Sec.~\ref{sub:con:cfa:approach}), we consider the case of $g(u) = 0$, $K = \mathrm{diag}(\kappa_1, \dots, \kappa_n)$ and $D = \mathrm{diag}(d_1, \dots, d_n)$ but with the hyperbolic coupling between the oscillators active (i.e., a full $W$ matrix).
Integrating the dynamics at a very small time step (i.e., $\delta t = \num{5e-5}~\si{s}$) with a high-order \gls{ODE} solver would give us a very accurate solution, but this is computationally infeasible in practice. We, therefore, regard this as the upper bound on the accuracy of the solution.
A feasible solution would be to implement either a high-order solver such as Tsit5 at a larger integration time-step, e.g., $\delta t = \num{1e-1}~\si{s}$) or a low-order solver with a slightly smaller integration time step, e.g., $\delta t = \num{5e-2}~\si{s}$). Therefore, we also benchmark these options.
We also benchmark an implementation specialized on the underdamped case (i.e., $\zeta_i <1$): \gls{CFA-UDCON}. This specialized implementation allows us to avoid using complex numbers in the algorithm and reduces the number of computations necessary for calculating the approximated solution.
As a result, we see a considerable increase in the sim-time to real-time factor.

\begin{figure}[ht]
    \centering
    \subfigure[Positions]{\includegraphics[width=0.49\textwidth, trim={5, 5, 5, 5}]{con/figures/results/cfa/coupled_oscillator_position.pdf}}
    % \hfill
    \subfigure[Velocities]{\includegraphics[width=0.49\textwidth, trim={5, 5, 5, 5}]{con/figures/results/cfa/coupled_oscillator_velocity.pdf}}
    \caption{Analysis of approximation error of CFA-CON: we compare the ground-truth solution of a \SI{40}{s} rollout of the \gls{CON} network consisting of three oscillators ($n=3$) with the \emph{CFA-CON} executed at a time step of $\delta t = \SI{0.1}{s}$ and a solution generated by integrating the \gls{ODE} at a time step of $\delta t = \SI{0.05}{s}$ with the Euler method.}
    \label{fig:con:cfa_con_comparison}
\end{figure}

\subsubsection{Integration error} 
We perform the integration error benchmark over $100$ different network configurations, all consisting of $50$ oscillators ($n=50$): First, we sample the natural frequency of the $i$th oscillator from a uniform distribution as $\omega_{\mathrm{n},i} \sim \mathcal{U}(\SI{0.05}{Hz}, \SI{0.5}{Hz})$, then we sample $\kappa_i \sim \mathcal{U}(\SI{0.2}{N\per m}, \SI{2}{N \per m})$ such that $K = \mathrm{diag}(\kappa_1, \dots, \kappa_n) \succ 0$, which lets us determine each mass $m_i = \frac{\kappa_i}{\omega_{\mathrm{n},i}^2}$.
Next, the damping ratio is determined as $\zeta_i \sim \mathcal{U}(0.1, 0.9)$ and $\zeta_i \sim \mathcal{U}(0.1, 2.0)$ for the underdamped and general case, respectively. As a result, $D = \mathrm{diag}(d_1, \dots, d_n) \succ 0$ with $d_i = 2 \, \zeta_i \, \sqrt{m_i \, \kappa_i}$ given.
Finally, by leveraging the Cholesky decomposition, we sample a $W \succ 0$  and $b_i \sim \mathcal{U}(-1, 1)$.
We compute the estimation error of all integrated trajectories with respect to the high-precision solution (i.e., Tsitouras' 5/4 method (Tsit5) at $\delta t = \num{5e-4}~\si{s}$).
For this, we compute the \gls{RMSE} for each \SI{60}{s} trajectory and then take the mean and standard deviation across the $100$ different network configurations.

\subsubsection{Simulation-Time to Real-Time Factor} 
The simulation vs. real-time factor is computed as the simulated rollout duration per second of computational time. For this, we let each method simulate a \SI{60}{s} trajectory for $100$ times and record the minimum run time on an Intel Core i7-10870H CPU (single core) over $10$ trials. Because of computational constraints, we simulated the trajectory  with the high-precision Tsit5 solver only $5$ times. 


\subsubsection{Results}
We also provide qualitative results for the integration accuracy in Fig.~\ref{fig:con:cfa_con_comparison}
The results in Table~\ref{tab:con:cfa_evaluation} show that \gls{CFA-CON} is \SI{30}{\percent} more accurate than the Euler integrator at half of the speed. Comparing against the Tsit5 integrator, \gls{CFA-CON} exhibits a 1.56x speed increase while being significantly less accurate.
For the underdamped case with $\zeta < 1$, the specialized implementation \gls{CFA-UDCON} is \SI{14.8}{\percent} faster and at the same time \SI{32}{\percent} more accurate than the Euler integrator. Furthermore, \gls{CFA-UDCON} is 3.7x faster and significantly less accurate than the Tsit5 integrator.
We can conclude that in the pure rollout setting (i.e., no backpropagation involved) for a generic \gls{CON}, the \gls{CFA-CON} does not show clear advantages to an appropriately tuned Euler or Tsit5 solver. However, the specialized version \gls{CFA-UDCON} demonstrates a 2.4x speed-up at no reduction of accuracy vs. \gls{CFA-CON} for underdamped oscillator networks.


\begin{table}[ht]
    \centering
    \begin{scriptsize}
    \begin{tabular}{c c c c r }
         \toprule
         \textbf{Method} & \textbf{RMSE} [m] $\downarrow$ & \textbf{RMSE} $\zeta < 1$ [m] $\downarrow$ & \textbf{Complexity} $\downarrow$ & \thead{\textbf{$\frac{\text{Sim. time}}{\text{Real time}}$}} $\uparrow$ \\
         \toprule
         \makecell{\gls{CON} with Tsit5\\ at $\delta t = \num{5e-5}~\si{s}$} & n/a & n/a & $\mathcal{O} \left (\frac{n^{\log_2 7} p h}{\delta t} \right) = \mathcal{O}(\num{3.5e11})$ & $5.68$x\\
         \midrule
         \makecell{\gls{CON} with Tsit5\\ at $\delta t = \num{1e-1}~\si{s}$} & $\mathbf{\num{5e-5} \pm \num{1e-5}}$ & $\mathbf{\num{8e-6} \pm \num{1e-5}}$ & $\mathcal{O} \left (n \frac{n^{\log_2 7} p h}{\delta t} \right) = \mathcal{O}(\num{1.8e8})$ & $11310$x\\
         \midrule
         \makecell{\gls{CON} with Euler\\ at $\delta t = \num{5e-2}~\si{s}$} & $\num{0.010} \pm \num{0.003}$ & $\num{0.022} \pm \num{0.005}$ & $\mathcal{O} \left (\frac{n^{\log_2 7} h}{\delta t} \right) = \mathcal{O}(\num{7.1e7})$ & $36500$x\\
         \midrule
         \makecell{\gls{CFA-CON} (our)\\ with $\delta t = \num{1e-1}~\si{s}$} & $\num{0.007} \pm \num{0.002}$ & $\num{0.015} \pm \num{0.003}$ & $\mathcal{O} \left (\frac{n^{\log_2 7} h}{\delta t} \right) = \mathcal{O}(\bf{\num{3.5e7}})$ & $17680$x\\
         \midrule
         \makecell{\gls{CFA-UDCON} (our)\\ with $\delta t = \num{1e-1}~\si{s}$} & n/a & $\num{0.015} \pm \num{0.003}$ & $\mathcal{O} \left (\frac{n^{\log_2 7} h}{\delta t} \right) = \mathcal{O}(\bf{\num{3.5e7}})$ & $\mathbf{41900}$\textbf{x}\\
         \bottomrule
    \end{tabular}
    \end{scriptsize}
    \vspace{0.5cm}
    \caption{Benchmarking of various methods for integrating the \gls{CON} dynamics.
    The \gls{RMSE} is computed with respect to the Tsitouras' 5/4 method (Tsit5) (i.e., extremely high accuracy but also extremely high computational complexity). 
    We denote with $n$ the number of oscillators in the network (in this case $n=50$), with $p$ the order of the numerical \gls{ODE} solver, and with $\delta t$ the time-step.
    When stating the complexity, we refer to $h = t_N-t_0$ as the rollout horizon in seconds. In this case, we report the results for a horizon of $h = \SI{60}{s}$.
    The \gls{RMSE} column states the \gls{RMSE} of the various integration strategies with respect to the \emph{CON with Tsit5 at $\delta t = \num{5e-5}$s} solution, which we consider to be the ground-truth.
    The \emph{RMSE $\zeta < 1$} computes the same metrics, but this time for a dataset that contains only underdamped oscillators.
    The \emph{$\frac{\text{Sim. time}}{\text{Real time}}$} column states the ratio between the duration of the simulation achieved (in seconds) per second of real-time (i.e., computational time).
    We report the mean and standard deviation of the \gls{RMSE} over $100$ different network configurations.
    }
    \label{tab:con:cfa_evaluation}
\end{table}
\section{Learning Control-Oriented Latent Dynamics from Pixels}\label{sec:con:learning_latent_space_dynamics}
We now move towards learning latent dynamical models based on \gls{CON} and \gls{CFA-CON}.
% , which we will leverage later in Sec~\ref{sec:latent_space_control} for model-based control. 
\glspl{CON} are an ideal fit for learning latent dynamics as they guarantee that the latent states stay bounded. 

\begin{figure}[t]
    \centering
    \includegraphics[width=1.0\linewidth]{con/figures/autoencoder/blockdiagram_autoencoder_v1_cropped.pdf}
    \caption{Exploiting \glspl{CON} for learning latent dynamics from pixels: We encode the initial observation $o(t_0)$ and the input $u(t)$ into latent space where we leverage the \gls{CON} to predict future latent states. Finally, we decode both the latent-space torques $\tau(t)$ and the predicted latent states $z(t)$.}
    \label{fig:con:blockdiagram_autoencoder}
\end{figure}

We assume to have access to observations in the form of images $o \in \mathbb{R}^{h_\mathrm{o} \times w_\mathrm{o} \times c_\mathrm{o}}$, where $c_\mathrm{o}$ denotes the number of channels. Please note that this could also be other high-dimensional observations such as LiDAR scans, point clouds, etc.
We now leverage an encoder-decoder architecture to map these high-dimensional observations into a compressed latent space: % parametrized by $z^\mathrm{T} & \dot{z}^\mathrm{T} \end{bmatrix}^\mathrm{T} \in \mathbb{R}^{2n_z}$.
The encoder $\Phi: \mathbb{R}^{h_\mathrm{o} \times w_\mathrm{o} \times c_\mathrm{o}} \to \mathbb{R}^{n_z} $ with $n_z \ll h_\mathrm{o} \,  w_\mathrm{o}$ identifies a low-dimensional latent representation $z \in \mathbb{R}^{n_z}$ of the images. The decoder $\Psi: \mathbb{R}^{n_z} \to \mathbb{R}^{h_\mathrm{o} \times w_\mathrm{o} \times c_\mathrm{o}}$ approximates the inverse operation by reconstructing an image $\hat{o} \in \mathbb{R}^{h_\mathrm{o} \times w_\mathrm{o} \times c_\mathrm{o}}$ based on the latent representation.
To promote the learning of a smooth and monotonic mapping into latent space, we specifically choose to implement the autoencoder here as a $\beta$-\gls{VAE}~\cite{kingma2014auto, higgins2017beta}.
Instead of just statically reconstructing the image $\hat{o}(t_k)$, we are interested in predicting future observations $\hat{o}(t_{k+l})$, where $l \in 1 \dots N$. For this, we train a \nth{2}-order dynamical model that is, when integrated, able to predict future latent representations $z(t_{k+l})$.
This requires us to define a latent state $\xi(t) = \begin{bmatrix}
    z^\mathrm{T}(t) & \dot{z}^\mathrm{T}(t)
\end{bmatrix}^\mathrm{T} \in \mathbb{R}^{2n_z}$ consisting of the latent representation and latent velocity $\dot{z}(t) \in \mathbb{R}^{n_z}$.

We now rely on \gls{CON} with $n = n_z$ oscillators to provide us with the latent state derivative $\dot{\xi}= f_\mathrm{w}(z(t), u(t))$, where we defined $\xi = y_\mathrm{w}$, and $z = x_\mathrm{w}$. To ensure stability, we make use of the Cholvesky decomposition to ensure that $M_\mathrm{w}, K_\mathrm{w}$ and $D_\mathrm{w}$ always remain positive definite (see Theorem~\ref{theorem:con:iss}).
It is important to note that we train the encoder, decoder, and dynamical model all jointly.
Please refer to Appendix~\ref{sec:apx-con:experimental_setup} for more implementation details.

% \begin{table}[ht]
%     \centering
%     \begin{scriptsize}
%     \begin{tabular}{c c c c c }
%          \toprule
%          \textbf{Model} & \textbf{RMSE PCC-NS-2} $\downarrow$ & \textbf{SSIM PCC-NS-2} $\uparrow$ & \textbf{RMSE CS} $\downarrow$ & \textbf{SSIM CS} $\uparrow$ \\
%          % \textbf{Model parameters} $\downarrow$ & \textbf{Training} $\frac{\mathbf{\text{steps}}}{\mathbf{\text{second}}}$ $\uparrow$ $\downarrow$
%          \midrule
%          RNN & $0.1373 \pm 0.0185$ & $0.9643 \pm 0.0077$ & $\mathbf{0.1011 \pm 0.0009}$ & $\mathbf{0.9776 \pm 0.0004}$ \\
%          GRU & $\mathbf{0.0951 \pm 0.0021}$ & $\mathbf{0.9814 \pm 0.0006}$ & $0.1125 \pm 0.0100$ & $0.9730 \pm 0.0040$\\
%          coRNN & $0.2504 \pm 0.0899$ & $0.8774 \pm 0.0857$ & $0.2537 \pm 0.0018$ & $0.8820 \pm 0.0024$\\
%          NODE & $0.1867 \pm 0.0561$ & $0.9279 \pm 0.0348$ & $0.2415 \pm 0.0021$ & $0.8946 \pm 0.0023$\\
%          MECH-NODE & $0.1035 \pm 0.0012$ & $0.9778 \pm 0.0004$ & $0.2494 \pm 0.0028$ & $0.8898 \pm 0.0016$\\
%          CON-S (our) & $0.0996 \pm 0.0012$ & $0.9792 \pm 0.0007$ & $0.1993 \pm 0.0646$ & $0.9218 \pm 0.0380$\\
%          CON-M (our) & $0.1008 \pm 0.0006$ & $0.9786 \pm 0.0003$ & $0.1063 \pm 0.0027$ & $0.9758 \pm 0.0011$\\
%          CFA-CON (our) & $0.1124 \pm 0.0025$ & $0.9734 \pm 0.0012$ & $0.1462 \pm 0.0211$ & $0.9730 \pm 0.0040$\\
%          \bottomrule
%     \end{tabular}
%     \end{scriptsize}
%     \caption{Benchmarking of \gls{CON} and \gls{CFA-CON} at learning latent dynamics for two datasets. The \emph{CS} dataset considers a continuum soft robot consisting of one segment with three constant planar strains. The \emph{PCC-NS-2} dataset contains \SI{2}{s} trajectories of a continuum soft robot made of two piecewise constant curvature segments. 
%     We choose the latent dimensions of the models as $n_z = 8$ and $n_z =12$ for the \emph{PCC-NS-2} and \emph{CS} datasets, respectively.
%     For the \glspl{RNN}, this corresponds to a hidden state dimension of $n_z=16$ and $n_z = 24$.
%     \emph{CON-S} and \emph{CON-M} are small and medium-sized versions of the \gls{CON} model, respectively. \emph{MECH-NODE} is a \gls{NODE} with prior knowledge about the mechanical structure of the system (i.e., $\frac{\mathrm{d}x}{\mathrm{d}t} = \dot{x}$).
%     We report the mean and standard deviation of the \gls{RMSE} and \gls{SSIM} metrics over three different random seeds.}
%     \label{tab:con:latent_dynamics_results}
% \end{table}
\begin{landscape}
\begin{table}
    \centering
    \begin{small}
    \setlength\tabcolsep{2pt}
    \begin{tabular}{c c c c c c c c}
         \toprule
         \textbf{Model} & \textbf{RMSE M-SP+F} $\downarrow$ & \textbf{RMSE S-P+F} $\downarrow$ & \textbf{RMSE D-P+F} $\downarrow$ & \textbf{RMSE CS} $\downarrow$ & \textbf{RMSE PCC-NS-2} $\downarrow$ & \textbf{RMSE PCC-NS-3} $\downarrow$ & \textbf{RMSE R-D} $\downarrow$ \\
         % \textbf{Model parameters} $\downarrow$ & \textbf{Training} $\frac{\mathbf{\text{steps}}}{\mathbf{\text{second}}}$ $\uparrow$ $\downarrow$
         \midrule
         RNN & $0.2739 \pm 0.0057$ & $0.2378 \pm 0.0352$ & $0.1694 \pm 0.0004$ & $\mathbf{0.1011 \pm 0.0009}$ & $0.1373 \pm 0.0185$ & $0.2232 \pm 0.0075$ & $0.3763 \pm 0.0374$\\
         GRU~\cite{cho2014learning} & $0.0267 \pm 0.0033$ & $0.1457 \pm 0.0078$ & $0.1329 \pm 0.0005$ & $0.1125 \pm 0.0100$ & $\mathbf{0.0951 \pm 0.0021}$ & $0.2148 \pm 0.0196$ & $0.3232 \pm 0.0368$\\
         coRNN~\cite{rusch2020coupled} & $\mathbf{0.0265 \pm 0.0002}$ & $0.1333 \pm 0.0044$ & $0.1324 \pm 0.0016$ & $0.2537 \pm 0.0018$ & $0.2504 \pm 0.0899$ & $0.2474 \pm 0.0018$ & $\mathbf{0.0741 \pm 0.0001}$\\
         NODE~\cite{chen2018neural} & $\mathbf{0.0264 \pm 0.0010}$ & $\mathbf{0.1260 \pm 0.0013}$ & $0.1324 \pm 0.0024$ & $0.2415 \pm 0.0021$ & $0.1867 \pm 0.0561$ & $0.3373 \pm 0.0565$ & $\mathbf{0.0738 \pm 0.0007}$\\
         MECH-NODE & $0.0328 \pm 0.0034$ & $0.1650 \pm 0.0205$ & $0.1710 \pm 0.0111$ & $0.2494 \pm 0.0028$ & $0.1035 \pm 0.0012$ & $0.1900 \pm 0.0024$ & N/A\\
         CON-S (our) & $0.0303 \pm 0.0053$ & $0.1303 \pm 0.0064$ & $0.1323 \pm 0.0018$ & $0.1993 \pm 0.0646$ & $0.0996 \pm 0.0012$ & $0.1792 \pm 0.0038$ & $0.1110 \pm 0.0160$\\
         CON-M (our) & $0.0303 \pm 0.0053$ & $0.1303 \pm 0.0064$ & $0.1323 \pm 0.0018$ & $0.1063 \pm 0.0027$ & $0.1008 \pm 0.0006$ & $\mathbf{0.1785 \pm 0.0023}$ & $0.1110 \pm 0.0160$\\
         CFA-CON (our) & $0.0313 \pm 0.0026$ & $0.1352 \pm 0.0073$ & $\mathbf{0.1307 \pm 0.0012}$ & $0.1462 \pm 0.0211$ & $0.1124 \pm 0.0025$ & $0.1803 \pm 0.0003$ & $0.1068 \pm 0.0059$\\
         \bottomrule
    \end{tabular}
    \end{small}
    \caption{Benchmarking of \gls{CON} and \gls{CFA-CON} at learning latent dynamics against baseline methods. 
    The first three datasets, based on ~\cite{botev2021priors}, contain samples of a mass-spring with friction (\emph{M-SP + F}), a single pendulum with friction (\emph{S-P + F}), and a double pendulum with friction (\emph{D-P + F}) (all without system inputs).
    The \emph{CS} dataset considers a continuum soft robot consisting of one segment with three constant planar strains. The \emph{PCC-NS-2} and \emph{PCC-NS-3} datasets contain trajectories of a continuum soft robot made of two and three piecewise constant curvature segments, respectively.
    We choose the latent dimensions of the models as $n_z = 4$, $n_z = 4$, and $n_z = 12$ for the \emph{M-SP + F}, \emph{S-P + F}, and \emph{D-P + F} datasets, and  $n_z = 8$, $n_z = 12$, and $n_z = 12$ for the \emph{PCC-NS-2}, \emph{PCC-NS-3}, and \emph{CS} soft robotic datasets.
    % For the \glspl{RNN}, this corresponds to a hidden state dimension of $n_z=16$ and $n_z = 24$.
    % \emph{CON-S} and \emph{CON-M} are small and medium-sized versions of the \gls{CON} model, respectively. \emph{MECH-NODE} is a \gls{NODE} with prior knowledge about the mechanical structure of the system (i.e., $\frac{\mathrm{d}x}{\mathrm{d}t} = \dot{x}$).
    We report the mean and standard deviation over three different random seeds.}
    \label{tab:con:latent_dynamics_results}
\end{table}

\begin{table}
    \centering
    \begin{small}
    \setlength\tabcolsep{2.5pt}
    \begin{tabular}{c|c|c|c|c|c|c|c|c|c}
        \toprule
        \textbf{Dataset} & $\mathbf{n_z}$ & \textbf{RNN} & \textbf{GRU}~\cite{cho2014learning} & \textbf{coRNN}~\cite{rusch2020coupled} & \textbf{NODE}~\cite{chen2018neural} & \textbf{MECH-NODE} & \textbf{CON-S (our)} & \textbf{CON-M (our)} & \textbf{CFA-CON (our)}\\
        \midrule
        M-SP+F~\cite{botev2021priors} & $4$ & $88$ & $248$ & $40$ & $3368$ & $3244$ & $\mathbf{34}$ & $\mathbf{34}$ & $\mathbf{34}$\\
        S-P+F~\cite{botev2021priors} & $4$ & $88$ & $248$ & $40$ & $3368$ & $3244$ & $\mathbf{34}$ & $\mathbf{34}$ & $\mathbf{34}$\\
        D-P+F~\cite{botev2021priors} & $12$ & $672$ & $1968$ & $348$ & $4404$ & $4032$ & $\mathbf{246}$ & $\mathbf{246}$ & $\mathbf{246}$\\
        CS & $12$ & $696$ & $2040$ & $\mathbf{336}$ & $4374$ & $4002$ & $1386$ & $8568$ & $8568$\\
        PCC-NS-2 & $8$ & $320$ & $928$ & $\mathbf{152}$ & $3856$ & $3062$ & $676$ & $7048$ & $7048$\\
        PCC-NS-3 & $12$ & $696$ & $2040$ & $\mathbf{336}$ & $4374$ & $4002$ & $1386$ & $8568$ & $8568$\\
        R-D~\cite{champion2019data} & $4$ & $\mathbf{20}$ & $52$ & $\mathbf{20}$ & $3064$ & - & $24$ & $24$ & $24$\\
        \bottomrule
    \end{tabular}
    \end{small}
    \caption{Number of trainable parameters for the each latent dynamic models depending on the respective dataset: mass-spring with friction (\emph{M-SP+F}), single pendulum with friction (\emph{S-P+F}), double pendulum with friction (\emph{D-P+F}), a constant strain soft robot (\emph{CS}), a piecewise constant curvature with two segments (\emph{PCC-NS-2}, a piecewise constant curvature soft robot with three segments (\emph{PCC-NS-3}, and reaction-diffusion (\emph{R-D}).}
    \label{tab:con:latent_dynamics_number_of_parameters}
\end{table}
\end{landscape}

\begin{figure}[ht]
    \centering
    \subfigure[RMSE vs. latent dimension $n_z$]{\includegraphics[width=0.45\textwidth, trim={5, 5, 5, 5}]{con/figures/results/latent_dynamics/pcc_ns-2/sweep_rmse_rec_dynamic_vs_n_z.pdf}}
    \subfigure[RMSE vs. model parameters]{\includegraphics[width=0.45\textwidth, trim={5, 5, 5, 5}]{con/figures/results/latent_dynamics/pcc_ns-2/sweep_rmse_rec_dynamic_vs_num_trainable_params.pdf}}
    \caption{Evaluation of prediction performance of the various models vs. the dimension of their latent representation $n_z$ and the number of trainable parameters of the dynamics model, respectively, on the \emph{PCC-NS-2} dataset. All hyperparameters are tuned for each model separately for $n_z=8$. The error bar denotes the standard deviation across three random seeds.} \label{fig:con:latent_dynamics_sweep_pcc_ns-2}
\end{figure}

\subsection{Training} 
It is important to remember that because we are using a $\beta$-\gls{VAE}~\cite{kingma2014auto, higgins2017beta}, the image encoding becomes stochastic, and the encoder neural network actually outputs $\mu_z(o), 2 \, \log(\sigma_z)(o) \in \mathbb{R}^{n_z}$. After executing the reparametrization trick as $z(t_k) \sim \mathcal{N}(\mu_z(t_k), \sigma_z^2(t_k))$, we formulate the loss function, evaluated on each trajectory consisting of $N$ time-steps, as
\begin{equation}\label{eq:training_loss}
\begin{split}
    \mathcal{L} =& \: \sum_{k=0}^{N} \left ( \underbrace{\frac{\mathrm{MSE}(o(t_k), \Psi(z(t_k)))}{N+1}}_{\text{Static image reconstruction loss}} + \beta \underbrace{\frac{\mathcal{D}_\mathrm{KL} \left ( (\mu_z(t_k), \sigma_z(t_k) \right )}{N+1}}_{\text{Kullback–Leibler divergence}} \right )\\
    &\: + \sum_{k=1}^{N} \left ( \lambda_{\vec{o}} \underbrace{ \frac{\mathrm{MSE}(o(t_k), \Psi(\hat{z}(t_k)))}{N}}_{\text{Dynamic image reconstruction loss}} + \lambda_{\mathrm{z}} \underbrace{\frac{\mathrm{MSE}(z(t_k), \hat{z}(t_k))}{N}}_{\text{Latent dynamics consistency loss}} \right ),
\end{split}
\end{equation}
where $\hat{z}(t_k)$ is predicted by $\hat{\xi}(t_k)  = \int_{t=t_0}^{t_k} f_\xi(\xi(t'), u(t')) \: \mathrm{d}t'$, and $\xi(t_0) = \begin{bmatrix}
    z^\mathrm{T}(t_0) & \dot{z}^\mathrm{T}(t_0)
\end{bmatrix}^\mathrm{T}$. Here, $z(t_0)$ is given by the encoder, and $\dot{z}(t_0)$ is approximated using finite differences in image-space. $\beta, \lambda_{\vec{o}}, \lambda_\mathrm{z} \in \mathbb{R}$ are loss weights.

\subsubsection{Estimation of the initial latent velocity}\label{sub:con:initial_latent_velocity_estimation}
For \nth{2}-order systems and when integrating the evolution of the latent state $\xi(t) = \begin{bmatrix}
    z^\mathrm{T}(t) & \dot{z}^\mathrm{T}(t)
\end{bmatrix}^\mathrm{T}$ in time, we need to have access to an initial latent velocity $\dot{z}(t_0)$ such that we can roll out the latent state $\xi(t)$ in time.
A naive approach to estimating such an initial latent velocity would be to encode multiple (at least two) images of the system at the start of the trajectory into latent space and then perform numerical differentiation (e.g., finite differences) in latent space. However, we found the resulting $\dot{z}(t_k)$ to be relatively noisy and susceptible to small encoding errors.
Instead, we propose to perform numerical differentiation in image space and then map this velocity into latent space using the encoder's Jacobian. First, we estimate the image-space velocity at $t_k$ using finite differences: $\dot{o}(t_k) \approx \frac{o(t_{k+1}) - o(t_{k-1})}{t_{k+1} - t_{k-1}}$. The latent velocity is then estimated as $\dot{z}(t_k) = \frac{\partial \Phi}{\partial o} (o(t_k)) \, \dot{o}(t_k)$, where $\frac{\partial \Phi}{\partial o}$ is obtained with forward-mode automatic differentiation.

\subsection{Models} 
We train the \gls{CON} with the input-to-forcing mapping $g(u) = B(u) \, u$, where $B(u)$ is parametrized by a \gls{MLP} with a hyperbolic tangent activation function applied in between layers. We report results for two variants of the \gls{CON} model: for the medium-sized \emph{CON-M} and small-sized \emph{CON-S}, the \gls{MLP} consists of five and two layers with a hidden dimension of $30$ and $12$, respectively. The model \gls{CFA-CON} uses the same architecture as \emph{CON-M}. We compare % the performance of our proposed models 
against several popular latent space model architectures:
The \gls{NODE} model uses a \gls{MLP} with an hyperbolic activation functions and predicts $\dot{\xi}(t) = f_\mathrm{NODE}(\xi(t), u(t))$. To make the comparison fair, we parametrize the NODE's \gls{MLP} in the same fashion as for \emph{CON-M}. The \emph{MECH-NODE} integrates prior knowledge towards learning \nth{2}-order mechanical \glspl{ODE} and, therefore, predicts $\ddot{z}(t) = f_\mathrm{MECH-NODE}(\xi(t), u(t))$. 
Furthermore, we consider multiple autoregressive models: \gls{RNN}, \gls{GRU}, and \gls{coRNN} and let them parameterize the following transition function: $\xi(t_{k+1}) = f_\mathrm{ar}(\xi(t_k), u(t_k))$
As common in the relevant literature~\cite{botev2021priors}, we allow the autoregressive models to perform multiple time step transitions before predicting the next sample.
For the autoencoder, we use a vanilla \gls{CNN}. More details can be found in Appendix~\ref{sec:apx-con:experimental_setup}.

\subsection{Datasets} 
We consider in total seven datasets that are based on simulations of unactuated mechanical systems, actuated continuum soft robots, and the reaction-diffusion dynamics. The first three, mechanical dataset are based on the work of Botev et al.~\cite{botev2021priors} and contain video sequences of a mass-spring system with friction (\emph{M-SP+F}), a single pendulum with friction (\emph{S-P+F}), and a double pendulum with friction (\emph{D-P+F}).
Continuum soft robots have theoretically infinite \gls{DOF}, evolve with highly nonlinear and often time-dependent dynamical behaviors, and are notoriously difficult to model from first principles~\cite{armanini2023soft}. For that reason, it is a very interesting proposition if we could learn latent-space dynamical models directly from video~\cite{thuruthel2023multi} and later leverage them for control~\cite{almanzor2023static}. 
Therefore, we generate three datasets based on the \gls{PCS} soft robot model. \emph{CS} considers one segment with constant strain and is modeled using three configuration variables. \emph{PCC-NS-2} and \emph{PCC-NS-3} only consider bending deformations and contain soft robots with two and three segments, respectively. 
As all previous examples exampled \glspl{ODE}, we strive to test the proposed approach also on a system that is governed by \glspl{PDE}. Specifically, we consider the Reaction-diffusion (\emph{R-D}) dataset as included in the SINDy Autoencoder paper~\cite{champion2019data}.
For all datasets, we generate images of size $32 \times 32 \mathrm{px}$ and subsequently normalize the pixels to the interval $[-1, 1]$. 
% We tune all hyperparameters for each model and dataset separately using Optuna~\cite{akiba2019optuna}.

\subsubsection{Unactuated mechanical datasets}
We consider multiple mechanical datasets based on a standard implementation included in the \emph{Toy Physics} category of the \emph{NeurIPS 2021 Track on Datasets and Benchmarks} publication by Botev et al.~\cite{botev2021priors}:  mass-spring with friction (M-SP+F), a single pendulum with friction (S-P+F), and a double pendulum with friction (D-P+F).
All datasets contain $5000$ system trajectories in the training set and $1000$ trajectories each in the validation and test set.
Each trajectory is generated by first randomly initializing the system, then rolling it out for $\SI{3}{s}$ using an Euler integrator with a time step size of $\SI{5}{ms}$. Samples are recorded at a rate of \SI{20}{Hz}  (i.e., a time step of \SI{0.05}{s}). As a result, each trajectory contains $60$ images of the system's state.
As all of these datasets are unactuated, we can deactivate the input-to-forcing mapping component from all models (e.g., set $g(u) = 0$ for the \gls{CON} model).

The \emph{M-SP+F} dataset contains motion samples of a damped harmonic oscillator with a mass of $\SI{0.5}{kg}$, a spring stiffness of \SI{2}{N \per m}, and a damping coefficient of \SI{0.05}{Ns \per m}. For each trajectory, the initial condition of the mass-spring is randomly sampled by combining a random $\mathrm{sign}(q)$ with a uniformly sampled $|q| \sim \mathcal{U}(\SI{0.1}{m}, \SI{1}{m})$. The position of the mass is rendered with a filled circle in a grayscale image.

The \emph{SP+F} and \emph{DP+F} datasets include the evolutions of a single-link pendulum and double-link pendulum, respectively, with a mass of \SI{0.5}{kg} attached to the end of each link, which has a length of \SI{1}{m}. The dataset considers a gravitational acceleration of $\SI{3}{m \per s^2}$. A rotational damper with coefficient \SI{0.05}{Nms \per rad} provides the friction. 
Similarly to the \emph{M-SP+F} dataset, both the sign and the absolute value of the initial configuration are randomly sampled, where $|q(0)| \sim \mathcal{U}(\SI{1.3}{rad}, \SI{2.3}{rad})$.
The position of each mass is rendered with a filled circle. For the single-link pendulum, this is done in grayscale, and for the double pendulum, each mass is rendered with a different color (i.e., blue and red).

\subsubsection{Actuated continuum soft robot datasets}\label{ssub:con:soft_robot_dataset}
The shape of slender and deformable rods can be approximated by considering the deformations along the 1D curve of the backbone~\cite{gazzola2018forward}. While this curve is still infinite-dimensional, it is possible to discretize the backbone into (many) segments with piecewise constant strain~\cite{renda2016discrete, gazzola2018forward}.
Accordingly, we describe the kinematics of a planar continuum soft robot consisting of $n_\mathrm{b}$ segments with the \gls{PCS} model~\cite{renda2016discrete}. We assume each segment has a length of \SI{100}{mm} and a diameter of \SI{20}{mm}.
The \gls{PCS} model assumes each segment to have constant strain. In the planar case, this means that the shape of the $i$th segment can be parametrized by $\xi_i = \begin{bmatrix}
    \kappa_{\mathrm{be},i} & \sigma_{\mathrm{sh},i} & \sigma_{\mathrm{ax},i}
\end{bmatrix}^\mathrm{T} \in \mathbb{R}^{3}$ where $\kappa_{\mathrm{be},i}$ is the bending strain (i.e., the curvature) in the unit \si{rad \per \meter}, $\sigma_{\mathrm{sh},i}$ is the shear strain  (dimensionless), and $\sigma_{\mathrm{ax},i}$ is the axial elongation strain (dimensionless).
The robot's configuration is then defined as $q = \begin{bmatrix}
    \xi_1^\mathrm{T} & \cdots & \xi_i^\mathrm{T} & \cdots & \xi_{n_\mathrm{b}}^\mathrm{T}
\end{bmatrix}^\mathrm{T}$.
In the case of \gls{PCC}, only the bending strain is active as shear strains and axial strains are neglected, and the configuration is now $q \in \mathbb{R}^{n_\mathrm{b}}$.
The \gls{PCS} model generates \gls{EOM} in the form of~\cite{della2023model}
\begin{equation}
    B(q) \, \ddot{q} + C(q,\dot{q}) \, \dot{q} + G(q) + K_\mathrm{q} \, q + D_\mathrm{q} \, \dot{q} = u(t),
\end{equation}
where $B(q) \succ 0$ and $C(q,\dot{q})$ are the inertia and Corioli matrices, respectively. $G(q)$ collects the gravitational forces, $K_\mathrm{q} \succ 0$ is the stiffness matrix, and $D_\mathrm{q} \succ 0$ contains the damping coefficients. 
$u(t) \in \mathbb{R}^{n_\mathrm{b}}$ is an external force acting on the generalized coordinates, and now $m = n_\mathrm{b}$.

% \begin{figure}
%     \centering
%     \includegraphics[width=0.5\columnwidth]{figures/soft_robot/katzschmann_planar_soft_robot.png}
%     \caption{A planar soft robot, consisting of six segments, that deforms with piecewise constant curvature. The behavior of the simulated robots included in datasets \emph{PCC-NS-2} and \emph{PCC-NS-3} resembles this pneumatically actuated robot. The source of the image is (Della Santina et al., 2020)~\cite{della2020model}.}
%     \label{fig:enter-label}
% \end{figure}

We derive the corresponding dynamics for a continuum soft robot of material density $\SI{600}{kg \per \meter^3}$, elastic modulus of \SI{20000}{Pa}, shear modulus of \SI{10000}{Pa}, and damping coefficients of \SI{0.00001}{Nm^2s} for bending strains, \SI{0.01}{Ns} for shear strains, and \SI{0.01}{Ns} for axial strains, respectively.
Gravity is pointing downwards.
The implementation of the dynamics in JAX~\cite{jax2018github} is based on the \emph{JSRM} library~\cite{stolzle2024experimental, stolzle2024guiding}, and we simulate the robot using a constant integration time step of \SI{0.1}{ms}.
We render grayscale images of the robot with a size of $32 \times 32 \mathrm{px}$ at a rate of \SI{50}{Hz} using OpenCV~\cite{opencv_library}. 
We generate $10000$ trajectories, each of duration \SI{2.0}{s} and a sampling time-step of $\SI{0.02}{s}$. We use \SI{60}{\percent} training, \SI{20}{\percent} validation, and \SI{20}{\percent} test split.
For each trajectory, we randomly sample a constant actuation/input $u \sim \mathcal{U}(-u_\mathrm{max}, u_\mathrm{max})$. We choose the maximum actuation magnitude to be equal to the sum of the contribution of the potential forces (i.e., elastic and gravitational forces): $u_\mathrm{max} = G(q_\mathrm{max}) + K \, q_\mathrm{max}$ with $q_{\mathrm{max},i} = \begin{bmatrix}
    5 \, \pi~\si{rad \per m}, 0.2, 0.2
\end{bmatrix}^\mathrm{T}$.

We generate three datasets based on this continuum soft robot model: in the \emph{CS} dataset, we consider one segment with all three planar strains active (i.e., bending, shear, and elongation). This results in three \gls{DOF} and six-state variables in the dynamical model. In the case of the \emph{PCC-NS-2} and \emph{PCC-NS-3} datasets, we base the dataset on a simulated system consisting of two planar bending segments, respectively. Each segment is parametrized using \gls{CC}~\cite{webster2010design, rosi2022sensing}, which results in two configuration variables and a state dimension of four.

\subsubsection{Unactuated PDE reaction-diffusion dataset}\label{ssub:con:reaction_diffusion_dataset}
 We consider the \nth{1}-order Reaction-diffusion (\emph{R-D}) \gls{PDE} on which (Champion et al, 2019)~\cite{champion2019data} evaluated their SINDy Autoencoder on. The \gls{PDE} of the high-dimensional lambda-omega reaction-diffusion system is defined as
\begin{equation}
\begin{split}
    \frac{\partial u}{\partial t} =& \: \left ( 1 - (u^2 + v^2) \right ) u + \beta \, (u^2 + v^2) \, v + d_1 \left ( \frac{\partial^2 u}{\partial q_1^2} + \frac{\partial^2 u}{\partial q_2^2} \right ),\\
    \frac{\partial v}{\partial t} =& \: -\beta (u^2 + v^2) \, u + (1 - (u^2 + v^2)) \, v + d_2 \left ( \frac{\partial^2 v}{\partial q_1^2} + \frac{\partial^2 v}{\partial q_2^2} \right ),
\end{split}
\end{equation}
where $u(t,q): \mathbb{R} \times \mathbb{R}^2 \to \mathbb{R}$ and $v(t,q): \mathbb{R} \times \mathbb{R}^2 \to \mathbb{R}$ are time-dependent two vector fields defined over the spatial domain $q \in \mathbb{R}^2$.
We choose the same system parameters and initial condition as Champion et al.~\cite{champion2019data}: $d_1, d_2 = 0.1$, and $\beta = 1$ and 
\begin{equation}
\begin{split}
    u(0,q) = \tanh \left ( \sqrt{q_1^2 + q_2^2} \, \cos \left ( \angle (q_1 + i q_2) - \sqrt{q_1^2 + q_2^2} \right ) \right ),\\
    v(0,q) = \tanh \left ( \sqrt{q_1^2 + q_2^2} \, \sin \left ( \angle (q_1 + i q_2) - \sqrt{q_1^2 + q_2^2} \right ) \right ).
\end{split}
\end{equation}

After discretizing the spatial domain into $32$ points along each dimension, we solve the \gls{PDE} with a MATLAB ODE45 solver the solution of $u(t,q)$ and $v(t,q)$ at each time step and grid point.
Subsequently, the solution is multiplied with a Gaussian centered at the origin~\cite{champion2019data}
\begin{equation}
\begin{split}
    \bar{u}(t,q) =& \: \exp(-0.01 \, (q_1^2 + q_2^2)) \, \bar{u}(t,q),\\
    \bar{v}(t,q) =& \: \exp(-0.01 \, (q_1^2 + q_2^2)) \, \bar{v}(t,q).
\end{split}
\end{equation}
We integrate the system from the specified initial condition for \SI{500}{s} and store samples at a time step of \SI{0.05}{s}. We divide the entire sequence into $99$ subsequences each containing $101$ samples. We train the models to predict these subsequences that have a horizon of $\SI{5.0}{s}$ each.

We stack the solution of $\bar{u}(t, q)$ and $\bar{v}(t,q)$ contained in the two grids $o_\mathrm{u}(t), o_\mathrm{v}(t) \in \mathbb{R}^{32 \times 32}$, respectively, to gather the images $o(t) \in \mathbb{R}^{32 \times 32 \times 2}$ containing two channels.
A sample sequence of the generated images is presented in Fig.~\ref{fig:con:latent_dynamics:sequence_of_stills:r_d:rollout2}.
We use \SI{60}{\percent} of the subsequences (i.e., $59$) as our training set, and employ \SI{20}{\percent} (i.e., $19$) for the validation and test sets, respectively. 

\subsection{Results}

\subsubsection{Results for unactuated mechanical datasets}
The results in Tab.~\ref{tab:con:latent_dynamics_results} show that the \emph{NODE} model slightly outperforms the \emph{CON} network on the \emph{M-SP+F} and \emph{S-P+F} datasets. However, as the datasets do not consider system inputs, we can remove the input mapping from all models (e.g., \emph{RNN}, \emph{GRU}, \emph{coRNN}, \emph{CON}, and \emph{CFA-CON}). With that adjustment, the \emph{CON} network has the fewest parameters among all models, particularly two orders of magnitude less than the NODE model. Therefore, we find it very impressive that the CON network is roughly on par with the NODE model. For the \emph{D-P+F} dataset, we can conclude that the \emph{CFA-CON} model offers the best performance across all methods. Finally, most of the time, the \emph{CON} \& \emph{CFA-CON} networks outperform the other baseline methods that have more trainable parameters.
Sequences of stills of the rollout of the \gls{CON} model trained on the \emph{M-SP+F}, and \emph{D-P+F} datasets are presented in Figs.~\ref{fig:con:latent_dynamics:sequence_of_stills:m-sp+f:rollout3} and \ref{fig:con:latent_dynamics:sequence_of_stills:d-p+f:rollout7}.

\subsubsection{Results for actuated continuum soft robot datasets} 
The results in Tab.~\ref{tab:con:latent_dynamics_results} show that \emph{CON-M} matches the performance of the state-of-the-art methods across all experiments. In the case of \emph{PCC-NS-3}, \emph{CON-M} even decreases the \gls{RMSE} error by \SI{6}{\percent} w.r.t. the closest baseline method (MECH-NODE).
Impressively, the performance is not reduced (but instead often even improved) compared to other models that offer a much larger design space for learning the dynamics (e.g. \gls{NODE}).
Furthermore, \emph{CON-S} and \emph{CFA-CON} often only exhibit slightly lower performance than \emph{CON-M}, even though they have significantly fewer parameters and consider an approximated solution, respectively.
Supplementary results (e.g., more evaluation metrics) can be found in Appendix~\ref{sec:apx-con:latent_dynamics_results}. Sequences of stills for the \gls{CON} model trained on the \emph{PCC-NS-3} dataset are provided in Fig.~\ref{fig:con:latent_dynamics:sequence_of_stills:pcc_ns-3:rollout4}.

We also conduct on the \emph{PCC-NS-2} dataset an analysis concerning the effect of the latent dimension on the performance (see Fig.~\ref{fig:con:latent_dynamics_sweep_pcc_ns-2}). For this experiment, all hyperparameters were tuned for $n_z=8$, and we observe that the \gls{CON} models have a much-improved consistency and smaller variance w.r.t. the baseline methods when the latent dimensionality is increased.

\subsubsection{Results for reaction-diffusion dataset}
To address the unactuated nature of the \emph{R-D} dataset, we remove, analog to the \emph{M-SP+F}, \emph{S-P+F}, and \emph{D-P+F} datasets, the input-to-state mapping parameters of the dynamical models (e.g., the $B(u)$ and $E(\tau)$ \glspl{MLP} for the \gls{CON} models).
Furthermore, the \gls{PDE} describing the system dynamics is of \nth{1}-order. Therefore, we leverage the \nth{1}-order versions of the latent dynamics, in particular for the \emph{coRNN}, \emph{CON}, \emph{CFA-CON} models, as specified in Apx.~\ref{sub:apx-con:1st_order_latent_dynamics}.
We report the \gls{RMSE} of the test set evaluations in Tab.~\ref{tab:con:latent_dynamics_results}. Furthermore, we also present a sequence of stills of the rollout of a trained latent dynamics \gls{CON} model in Fig.~\ref{fig:con:latent_dynamics:sequence_of_stills:r_d:rollout2}.
We find it impressive that \gls{CON} with its strong stability guarantees can accurately model the dynamics of a high-dimensional \gls{PDE} system.
Still, these initial results show that, for a comparable number of model parameters, the \gls{coRNN} model exhibits a \SI{32}{\percent} better performance than the \gls{CON} model. Therefore, in particular as \gls{coRNN} and \gls{CON} are both oscillatory networks, it would be interesting to analze in future work which characteristic gives \gls{coRNN} the most performance benefits compared to \gls{CON} on this \gls{PDE} datasets.

\begin{figure}[hb]
    \centering
    \subfigure{\includegraphics[width=0.160\columnwidth]{con/figures/results/latent_dynamics/m-sp+f/sequence_of_stills/rollout_3_target_0.00.png}}
    \subfigure{\includegraphics[width=0.160\columnwidth]{con/figures/results/latent_dynamics/m-sp+f/sequence_of_stills/rollout_3_target_0.55.png}}
    \subfigure{\includegraphics[width=0.160\columnwidth]{con/figures/results/latent_dynamics/m-sp+f/sequence_of_stills/rollout_3_target_1.10.png}}
    \subfigure{\includegraphics[width=0.160\columnwidth]{con/figures/results/latent_dynamics/m-sp+f/sequence_of_stills/rollout_3_target_1.65.png}}
    \subfigure{\includegraphics[width=0.160\columnwidth]{con/figures/results/latent_dynamics/m-sp+f/sequence_of_stills/rollout_3_target_2.20.png}}
    \subfigure{\includegraphics[width=0.160\columnwidth]{con/figures/results/latent_dynamics/m-sp+f/sequence_of_stills/rollout_3_target_2.75.png}}
    \\
    \setcounter{subfigure}{0}
    \subfigure[t=\SI{0.0}{s}]{\includegraphics[width=0.160\columnwidth]{con/figures/results/latent_dynamics/m-sp+f/sequence_of_stills/rollout_3_pred_0.00.png}}
    \subfigure[t=\SI{0.55}{s}]{\includegraphics[width=0.160\columnwidth]{con/figures/results/latent_dynamics/m-sp+f/sequence_of_stills/rollout_3_pred_0.55.png}}
    \subfigure[t=\SI{1.10}{s}]{\includegraphics[width=0.160\columnwidth]{con/figures/results/latent_dynamics/m-sp+f/sequence_of_stills/rollout_3_pred_1.10.png}}
    \subfigure[t=\SI{1.65}{s}]{\includegraphics[width=0.160\columnwidth]{con/figures/results/latent_dynamics/m-sp+f/sequence_of_stills/rollout_3_pred_1.65.png}}
    \subfigure[t=\SI{2.20}{s}]{\includegraphics[width=0.160\columnwidth]{con/figures/results/latent_dynamics/m-sp+f/sequence_of_stills/rollout_3_pred_2.20.png}}
    \subfigure[t=\SI{2.75}{s}]{\includegraphics[width=0.160\columnwidth]{con/figures/results/latent_dynamics/m-sp+f/sequence_of_stills/rollout_3_pred_2.75.png}}
    \caption{Prediction sequence of a \gls{CON} model with latent dimension $n_z=4$ trained on the damped harmonic oscillator (\emph{M-SP+F}) dataset~\cite{botev2021priors}. 
    \textbf{Top row:} Ground-truth evolution of the system. \textbf{Bottom row:} Predictions of the \emph{CON} model. \newline
    The prediction model is given three images centered around $t=0$ for encoding the initial latent $z(0)$ and estimation of the initial latent velocity $\dot{z}(0)$. Subsequently, we roll out the autonomous network dynamics (i.e., unforced) and compare the decoded predictions with the ground-truth evolution of the system.  
    }\label{fig:con:latent_dynamics:sequence_of_stills:m-sp+f:rollout3}
\end{figure}

\begin{figure}[hb]
    \centering
    \subfigure{\includegraphics[width=0.160\columnwidth]{con/figures/results/latent_dynamics/d-p+f/sequence_of_stills/rollout_7_target_0.00.png}}
    \subfigure{\includegraphics[width=0.160\columnwidth]{con/figures/results/latent_dynamics/d-p+f/sequence_of_stills/rollout_7_target_0.55.png}}
    \subfigure{\includegraphics[width=0.160\columnwidth]{con/figures/results/latent_dynamics/d-p+f/sequence_of_stills/rollout_7_target_1.10.png}}
    \subfigure{\includegraphics[width=0.160\columnwidth]{con/figures/results/latent_dynamics/d-p+f/sequence_of_stills/rollout_7_target_1.65.png}}
    \subfigure{\includegraphics[width=0.160\columnwidth]{con/figures/results/latent_dynamics/d-p+f/sequence_of_stills/rollout_7_target_2.20.png}}
    \subfigure{\includegraphics[width=0.160\columnwidth]{con/figures/results/latent_dynamics/d-p+f/sequence_of_stills/rollout_7_target_2.75.png}}
    \\
    \setcounter{subfigure}{0}
    \subfigure[t=\SI{0.0}{s}]{\includegraphics[width=0.160\columnwidth]{con/figures/results/latent_dynamics/d-p+f/sequence_of_stills/rollout_7_pred_0.00.png}}
    \subfigure[t=\SI{0.55}{s}]{\includegraphics[width=0.160\columnwidth]{con/figures/results/latent_dynamics/d-p+f/sequence_of_stills/rollout_7_pred_0.55.png}}
    \subfigure[t=\SI{1.10}{s}]{\includegraphics[width=0.160\columnwidth]{con/figures/results/latent_dynamics/d-p+f/sequence_of_stills/rollout_7_pred_1.10.png}}
    \subfigure[t=\SI{1.65}{s}]{\includegraphics[width=0.160\columnwidth]{con/figures/results/latent_dynamics/d-p+f/sequence_of_stills/rollout_7_pred_1.65.png}}
    \subfigure[t=\SI{2.20}{s}]{\includegraphics[width=0.160\columnwidth]{con/figures/results/latent_dynamics/d-p+f/sequence_of_stills/rollout_7_pred_2.20.png}}
    \subfigure[t=\SI{2.75}{s}]{\includegraphics[width=0.160\columnwidth]{con/figures/results/latent_dynamics/d-p+f/sequence_of_stills/rollout_7_pred_2.75.png}}
    \caption{Prediction sequence of a \gls{CON} model with latent dimension $n_z=12$ trained on the double pendulum with friction (\emph{D-P+F}) dataset~\cite{botev2021priors}. 
    \textbf{Top row:} ground-truth evolution of the system. \textbf{Bottom row:} predictions of the \emph{CON} model. \newline
    The prediction model is given three images centered around $t=0$ for encoding the initial latent $z(0)$ and estimation of the initial latent velocity $\dot{z}(0)$. Subsequently, we roll out the autonomous network dynamics (i.e., unforced) and compare the decoded predictions with the ground-truth evolution of the system.  
    }\label{fig:con:latent_dynamics:sequence_of_stills:d-p+f:rollout7}
\end{figure}

\begin{figure}[hb]
    \centering
    \subfigure{\includegraphics[width=0.160\columnwidth]{con/figures/results/latent_dynamics/pcc_ns-3/sequence_of_stills/rollout_4_target_0.00.png}}
    \subfigure{\includegraphics[width=0.160\columnwidth]{con/figures/results/latent_dynamics/pcc_ns-3/sequence_of_stills/rollout_4_target_0.36.png}}
    \subfigure{\includegraphics[width=0.160\columnwidth]{con/figures/results/latent_dynamics/pcc_ns-3/sequence_of_stills/rollout_4_target_0.72.png}}
    \subfigure{\includegraphics[width=0.160\columnwidth]{con/figures/results/latent_dynamics/pcc_ns-3/sequence_of_stills/rollout_4_target_1.08.png}}
    \subfigure{\includegraphics[width=0.160\columnwidth]{con/figures/results/latent_dynamics/pcc_ns-3/sequence_of_stills/rollout_4_target_1.44.png}}
    \subfigure{\includegraphics[width=0.160\columnwidth]{con/figures/results/latent_dynamics/pcc_ns-3/sequence_of_stills/rollout_4_target_1.80.png}}
    \\
    \setcounter{subfigure}{0}
    \subfigure[t=\SI{0.0}{s}]{\includegraphics[width=0.160\columnwidth]{con/figures/results/latent_dynamics/pcc_ns-3/sequence_of_stills/rollout_4_pred_0.00.png}}
    \subfigure[t=\SI{0.36}{s}]{\includegraphics[width=0.160\columnwidth]{con/figures/results/latent_dynamics/pcc_ns-3/sequence_of_stills/rollout_4_pred_0.36.png}}
    \subfigure[t=\SI{0.72}{s}]{\includegraphics[width=0.160\columnwidth]{con/figures/results/latent_dynamics/pcc_ns-3/sequence_of_stills/rollout_4_pred_0.72.png}}
    \subfigure[t=\SI{1.08}{s}]{\includegraphics[width=0.160\columnwidth]{con/figures/results/latent_dynamics/pcc_ns-3/sequence_of_stills/rollout_4_pred_1.08.png}}
    \subfigure[t=\SI{1.44}{s}]{\includegraphics[width=0.160\columnwidth]{con/figures/results/latent_dynamics/pcc_ns-3/sequence_of_stills/rollout_4_pred_1.44.png}}
    \subfigure[t=\SI{1.80}{s}]{\includegraphics[width=0.160\columnwidth]{con/figures/results/latent_dynamics/pcc_ns-3/sequence_of_stills/rollout_4_pred_1.80.png}}
    \caption{Prediction sequence of a forced \gls{CON} model with latent dimension $n_z=12$ trained on the soft robotic \emph{PCC-NS-3} dataset containing trajectories of a simulated piecewise constant curvature robot with three segments. 
    \textbf{Top row:} Ground-truth evolution of the system. \textbf{Bottom row:} Predictions of the \emph{CON-M} model. \newline
    The prediction model is given three images centered around $t=0$ for encoding the initial latent $z(0)$ and estimation of the initial latent velocity $\dot{z}(0)$. Subsequently, we roll out the autonomous network dynamics (i.e., unforced) and compare the decoded predictions with the ground-truth evolution of the system.  
    }\label{fig:con:latent_dynamics:sequence_of_stills:pcc_ns-3:rollout4}
\end{figure}

\begin{figure}[hb]
    \centering
    \subfigure{\includegraphics[width=0.160\columnwidth]{con/figures/results/latent_dynamics/r-d/sequence_of_stills/rollout_2_target_0.00.png}}
    \subfigure{\includegraphics[width=0.160\columnwidth]{con/figures/results/latent_dynamics/r-d/sequence_of_stills/rollout_2_target_1.00.png}}
    \subfigure{\includegraphics[width=0.160\columnwidth]{con/figures/results/latent_dynamics/r-d/sequence_of_stills/rollout_2_target_2.00.png}}
    \subfigure{\includegraphics[width=0.160\columnwidth]{con/figures/results/latent_dynamics/r-d/sequence_of_stills/rollout_2_target_3.00.png}}
    \subfigure{\includegraphics[width=0.160\columnwidth]{con/figures/results/latent_dynamics/r-d/sequence_of_stills/rollout_2_target_4.00.png}}
    \subfigure{\includegraphics[width=0.160\columnwidth]{con/figures/results/latent_dynamics/r-d/sequence_of_stills/rollout_2_target_4.90.png}}
    \\
    \setcounter{subfigure}{0}
    \subfigure[t=\SI{0.0}{s}]{\includegraphics[width=0.160\columnwidth]{con/figures/results/latent_dynamics/r-d/sequence_of_stills/rollout_2_pred_0.00.png}}
    \subfigure[t=\SI{1.0}{s}]{\includegraphics[width=0.160\columnwidth]{con/figures/results/latent_dynamics/r-d/sequence_of_stills/rollout_2_pred_1.00.png}}
    \subfigure[t=\SI{2.0}{s}]{\includegraphics[width=0.160\columnwidth]{con/figures/results/latent_dynamics/r-d/sequence_of_stills/rollout_2_pred_2.00.png}}
    \subfigure[t=\SI{3.0}{s}]{\includegraphics[width=0.160\columnwidth]{con/figures/results/latent_dynamics/r-d/sequence_of_stills/rollout_2_pred_3.00.png}}
    \subfigure[t=\SI{4.0}{s}]{\includegraphics[width=0.160\columnwidth]{con/figures/results/latent_dynamics/r-d/sequence_of_stills/rollout_2_pred_4.00.png}}
    \subfigure[t=\SI{5.0}{s}]{\includegraphics[width=0.160\columnwidth]{con/figures/results/latent_dynamics/r-d/sequence_of_stills/rollout_2_pred_4.90.png}}
    \caption{Prediction sequence of an unforced, \nth{1}-order \gls{CON} model with latent dimension $n_z=4$ trained on the reaction-diffusion (\emph{R-D}) dataset. 
    \textbf{Top row:} Ground-truth evolution of the system. \textbf{Bottom row:} Predictions of the \emph{CON-M} model.
    We roll out the autonomous, \nth{1}-order network dynamics (i.e., unforced) and compare the decoded predictions with the ground-truth evolution of the system.  
    }\label{fig:con:latent_dynamics:sequence_of_stills:r_d:rollout2}
\end{figure}
\section{Exploiting the Dynamic Structure for Latent Space Control}\label{sec:con:latent_space_control}

We consider the problem setting of guiding the system towards a desired observation $o^\mathrm{d}$ by providing a sequence of inputs $u(t_k)$ such that at time $t_N$, the actual observation $o(t_N)$ matches $o^\mathrm{d}$.
% Performing control in latent space drastically reduces the algorithmic requirements as $n_z \ll h_\mathrm{o} \, w_\mathrm{o}$.
A relatively simple way would be to encode the desired observation into latent space $z^\mathrm{d} = \Phi(o^\mathrm{d})$ and then to design a feedback controller (e.g., PID) in latent space: $g(u) = \mathrm{PID}(z^\mathrm{d}-z(t), -\dot{z})$.
Unfortunately, several challenges appear: first, it is not clear how the latent-space force $\tau = g(u)$ can be mapped back to an actual input $u(t)$ as the inverse input-to-forcing mapping $g^{-1}$ is generally not known.
Furthermore, relying entirely on a PID controller has several well-known drawbacks, such as poor and slow transient behavior, steady-state errors (in case the integral gain is chosen to be zero), and instability for high proportional and integral gains.
% Another option would be to formulate a \gls{MPC} optimization problem~\cite{lenz2015deepmpc, hafner2019learning, hewing2020learning, alora2023robust} and find the optimal input sequence on the chosen horizon to guide the system towards $o^\mathrm{d}$. This has disadvantages, too, as the computational requirements are considerable, and it can be difficult to tune cost functions in latent space.
% In this work, we propose a control strategy that can solve or sidestep all the challenges mentioned above.
We take inspiration from potential shaping strategies~\cite{bloch2001controlled, ortega2021pid}, which are widely used for effectively controlling (elastic) robots, and, therefore, combine a feedforward term compensating the latent-space potential forces with an integral-saturated, PID-like feedback term. For mapping the desired forcing $\tau$ back to an input $u(t)$, we train an forcing decoder $\eta: \mathbb{R}^{n} \to \mathbb{R}^{m}$ that approximates $g^{-1}$. Specifically, we consider here the structure $u = \eta(\tau) = E(\tau) \, \tau$, where $E \in \mathbb{R}^{m \times n}$ is parameterized by an \gls{MLP}.

\subsection{Control strategy}
The latent-space control law is given by
\begin{equation}\label{eq:con:controller}
    \tau(t) = g(u) = \underbrace{K_\mathrm{w} \, z^\mathrm{d} + \tanh(z^\mathrm{d} + b)}_{\substack{\text{Feedforward term:}\\ \text{compensation of potential forces}}} + \underbrace{K_\mathrm{p} \, (z^\mathrm{d} - z) - K_\mathrm{d} \, \dot{z} + K_\mathrm{i} \int_0^t \tanh(\upsilon (z^\mathrm{d}(t') - z(t')))\mathrm{d}t'}_{\text{Feedback term: P-satI-D}}
\end{equation}
where $K_\mathrm{p}, K_\mathrm{i}, K_\mathrm{d} \in \mathbb{R}^{n \times n}$ are the proportional, integral, and derivative control gains, respectively.
As integral terms can often lead to instability when applied to nonlinear systems~\cite{stolzle2024experimental}, we adopt an integral term saturation~\cite{pustina2022p} with the associated dimensionless gain $\upsilon \in \mathbb{R}$, which ensures that the integral error added at each time step is bounded to the interval $(-1, 1)$.
% 
Subsequently, $\tau$ is  decoded to the input as $u(t) = \eta(\tau) = E(\tau) \, \tau$.
For training this decoder, we add a reconstruction loss to \eqref{eq:training_loss}: $\mathcal{L}_{\mathrm{u}}(t_k) = \lambda_{\mathrm{u}} \, \mathrm{MSE}(u(t_k), \hat{u}(t_k)) = \lambda_{\mathrm{u}} \, \mathrm{MSE} \left (u(t_k), \eta(g(u(t_k))) \right )$.

\begin{figure}[t]
    \centering
    \subfigure[Samples of used datasets]{\includegraphics[width=0.27\linewidth]{con/figures/datasets_visualization/datatsets_visualization_v2_cropped_compressed.pdf}\label{fig:con:datasets_visualization}}
    \hfill
    \subfigure[Blockscheme of model-based control in latent space]{\includegraphics[width=0.69\linewidth]{con/figures/control/blockdiagram_latent_space_control_v1_cropped.pdf}\label{fig:con:blockdiagram_latent_space_control}}
    \caption{\textbf{Panel (a):} Samples of some of the datasets used as part of the experimental verification, specifically for the results reported in Tab.~\ref{tab:con:latent_dynamics_results}. The real-world Reaction-Diffusion image is adopted from~\cite{epstein2016reaction}. \textbf{Panel (b):} Model-based control in latent space by exploiting the physical structure of the CON model.}
\end{figure}

\subsection{Implementation of closed-loop control}
We generate a trajectory of $7$ setpoints, where $q^\mathrm{d}(t_j) \in \mathbb{R}^{n_\mathrm{q}}$ is a sampled configuration of the actual system.
Then, we render an image $o^\mathrm{d}(t_j)$ that represents the target observation for the controller and encode it into latent space to retrieve $z^\mathrm{d} \in \mathbb{R}^{n_\mathrm{z}}$.
At time step $k$, we render an image $o(t_k)$ of the robot's current configuration $q(t_k)$ and encode the image. 
% Analog to Section~\ref{sec:con:learning_latent_space_dynamics}, we apply finite differences in image space to retrieve an estimate of $\hat{\dot{o}}$ and then project into latent space to capture the latent velocity $\dot{z}$. % that is needed by the control law in case the derivative feedback terms are active.
Subsequently, we evaluate the control law and apply the decoder $u(t_k) = \eta(\tau(t_k)))$, which is finally passed to the simulator that integrates the ground-truth dynamics to the next time-step $t_{k+1}$ considering the actuation $u(t_k)$.
The controller runs at \SI{100}{Hz}, and we simulate the ground-truth dynamics with a \emph{Dopri5} \gls{ODE} integrator at a time-step of \num{5e-4}~\si{s}.

\subsection{Latent-space control of a damped harmonic oscillator}

We consider an actuated version of the \emph{M-SP+F} dataset (i.e., a damped harmonic oscillator) and denote it with \emph{M-SP+F+A}. All system, trajectory sampling and rendering parameters remain the same, except that for each trajectory in the dataset we randomly sample a forcing $u \sim \mathcal{U}(-\SI{1}{N}, \SI{1}{N})$.

We train a \gls{CON} model with latent dimension $n_z = 1$ over three random seeds on the \emph{M-SP+F+A} dataset. This means that the network consists of a single oscillator.
From the three different random seeds, we choose the model that achieves the best validation loss, which results in an RMSE of $0.0327$, a PSNR of $5.99$, and SSIM of $0.9796$ on the test set.

Fig.~\ref{fig:con:control:m-sp+f+a:analysis} shows how the encoder learns an almost linear relationship between the actual configuration of the system and the predicted latent space representation. Furthermore, we notice that both the ground-truth and the learned potential energy are convex and exhibit a global minimum at $q=\SI{0}{m}$.

We compare the performance of \emph{P-satI-D}, \emph{D+FF}, and \emph{P-satI-D+FF} controllers based on the \gls{CON} model in Fig.~\ref{fig:con:control:m-sp+f+a:sequences_comparison}.
For the \emph{P-satI-D} controller, we choose the control gains $K_\mathrm{p} = 10, K_\mathrm{i}=10, K_\mathrm{d} = \num{5}, \upsilon = 1$. The \emph{D+FF} controller uses $K_\mathrm{d} = \num{3.5}$. Finally, the \emph{P-satI-D+FF} is configured with $K_\mathrm{p} = 2, K_\mathrm{i}=0.3, K_\mathrm{d} = \num{3.5}, \upsilon = 1$.
The results show that the \emph{P-satI-D+FF} controller exhibits thanks to its feedforward term no overshooting and a faster response time than the pure feedback controller \emph{P-satI-D}. The high accuracy of the feedfoward term can be seen from the performance of the \emph{D+FF} controller, that only exhibits relatively small steady-state error. Adding small proportional and integral feedback actions in the \emph{P-satI-D+FF} controller keeps the compliance high while removing the steady-state error and reducing the response time.

Finally, we visualize the behavior of the \emph{P-satI-D+FF} controller as a sequence of stills in Fig.~\ref{fig:con:control:m-sp+f+a:con_PsatID+FF:sequence_of_stills}.


\begin{figure}[ht]
    \centering
    \subfigure[Configuration $q$ to latent $z$ mapping]{\includegraphics[width=0.49\columnwidth, trim={5, 10, 5, 5}]{con/figures/results/control/m-sp+f+a/analysis/mapping_configuration_to_latent_space.pdf}}
    \subfigure[Ground-truth and learned potential energy $\mathcal{U}$]{\includegraphics[width=0.49\columnwidth, trim={5, 10, 5, 5}]{con/figures/results/control/m-sp+f+a/analysis/potential_energy_landscape_q.pdf}}
    \caption{
    \textbf{Panel (a):} Learned mapping from configuration to latent space for the CON model with $n_z$ (i.e., consisting of a single oscillator) trained on the actuated damped harmonic oscillator (\emph{M-SP+F+A}) dataset. \textbf{Panel (b):} The blue line represents the ground-truth potential energy of the damped harmonic oscillator. The orange line represents the learned potential energy of the CON model evaluated vs. the system configuration by rendering and subsequently encoding into latent space each configuration value.
    }\label{fig:con:control:m-sp+f+a:analysis}
\end{figure}

\begin{figure}[ht]
    \centering
    \subfigure[Configuration $q(t) \in \mathbb{R}^2$]{\includegraphics[width=0.49\columnwidth, trim={5, 10, 5, 5}]{con/figures/results/control/m-sp+f+a/analysis/setpoint_control_sequences_q.pdf}}
    \subfigure[Latent representation $z(t) \in \mathbb{R}^2$]{\includegraphics[width=0.49\columnwidth, trim={5, 10, 5, 5}]{con/figures/results/control/m-sp+f+a/analysis/setpoint_control_sequences_z.pdf}}\\
    \subfigure[Control input $u(t) \in \mathbb{R}^2$]{\includegraphics[width=0.49\columnwidth, trim={5, 10, 5, 5}]{con/figures/results/control/m-sp+f+a/analysis/setpoint_control_sequences_u.pdf}}
    \subfigure[Potential energy $U(t) \in \mathbb{R}$]{\includegraphics[width=0.49\columnwidth, trim={5, 10, 5, 5}]{con/figures/results/control/m-sp+f+a/analysis/setpoint_control_sequences_potential_energy.pdf}}
    \caption{Latent-space control of an actuated damped harmonic oscillator (\emph{M-SP+F+A}) following a sequence of setpoints. 
    We compare multiple controllers based on a trained \gls{CON} network with $n_z=1$. The \gls{CON} model weights are initialized using a random seed of 0.
    The blue line represents a pure feedback controller (\emph{P-satI-D}). The orange line visualizes the behavior of a feedforward controller with only a damping term applied in feedback (\emph{D+FF}). The green line shows the performance of our proposed combination of feedback and feedforward terms (\emph{P-satI-D+FF}).
    The dotted and solid lines show the reference and actual values, respectively.
    For each setpoint, we randomly sample a desired shape $q^\mathrm{d}$ and render the corresponding image $o^\mathrm{d}$. This image is then encoded to a target latent $z^\mathrm{d}$. The controller then computes a latent-space torque $F^\mathrm{d}$, which is decoded to an input $u$. Finally, we provide this input to the simulator, which performs a roll-out of the closed-loop dynamics.
    Important: The robot's configuration (i.e., the first-principle, minimal-order state) is solely used for generating a target image and simulating the closed-loop system. 
    }\label{fig:con:control:m-sp+f+a:sequences_comparison}
\end{figure}

\begin{figure}[hb]
    \centering
    \subfigure[t=\SI{0.0}{s}]{\includegraphics[width=0.16\columnwidth]{con/figures/results/control/m-sp+f+a/con_psatid+ff/sequences_of_stills/setpoint_sequence_controlled_rollout_actual_0.00.png}}
    \subfigure[t=\SI{0.5}{s}]{\includegraphics[width=0.16\columnwidth]{con/figures/results/control/m-sp+f+a/con_psatid+ff/sequences_of_stills/setpoint_sequence_controlled_rollout_actual_0.50.png}}
    \subfigure[t=\SI{1.0}{s}]{\includegraphics[width=0.16\columnwidth]{con/figures/results/control/m-sp+f+a/con_psatid+ff/sequences_of_stills/setpoint_sequence_controlled_rollout_actual_1.00.png}}
    \subfigure[t=\SI{1.5}{s}]{\includegraphics[width=0.16\columnwidth]{con/figures/results/control/m-sp+f+a/con_psatid+ff/sequences_of_stills/setpoint_sequence_controlled_rollout_actual_1.50.png}}
    \subfigure[t=\SI{2.0}{s}]{\includegraphics[width=0.16\columnwidth]{con/figures/results/control/m-sp+f+a/con_psatid+ff/sequences_of_stills/setpoint_sequence_controlled_rollout_actual_2.00.png}}
    \subfigure[Target $0-4.9\si{s}$]{\includegraphics[width=0.16\columnwidth]{con/figures/results/control/m-sp+f+a/con_psatid+ff/sequences_of_stills/setpoint_sequence_controlled_rollout_desired_0.00.png}}
    \\
    \subfigure[t=\SI{5.00}{s}]{\includegraphics[width=0.16\columnwidth]{con/figures/results/control/m-sp+f+a/con_psatid+ff/sequences_of_stills/setpoint_sequence_controlled_rollout_actual_5.00.png}}
    \subfigure[t=\SI{5.5}{s}]{\includegraphics[width=0.16\columnwidth]{con/figures/results/control/m-sp+f+a/con_psatid+ff/sequences_of_stills/setpoint_sequence_controlled_rollout_actual_5.50.png}}
    \subfigure[t=\SI{6.0}{s}]{\includegraphics[width=0.16\columnwidth]{con/figures/results/control/m-sp+f+a/con_psatid+ff/sequences_of_stills/setpoint_sequence_controlled_rollout_actual_6.00.png}}
    \subfigure[t=\SI{6.5}{s}]{\includegraphics[width=0.16\columnwidth]{con/figures/results/control/m-sp+f+a/con_psatid+ff/sequences_of_stills/setpoint_sequence_controlled_rollout_actual_6.50.png}}
    \subfigure[t=\SI{7.5}{s}]{\includegraphics[width=0.16\columnwidth]{con/figures/results/control/m-sp+f+a/con_psatid+ff/sequences_of_stills/setpoint_sequence_controlled_rollout_actual_7.00.png}}
    \subfigure[Target $5-9.9\si{s}$]{\includegraphics[width=0.16\columnwidth]{con/figures/results/control/m-sp+f+a/con_psatid+ff/sequences_of_stills/setpoint_sequence_controlled_rollout_desired_5.00.png}}
    \caption{Sequence of closed-loop control of an actuated damped harmonic oscillator (\emph{M-SP+F+A}) with a \emph{P-satI-D+FF}controller  based on a trained \gls{CON} with $n_z=1$.
    \textbf{Columns 1-4:} show the actual behavior of the closed-loop system. \textbf{Column 5:} demonstrates the target image that the control sees for all time instances in the row.
    }\label{fig:con:control:m-sp+f+a:con_PsatID+FF:sequence_of_stills}
\end{figure}

\subsection{Latent-space control of a two segment PCC soft robot}

\subsubsection{Experimental setup.}
% After initializing the neural network weights with three different random seeds, 
We train a \gls{CON} model with two latent variables ($n_z = 2$) on the \emph{PCC-NS-2} dataset.
Analog to the input encoder mapping $B(u)$, the forcing decoder mapping $E(\tau)$ is parametrized by an \gls{MLP} consisting of five layers with hidden dimension $30$ and a hyperbolic tangent activation function.
The \emph{CON} model achieves an \gls{RMSE} of $0.1628$ on the test set.
\subsubsection{Model selection}
For the control experiments, we train instances of the \emph{MECH-NODE} and \emph{CON-M} models with latent dimension $n_z = 2$ and with neural network weights initialized with three different random seeds. 
We found that model-based control does not perform as well when the latent stiffness $\Gamma_\mathrm{w}$ (as visualized in Fig.~\ref{fig:con:control:pcc_ns-2:potential_energy_landscape_z}) is significantly larger along one of the Eigenvectors than along the other one. Therefore, we evaluate the Eigenvalues of the learned stiffness matrix in $\mathcal{W}$-coordinates after training: $\lambda_{1,2}(\Gamma_\mathrm{w})$. Particularly, we choose the seed that minimizes the normalized standard deviation of the Eigenvalues: $\mathrm{seed} = \arg\min \frac{\sigma_\lambda}{\mu_\lambda}$, where
\begin{equation}
\begin{split}
    \mu_\lambda = \frac{\lambda_1(\Gamma_\mathrm{w}) + \lambda_2(\Gamma_\mathrm{w})}{2},
    \quad
    \sigma_\lambda = \sqrt{\frac{(\lambda_1(\Gamma_\mathrm{w})-\mu_\lambda)^2 + (\lambda_2(\Gamma_\mathrm{w})-\mu_\lambda)^2}{2}}.
\end{split}
\end{equation}


We benchmark two controllers on the simulated continuum soft robot consisting of two segments: (i) a pure \emph{P-satI-D} controller (i.e., the feedback term in \eqref{eq:con:controller}) that leverages the smooth mapping into the latent representation enabled by the \gls{CON} dynamic model and the $\beta$-\gls{VAE}, and (ii) a \emph{P-satI-D+FF} (i.e., \eqref{eq:con:controller}) that exploits the structure of the \gls{CON} dynamics by compensating for the potential forces.
The stable closed-loop system dynamics made the control gain tuning very easy, and we selected $K_\mathrm{p} = 1, K_\mathrm{i}=2, K_\mathrm{d} = 0.02, \upsilon = 1$ for the \emph{P-satI-D} controller and $K_\mathrm{p} = 0, K_\mathrm{i}=2, K_\mathrm{d} = 0.05, \upsilon = 1$ for the \emph{P-satI-D+FF} controller, respectively.

Furthermore, we compare the control performance of our model-based controllers with a baseline control strategy based on the \emph{MECH-NODE} ($n_z = 2$) that achieves an error of $0.1104$ on the test set.
For \emph{MECH-NODE}, we chose the model with the lowest validation loss (seed $0$)
First, we utilize the same P-satI-D feedback controller as for the \gls{CON} model to generate the control action $\tau(t)$ in latent space. 
As the MECH-NODE uses an \gls{MLP} to parameterize the function $\dot{\xi} = f_\xi(\xi, u)$, we cannot easily map $\tau(t)$ into an input $u(t)$. 
Therefore, we linearize the latent space dynamics w.r.t to the input as $f_{\xi,\mathrm{ac}}(\xi, u) = f_\xi(\xi, 0) + A(\xi) \, u$, where $A(\xi) = \frac{\partial f_\xi}{\partial u}(\xi, 0)$ is computed using autodiff. Then, $u(t) = A^\mathrm{T}(\xi) \, \tau(t)$. After tuning the control gains, we choose  $K_\mathrm{p} = 0.001, K_\mathrm{i}=0.02, K_\mathrm{d} = \num{1e-5}, \upsilon = 1$.
% Therefore, we leverage auto differentiation to linearize the system dynamics w.r.t. the input $u$

\subsubsection{Potential energy landscape}
When leveraging (learned) dynamical models for setpoint regulation, it is essential to accurately estimate the potential energy as this dictates the efficacy of the feedforward terms. Therefore, we qualitatively evaluate the potential energy landscape of the \gls{CON} latent dynamic model.

In Fig.~\ref{fig:con:control:pcc_ns-2:potential_energy_landscape_z}, we can see how \gls{CON} contains a single, isolated, and globally asymptotically stable equilibrium as proven in Section~\ref{sec:con:con}, respectively. 
% 
Furthermore, we want to verify that the learned potential corresponds to the actual potential energy of the simulated system.
An autonomous continuum soft robot with the tip pointing downwards in a straight configuration exhibits an isolated, globally asymptotically stable equilibrium at $q = 0$ (i.e., zero strains)~\cite{stolzle2021piston}.
For this purpose, we can compare the learned potential energy field in Fig.~\ref{fig:con:control:pcc_ns-2:potential_energy_landscape_q} with the ground-truth potential energy field in Fig.~\ref{fig:con:control:pcc_ns-2:potential_energy_landscape_q_gt}.
We confirm, based on Fig.~\ref{fig:con:control:pcc_ns-2:potential_energy_landscape_q}, that, indeed, the learned potential also has its minimum close to/at $q=0$. Although the field is shaped slightly differently, the potential forces are clearly pointing inwards towards the global attractor.

\begin{figure}[ht]
    \centering
    \subfigure[Learned potential energy as a function of $z$]{\includegraphics[width=0.5\columnwidth, trim={5, 10, 5, 5}]{con/figures/results/control/pcc_ns-2/potential_energy_landscape/potential_energy_landscape_z.pdf}\label{fig:con:control:pcc_ns-2:potential_energy_landscape_z}}\\
    \subfigure[Learned potential energy as a function of $q$]{\includegraphics[width=0.49\columnwidth, trim={20, 10, 20, 20}]{con/figures/results/control/pcc_ns-2/potential_energy_landscape/potential_energy_landscape_q.pdf}\label{fig:con:control:pcc_ns-2:potential_energy_landscape_q}}
    \subfigure[Ground-truth potential energy as a function of $q$]{\includegraphics[width=0.49\columnwidth, trim={20, 10, 20, 20}]{con/figures/results/control/pcc_ns-2/potential_energy_landscape/potential_energy_landscape_q_gt.pdf}\label{fig:con:control:pcc_ns-2:potential_energy_landscape_q_gt}}
    \caption{Potential energy landscapes of a \gls{CON} with $n_z=2$ trained to learn the latent space dynamics of a continuum soft robots (simulated with two \gls{PCC} segments). \textbf{Panel (a):} Here, we visualize the learned potential energy of \gls{CON} using the color scale as a function of the latent representation $z = x_\mathrm{w} \in \mathbb{R}^2$. The arrows denote the gradient of the potential field $\frac{\partial \mathcal{U}}{\partial z}$ (i.e., the potential force), with the magnitude of the gradient expressed as the length of the arrow. \textbf{Panel (b):} Again, we display the learned potential energy of \gls{CON} using the color scale, but in this case, as a function of the configuration $q \in \mathbb{R}$ of the robot (that is hidden from the model). First, we render an image $o$ of the shape of the robot for each configuration $q = \begin{bmatrix}
        q_1 & q_2
    \end{bmatrix}^\mathrm{T} \in \mathbb{R}^2$. Then, we encode the image into latent space as $z = \Phi(o)$. This allows us then to compute the potential energy $\mathcal{U}(z)$ of the \gls{CON} latent dynamics model. \textbf{Panel (c):} Here, we display the potential energy and its associated potential forces of the actual (i.e., simulated) system.}
\end{figure}

\begin{figure}[ht]
    \centering
    \subfigure[Config. for P-satI-D with MECH-NODE.]{\includegraphics[width=0.45\columnwidth, trim={5, 10, 5, 5}]{con/figures/results/control/pcc_ns-2/mech_node_psatid/setpoint_control_sequence_q.pdf}}
    \subfigure[Latent for P-satI-D with MECH-NODE.]{\includegraphics[width=0.45\columnwidth, trim={5, 10, 5, 5}]{con/figures/results/control/pcc_ns-2/mech_node_psatid/setpoint_control_sequence_z.pdf}}\\
    \subfigure[Configuration for P-satI-D with \gls{CON}.]{\includegraphics[width=0.45\columnwidth, trim={5, 10, 5, 5}]{con/figures/results/control/pcc_ns-2/con_psatid/setpoint_control_sequence_q.pdf}}
    \subfigure[Latent for P-satI-D with \gls{CON}.]{\includegraphics[width=0.45\columnwidth, trim={5, 10, 5, 5}]{con/figures/results/control/pcc_ns-2/con_psatid/setpoint_control_sequence_z.pdf}}\\
    \subfigure[Configuration for P-satI-D+FF with \gls{CON}.]{\includegraphics[width=0.45\columnwidth, trim={5, 10, 5, 5}]{con/figures/results/control/pcc_ns-2/con_psatid+ff/setpoint_control_sequence_q.pdf}}
    \subfigure[Latent for P-satI-D+FF with \gls{CON}.]{\includegraphics[width=0.45\columnwidth, trim={5, 10, 5, 5}]{con/figures/results/control/pcc_ns-2/con_psatid+ff/setpoint_control_sequence_z.pdf}}\\
    \caption{Latent-space control of a continuum soft robot (simulated using two piecewise constant curvature segments) at following a sequence of setpoints: The upper two rows show the performance of a pure P-satI-D feedback controller operating in latent space $z$ learned with the MECH-NODE and \gls{CON} models, respectively. The lower row displays the results for a latent space controller based on the \gls{CON} model that additionally also compensates for the learned potential forces.}\label{fig:con:latent_space_control_results}
\end{figure}

\begin{figure}[hb]
    \centering
    % image size: 575x575px
    \subfigure[t=\SI{0.0}{s}]{\includegraphics[width=0.16\columnwidth]{con/figures/results/control/pcc_ns-2/con_psatid+ff/sequences_of_stills/setpoint_sequence_controlled_rollout_actual_0.00.png}}
    \subfigure[t=\SI{0.2}{s}]{\includegraphics[width=0.16\columnwidth]{con/figures/results/control/pcc_ns-2/con_psatid+ff/sequences_of_stills/setpoint_sequence_controlled_rollout_actual_0.20.png}}
    \subfigure[t=\SI{0.4}{s}]{\includegraphics[width=0.16\columnwidth]{con/figures/results/control/pcc_ns-2/con_psatid+ff/sequences_of_stills/setpoint_sequence_controlled_rollout_actual_0.40.png}}
    \subfigure[t=\SI{0.6}{s}]{\includegraphics[width=0.16\columnwidth]{con/figures/results/control/pcc_ns-2/con_psatid+ff/sequences_of_stills/setpoint_sequence_controlled_rollout_actual_0.60.png}}
    \subfigure[t=\SI{0.8}{s}]{\includegraphics[width=0.16\columnwidth]{con/figures/results/control/pcc_ns-2/con_psatid+ff/sequences_of_stills/setpoint_sequence_controlled_rollout_actual_0.80.png}}
    \subfigure[Target $0-4.9\si{s}$]{\includegraphics[width=0.16\columnwidth]{con/figures/results/control/pcc_ns-2/con_psatid+ff/sequences_of_stills/setpoint_sequence_controlled_rollout_desired_0.00.png}}
    \\
    \subfigure[t=\SI{5.00}{s}]{\includegraphics[width=0.16\columnwidth]{con/figures/results/control/pcc_ns-2/con_psatid+ff/sequences_of_stills/setpoint_sequence_controlled_rollout_actual_5.00.png}}
    \subfigure[t=\SI{5.2}{s}]{\includegraphics[width=0.16\columnwidth]{con/figures/results/control/pcc_ns-2/con_psatid+ff/sequences_of_stills/setpoint_sequence_controlled_rollout_actual_5.20.png}}
    \subfigure[t=\SI{5.4}{s}]{\includegraphics[width=0.16\columnwidth]{con/figures/results/control/pcc_ns-2/con_psatid+ff/sequences_of_stills/setpoint_sequence_controlled_rollout_actual_5.40.png}}
    \subfigure[t=\SI{5.6}{s}]{\includegraphics[width=0.16\columnwidth]{con/figures/results/control/pcc_ns-2/con_psatid+ff/sequences_of_stills/setpoint_sequence_controlled_rollout_actual_5.60.png}}
    \subfigure[t=\SI{5.8}{s}]{\includegraphics[width=0.16\columnwidth]{con/figures/results/control/pcc_ns-2/con_psatid+ff/sequences_of_stills/setpoint_sequence_controlled_rollout_actual_5.80.png}}
    \subfigure[Target $5-9.9\si{s}$]{\includegraphics[width=0.16\columnwidth]{con/figures/results/control/pcc_ns-2/con_psatid+ff/sequences_of_stills/setpoint_sequence_controlled_rollout_desired_5.00.png}}
    \caption{Sequence of closed-loop control of a continuum soft robot consisting of two constant curvature segments with the \emph{P-satI-D+FF} based on a trained \gls{CON} with $n_z=2$.
    \textbf{Columns 1-4:} show the actual behavior of the closed-loop system. \textbf{Column 5:} demonstrates the target image that the control sees for all time instances in the row.
    }\label{fig:con:control:pcc_ns-2:con_PsatID+FF:sequence_of_stills}
\end{figure}

\subsubsection{Results} 
% We report the mean and standard deviation over three different model initializations in the following.
As an evaluation metric, we consider the \gls{RMSE} between the actual and the reference trajectory.
% settling time~\cite{tay2012high} required to reach and remain within \SI{20}{\percent} of the step averaged across all setpoints
The \emph{P-satI-D} applied to the \emph{MECH-NODE} model (baseline) achieves an \gls{RMSE} of \SI{2.88}{rad \per m} w.r.t. to the desired configuration $q^\mathrm{d}$ (but unknown to the algorithm).
The \emph{P-satI-D} \gls{CON} controller, which does not exploit the learned latent dynamics for control, exhibits an \gls{RMSE} of \SI{4.08}{rad \per m} w.r.t. to the desired configuration $q^\mathrm{d}$.
The \emph{P-satI-D+FF} controller exhibits an \gls{RMSE} of \SI{2.12}{rad \per m} w.r.t. to the desired configuration $q^\mathrm{d}$.
We also visualize the closed-loop trajectories in Fig.~\ref{fig:con:latent_space_control_results} and as sequences of stills in Fig.~\ref{fig:con:control:pcc_ns-2:con_PsatID+FF:sequence_of_stills}.
We conclude that the nicely structured latent space generated by the $\beta$-VAE allows the \emph{P-satI-D} controller to effectively regulate the system towards the setpoint, although the response time is rather slow. The \emph{P-satI-D+FF} controller is able to exploit the structure of the \gls{CON} model through its potential shaping feedforward term.
With that, \emph{CON P-satI-D+FF} exhibits a faster response time and a \SI{26}{\percent} lower \gls{RMSE} than the \emph{MECH-NODE P-satI-D} baseline.
% Importantly, the closed-loop system remained stable in all experiments we conducted.

\section{Conclusion}\label{sec:backstepping:conclusion}
%
This chapter proposed a model for soft robots actuated using pneumatic fluidic drive cylinders and introduced a model-based controller to take actuators' dynamics into account. The stability of this backstepping-based control strategy was proven using a Lyapunov argument.  As an example of application, model and control strategy have been specialized for the planar \gls{PCC}-case. We also proposed a coupling-aware extension of the standard hierarchical PID strategy as a middle-ground solution. %Our simulations show the feasibility and convergence properties of the nonlinear backstepping controller for the set-point control. % of a \gls{PCC}-modeled soft robotic arm consisting of three segments. 
% While the backstepping controller is agnostic to any changes in the actuation system parameters, the PID controller would require an extensive re-tuning of its gains.
%
Future work will focus on applying this strategy to more sophisticated models and controllers and on experimental validation in a lab environment. %Also, we will consider different kind of actuation dynamics.

\section*{Afterword}
In this chapter, we introduced \glsxtrfull{CON} and leveraged them for learning mechanically interpretable latent dynamics of physical systems.
The ability to derive kinetic and potential energy terms of the \gls{CON} i) allows us to provide strong stability guarantees (e.g., \gls{ISS}), and ii) enables us to apply computationally efficient model-based control strategies established in the (soft) robotics community such as P-satI-D+feedforward directly in latent space.
% Even though, compared to Chapter~\ref{chp:pcsregression}, the learned dynamics do not map one-to-one to a physics-based soft robot model anymore, the latent dynamics having a physical (in this case mechanical) interpretation allows us a) enforce similar stability characteristics in the learned system that the original physical system also exhibits, and b) to apply control strategies that were originally developed to be used with fully physics-derived models. These characteristics and features motivate the benefits of \emph{grey-box}, specifically ones that exhibit physical structure and stability guarantees.
Although the learned dynamics no longer map directly to a physics-based soft robot model as they did in Chapter~\ref{chp:pcsregression}, the fact that the latent dynamics retain a physical (in this case, mechanical) interpretation offers several advantages: (a) it enables us to enforce stability characteristics in the learned system similar to those observed in the original physical system, and (b) it allows the application of control strategies originally designed for fully physics-derived models. These attributes underscore the benefits of \emph{grey-box} approaches, particularly those that incorporate physical structure and provide stability guarantees.

% However, the presented strategy of combining an encoding into latent space with a latent dynamical system that exhibits certain physical structures and stability guarantees is not only useful for modeling but also for motion planning. In recent years, there has been an increasing community trend to learn motion policies from demonstration with stability guarantees by combining diffeomorphic encoders with latent dynamics parametrized by stable dynamical systems in the framework of \gls{SMP}~\citep{rana2020euclideanizing, perez2023stable, zhi2024teaching}. In Chapter~\ref{chp:osmp}, we extend this approach to learning period motions and demonstrate on multiple real (soft) robots how this approach is able to generate natural, compliant, and safe robot behavior.
The proposed approach of integrating an encoding into a latent space with a latent dynamical system that incorporates specific physical structures and stability guarantees proves valuable not only for learning models but also for motion planning. In recent years, there has been a growing trend in the research community to develop motion policies with stability guarantees by leveraging diffeomorphic encoders and latent dynamics represented by stable dynamical systems within the framework of \gls{SMP}\citep{rana2020euclideanizing, perez2023stable, zhi2024teaching}. In Chapter~\ref{chp:osmp}, we expand this method to learn periodic motions and demonstrate its ability to generate natural, compliant, and safe behavior across various real (soft) robots.

