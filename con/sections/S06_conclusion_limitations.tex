\section{Conclusion and Limitations}\label{sec:con:conclusion_and_limitations}

\subsection{Conclusion} In this work, we propose a new formulation for a coupled oscillator that is inherently input-to-state stable. Additionally, we identify a closed-form approximation, that is
% , for the same or smaller computational costs in situations where the dynamics are dominated by linear, decoupled terms, 
able to simulate the network dynamics more accurately compared to numerical \gls{ODE} integrators with similar computational costs.
% We leverage the network to learn the latent dynamics of complex nonlinear mechanical systems, in this case specifically a continuum soft robot, directly from pixels.
When learning latent dynamics with \gls{CON}, we observe that the performance is on par or slightly better compared to SoA methods such as \glspl{RNN}, \glspl{NODE}, etc., even though we constrained the solution space to a \gls{ISS}-stable coupled oscillator structure. Furthermore, we point out that the performance of the \gls{CON} models is more consistent across latent dimensions compared to the baselines and improved when not specifically tuned for a given dimension.
Furthermore, as seen in Tab.~\ref{tab:apx-con:latent_dynamics_results:pcc_ns-2}, the closed-form approximation achieves, with the same number of model parameters, similar accuracies and double the training speed w.r.t. to the continuous-time model.
Finally, we demonstrate that even a simple PID-like latent-space controller can effectively regulate the system to a setpoint. By exploiting the network structure and compensating for potential forces, regulation performance can be greatly improved, and response time decreased by more than \SI{55}{\percent}.

\subsection{Limitations}
% While we think our proposed method shows great potential and opens interesting avenues for future research, there are currently certain limitations. For example, our stability requirements ask for $M_\mathrm{w}, K_\mathrm{w} \succ 0$ such that there exists only a unique and isolated equilibrium. This might be too restrictive for systems that experience multiple equilibria and could be alleviated by relaxing the conditions on the network parameters while still preserving the physical structure.
While we think our proposed method shows great potential and opens interesting avenues for future research, there exist certain limitations. For example, the proposed method of learning (latent) dynamics implicitly assumes that the underlying system adheres to the Markov property (e.g., the full state of the system is observable), that a system with mechanical structure can approximate it, and that it has an isolated, globally asymptotically stable equilibrium. This is, for example, the case for many mechanical systems (e.g., some continuum soft robots, deformable objects, and elastic structures) with continuous dynamics, convex elastic behavior, dissipation, and whose time-dependent effects (e.g., viscoelasticity, hysteresis) are negligible. Even if these conditions are not met globally, the method can be applied to model the local behavior around an asymptotic equilibrium point of the system (e.g., robotic manipulators, legged robots) with added stability benefits for out-of-distribution samples. Alternatively, the method could be extended to relax some of these assumptions, e.g., by allowing for multiple equilibria, zero damping, or by incorporating additional terms to capture discontinuous dynamics (e.g., stick-slip models) or period motions (e.g., limit cycles such as the Van der Pol oscillator). The proposed method might not be suitable for some physical systems, such as nonholonomic systems, partially observable systems, or systems with non-Markovian properties. Examples of such systems include mobile robots and systems with hidden states or delayed observations. % Finally, applying this method to non-physical systems, such as financial systems, social networks, or other complex systems, is out of the scope of this work.

Furthermore, the approximated closed-form solution shows the best integration for situations where linear, decoupled dynamics dominate the transient. For dominant nonlinear, coupled forces, the performance of \gls{CFA-CON} degrades, and it might be better to revert to numerical integration of the \gls{CON} ODE.
Finally, the control works exceptionally well in the setting where the latent dimension equals the input dimension. We hypothesize that this enables the method to identify a diffeomorphism between the input and the latent-space forcing. 
Still not investigated is how the performance could degrade if $n_z > m$ (or $n_z < m$ for that matter).