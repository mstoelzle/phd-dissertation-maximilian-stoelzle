% \section{Conclusion}
% In this work, we present a data-driven method that utilizes the PCS strain model to derive low-dimensional kinematic and dynamic models for continuum soft robots from discrete backbone pose measurements, outperforming ML-based models like neural networks by maintaining the physical robot structure. This enhancement improves data efficiency and performance beyond the training set, allowing for direct and effective model-based control design. 
% %Unlike traditional methods that require extensive expert input and complex identification processes, our approach automates the determination of key model parameters using a robust linear least-squares regression, providing significant efficiency gains. 
% %We validated our learned models through simulations that demonstrated their ability to accurately predict robot dynamics and control from minimal data inputs, showing superior accuracy and generalization capabilities compared to existing methods. 
% Future work will explore expanding this approach to 3D models and real-world applications, aiming to further refine the actuation matrix for underactuated systems.

\section{Conclusion}
In this work, we propose a data-driven approach leveraging the \gls{PCS} strain model to identify low-dimensional kinematic and dynamic models for continuum soft robots directly from discrete pose measurements of the backbone's shape.
Compared to \gls{ML}-based models (e.g., neural networks, symbolic regression), we preserve the physical structure of the continuum soft robot model, leading to a significantly improved training data efficiency and out-of-training-distribution performance.
Furthermore, our method preserves the physical structure, enabling fast and efficient model-based control design.
Compared to deriving and formulation continuum soft robot models by hand using \gls{PCC}, \gls{PCS}, etc. approximations, our approach requires (i) less expert knowledge as the number of segments, segment lengths, active strains, etc. are automatically determined and (ii) we can regress all dynamic parameters with closed-form linear least-squares which is more efficient and robust than traditionally used system identification procedures (e.g., constrained nonlinear least squares, determination of parameters using mechanical testing equipment).

We verified both the \emph{Kinematic Fusion} and the \emph{Dynamic Regression and Sparsification} algorithms in simulation. The \emph{Kinematic Fusion} method can automatically and accurately determine the number of planar \gls{PCS} segments and their respective length purely based on extracted SE(2) poses of the backbone shape. 
For continuum soft robots whose shape by definition cannot be represented by the \gls{PCS} model (e.g., soft robots exhibiting affine curvature), we formulated a Pareto front between the \gls{DOF} of the model and the shape reconstruction accuracy. This enables the user to easily choose the best kinematic model for a given computational budget.
We showed that our proposed method is able to derive accurate dynamic models from just \SI{4}{s} of video recordings and can generalize to out-of-distribution actuation sequences, which is not the case for the \gls{ML}-based baseline methods that we considered (e.g., \gls{NODE}, \gls{LNN}~\citep{liu2024physics}, and \gls{CON}~\citep{stolzle2024input}).
Furthermore, even for samples inside the training distribution, our method exhibits a more than \SI{70}{\percent} lower shape prediction error than the baselines.
Finally, we demonstrated that the physical structure of the dynamical model that remains intact during the regression allows us to leverage the learned dynamics within closed-form model-based control in a plug-and-play fashion.
For future work, it would be interesting to validate the proposed approach on both 3D (i.e., $SE(3)$ input poses) and real-world data. Furthermore, it would be valuable to identify ways to regress a possibly underactuated actuation matrix $A(q) \in \mathbb{R}^{n_\mathrm{q} \times m}$, where $m$ is the number of actuators.