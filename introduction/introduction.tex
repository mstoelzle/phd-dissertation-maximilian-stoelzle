\chapter{Introduction}
\label{chp:introduction}

\textit{
% This chapters positions the dissertation within the societal context and specifically the challenges that we are facing when bringing robotics into human-centered environments and for robots to assist us with activities of daily living.
% Firstly, we explain how soft robots with their intrinsic safety can contribute to addressing this challenge, although they currently face performance limitations, in particular with respect to their precision.
% Subsequently, we motivate how emplyoing modern machine learning methods with their significant expressiveness could enable learning much more accurate dynamic models of the complex behavior that soft robots exhibit. In order to preserve insight into the decision making, stability guarantees, and the computational efficiency of the controller, we aim to integrate physical structure into learned models such that we can leverage them within closed-form model-based controllers.
% Next, we present the research questions that this thesis aims to tackle including the contributions that this thesis makes to contribute to address those research questions.
% Finally, we outline the structure of this thesis and connect chapters to relevant topics, the contributions, and the strength of physical priors within the used models.
This chapter situates the dissertation within the broader societal context, addressing the challenges of integrating robotics into human-centered environments and assisting with daily living activities. First, we explain how soft robots, with their inherent safety, can help overcome these challenges despite current performance limitations—particularly regarding precision. Next, we argue that employing modern machine learning methods, with their significant expressiveness, could enable the development of far more accurate dynamic models of the complex behaviors exhibited by soft robots. To maintain transparency in decision-making, ensure stability guarantees, and preserve computational efficiency in the controller, our approach integrates physical structure into the learned models for use in closed-form model-based controllers. We then present the research questions this thesis aims to tackle, along with the corresponding contributions. Finally, we outline the structure of the thesis, connecting each chapter to the relevant topics, contributions, and the strength of physical priors within the employed models.
}

%% Start the actual chapter on a new page.
\newpage

\section{Motivation}\label{sec:introduction:motivation}
% \dropcap{T}his is a introductory page.

% The last decades have shown great progress in robotics, with performant, precise, and powerful rigid robots now being widely adopted~\citep{todd1996fundamentals}, particularly in manufacturing scenarios where repetitive movements are prevalent within assembly line manufacturing, but more and more also in domains that require a larger degree of autonomy, such as warehouse logistics, automated inspections, etc.
% However, to address pressing societal challenges, there is a growing demand for robotic systems that are specifically designed for and adaptable to human-centered environments (e.g., homes and public spaces)~\citep{nahavandi2019industry, chibani2013ubiquitous, royakkers2015literature, he2021challenges}.
% To unlock their full potential, robots must be designed with inherent physical compliance to operate safely around humans in dynamic, unconstrained, unpredictable settings.
% Such requirement aligns with Asimov's First Law--robots must never harm humans--making safety fundamental to their design and deployment~\citep{villani2018survey}.
\dropcap{O}ver the past few decades, the domain of robotics has made remarkable strides, leading to the widespread adoption of high-performance, precise, and powerful rigid robots~\citep{todd1996fundamentals}. While these systems have traditionally excelled in manufacturing environments with repetitive tasks—such as assembly lines—they are increasingly finding use in applications that require more autonomy, including warehouse logistics and automated inspections. Yet, in addressing contemporary societal needs, there is a heightened demand for robots specifically crafted for human-centered settings, e.g., homes and public spaces~\citep{nahavandi2019industry, chibani2013ubiquitous, royakkers2015literature, he2021challenges}. Fully realizing the potential of such robots calls for integrating inherent compliance, enabling safe interaction with humans in dynamic, unconstrained, and unpredictable contexts. This requirement aligns with Asimov’s First Law~\citep{asimov1941three} — robots must never harm humans, making safety a paramount consideration in both their design and deployment~\citep{villani2018survey}.

% While safety is traditionally ensured through computational control policies, this approach is vulnerable to perception errors and often results in overly cautious behavior that limits robot performance. Soft robotics presents a promising alternative by establishing passive compliance throughout the entire robot body with material softness. This embodied intelligence is inherently resistant to perception or control
% errors. Recent years have witnessed remarkable progress in
% soft robotics, with researchers developing new designs, smart
% materials, actuators, sensors, models, and control approaches.

% Existing strategies mostly try to ensure safety via computational intelligence; for example, they advanced control mechanisms such as safety filters or impedance controllers, real-time collision detection, and predefined safety zones~\citep{zhao2024potential}. 
% For this task, they rely on sophisticated sensors, algorithms, and extensive pre-programming to anticipate and avoid hazards \citep{fragapane2021planning}.
% For example, collision detection enhances safety by stopping or slowing the robot upon contact. While reducing the worst-case injury likelihood, this approach is inherently reactive and cannot fully prevent injuries.
% However, in case of sensor failure or errors in the perception and/or reasoning pipeline, safety cannot be guaranteed anymore.
% There has also been work to improve safety through hardware.
% For example, \gls{Cobot}~\citep{el2019cobot}, designed to safely interact with humans, often use series elastic actuators to decouple the actuator from the link dynamics~\citep{pratt1995series} and aim to reduce the link inertia~\citep{albu2007dlr}.
% Yet, their rigid body still poses significant risks to nearby humans~\citep{haddadin2013towards}.
% In order to reduce safety risks, design standards such as ISO/TS 15066:2016~\citep{iso2016collaborative} demand that \gls{cobot} move slowly enough to stop in time to prevent collisions, limiting their effectiveness and performance~\citep{ajoudani2018progress, lucci2020combining}.
% In conclusion, the current strategies pose high requirements on perception systems and significantly reduce the robots/cobots' efficiency and effectiveness in the context of human-centered environments.
Existing approaches to ensuring safety largely depend on computational intelligence~\citep{ahn2024autort, sermanet2025generatingrobotconstitutions}, employing advanced control strategies, e.g., safety filters~\citep{ames2016control} and impedance controllers~\citep{khatib1987unified}, real-time collision detection~\citep{haddadin2017robot}, and predefined safety zones~\citep{zhao2024potential}. These methods leverage sophisticated sensing, algorithms, and extensive pre-programming to predict and mitigate potential hazards~\citep{fragapane2021planning}. For instance, collision detection improves safety by stopping or slowing the robot upon contact. Although this measure helps reduce the risk of serious injury, it is inherently reactive and cannot entirely eliminate harm. Moreover, any sensor malfunction or failure in the perception or decision-making pipeline undermines safety guarantees.
Hardware-based solutions have also been developed to enhance safety. For example, \glspl{Cobot}~\citep{el2019cobot}, which are designed for safe human-robot interaction, typically incorporate series elastic actuators to isolate actuator dynamics from the robot’s links~\citep{pratt1995series} and minimize link inertia~\citep{albu2007dlr}. Still, their rigid structures pose considerable risks to nearby individuals~\citep{haddadin2013towards}. To address these dangers, standards such as ISO/TS 15066:2016~\citep{iso2016collaborative} mandate that \glspl{Cobot} move slowly enough to stop before colliding with a person, thereby constraining performance and effectiveness~\citep{ajoudani2018progress, lucci2020combining}. Ultimately, current safety measures place heavy demands on perception systems and substantially reduce the efficiency and capabilities of robots and \glspl{Cobot} in human-centered environments.

\begin{figure}[ht]
    \centering
    \includegraphics[width=0.8\linewidth]{introduction/figures/thesis_topics_venn_v2.pdf}
    \caption{Topics covered in this thesis: leveraging learned models for the control of soft robots.}
    \label{fig:introduction:topics_venn}
\end{figure}

% Interestingly, soft robotics redefines safety from the ground up. 
% In these robots~\citep{rus2015design, laschi2016soft}, safety is not an add-on or managed through computational stacks but is embedded in the material and structural properties of the robot itself. 
% Their compliant nature allows soft robots to interact safely with humans and operate in sensitive environments where safety is essential, such as personal assistance, caregiving, and handling delicate objects and produce~\citep{abidi2017intrinsic}. 
% Recent years have witnessed remarkable progress in soft robotics~\citep{yasa2023overview}, with researchers developing new designs~\citep{laschi2012soft, hawkes2017soft, guan2023trimmed, katzschmann2018exploration, tolley2014resilient}, smart materials~\citep{terryn2017self, mazzolai2022roadmap}, actuators~\citep{shepherd2013using, vasios2020harnessing, lipton2018handedness, gravert2024low, wehner2016integrated, aubin2022towards}, sensors~\citep{larson2016highly, thuruthel2019soft, truby2020distributed}, models~\citep{renda2018discrete, boyer2020dynamics, renda2020geometric}, and control~\citep{thuruthel2018model, della2020model, jitosho2023reinforcement, pustina2024input} approaches. 
% Despite their potential to revolutionize human-robot interaction~\citep{jorgensen2022soft}, soft robots face significant challenges hindering their full integration into practical applications~\citep{hawkes2021hard}. 
% Particularly, current soft robots often exhibit imprecise and often oscillatory motion~\citep{mazzolai2022roadmap, majidi2014soft, hawkes2017soft}.
% The reason is that modeling~\citep{armanini2023soft} and control~\citep{della2023model} of continuum soft robots presents significant challenges due to their infinite degrees of freedom, complex nonlinear dynamics, and time-dependent behaviors such as hysteresis.
Soft robotics reimagines safety from the ground up by embedding it into the robot’s fundamental mechanical design. Rather than treating safety as an add-on or relying solely on computational layers, soft robots incorporate safety through the choice of materials and structural configurations~\citep{rus2015design, laschi2016soft}. Their inherent compliance facilitates safe interactions with humans\footnote{
% We note, though, that the inherent safety of soft robots has, to the best of the author's knowledge, not yet rigorously quantified, experimentally analyzed and validated, for example, within controlled user studies, or compared to the safety of (collaborative) rigid manipulators. As a first step towards quantifying the (added) safety of soft robots, we propose a safety metric in Chapter~\ref{chp:safetymetric} of this thesis.
To the best of our knowledge, the inherent safety of soft robots has not yet been rigorously quantified, experimentally analyzed, or validated in literature—for instance, through controlled user studies or by comparing them to (collaborative) rigid manipulators. In Chapter~\ref{chp:safetymetric}, we introduce a safety metric as a first step toward quantifying the (added) safety of soft robots.
} 
and makes them ideal for safety-critical settings like personal assistance, caregiving, and handling delicate items~\citep{abidi2017intrinsic, yumbla2021human}.
Recent advances in the field have been substantial~\citep{yasa2023overview}, with innovations in soft robot designs~\citep{laschi2012soft, hawkes2017soft, guan2023trimmed, katzschmann2018exploration, tolley2014resilient}, smart materials~\citep{terryn2017self, mazzolai2022roadmap}, actuators~\citep{shepherd2013using, vasios2020harnessing, lipton2018handedness, gravert2024low, wehner2016integrated, aubin2022towards}, sensors~\citep{larson2016highly, thuruthel2019soft, truby2020distributed}, modeling methods~\citep{renda2018discrete, boyer2020dynamics, renda2020geometric}, and control techniques~\citep{thuruthel2018model, della2020model, jitosho2023reinforcement, pustina2024input}. Despite this progress and the potential to transform human-robot interaction~\citep{jorgensen2022soft}, widespread practical adoption of soft robots remains a challenge~\citep{hawkes2021hard}. In particular, many soft robots still struggle with imprecise, often oscillatory motion~\citep{mazzolai2022roadmap, majidi2014soft, hawkes2017soft}. This is largely because modeling~\citep{armanini2023soft} and controlling~\citep{della2023model} continuum soft robots from first principles is inherently difficult; they exhibit infinite degrees of freedom, complex nonlinear dynamics, and time-dependent phenomena such as hysteresis~\citep{armanini2023soft}, and might make extensive large-area multi-point contact with the environment.
% In summary, we find that current soft robots are not sufficiently capable, and specifically precise, which, similarly to safety-aware controllers for rigid robots that instill a very cautious and slow motion behavior, we are again paying performance for safety.
In summary, our findings indicate that current soft robots lack sufficient capability—particularly in precision—and, much like safety-aware controllers for rigid robots that enforce cautious, slow motions, we are again sacrificing performance for safety.
Therefore, we see a pressing need for novel motor control strategies for soft robots that combine the inherent compliance and embodied intelligence of soft robots with the precise motion characteristic of rigid robotic manipulators

\begin{figure}[ht]
    \centering
    \includegraphics[width=1.0\linewidth]{introduction/figures/model_based_control_with_physical_models_v1_cropped.pdf}
    \caption{\textbf{Existing Approach: Model-Based Control with Physical Models.}
    % Derivation of physical models from first principles and subsequent usage within closed-form model-based control schemes, such as PID+energy-shaping~\cite{della2023model}.
    % Expert knowledge is needed to derive kinematic and dynamical models for soft robots from first principles. It is often possible to describe the dynamics in a control-oriented Euler-Lagrangian form. Subsequently, the model knowledge can be leveraged within closed-form controllers consisting of a feedback term (e.g., a PID-like controller) and a model-based term (e.g., an energy-shaping feedforward) that are tracking a given reference~\citep{della2023model}. Finally, the control input needs to be mapped into in an actuation that can be applied to the continuum soft robot in closed loop.
    Expert knowledge is required to derive kinematic and dynamic models for soft robots from first principles~\citep{armanini2017elastica}. The dynamics can often be expressed in a control-oriented Euler-Lagrangian framework. This model knowledge is then integrated into closed-form controllers that combine an error-based feedback term (e.g., a PID-like controller) with a model-based term (e.g., an energy-shaping feedforward) to track a given reference~\citep{della2020model, caasenbrood2021energy, della2023model}. Finally, the control input must be translated into an actuation that exerts forces and torques on the continuum soft robot in a closed loop.
    }
    \label{fig:introduction:model_based_control_with_physical_models}
\end{figure}

\begin{figure}[ht]
    \centering
    \includegraphics[width=0.75\linewidth]{introduction/figures/controller_learning_v1_cropped.pdf}
    \caption{\textbf{Existing Approach: Direct Controller Learning.} 
    Directly learning the controller via \gls{RL}~\citep{morimoto2021model, jitosho2023reinforcement, alessi2024pushing} or \gls{ILC}~\citep{hofer2019iterative, pierallini2023provably} by interacting with the soft robotic system and optimizing the control policy based on reward/loss/error signals.
    }
    \label{fig:introduction:direct_controller_learning}
\end{figure}

\begin{figure}[ht]
    \centering
    \includegraphics[width=0.9\linewidth]{introduction/figures/optimal_control_learned_models_v1_cropped.pdf}
    \caption{\textbf{Emerging Approach: Optimal Control with Learned Models.} 
    First, the dynamic model is learned via \gls{ML} approaches~\citep{gillespie2018learning, xie2023dynamic, alora2023data, alora2023robust, kasaei2023data, liu2024physics, chen2024data, marques2024visuo}, and subsequently, the control input is devised via optimal control techniques, such as \gls{LQR}~\citep{bruder2020data, bruder2024koopman} or \gls{MPC}~\citep{gillespie2018learning, hewing2020learning, alora2023data, alora2023robust}, by iteratively predicting the evolution of the closed-loop system for a given control input sequence over a horizon $H$ and using the error/loss between the predicted and desired behavior to optimize the control input.
    }
    \label{fig:introduction:optimal_control_with_learned_models}
\end{figure}

% At the moment, we can identify three dominant approaches for controlling the complex behavior of soft robots~\citep{thuruthel2018control, della2023model}:
% (1) leveraging fully physics-based models~\citep{armanini2023soft} (e.g., strain models~\citep{alessi2024rod}) for control, often using nonlinear feedback (e.g., PD+~\citep{della2020model}) or feedback+feedforward (e.g., PD+energy-shaping~\citep{della2023model, caasenbrood2023control} control schemes, (2) learning a model and subsequently exploiting it for control, be it via optimal control (e.g., \gls{LQR}~\citep{bruder2020data, haggerty2023control}, \gls{MPC}~\citep{gillespie2018learning, thuruthel2017learning, alora2023robust, alora2023data}, gradient-descent-based optimization~\citep{bern2020soft}), or model-based \gls{RL}~\citep{thuruthel2018model, centurelli2022closed}, or (3) directly learning the controller via model-free \gls{RL}~\citep{morimoto2021model, jitosho2023reinforcement, alessi2024pushing} or \gls{ILC}~\citep{hofer2019iterative, pierallini2023provably}.
% However, all strategies exhibit significant deficiencies and shortcomings:
At present, two main approaches can be distinguished for controlling the complex behavior of soft robots~\citep{thuruthel2018control, della2023model}. The first (1), visualized in Fig.~\ref{fig:introduction:model_based_control_with_physical_models}, relies on fully physics-based models~\citep{armanini2023soft}, for example using strain models~\citep{alessi2024rod}, and typically involves nonlinear feedback, e.g., PD+~\citep{della2020model}, or error-based feedback+feedforward, e.g., PID+energy-shaping~\citep{della2023model, caasenbrood2023control, soleti2025model}, control schemes. % The second (2) consists in learning a model and then exploiting it for control, whether through optimal control methods (e.g., \gls{LQR}~\citep{bruder2020data, haggerty2023control}, \gls{MPC}~\citep{gillespie2018learning, thuruthel2017learning, alora2023robust, alora2023data}, or gradient-descent-based optimization~\citep{bern2020soft, marques2024visuo}) or via model-based \gls{RL}~\citep{thuruthel2018model, centurelli2022closed}. 
The second (2) approach, visualized in Fig.~\ref{fig:introduction:direct_controller_learning}, is to learn the controller directly in a model-free fashion~\citep{george2017learning}, for example, through model-free \gls{RL}~\citep{morimoto2021model, jitosho2023reinforcement, alessi2024pushing} or \gls{ILC}~\citep{hofer2019iterative, pierallini2023provably}. However, both of these strategies exhibit significant drawbacks.

% (1) the derivation and design of fully physics-based models requires a lot of expert knowledge~\citep{stella2023science} and many simplifying assumptions, such as that the continuum soft robot's body is \emph{slender} (i.e., the backbone radius is much smaller than its length)~\citep{cosserat1909theorie} and can be approximated effectively by a low-dimensional kinematic parametrization~\citep{armanini2023soft}, backbone cross-sections preserve a constant area~\citep{gazzola2018forward}, the material is isotropic, exhibits Hookean behavior and does not display time-dependent characteristics such as hysteresis, etc. As a result, fully physics-based models do not sufficiently capture the complex behavior of advanced soft robots, which in turn leads to unsatisfactory of control performance.
% Approach (2) - learning linear dynamical models, for example, following a Koopman approach~\citep{bruder2020data, bruder2024koopman} such that they can be controlled via \gls{LQR} is often unstable, exhibits an inherent trade-off between dimensionality and expressiveness, and cannot accurately capture the behavior of non-linearizable systems~\citep{cenedese2022data}. On the other hand, learning the dynamics using nonlinear functions, and in particular \glspl{DNN}, for example, with \glspl{MLP}, \glspl{RNN}~\citep{thuruthel2018model, sun2022physics}, \glspl{LSTM}~\citep{xie2023dynamic}, or \glspl{NODE}~\citep{kasaei2023data} is a more expressive approach, but control becomes computationally very expensive techniques such as \gls{MPC} or model-based \gls{RL} need to be used. Furthermore, we usually lack a physical interpretation of the learned model, preventing us from analyzing the stability of the open-loop or closed-loop system using standard techniques, such as Lyapunov arguments~\citep{khalil2002nonlinear}.
% The third existing approach, (3), in the case of \gls{RL} sample inefficient and lacks stability guarantees, and in the case of \gls{ILC} can only learn repetitive trajectories~\citep{bristow2006survey}.
% In particular, the sample inefficiency is a huge issue for soft robots as they exhibit changing material properties rendering previously learned controllers to be suboptimal, and a currently a limited lifetime~\citep{yasa2023overview} - which might mean that the soft robot might be damaged or broken before the (\gls{RL}) controller is fully trained.
Developing fully physics-based models, as used in (1), demands substantial expert knowledge~\citep{stella2023science} and often hinges on strong simplifying assumptions, such as the continuum soft robot being slender, i.e., the backbone radius is much smaller relative to its length~\citep{cosserat1909theorie}, allowing for a low-dimensional kinematic parametrization~\citep{armanini2023soft}, preserving a constant backbone cross-section~\citep{gazzola2018forward}, and assuming isotropic Hookean materials without time-dependent effects like hysteresis. These assumptions fail to capture the intricate physics of many soft robots, undermining control performance. 
% Furthermore, it seems currently infeasible to model the joint dynamics of a soft robot interacting intensively with the world around it (e.g., collaborating with humans, being in contact with the environment over large-areas at multiple points along the body, predicting the evaluation of the world in the spirit of world models~\citep{ha2018world}) accurately just using first principles from physics therefore motivating the need to adapt learning-based solutions.
Furthermore, accurately modeling from physics using only first principles the joint dynamics of a soft robot that interacts intensively with the environment—whether by collaborating with humans, making extensive large-area multi-point contact with its surroundings, or predicting the environment dynamic evaluation in the spirit of world models~\citep{ha2018world}—appears currently infeasible, thereby motivating the need for learning-based solutions.
Discussing approach (2): Model-free \gls{RL}, meanwhile, tends to be sample inefficient and does not guarantee stability, whereas \gls{ILC} can only be used for learning repetitive trajectories~\citep{bristow2006survey}. The sample inefficiency is particularly problematic for soft robots since their material properties change over time—rendering previously learned controllers less effective—and their limited lifespans~\citep{yasa2023overview} risk the robot being damaged or failing altogether before a controller learned through \gls{RL} is fully trained.
The lack of insight into the decision-making process and the lack of stability guarantees can cause safety issues, particularly when frequently operating in close proximity to humans.

% To counteract the disadvantages of approaches (1) and (2), there have been in recent years steps to combine the benefits of \gls{ML} and model-based control by first learning a model, for example, using a \gls{NN}, and subsequently exploiting it for control.
% However, this increase of model expressiveness through \gls{ML} techniques has come at the cost of computational control efficiency as closed-form controllers are usually not available, and we need to resort to optimization-based strategies.
To overcome the limitations of approaches (1) and (2), research in recent years has combined the strengths of \gls{ML} and model-based control by first learning a model—often using a \gls{NN}—and then exploiting it for control~\citep{gillespie2018learning, thuruthel2018model, bruder2020data, bruder2024koopman, alora2023data, alora2023robust, chen2024data}. However, this enhanced model expressiveness through \gls{ML} techniques comes at the expense of computational efficiency, as closed-form controllers are typically unavailable, and one must rely on optimal control/optimization-based strategies (see Fig.~\ref{fig:introduction:optimal_control_with_learned_models}).
For example, learning linear dynamical models—such as those based on a Koopman approach~\citep{bruder2020data, bruder2024koopman} often proves unstable, suffers from the trade-off between model dimensionality and expressiveness, and cannot accurately represent non-linearizable systems~\citep{cenedese2022data}. On the other hand, using nonlinear functions (notably \glspl{DNN}, including \glspl{MLP}, \glspl{RNN}~\citep{thuruthel2018model, sun2022physics}, \glspl{LSTM}~\citep{xie2023dynamic}, or \glspl{NODE}~\citep{kasaei2023data} provides more expressive models but requires computationally heavy control techniques such as optimal control with \gls{MPC}~\citep{gillespie2018learning, aswani2013provably, kabzan2019learning, hewing2020learning, alora2023data, alora2023robust} or \glspl{CBF}~\citep{taylor2020learning}, or model-based \gls{RL}~\citep{thuruthel2018model} as they typically lack a clear physical structure that would allow, for example, for energy shaping-based or feedback-linearization control~\citep{khalil2002nonlinear} and that would permit stability analyses of both the open-loop and closed-loop system by conventional means, such as Lyapunov arguments~\citep{khalil2002nonlinear}. 
% The high computational demand required for \gls{MPC} limits the maximum control frequence that can be realized in practice - therefore also limiting the ability to control highly dynamic behaviors effectively.
The high computational demand associated with \gls{MPC} limits the maximum control frequency achievable in practice, thereby restricting the effective control of highly dynamic behaviors.

\begin{figure}[ht]
    \centering
    \includegraphics[width=0.9\linewidth]{introduction/figures/model_based_control_with_learned_models_v2_cropped.pdf}
    \caption{\textbf{Core Contribution of this Thesis.} 
    % Closed-form model-based control with learned models that exhibit a physical structure.
    % The core contribution of this thesis is to combine learned models with closed-form controllers. To make this possible, we learn the dynamical models with a physical structure which makes it possible for us to derive the potential and kinetic energy of the learned system. Apart from the specific structure of the dynamical model, the model learning follows a standard approach as established in literature by predicting the future evolution of the soft robotic system, comparing the prediction with a dataset of actual motion data, and using this prediction error to optimize the free parameters of the learned dynamical model.
    % Moving to control, we reshape the potential energy of the closed-loop system by adding a feedforward term that regulates the system towards a given reference~\citep{della2023model}. This potential shaping feedforward term is complemented by an integral-saturated PID feedback term, named P-satI-D~\citep{pustina2022p}, that rejects disturbances and compensates for modeling errors. Finally, the control input is mapped into an actuation, which is potentially also learned.
    % The insight into the potential and kinetic energy systems enables us to analyze the stability and convergence characteristics of both open-loop and closed-loop systems using Lyapunov arguments~\citep{khalil2002nonlinear}.
    The primary contribution of this thesis is the integration of learned models with closed-form controllers. To enable this, we learn dynamical models with a physical structure that allows us to derive the system’s potential and kinetic energy. Beyond the specific dynamical model structure, the learning process follows a well-established approach in the literature: we predict the future evolution of the soft robotic system, compare these predictions with actual motion data, and use the resulting prediction error to optimize the free parameters of the learned model.
    % 
    On the control side, we modify the closed-loop system’s potential energy by incorporating a feedforward term that steers the system toward a designated reference~\citep{della2023model}. This potential shaping feedforward is complemented by an integral-saturated PID feedback term, known as P-satI-D~\citep{pustina2022p}, which rejects disturbances and compensates for modeling errors. Finally, the control input is translated into an actuation signal, where this mapping may also be learned. This insight into the potential and kinetic energy permits an analysis of the stability and convergence properties of both open-loop and closed-loop systems using Lyapunov arguments~\citep{khalil2002nonlinear}.
    }
    \label{fig:introduction:model_based_control_with_learned_models}
\end{figure}

% \begin{figure}[ht]
%     \centering
%     \subfigure[Closed-form Control with Physical Models]{\includegraphics[width=0.49\linewidth]{introduction/figures/model_based_control_with_physical_models_v1_cropped.pdf}\label{fig:introduction:model_based_control_with_physical_models}}\\
%     \subfigure[Learning the Controller]{\includegraphics[width=0.44\linewidth]{introduction/figures/controller_learning_v1_cropped.pdf}\label{fig:introduction:direct_controller_learning}}
%     \hfill
%     \subfigure[Closed-form Control with Learned Models]{\includegraphics[width=0.50\linewidth]{introduction/figures/model_based_control_with_learned_models_v2_cropped.pdf}\label{fig:introduction:model_based_control_with_learned_models}}
%     \caption{Overview of existing soft robot control approaches (Panels (a)-(c)) and the control strategy proposed in this thesis (Panel (d)).
%     \textbf{Panel (a):} Derivation of physical models from first principles and subsequent usage within closed-form model-based control schemes, such as PID+energy-shaping~\cite{della2023model}.
%     \textbf{Panel (c):} Directly learning the controller via \gls{RL}~\citep{morimoto2021model, jitosho2023reinforcement, alessi2024pushing} or \gls{ILC}~\citep{hofer2019iterative, pierallini2023provably} by interacting with the soft robotic system and optimizing the control policy based on reward/loss/error signals.
%     }
%     \label{fig:introduction:soft_robot_control_approaches}
% \end{figure}

% In this thesis, we aim to avoid all these deficiencies by combining learned models with closed-form control (see Fig.~\ref{fig:introduction:model_based_control_with_learned_models}).
% By integrating learned models into established feedback+feedforward control strategies, we combine the best of both worlds: learning models require less expert knowledge and allow us to capture mode complex dynamical effects while the closed-form feedback+feedforward controller is computationally extremely efficient.
% However, in order to apply feedback+feedforward control strategies, such as PD+energy-shaping~\citep{della2023model, caasenbrood2023control}, we need to be able to inspect the kinetic and potential energies of the learned model, which is generally not possible for existing popular \gls{ML} architectures such as \glspl{RNN}, \glspl{MLP}, or \glspl{NODE}.
% Therefore, we notice that the learned models need to exhibit a physical structure in order for us to be able to apply common model-based control schemes, such as PD+ or PD+feedforward, and to guarantee the stability of the closed-loop system using Lyapunov arguments.
In this thesis, we seek to circumvent the aforementioned shortcomings by integrating learned models with closed-form control, as visualized in Fig.~\ref{fig:introduction:topics_venn}. By embedding learned models within established closed-form model-based control strategies, such as nonlinear feedback~\citep{della2020model} or error-based feedback+energy shaping~\citep{della2023model, caasenbrood2023control}, as visualized in Fig.~\ref{fig:introduction:model_based_control_with_learned_models}, we benefit from both paradigms: learned models demand less specialized expertise while capturing more complex dynamical behaviors, and the closed-form model-based controller remains computationally very efficient. However, to apply such approaches, it is necessary to inspect the kinetic and potential energies of the learned model—something that existing popular \gls{ML} architectures like \glspl{RNN}, \glspl{MLP}, or \glspl{NODE} generally do not allow. Consequently, it becomes clear that the learned models must exhibit a physical structure in order to employ common closed-form model-based control schemes and to secure closed-loop stability via Lyapunov arguments.

% In recent years, there has been an exciting development in this area with the nascent literature on \glspl{LNN}~\citep{lutter2019deep, cranmer2020lagrangian, liu2024physics} and \glspl{HNN}~\citep{greydanus2019hamiltonian, liu2024physics} which both explicitly learn the kinetic and potential energy of the system. However, they also exhibit significant drawbacks: the derivation of the \gls{EOM} requires the online computation of higher-order derivatives, which is computationally expensive (particularly during training but also during inference) and increases (GPU) memory requirements~\citep{lutter2019deep}. Furthermore, in a naive implementation, they do not exhibit any formal stability guarantees, and the computation of higher-order derivatives can introduce numerical stiffness or instability if not carefully managed (e.g., needing high-precision auto differentiation~\citep{rumelhart1986learning} or careful hyperparameter tuning). Finally, we identify a need for learning models that incorporate inductive biases on the specific dynamics exhibited by soft robots.
Recent advances in this direction include the emerging literature on \glspl{LNN}~\citep{lutter2019deep, cranmer2020lagrangian, liu2024physics} and \glspl{HNN}~\citep{greydanus2019hamiltonian, liu2024physics}, both of which explicitly learn the system’s kinetic and potential energies. Nevertheless, these approaches still present substantial challenges: deriving the \gls{EOM} necessitates real-time higher-order derivatives, which is computationally demanding—particularly during training, but also at inference—and escalates (GPU) memory usage~\citep{lutter2019deep}. Moreover, without careful implementation, they do not offer formal stability guarantees, and the computation of higher-order derivatives can cause numerical stiffness or instability~\citep{greydanus2019hamiltonian} unless managed through precise auto-differentiation~\citep{rumelhart1986learning} or meticulous hyperparameter tuning. Finally, there is a clear need for learned models with additional inductive bias tailored to the specific dynamics of soft robots.

% In conclusion, we tackle in this thesis the problem setting of achieving closed-form control of soft robots, ensuring both precise and safe behavior. The key research challenge here is to integrate physical structure into learned models in order to allow exploitation via established PID-like+energy-shaping controllers and stability analysis via Lyapunov arguments.
In conclusion, this thesis addresses the central challenge of achieving closed-form control of soft robots while maintaining accurate and safe operation. The key research goal is to advance our understanding of the physical structure of soft robots and incorporate it into learned models, thereby enabling the use of effective closed-form controllers and allowing for stability verification via Lyapunov-based methods. In the following section, we will pose the principal research question directly connecting to this identified challenge.


\section{Research Questions}\label{sec:introduction:research_questions}
In this section, we lay out the research questions that guided the research conducted as part of this thesis.
The principal \gls{RQ} that we pose in this thesis is
% \begin{large}
\begin{titled-frame}{\textbf{Principal Research Question}}
    % How can we integrate physical structure and stability guarantees into the sensing and control of soft robots?
    % \noindent How can we effectively leverage learned models to achieve computationally efficient and safe control of soft robots?
    % \noindent How can we computationally efficiently control soft robots via learned models to achieve precise and safe motion behavior?
    % \noindent How can we efficiently control soft robots using learned models, ensuring precise and safe behavior?
    \noindent How can we leverage learned models that integrate physical structure to efficiently control soft robots, ensuring both precise and safe behavior?
\end{titled-frame}
% \end{large}
This principal \gls{RQ} investigates how we can develop (dynamical) models for soft robots with a physical structure that enables the use of closed-form feedback and feedforward control strategies, thereby addressing the research gap identified in Sec.~\ref{sec:introduction:motivation}. To make answering this principal \gls{RQ} more tractable, we break it down into several key \glspl{RQ}, which we list below.

\begin{researchquestion}\label{rq:soft_robotic_safety}
    % What makes soft robots safer than rigid robots, and when are they safe enough?
    % What makes soft robots safer than rigid robots, and under what conditions can they considered safe enough?
    How do we quantify safety in soft robotics?
\end{researchquestion}
% As referenced in Sec.~\ref{sec:introduction:motivation}, one of the key motivators behind soft robotics is that their passive compliance would increase/guarantee safety, which is crucial for deploying soft robots in human-centric environments.
% While a few papers argue this connection between material softness and safety using toy models, such as the Timoshenko beam~\citep{abidi2017intrinsic}, the safety of soft robots has been so far neither a) quantified with a safety metric, b) analyzed in thorough experimental user studies, and c) compared to rigid robots in a quantitative fashion.
% Therefore, we argue that in order to justify the continued investment into soft robotics research, the community needs to start quantifying the added safety of soft robots.
% Or framed differently, quantifying the safety of soft robots would allow us to certify that they meet the safety requirements for specific human-centric applications.
% This \gls{RQ}, which we investigate in Chapter~\ref{chp:safetymetric}, connects to the principal \gls{RQ} by analyzing the safety of the closed-loop system.
As highlighted in Sec.\ref{sec:introduction:motivation}, a key driver of soft robotics is the belief that their passive compliance enhances or guarantees safety, a critical factor for deploying soft robots in human-centric environments~\citep{rus2015design, mengaldo2022concise}. While some studies link material softness to safety using simplified models, such as the Timoshenko beam\citep{abidi2017intrinsic}, the safety of soft robots has yet to be: a) quantified using a safety metric, 
% b) rigorously evaluated through comprehensive experimental user studies, 
or b) compared quantitatively to rigid robots.
To justify continued investment in soft robotics research, we argue that the field must begin quantifying the safety benefits of soft robots. Alternatively, framing the issue differently, quantifying safety would allow us to certify that soft robots meet the safety requirements for specific human-centric applications.

% This \gls{RQ}, which we explore in Chapter~\ref{chp:safetymetric}, ties into the primary \gls{RQ} by evaluating the safety of the closed-loop system.
% The principal \gls{RQ} requires us to ensure both precise and safe behavior. While precision is relatively easy to measure (e.g., the distance to the reference), quantifying the safety is much harder, and this research question will help with that.
The principal \gls{RQ} mandates that we ensure both precise and safe behavior. While precision can be measured relatively easily (e.g., by assessing the distance to a reference), quantifying safety is considerably more challenging—and this research question is designed to tackle that issue.

\begin{researchquestion}\label{rq:shape_sensing}
    % How can we make shape sensing for soft robots more robust and effective by integrating prior (kinematic) knowledge?
    How can we enhance the robustness and effectiveness of shape sensing for soft robots by leveraging prior (kinematic) knowledge?
\end{researchquestion}
Perceiving its own state (\emph{Proprioception}) and the state of the environment (\emph{Exterioception}) are two fundamental tasks for any autonomous robot~\citep{thuruthel2019soft}.
Despite its importance, perception algorithms for soft robots have been relatively less investigated.

% State information is essential for any feedback control policy. More specifically, most soft robot controllers require access to measurements of the configuration space, which, irrespective of the chosen kinematic parametrization, usually approximates the shape of the soft robotic backbone.
% Compared to rigid manipulators, where joint encoders are sufficient to get a good estimate of the robot's state, shape sensing for soft robots is particularly challenging for various reasons, with two primary ones being that a) we would need to measure infinite degrees of freedom theoretically and b) the possibly rigid sensors and electronics should not obstruct the natural softness of the soft robot.
% While there has been a major community effort in recent years toward developing new proprioception \& shape-sensing concepts for soft robots, interpreting the sensor measurements and mapping them to configuration variables remains a significant challenge. Specifically, existing proprioception approaches only rarely~\citep{stella2023soft} leverage prior knowledge, such as the kinematic model, to make the problem more tractable and the algorithms more robust.
% Therefore, we investigate in Chapters~\ref{chp:srslam} \& \ref{chp:promasens} how we can use kinematic priors as an inductive bias when implementing shape sensing algorithm, which would, in turn, improve the performance of feedback controllers, therefore connecting to the principal \gls{RQ}.
In this thesis, we specifically focus on \emph{proprioception}, as accurate information about its own shape is vital for any feedback control strategy. 
In most soft robot control schemes, measuring the configuration space, which is commonly an approximation of the robot’s backbone shape, is essential, regardless of the chosen kinematic parametrization. Unlike rigid manipulators, where joint encoders suffice to estimate the robot’s state, shape sensing in soft robots introduces unique challenges. Two major factors are (a) the theoretically infinite number of degrees of freedom that would need to be measured and (b) the need to avoid compromising the robot’s inherent softness by integrating stiff sensors or electronics.
%
Although the field has made considerable progress in developing new proprioception and shape-sensing methods for soft robots, interpreting sensor readings and mapping them to configuration variables remains a significant hurdle. Notably, existing approaches to proprioception only rarely~\citep{stella2023soft} incorporate prior knowledge, such as a kinematic model, which could make these algorithms more robust and performant.
%
For these reasons, Chapters~\ref{chp:srslam} \& \ref{chp:promasens} investigate how to employ kinematic priors as an inductive bias in shape-sensing algorithms. 
% This integration ultimately boosts the effectiveness of feedback controllers and, thereby, links directly to the principal \gls{RQ} of this thesis.

\begin{researchquestion}\label{rq:actuation_models}
    % Which advanced actuation models are required for exploiting the dynamic behavior of soft robots, and how can we leverage them for control? 
    An aspect of the physical priors that needed a closer look: Which advanced actuation models are necessary for exploiting the dynamic behavior of soft robots, and how can they be utilized for control?
\end{researchquestion}
% Over the last decade, there has been tremendous progress on the modeling~\citep{armanini2023soft} and the control~\citep{della2023model} of soft robots.
% To this day, most of the research has focused on identifying expressive and, at the same, low-dimensional, kinematic, and dynamic models that accurately capture the behavior of the soft robot's backbone shape. Such control-oriented models have proven to be key for improving the performance and effectiveness of model-based controllers.
% In these models, the actuation of the robot is mostly incorporated by the inclusion of actuation-affine terms in the \gls{EOM}, either through a configuration-dependent or even a constant actuation matrix.
% In this thesis, however, we argue that this approach is too simplistic and that more accurate and advanced actuation models are necessary for fully infusing highly dynamic motions into soft robots.
% Indeed, the actuation of soft robots from their rigid counterparts where usually electric actuators directly attack at the joints (i.e., the \glspl{DOF} of the robot). Instead, the actuators of soft robots, be they tendon-driven, pneumatically-actuated, etc., are distributed over the soft robot's body or even contained in its base and apply the actuation forces and torques indirectly to large parts of the soft robot's body, affecting and coupling many \gls{DOF}. 
% Furthermore, modeling and accounting for the dynamics of the actuator itself would allow us to decrease the response time and improve the transient behavior of the soft robot.
% Therefore, Chapters~\ref{chp:hsamodel} \& \ref{chp:backstepping} investigate how we can develop more accurate physics-based actuation models for soft robots considering the cases of a complex auxetic metamaterial-based and pneumatic piston-driven actuation.
% Subsequently, we probe how these advanced actuation models can be exploited within model-based control schemes, ranging from task-space control in Chapter~\ref{chp:hsacontrol} to backstepping control in Chapter~\ref{chp:backstepping}.
% This research aims to highlight the limitations of fully physics-based models regarding their expressiveness in fully capturing the dynamical behavior of soft robots and the needed expert knowledge to design them, motivating the exploration of more expressive and adaptable neural network-based alternatives.
% Furthermore, this gained knowledge about advanced physics-based actuation models can be used in the future as an inductive bias when learning models.
Over the past decade, soft robotics has seen remarkable advances in both modeling~\citep{armanini2023soft} and control~\citep{della2023model}. Much of this research has focused on creating expressive yet low-dimensional models (kinematic and dynamic) that accurately capture a soft robot’s backbone shape, as such models are central to improving the performance of model-based controllers.
%
In many cases, actuation is introduced into these models via actuation-affine terms in the \gls{EOM}, typically through a configuration-dependent or constant actuation matrix~\citep{della2023model}. We argue, however, that this design decision is too restrictive and that more advanced actuation models are required to fully harness the high dynamic capabilities of soft robots. Indeed, the actuators in soft robots—be they tendon-driven, pneumatically powered, or otherwise—often span large parts of the robot’s body or reside in its base, indirectly applying forces and torques that affect multiple \glspl{DOF}. This is fundamentally different from rigid robots, where actuators commonly drive joints directly.
%
Moreover, explicitly modeling the actuator dynamics can reduce response time and improve transient performance. Consequently, in Chapters~\ref{chp:hsamodel} \& \ref{chp:backstepping}, we introduce more accurate physics-based actuation models for soft robots, illustrating the concept with both complex auxetic metamaterial-based and pneumatic piston-driven actuators. We also explore how these advanced models can be applied in model-based control schemes, from task-space control (Chapter~\ref{chp:hsacontrol}) to backstepping control (Chapter~\ref{chp:backstepping}). Through this work, we highlight the limitations of purely physics-based approaches in fully capturing a soft robot’s dynamic behavior, as well as the extensive domain expertise required to design such models—thereby motivating the study of more expressive, neural network-based alternatives. Additionally, these advanced physics-based models can later serve as an inductive bias when learning data-driven models.

\begin{researchquestion}\label{rq:physical_structure_learned_models}
    % Which physical structure do learned models need to exhibit to allow us to apply closed-form model-based control strategies with stability guarantees?
    Which physical structure should learned models exhibit to allow for provably-stable closed-form model-based control?
\end{researchquestion}
% To tackle this \gls{RQ}, we aim to learn about soft robotic models in a data-driven fashion. This would (potentially) increase the expressiveness and accuracy of the model while simultaneously lowering the expert knowledge needed for designing and tuning that is usually required when deriving the models from first principles.
% There already exists literature on establishing soft robot models using machine learning approaches~\citep{armanini2023soft, kim2021review, chen2024data}, spanning from neural network-based approaches~\citep{thuruthel2017learning} such as \glspl{MLP}, \glspl{RNN}~\citep{schafke2024learning} or \glspl{NODE} to \gls{GP}-based approaches~\citep{sabelhaus2021gaussian}.
% However, these models all lack a (physical) structure, which prevents us from leveraging effective feedforward+feedback control concepts and analyzing the stability characteristics of the closed-loop system. Instead, existing work that learned models had to employ computationally expensive \gls{MPC}~\citep{gillespie2018learning, alora2023robust, schafke2024learning} or \gls{RL}~\citep{thuruthel2018model} for control.
% Only very recently, we have seen initial works that establish a physical structure in the learned model: for example, the Koopman operator theory learns linear dynamics, which allows for the application of efficient \gls{LQR}-based control schemes~\citep{bruder2020data}. \glspl{LNN} and \glspl{HNN}~\citep{lutter2019deep} directly learn the energy of the system, thus directly providing us with energy terms that we can exploit for control and stability analysis.
% Still, there remains the need for model learning approaches that incorporate priors tailored to the specific soft robot dynamics.
% Therefore, we tackle in Chapters~\ref{chp:pcsregression} \& \ref{chp:con} the \gls{RQ} of how we can learn models for soft robots that incorporate physical structure such that we can leverage them within closed-form model-based control strategies such as energy-shaping feedforward controllers with PID-like feedback terms.
% This directly helps us answer the principal \gls{RQ} of this thesis.
To address this question, we propose learning soft robot models in a data-driven fashion. This approach can increase both expressiveness and accuracy while reducing the amount of expert knowledge needed when deriving models from first principles. Various machine learning techniques already exist for this purpose~\citep{armanini2023soft, kim2021review, chen2024data}, including neural networks~\citep{thuruthel2017learning} (e.g., \glspl{MLP}, \glspl{RNN}~\citep{schafke2024learning}, \glspl{NODE}~\citep{chen2018neural, kidger2021neural}) and \gls{GP}-based methods~\citep{sabelhaus2021gaussian}. Yet most of these models lack a (physical) structure, preventing us from applying efficient closed-form model-based control methods or from conducting a formal stability analysis of the closed-loop system. As a result, approaches such as \gls{MPC}~\citep{gillespie2018learning, alora2023robust, schafke2024learning}, optimal control, or \gls{RL}~\citep{thuruthel2018model} are often necessary for exploiting the learned models for control, coming with the cost of computational overhead limiting the maximum control frequencies.
%
Recent research has taken steps toward incorporating physical structure into learned models. For example, Koopman operator theory learns linearized dynamics, permitting \gls{LQR}-based control~\citep{bruder2020data}, and \glspl{LNN}/\glspl{HNN}~\citep{lutter2019deep} learn a system’s energy function for use in control or stability analysis. However, more specialized priors are still needed to reflect the unique dynamics of soft robots. Accordingly, in Chapters~\ref{chp:pcsregression} \& \ref{chp:con}, we examine how to learn models with embedded physical structures that enable closed-form model-based control strategies, such as energy-shaping feedforward control augmented by PID-like feedback terms. This endeavor directly supports the principal \gls{RQ} of this thesis.


\begin{researchquestion}\label{rq:compliant_motion_behaviors}
    How can we generate compliant and precise motion behaviors for soft robots?
\end{researchquestion}
% A compliant structure (provided by the material softness) and a compliant low-level controller (e.g., an impedance controller) alone are not sufficient for generating safe and compliant motions around humans as the high-level motion policy providing references and setpoints to the low-level controller can also instill unstable, unnatural/unexpected, or unsafe behavior into the soft robotic system.
% Adding on to this point, motion policies/planning tailored towards soft robots is an underexplored topic in general.
% Therefore, we investigate in Chapters~\ref{chp:braincontrol} \& \ref{chp:osmp} techniques for providing references to low-level controllers that instill compliant, natural, stable, and safe motion.
% Specifically, we consider in Chapter~\ref{chp:braincontrol} \glspl{BMI} and in Chapter~\ref{chp:osmp} motion policies learned from demonstration for communicating setpoints to the low-level controller.
% This research conducted as part of this \gls{RQ} contributes towards the principal \gls{RQ} by establishing safe holistic, and not \emph{just} low-level, control policies for soft robots.
A compliant structure, by virtue of the robot’s soft materials, together with a compliant low-level controller (e.g., impedance control), does not inherently guarantee safe or compliant motion in human-centric settings. High-level motion policies that provide references to the low-level controller may still introduce unstable, unnatural, or unsafe motions. Furthermore, developing motion policies specifically suited to soft robots is, on the whole, an underexplored field.
%
To address this, Chapters~\ref{chp:braincontrol} \& \ref{chp:osmp} introduce methods aimed at producing references that promote compliant, natural, stable, and safe motion. Specifically, Chapter~\ref{chp:braincontrol} investigates \glspl{BMI}, and Chapter~\ref{chp:osmp} examines learning motion policies from demonstration to supply setpoints to the low-level controller. This work contributes to the principal \gls{RQ} by fostering a safe, holistic approach—beyond mere low-level control—to operating soft robots.

\section{Thesis Contributions}\label{sec:introduction:contributions}

In the following, we will detail the key contributions made by this dissertation.
We start with the core contribution, which addresses the \emph{principal \gls{RQ}}.

% \begin{large}
\begin{titled-frame}{\textbf{Core Contribution}}
% \noindent Safe Closed-form Control of Soft Robots with Learned Models
\noindent Safe and Precise Soft Robots via Closed-form Control with Learned Models
% compliant, stable, and efficient?
\end{titled-frame}
% \end{large}
% In this thesis, we devised two approaches for learning kinematic and dynamic models of soft robots that exhibit a physical structure, which we go into more detail about when discussing Contribution~\ref{contrib:learned_models}. As these learned models press well-defined kinetic and potential energy terms, we can leverage well-established feedback+feedforward control strategies originally developed for rigid manipulators~\citep{kelly1995tuning, kelly1996class, kelly1998global, sciavicco2012modelling}.
% Specifically, we combine a potential-shaping feedforward term~\citep{della2023model} with an integral-saturated PID~\citep{pustina2022p} as the feedback term. This control strategy allows us to exploit the learned model knowledge in the feedforward term while simultaneously rejecting disturbances and compensating modeling errors with the feedback term.
% Furthermore, instead of devising the controller via computationally expensive strategies such as \gls{RL} or \gls{MPC} as it typically needs to be done for learned models, the control input is available in closed form.
% Also, this strategy would allow us to inspect the stability of the closed-loop system and prove its stability using Lyapunov arguments~\citep{khalil2002nonlinear}, as we have already done for the open-loop system~\citep{stolzle2024input}.
% These stability guarantees enabled by the interpretability of the learned model are already one step towards ensuring \emph{safe control}. To advance and quantify the safety of soft robot control, we furthermore i) devise a safety metric that allows an analysis of the injury severity that could be caused by the closed-loop system and, particularly, establishing limits on the magnitude of the (proportional) feedback gains to meet the given safety requirements, and ii) develop stable high-level strategies for generating compliant motion behaviors by providing references/setpoints to the low-level controller.
In this thesis, we introduce two approaches for learning kinematic and dynamic models of soft robots that incorporate a physical structure (discussed in more detail under Contribution~\ref{contrib:learned_models}). Because these learned models encode well-defined kinetic and potential energy terms, we can leverage established error-based feedback+energy-shaping control strategies originally designed for rigid manipulators~\citep{kelly1995tuning, kelly1996class, kelly1998global, sciavicco2012modelling}. Specifically, we combine a potential-shaping feedforward term~\citep{della2023model} with an integral-saturated PID~\citep{pustina2022p} for feedback. This setup allows us to utilize the learned model within the feedforward component while still handling disturbances and compensating for modeling inaccuracies via the feedback term.
% 
Moreover, instead of relying on computationally expensive methods such as \gls{RL} or \gls{MPC}, which are often necessary for learned models, our controller provides a closed-form solution. This makes it possible to analyze the characteristics of the closed-loop system and formally prove stability using Lyapunov arguments~\citep{khalil2002nonlinear}, as already demonstrated for the open-loop case~\citep{stolzle2024input}. These stability guarantees—supported by the interpretability of the learned model—constitute an initial step toward \emph{safe control}.
% 
To further promote and quantify safety, we i) propose a safety metric that evaluates potential injury severity from the closed-loop system dynamics and specifically establishes bounds on the proportional feedback gains to ensure safety requirements and ii) develop stable high-level strategies to generate compliant motion behaviors by providing references/setpoints to the low-level controller.

This core contribution is enabled by multiple key contributions, which we detail in the following.

\begin{contribution}\label{contrib:safety_metric}
    A First Quantitative Safety Metric for Soft Robots
\end{contribution}
% In order to answer \gls{RQ}~\ref{rq:soft_robotic_safety}, we devise the first quantitative safety metric for soft robots in Chapter~\ref{chp:safetymetric}.
% To achieve this goal, we first investigate potential applications in the realm of design, control, and certification of such a safety metric. Subsequently, we state a list of requirements that a safety metric would need to (ideally) meet in order to be suitable for these applications.
% Finally, based on the existing, standardized criterion for collaborative robots~\citep{iso2016collaborative}, which considers the maximum contact experienced during a collision between a robot and human as a proxy for injury severity, we devise a quantitative safety metric that takes into account the particular characteristics of soft robots, such as dynamics that consider the continuous deformability of the backbone, elasticity, and contact anywhere along its body.
% The proposed safety metric comes in two flavors: the \gls{SRISC} models the injury severity for a given contact geometry, soft robot configuration, and velocity. The \gls{SRDHC} assess the safety of a soft robotic design under the given operating conditions, such as maximum velocity, maximum actuation torques, etc.
% Finally, we give recommendations for the safe design of soft robotic manipulators.
To address \gls{RQ}~\ref{rq:soft_robotic_safety}, we introduce the first quantitative safety metric for soft robots in Chapter~\ref{chp:safetymetric}. We begin by examining possible uses of such a metric in design, control, and certification. From there, we define a set of requirements that the metric should ideally satisfy to be suitable for these applications.
% 
Building on the existing, standardized criterion for collaborative robots~\citep{iso2016collaborative}—which uses maximum contact pressure during a collision with a human as a proxy for injury severity—we develop a metric that considers the distinctive features of soft robots, including their continuous deformability, elasticity, and the potential for contact along the entire robot body. The proposed safety metric has two variants: the \glspl{SRISC} models injury severity for a given contact geometry, robot configuration, and velocity, while the \glspl{SRDHC} assesses how safe a soft robotic design is under certain operating conditions (e.g., maximum velocity and actuation torques). 
% Finally, we provide recommendations for the safe design of soft robotic manipulators.

% This contribution demonstrates a pathway how in the future the safety of soft robots, and essential goal of this thesis, could be quantified, optimized, ensured via control, and certified.
% Furthermore, this contribution reveals the effect that controllers have on the safety of the closed-loop system - an often overlooked aspect in the domain of soft robots. Therefore, it informs us how we should design compliant controllers that do not compromise the safety of soft robots, by, for example, eliminating integral terms, and minimizing feedback gains, instead relying on an effective feedforward term that requires an accurate dynamical model of the soft robots. 
This contribution outlines a pathway for quantifying, optimizing, ensuring through control, and ultimately certifying the safety of soft robots—a central objective of this thesis. Moreover, it highlights the significant impact controllers have on the safety of the closed-loop system, an aspect often overlooked in soft robotics. Consequently, it guides us toward designing compliant controllers that maintain safety, for instance, by eliminating integral terms and minimizing feedback gains, while instead relying on effective feedforward strategies based on an accurate dynamic model of soft robots.

% \subsection{Leveraging Kinematic Models for Soft
% Robot Shape Sensing}
\begin{contribution}\label{contrib:kinematic_models_shape_sensing}
    Leveraging Kinematic Models for Soft Robot Shape Sensing    
\end{contribution}
% This contribution towards answering \gls{RQ}~\ref{rq:shape_sensing} enables more robust and accurate shape sensing for soft robots by exploiting existing kinematic model knowledge. This improved state estimation is required for the deployment of feedback controllers, and it thus directly contributes to the core contribution of this thesis.
% Below, we detail two approaches that make up this contribution.

% In the first, fully learning-free, method presented in Chapter~\ref{chp:srslam}, we leverage a monocular visual \gls{SLAM} algorithm to obtain multiple pose measurements distributed along the soft robotic backbone from camera images. Subsequently, we formulate an optimization problem that projects these samples of the soft robot's shape onto the kinematic model to obtain configuration estimates. This helps to limit the effects of estimation drift on the shape belief and ensures that we preserve consistency between the pose estimates and the kinematic model. 
% We verify the proposed algorithm both with a simulated \gls{CC} segment and on a real soft segment with an attached low-cost Raspberry Pi camera.
% A major benefit of this approach is that very good shape reconstruction performance is enabled by the combination of four already very established components: 1) low-cost monocular cameras, 2) \gls{vSLAM}, 3) the \gls{PCC}~\citep{webster2010design} kinematic model for soft robots, and 4) optimization based on the Levenberg-Marquardt algorithm~\citep{levenberg1944method, marquardt1963algorithm}.
% In the future, shape estimation performance could be (likely) very easily improved by, for example, replacing the used \gls{vSLAM} algorithm with the current \gls{SOTA} algorithm or by using more expressive kinematic parametrizations (e.g., \gls{GVS}~\citep{renda2020geometric}).


% The second approach appears in Chapter~\ref{chp:promasens} and considers shape proprioception based on magnetic sensors. 
% Specifically, we propose to embed multiple magnets and magnetic sensors into 
% As magnetic fields in 3D can be very complex, a pure physics-based approach is not sufficient anymore. Therefore, we employ modern \gls{ML} approaches to aid us in the interpretation of the magnetic sensor readings. However, instead of learning the mapping from sensor readings to configuration estimates end-to-end, we leverage the kinematic model to simplify the learning problem and improve the data efficiency. Specifically, we use the kinematic model describing the backbone shape to parametrize the spatial relationship between a magnetic sensor and each magnet through a set of a few variables. 
% This then allows us to efficiently and robustly learn to predict the measurements of the magnetic sensor using a neural network. During inference, we then formulate an optimization problem that aims to estimate the soft robot shape by matching the actual and the predicted sensor measurements, which we solve through gradient descent.
% This approach demonstrates how existing prior knowledge, such as a kinematic model, can be used to reduce the size of the \emph{black-box}.

This contribution, which addresses \gls{RQ}~\ref{rq:shape_sensing}, improves the robustness and accuracy of shape sensing in soft robots by harnessing existing kinematic model knowledge.
% We accomplish this by solving nonlinear optimization problems that align the sensor measurements with the backbone shapes attainable by the kinematic model.
We achieve this by solving nonlinear optimization problems that align sensor measurements with the backbone shapes predicted by the kinematic model.
Enhanced state estimation is critical for deploying feedback controllers, thus directly supporting the thesis’s core contribution.
Below, we outline two distinct approaches.

The first method presented in Chapter~\ref{chp:srslam} combines monocular cameras embedded into the soft robot surveilling the environment with \gls{vSLAM} and a projection onto the kinematic model.
Specifically, the \gls{vSLAM} algorithm estimates multiple poses along the soft robot’s backbone based on the camera images. We then solve an optimization problem that projects these pose measurements onto the kinematic model, producing configuration estimates. This strategy mitigates estimation drift and preserves consistency between pose estimates and the kinematic model. We verify the algorithm both in simulation and on a real soft segment equipped with a low-cost Raspberry Pi camera. Its success stems from blending four well-established elements: (1) low-cost monocular cameras, (2) \gls{vSLAM}, (3) the \gls{PCC}~\citep{webster2010design} kinematic model, and (4) Levenberg-Marquardt optimization~\citep{levenberg1944method, marquardt1963algorithm}. Future work could further boost performance simply by, for example, using more advanced \gls{vSLAM} methods or more expressive kinematic parametrizations such as \gls{GVS}~\citep{renda2020geometric}.

The second method, included in Chapter~\ref{chp:promasens}, centers on shape proprioception with magnetic sensors. We embed multiple magnets and magnetic sensors in the robot, which yields complex 3D magnetic fields that are not sufficiently modeled by purely physics-based approaches. Consequently, we combine modern \gls{ML} techniques with a physics-based kinematic model rather than learning the entire sensor-to-configuration mapping in an end-to-end fashion. Specifically, the kinematic model represents the backbone shape and parameterizes the spatial relationship between each sensor and magnet using just a few variables. We then train a neural network to predict sensor readings based on these parameters. During inference, we solve an optimization problem via gradient descent that matches the actual sensor outputs to the model’s predictions, thereby recovering the robot’s shape. This approach illustrates how incorporating a known kinematic model reduces the size of the \emph{black-box} that needs to be learned in a data-driven approach.


\begin{contribution}\label{contrib:actuation_models}
    Soft Robot Control with Advanced Physics-Based Actuation Models
\end{contribution}
% We address \gls{RQ}~\ref{rq:actuation_models} by deriving advanced actuation models from first principles and subsequently leveraging them in model-based controllers by considering two cases, namely \gls{HSA} robots, and pneumatic piston-driven soft robots. We detail each case in the following paragraphs.
% \textcolor{orange}{How does this contribute to the core contribution?}

% \gls{HSA} robots~\citep{lipton2018handedness, chin2018compliant} consist of multiple (usually four) \gls{HSA} rods that are composed in a parallel fashion and connected at the tip. An elongation of the rods causes the robot to elongate and/or bend. However, instead of directly applying axial forces and rotational torques to infuse elongation and bending strains as it is usually done with most soft robot actuation techniques, these deformations are generated by the auxetic metamaterial of the \gls{HSA} rods: applying a torsional torque, for example by attaching an electric servo motor, at the base while constraining the tip causes the rods to twist which in turn is simultaneously transformed by the auxetic metamaterial into elongation of the rod.
% Furthermore, the applied torsional torques can cause the entire \gls{HSA} segment to twist, which is a rare deformation for soft robots.
% This complex actuation technique calls for the development of completely new, specifically actuation, models. However, while related work in recent years has proposed new designs~\citep{good2022expanding, good2025torque}, fabrication techniques~\citep{truby2021recipe}, proprioception algorithms~\citep{zhang2022vision}, and applications~\citep{chen2024real} for \gls{HSA} robots, accurate models~\citep{garg2022kinematic} and control concepts are still missing. In Chapters~\ref{chp:hsamodel} \& \ref{chp:hsacontrol}, we propose and experimentally verify several models and controllers for \gls{HSA} robots, respectively. In the following, we will introduce the contributions in more detail.
% In Chapter~\ref{chp:hsamodel}  of this thesis, we propose new kinematic, dynamic, and actuation models that are tailored to \gls{HSA} robots.
% First, we propose a novel kinematic parameterization, the \gls{SPCS} model, that is able to capture the 3D shape of \gls{HSA} rods with the least possible \gls{DOF}.
% For this purpose, we combine the existing \gls{CS} and \gls{PCS} models~\citep{renda2018discrete} and keep certain strains constant over the entire \gls{HSA} rod length while allowing other strains to vary in a piecewise constant fashion.
% Secondly, as existing work had already experimentally established that the mechanical characteristics, such as stiffness and rest length, of \gls{HSA} rods vary as a function of the actuation (i.e., the twist angle of the rod at the base), we develop a model that is able to capture this particular actuation characteristic and allows easy integration into existing dynamical models/simulators that are based on the \gls{DCM}.
% Thirdly, we implement the second contribution into an extension of PyElastica~\citep{naughton2021elastica} that is able to simulate the behavior of \gls{HSA} robots in 3D.
% Fourthly, we derive a kinematic parameterization based on the \gls{CS} model that accurately captures the deformation of planar \gls{HSA} robots with just three configuration variables. In this realm, we also state for the first time a closed-form solution for the inverse kinematics of a \gls{CS} robot.
% Fifthly, we devise a dynamical model for planar \gls{HSA} robots in Euler-Lagrangian form.
% Subsequently, we exploit in Chapter~\ref{chp:hsacontrol} the proposed models for control. Specifically, we propose two control approaches:
% Firstly, we devise a configuration-space P-satI-D+potential shaping regulator that combines an integral-saturated PID controller~\citep{pustina2022p} with compensation of steady-state forces at the desired setpoint, and that is able to move the end-effector of the planar \gls{HSA} robot towards a desired position in operational space.
% A core challenge here was to develop the necessary to make this more complex actuation model usable within the feedback+feedforward strategy. This, for example, required i) linearization of the actuation terms, 2) a mapping into actuation coordinates~\citep{pustina2024input}, and 3) solving the static inversion problem~\citep{della2025pushing}.
% Secondly, we propose and experimentally verify a operational space impedance controller that cancels the existing soft robot dynamics based on the model knowledge and allows the shaping of the stiffness field through a Cartesian-space PD feedback term. The key challenge here was a) to derive the operational-space dynamics of the \gls{HSA} robot~\citep{khatib1987unified, della2020model} and b) to map desired generalized torques into the underactuated control input.


% Pneumatic/fluidic actuation for soft robots is widely adopted for its fast response time and high and distributed forces/torques~\citep{marchese2015recipe, zaidi2021actuation}.
% While there exist several options for fluidic pressure supplies, such as rotary pumps or high-pressure supplies combined with valve-enabled pressure actuators, fluidic drive cylinders/pistons~\citep{marchese2014design, marchese2016design, parlikar2024concept, malas2024novel} have the advantages of i) a direction relationship between soft robot deformation and piston position and ii) a closed-volume system~\citep{marchese2016design}.
% However, there exists only relatively little research on modeling the behavior of such pneumatic piston-driven soft robots~\citep{marchese2014design, xavier2020modelling}.
% Furthermore, existing model-based soft robot controllers~\citep{della2020model, della2023model} do not take the dynamics of the actuator into account but instead rely on a cascaded control scheme where an outer loop running a (configuration-space) soft robot controller at approximately \SI{100}{Hz} passes pressure references to a faster inner PID-like controller regulating the piston at approximately \SI{1}{kHz}~\citep{marchese2014design}.
% However, this control scheme is only effective when the delay between the time when a pressure setpoint is set and finally reached is relatively small, which is only the case when either i) the electric drive of the piston is very powerful, fast,, and precise, and/or ii) the configuration/pressure setpoints vary relatively slowly.
% Consequently, either very performance, and with that, costly piston actuators are required, or the speed of the dynamic behavior of the soft robot is inherently limited.
% We aim to resolve these issues in Chapter~\ref{chp:backstepping} by modeling the potential energy stored in the fluidic as a function of the soft robot configuration and the piston positions. This allows us to formulate the coupled dynamics between the soft robotic and the piston system in Euler-Lagrangian form. Finally, we employ a backstepping approach~\citep{kokotovic1992joy, lozano1992adaptive, khalil2002nonlinear} to a nonlinear model-based feedback controller for the coupled system that exploits both the soft robot and the piston dynamics.

We address \gls{RQ}~\ref{rq:actuation_models} by deriving advanced actuation models from first principles and then embedding them in model-based controllers. Two case studies guide our work: \gls{HSA} robots and pneumatic piston-driven soft robots. These advanced actuation models feed into the thesis’s core contribution by informing us about the available physical priors for learning models.

\paragraph{HSA Robots.}
\glspl{HSA} robot~\citep{lipton2018handedness, chin2018compliant} typically feature several \gls{HSA} rods arranged in parallel and joined at the tip. Applying torsional torque at the base (e.g., via a servo motor) twists the rods, which in turn elongates and/or bends the robot through the rod’s auxetic metamaterial. This complex actuation mechanism can also induce twisting along the robot’s length, a rare deformation for soft robots. Despite ongoing research into new designs~\citep{good2022expanding, good2025torque}, fabrication processes~\citep{truby2021recipe}, and proprioception~\citep{zhang2022vision}, accurate models~\citep{garg2022kinematic} and control methods are lacking. In Chapters~\ref{chp:hsamodel} and \ref{chp:hsacontrol}, we propose and experimentally validate various models and controllers for \gls{HSA} robots.
%
Specifically, Chapter~\ref{chp:hsamodel} introduces a suite of kinematic, dynamic, and actuation models. First, we develop the \gls{SPCS} model to parameterize the 3D shape of \gls{HSA} rods minimally by fusing the \gls{CS} and \gls{PCS} approaches~\citep{renda2018discrete}. Next, we derive an actuation model that captures how the rod’s stiffness and rest length vary with twist angle, allowing us to integrate this phenomenon into existing \gls{DCM}-based simulators. We then implement our findings in a PyElastica~\citep{naughton2021elastica} extension, enabling 3D simulations of \gls{HSA} robots. For planar \gls{HSA} robots, we detail a simpler kinematic model based on \gls{CS}, culminating in a closed-form inverse kinematics solution. Finally, we present a dynamic model in Euler-Lagrange form suitable for planar \gls{HSA} manipulators.
%
Chapter~\ref{chp:hsacontrol} leverages these models to develop two control schemes. The first is a configuration-space P-satI-D plus potential-shaping regulator, integrating an integral-saturated PID controller~\citep{pustina2022p} with force compensation at the setpoint to guide the end-effector to a desired location in task space. 
% Such a control strategy of making a complex, underactuated, and non-affine control term more amendable for control could, in the future, also be important when devising closed-form controllers based on learned models - for example, when the latent space is higher-dimensional than the actuation space.
In the future, a strategy that renders a complex, underactuated, and non-affine control term more amenable to control may also prove valuable when developing closed-form controllers based on learned models—for instance, if the latent space exceeds the dimension of the actuation space.
The second is a operational space impedance controller that cancels the soft robot’s inherent dynamics and uses a Cartesian PD feedback term to shape the desired stiffness field. Key challenges include mapping generalized torques into the underactuated control input, linearizing actuation terms, and solving the static inversion problem~\citep{della2025pushing}.

\paragraph{Pneumatic Actuation Dynamics \& Backstepping Control.}
Pneumatic actuators are popular in soft robotics for their quick response and distributed force generation~\citep{marchese2015recipe, zaidi2021actuation}, with some variants employing fluidic pistons~\citep{marchese2014design, marchese2016design, parlikar2024concept, malas2024novel}. Such piston-driven actuation has at least two advantages: (1) the robot’s deformation is directly related to the piston position, and (2) they are closed fluid systems allowing easy use of the ideal gas law~\citep{marchese2016design}. However, relatively little work has been done on modeling these systems~\citep{marchese2014design, xavier2020modelling}, and existing controllers often adopt a cascaded approach that does not consider actuator dynamics, as a PID controller unaware of the soft robot dynamics is usually employed to control the piston. This is viable only if the fluidic system can quickly achieve pressure setpoints, necessitating high-performance (often costly) actuators or slow robot motions.
%
Chapter~\ref{chp:backstepping} addresses this limitation by modeling the potential energy stored in the fluid as a function of the robot’s configuration and the piston position, thereby coupling the piston and soft robot dynamics in Euler-Lagrange form. We then apply a backstepping approach~\citep{kokotovic1992joy, lozano1992adaptive, khalil2002nonlinear} to design a nonlinear, model-based feedback controller for the entire system, enabling tighter integration of actuator and robot dynamics.
In the future, such an actuation model could serve as a structural prior when learning models, such as learning the potential energy stored in the fluid analog to \glspl{LNN}~\citep{lutter2019deep}, therefore connecting to the core contribution. Furthermore, applying the backstepping procedure to a learned model appears to be an unexplored but interesting research avenue.

% \begin{contribution}
%     Techniques for Learning Soft Robot Models that Enable Stable Control in Closed-Form
% \end{contribution}

\begin{contribution}\label{contrib:learned_models}
    Integrating Physical Structure and Stability Guarantees into Learned Models
\end{contribution}
% In this thesis, we propose two approaches for learning control-oriented dynamical models of soft robots that expose the necessary (physical) structure to allow for closed-form control via energy shaping and stability analysis of the open- and closed-loop system via Lyapunov arguments~\citep{khalil2002nonlinear}, addressing \gls{RQ}~\ref{rq:physical_structure_learned_models}. We will detail both approaches in more detail below.
In this thesis, addressing \gls{RQ}~\ref{rq:physical_structure_learned_models}, we propose two approaches for learning control-oriented dynamical models of soft robots that incorporate the necessary (physical) structure to facilitate closed-form control through energy shaping, as well as stability analysis of both the open- and closed-loop systems using Lyapunov arguments~\citep{khalil2002nonlinear}.
% We achieve this by integrating physics-based dynamical models into the learning algorithm which determines the free parameters of the dynamics and optimally optimizes over a change of coordinates, such as an encoding into latent space.
% We outline the two approaches in detail below.
We achieve this by incorporating physics-based dynamical models into the learning algorithm, which then identifies the free parameters of the dynamics and optionally optimizes over a change of coordinates—such as encoding into latent space. We detail the two approaches below.

% In Chapter~\ref{chp:pcsregression}, we devise an algorithm that learns a soft robot strain model, specifically a \gls{PCS}-based model, directly from data of the shape evolution of the soft robot's backbone.
% While \gls{PCS}~\citep{renda2018discrete} and similar kinematic parameterizations are already very much established in the literature, their design requires much expert knowledge and/or trial-and-error. For example, the modeling engineer needs to determine the number of segments, the length of each segment, and the active strains for each segment.
% Finally, the dynamic system parameters identification procedure via nonlinear optimization can be computationally quite demanding and possibly badly conditioned. 
% Instead, our proposed approach, composed of two algorithms, can directly identify all necessary parameters from the evolution of pose samples approximating the soft robot's shape.
% The first algorithm identifies a suitable kinematic parametrization, including the number of segments and their lengths, using an iterative procedure by analyzing the strain progression along the backbone. 
% The second algorithm subsequently derives the physics-based dynamical model from first principles based on the previously identified kinematic parametrization and represents the Euler-Lagrangian dynamic using a a sum of basis functions. This allows the dynamic parameters to be identified in closed form using linear least-squares. Additionally, the algorithm leverages heuristics, such as the identified strain stiffness, to neglect certain strains and, thus, iteratively reduce the \gls{DOF} of the model, making it more efficient and suitable for control.
% We verify the proposed approach in simulation and benchmark the performance at predicting the future evolution of the soft robot over long horizons against \gls{SOTA} \gls{ML} approaches such as \glspl{RNN}, \glspl{NODE}, etc.
In Chapter~\ref{chp:pcsregression}, we introduce an algorithm designed to learn a soft robot strain model, specifically a \gls{PCS}-based model, directly from data representing the shape evolution of the soft robot’s backbone.
Although \gls{PCS}~\citep{renda2018discrete} and similar kinematic parametrizations~\citep{alessi2024rod} are well-established in the literature, their design often requires significant expert knowledge and/or trial-and-error. For instance, the modeling engineer must determine the number of segments, the length of each segment, and the active strains for each segment.
Additionally, identifying dynamic system parameters through nonlinear optimization can be computationally intensive and potentially ill-conditioned.
Our proposed approach, comprising two algorithms, circumvents these challenges by directly identifying all necessary parameters from pose samples approximating the soft robot’s shape evolution.
The first algorithm uses an iterative procedure to determine an appropriate kinematic parameterization, including the number and length of the segments, by analyzing the strain progression along the backbone.
The second algorithm then derives a physics-based dynamical model from first principles using the previously identified kinematic parameterization, representing the Euler-Lagrangian dynamics as a sum of basis functions. This enables dynamic parameter identification in closed form through linear least-squares.
Additionally, the algorithm employs heuristics, such as identified strain stiffness, to neglect certain strains, iteratively reducing the \gls{DOF} of the model, thereby improving efficiency and control suitability.
We validate the proposed approach through simulations and benchmark its performance in predicting the soft robot’s future shape evolution over long horizons against \gls{SOTA} \gls{ML} methods, including \glspl{RNN}, \glspl{NODE}, and others.

The second approach, presented in Chapter~\ref{chp:con}, considers the problem setting of learning the dynamics of physical systems, and specifically, of continuum soft robots from high-dimensional observations, such as sequences of images.
As learning the dynamics directly in image space is considered to be intractable, an established procedure is to employ an autoencoder, such as a \gls{VAE}~\citep{kingma2014auto}, and then learning the dynamics of a lower dimensional space, referred to as the latent space.
Contrary to existing literature, which mostly uses \glspl{MLP}, \glspl{RNN} or \glspl{NODE}, we propose to learn the latent dynamics with a network of coupled oscillators that consists of damped harmonic oscillators that are connected by a nonlinear potential.
Crucially, this now allows us to assign a mechanical interpretation to each latent variable - the position of one of the harmonic oscillators.
Furthermore, the physical structure of the \glsxtrfull{CON} dynamics, in particular the network's energy terms, allow us to derive very strong stability guarantees, such as \glsxtrfull{GAS} for the unactuated and \glsxtrfull{ISS} for the actuated latent dynamical system, using Lyapunov arguments~\citep{khalil2002nonlinear}.
Finally, we propose an approximated closed-form solution to the rollout of the \gls{CON} dynamics that enables a speed-up of both training and inference.
We benchmark the proposed approach against \gls{SOTA} \gls{ML} approaches extensively on rendered image sequences of various simulated systems, including different simulated soft robots.


\begin{contribution}\label{contrib:model_based_control_with_learned_models}
    Exploiting Learned Models for Closed-Form Model-Based Control
\end{contribution}
% Also contributing to ~\gls{RQ}~\ref{rq:physical_structure_learned_models} and building on Contribution~\ref{contrib:model_based_control_with_learned_models}, we devised in this a strategy for closed-form regulation based on learned models.
% Here, we build on the seminal work that devised feedback+feedforward controllers for rigid manipulators~\citep{kelly1996class, kelly1997pd, kelly1998global, sciavicco2012modelling}, later extended to the physics-based model control of soft robots by accounting for linear elastic forces~\citep{della2020model, della2023model} and nonlinear elastic forces demonstrated in Chapter~\ref{chp:hsacontrol}.
% As the feedback term, we employ a P-satI-D~\citep{pustina2022p} that, compared to PD+ controllers~\citep{della2020model}, has an improved ability to compensate for model errors through the integral action while its saturation helps to decrease the likelihood of instability~\citep{pustina2022p}.
% An important underlying assumption, similar to fully physics-based models, is that we require the state space to be continuous. This means that special techniques~\citep{maithripala2015intrinsic} are required to deal with other geometric spaces, such as Lie groups, which do not directly fulfill this property.
% As the feedforward term, we choose to compensate the potential forces at the desired robot state, which requires the potential field to be (locally) convex~\citep{della2023model} in order that we need to rely on the integral action as little as possible. This strategy can be intuitively interpreted such that we reshape the potential energy of the closed-loop system to exhibit its (local) minimum at the setpoint.
% The result is a very simple but effective control strategy.
% Having already employed this P-satI-D+potential shaping control strategy on the entirely physics-based \gls{HSA} model as part of Contribution~\ref{contrib:actuation_models}, we verify additionally in simulation for both the learned strain-based model in Chapter~\ref{chp:pcsregression} and for latent space control based on \gls{CON} dynamics in Chapter~\ref{chp:con}.
Additionally contributing to \gls{RQ}\ref{rq:physical_structure_learned_models} and building on Contribution~\ref{contrib:model_based_control_with_learned_models}, we here propose a closed-form regulation strategy based on learned models. In doing so, we draw on seminal work that introduced closed-form model-based controllers for rigid manipulators~\citep{kelly1996class, kelly1997pd, kelly1998global, sciavicco2012modelling}, subsequently adapted for the physics-based control of soft robots by considering both linear~\citep{della2020model, della2023model} and, as shown in Chapter~\ref{chp:hsacontrol}, nonlinear elastic forces~\citep{borja2022energy}.
The controller combines an integral-saturated PID (P-satI-D)~\citep{pustina2022p} with an energy-shaping feedforward term. We motivate the considerations behind this strategy in the following.

For the error-based feedback component, we use a P-satI-D controller~\citep{pustina2022p}, which offers stronger robustness against modeling errors via its integral action while saturation reduces the risk of instability~\citep{pustina2022p}. Compared to PD feedback laws~\citep{della2020model}, this approach is more effective at mitigating unmodeled or wrongly modeled dynamics.
%
A key assumption—akin to purely physics-based models—is that the state space must be continuous. Consequently, specialized methods~\citep{maithripala2015intrinsic} are necessary for other geometric spaces (e.g., Lie groups), where continuity in this sense does not hold directly.

For the feedforward portion, we compensate the potential forces at the target state, requiring the potential field to be (locally) convex~\citep{borja2022energy, della2023model}. The aim is to rely on the integral action as little as possible, effectively reshaping the closed-loop system’s potential energy so its (local) minimum is at the desired setpoint. 
% Furthermore, the goal needs to be reachable by the control, which means that the setpoint is a reachable equilibrium of the closed-loop system. 
% In the underactuation, it is additionally important for the null-dynamics to be asymptotically stable in order to obtain a stable control behavior~\citep{borja2022energy}.
Additionally, the control must be able to reach the goal, which requires the setpoint to be a reachable equilibrium of the closed-loop system. In the underactuated case, it is also crucial for the null dynamics to be asymptotically stable in order to guarantee a stable control response~\citep{borja2022energy}.
% The benefit of compensating the dynamics (specifically, here, the static forces) at the target state instead of canceling the dynamics at the current state is increased robustness against mismatches between the (learned) model and the actual dynamics as unmodeled dynamics are less likely to become dominant.
Compensating for the dynamics—specifically the static forces—at the target state, rather than canceling them at the current state, enhances robustness against mismatches between the learned model and the actual dynamics, as unmodeled dynamics are less likely to dominate in the closed-loop system.

This combination of integral-saturated PID feedback action and potential shaping feedforward action yields a straightforward yet powerful control approach.
%
Having already applied this P-satI-D+potential shaping strategy to a fully physics-based \gls{HSA} model as part of Contribution~\ref{contrib:actuation_models}, we further validate it in simulation with the learned strain-based model in Chapter~\ref{chp:pcsregression} and for latent space control using \gls{CON} dynamics in Chapter~\ref{chp:con}.


\begin{contribution}\label{contrib:motion_behaviors}
    Beyond Low-Level Control: Generating Compliant Motion Behaviors for Soft Robots
\end{contribution}

% We contribute towards \gls{RQ}~\ref{rq:compliant_motion_behaviors} by proposing two approaches that generate compliant motion behavior for soft robots: First, we present Chapter~\ref{chp:braincontrol} a \gls{BMI} approach for guiding a low-level impedance control with motor imagery based on measurements by a wearable \gls{EEG} device. Second, we show in Chapter~\ref{chp:osmp} how a neural motion policy parametrized by a learned dynamical system with stability guarantees can be used for learning complex motion behaviors from demonstration can be used as a reference provider for a low-level soft robot controller. 
% \gls{RQ}~\ref{rq:compliant_motion_behaviors} ensures that not the safety enabled by the soft robot's body and compliant low-level controller is not jeopardized by the high-level motion behavior.
% In the following two paragraphs, we will introduce both lines of research in more detail.

% \glspl{BMI} promise barrier-free and physical-interaction-free operation of machines, and specifically robots, by analyzing the neural activity of the user and are particularly an interesting choice for allowing robots to assist impaired or elderly people with \gls{ADL}.
% However, currently, the classification of motor imagery from few-channel, wearable EEG devices exhibits relatively low accuracy for more than two classes~\citep{arpaia2022non, lee2024noir}. This can cause significant safety concerns when operating high-inertia, fast-moving, rigid robots using motor imagery. Therefore, soft robots, with their passive compliance, seem like a promising avenue as they would allow errors in the \gls{BMI} to occur without jeopardizing safety.
% Despite this promise, guiding soft robotic manipulators with \glsxtrfull{BMI} has not yet been explored in literature.
% In Chapter~\ref{chp:braincontrol}, we develop a \gls{BCI} protocol that allows users, for the first time, to operate soft robotic manipulators by imagining motor movements.
% As an \gls{EEG} device, we use a wearable cap with three channels, which promises in the future to take the operation with brain signals out of stationary lab environments.
% As achieving (relatively) high classification accuracies of the motor imagery using \gls{LDA} is currently only feasible on binary classification accuracies, an emphasis of the research was to identify a protocol that allows binary classifications to precisely and robustly control the movement of the soft robot end-effector in a plane.
% We achieve this by running two classifiers in parallel, one of which switches the coordinate axis of motion based on the classification of yaw clinching and another that controls the direction of motion (i.e., positive or negative sign) along the active coordinate axis by classifying motor imagery. 
% We experimentally verify the proposed \gls{BMI} protocol with a planar \gls{HSA} robot by relaying the operational space references generated by the \gls{BMI} system as setpoints/attractors to the compliant impedance controller developed as part of Contribution~\ref{contrib:actuation_models}.
% We quantitatively evaluate the brain signal-guided \emph{control} by visually projecting operational space goals stemming from the sequence of step functions onto a screen behind the \gls{HSA} robot.
% The operator then aims to guide the soft robot's end-effector toward the goal as fast as possible with motor imagery.
% We benchmark the \gls{BMI} interface against i) a very established \gls{HCI} - a keyboard, and ii) the low-level impedance controller directly having access to the goals, which we consider to be privileged information in this case.
% Finally, we also tackle the problem setting of assisting humans with a simple \glsxtrfull{ADL} by guiding the end-effector of a \gls{HSA} robot with brain signals to release hairspray from a bottle, which showcases the compliance and intelligence of the integrated system consisting of body, low-level motor control, and motion guided by the \gls{BMI} system.

% \glsxtrfull{DMP} parameterize motion policies with dynamical systems~\citep{ijspeert2013dynamical, saveriano2023dynamic} and allow for efficient learning of complex motions from demonstration (e.g., kinesthetic teaching, biomimetics, teleoperation, etc.).
% Particularly interesting is the case where the dynamic system does not exhibit an explicit dependence on time, as this allows for natural and compliant behavior even under perturbation~\citep{ijspeert2013dynamical}.
% However, the use of \glspl{DMP} within the realm of soft robotic manipulators has not yet been investigated.
% In Chapter~\ref{chp:osmp}, we propose an approach that can learn periodic motions from demonstration and track these demonstrations in a provably stable and compliant fashion without a time dependence.
% We accomplish this by building on the trailblazing literature on \glsxtrfull{SMP}~\citep{ijspeert2013dynamical, rana2020euclideanizing, perez2023stable} and propose a new approach that combines a bijective encoder with latent dynamics governed by a supercritical Hopf bifurcation. The \emph{simple} formulation of the latent dynamics allows us to prove their stability easily, and the expressiveness and diffeomorphism enabled by the Euclideanizing flows~\citep{dinh2016density, rana2020euclideanizing} allows us to learn complex motions while giving us the ability transfer these stability guarantees back into the space where the demonstration was provided.
% In practice, we map the current position of the robot in oracle space into latent space where we evaluate the \nth{1} order dynamics of the Hopf bifurcation, providing us with a latent velocity.
% Subsequently, we evaluate the (analytical) inverse Jacobian of the encoder to map the latent velocity into a desired velocity in oracle space, which would serve as a reference for the low-level motor controller of the system.
% We extensively validate this approach experimentally on a helicoid soft robot~\citep{guan2023trimmed}, a turtle robot swimming in a pool, a UR5 manipulator cleaning a whiteboard, and a Kuka \gls{Cobot} in human-contact-rich scenarios.
% The presented approach serves as an example of how compliant and natural motion policies for soft robots can be effectively learned from demonstrations of complex behavior.

Turning to \gls{RQ}~\ref{rq:compliant_motion_behaviors}, we propose two methods for generating compliant soft robot motion at a higher control level. Chapter~\ref{chp:braincontrol} covers a \gls{BMI}-based protocol that uses wearable \gls{EEG} to supply references to a low-level impedance controller, while Chapter~\ref{chp:osmp} focuses on learning motion policies from demonstration via a dynamical system. These strategies ensure that high-level policies do not undermine the inherent safety offered by the soft robot body and compliant low-level controllers.

\paragraph{Guiding Soft Robots via Motor Imagery.}
\glspl{BMI} allow users to operate machines through neural signals—particularly appealing for assisting individuals with limited mobility. Yet classification accuracies for motor imagery on wearable EEG devices are low for more than two classes~\citep{arpaia2022non, lee2024noir}, raising safety issues if used with fast, high-inertia rigid robots. Soft robots, by contrast, offer a passive compliance that can reduce risks from erroneous \gls{BMI} commands. However, controlling soft robots with \gls{BMI} has not yet been explored.
%
In Chapter~\ref{chp:braincontrol}, we introduce a brain-computer interface protocol that, for the first time, lets users operate a soft robotic manipulator using motor imagery. We employ a wearable, three-channel \gls{EEG} cap, aiming to in the future transition brain-controlled manipulations beyond traditional lab settings. Since robust motor imagery classification beyond two classes remains challenging, we devised an effective protocol that combines two parallel binary classifiers: one to select the active coordinate axis (based on yaw clinching) and another to determine movement direction along this axis (based on motor imagery). We validate our approach on a planar \gls{HSA} robot, where the \gls{BMI}-derived commands serve as setpoints for a compliant impedance controller (developed in Contribution~\ref{contrib:actuation_models}).
% We quantitatively evaluate the brain signal-guided \emph{control} by visually projecting operational space goals stemming from the sequence of step functions onto a screen behind the \gls{HSA} robot.
In a quantitative evaluation of the brain signal-driven control, we visually project step-function-based operational space goals onto a screen behind the \gls{HSA} robot.
% 
For benchmarking, we compare this \gls{BMI} system against (1) a keyboard interface and (2) a privileged scenario where the low-level controller has direct access to goal positions. Additionally, we show how this scheme can assist with a real-world \gls{ADL} task—guiding the robot’s end-effector with brain signals to spray hairspray—highlighting the synergy between the soft robot’s compliance, a suitable low-level controller, and higher-level \gls{BMI}-based commands.

\paragraph{Learning Stable Period Motions from Demonstration.}
\glspl{DMP} represent a well-known framework for learning complex motions from demonstrations~\citep{ijspeert2013dynamical, saveriano2023dynamic}. In soft robotics, adopting time-invariant \glspl{DMP} is particularly appealing for ensuring compliance under perturbations~\citep{ijspeert2013dynamical}, yet this approach remains unexplored. In Chapter~\ref{chp:osmp}, we present a method for learning periodic motions from demonstrations and reproducing them in a provably stable and compliant manner—without explicit time dependence.
% 
Our framework extends \glsxtrfull{SMP}~\citep{ijspeert2013dynamical, rana2020euclideanizing, perez2023stable} by combining a bijective encoder with latent dynamics governed by a supercritical Hopf bifurcation. The Hopf system’s simple latent dynamics facilitate stability proofs, while the diffeomorphism introduced by Euclideanizing flows~\citep{dinh2016density, rana2020euclideanizing} enables us to learn complex motions and transfer stability guarantees back to the original demonstration space. Practically, we encode the robot’s current position into the latent space, evaluate the Hopf system to obtain a latent velocity, and then map that velocity back to the demonstration space via the encoder’s analytical inverse Jacobian—yielding a velocity reference for a low-level motor controller, such as the configuration or operational space controllers presented in Chapter~\ref{chp:hsacontrol}.
% 
We conduct extensive experiments with diverse systems: a helicoid soft robot~\citep{guan2023trimmed}, a swimming turtle robot, a UR5 manipulator for whiteboard cleaning, and a Kuka \gls{Cobot} for tasks involving human contact. The results demonstrate that our approach can learn intricate motion patterns while preserving compliance, making it a promising path forward for soft robot motion planning.
\section{Thesis Outline}\label{sec:introduction:outline}

\begin{figure}[ht]
    \centering
    \includegraphics[width=1.0\linewidth]{introduction/figures/thesis_outline_tree_v2.pdf}
    \caption{
    % Outline of this thesis with a focus on the contributions. The background colors of the blocks refer to the parts of this thesis. With \emph{C.X}, we refer to the various contributions. Specifically, the dashed frames encapsulate the chapters corresponding to the respective contribution.
    Outline of the thesis emphasizing its contributions. The block background colors denote the various parts of the thesis, while \emph{C.X} refers to the $X$th contribution. Dashed frames highlight the chapters corresponding to each contribution.
    }
    \label{fig:introduction:thesis_outline_tree}
\end{figure}

\begin{figure}[ht]
    \centering
    \includegraphics[width=1.0\linewidth]{introduction/figures/thesis_outline_house_v2.pdf}
    \caption{
    % Outline of this thesis with a focus on the physical priors of the models and the application area of the model (e.g., proprioception/shape sensing, control, motion generation/policies, and \gls{HRI}).
    Outline of this thesis, highlighting the models’ physical priors and their application areas, including proprioception/shape sensing, control, motion generation/policies, and \gls{HRI}.
    }
    \label{fig:introduction:thesis_outline_house}
\end{figure}

In the following, we will outline the structure of this thesis, which is also visualized in Figs.~\ref{fig:introduction:thesis_outline_tree}-\ref{fig:introduction:thesis_outline_house}.

% In Chapter~\ref{chp:background}, we provide background on the existing work on kinematic parametrizations of the backbone shape of soft robots, the derivation of the forward kinematics, and dynamical model~\citep{armanini2023soft, alessi2024rod}. Finally, we reference established model-based control approaches for continuum soft robots.
Chapter~\ref{chp:background} aims to provide background on the existing literature on modeling and control of soft robots and introduce specific kinematic and dynamic models, model-based controllers, and other concepts that are reoccurring throughout this thesis.
For example, this includes the \gls{PCS} kinematic parametrization, Euler-Lagrangian dynamic models of soft robots, and P-satI-D~\citep{pustina2022p}+potential shaping~\citep{della2023model} controllers.

Chapter~\ref{chp:safetymetric} incorporating Contribution~\ref{contrib:safety_metric} analyzes the inherent safety of soft robots and recognizes the need for a quantitative safety metric that encompasses both the embodied and computational intelligence of the robotic system. After identifying interesting applications of a safety metric, we devise the requirements such a metric would need to meet. We then propose the first quantitative safety metric for soft robotic manipulators. Finally, we give recommendations for the safe design of soft robotic systems.

% At the outset of working on this thesis, there was still a lack of understanding of some fundamental characteristics of the behavior of soft robots.
% Examples include (1) detailed knowledge about the actuation behavior, such as the coupled dynamics between soft robots and pneumatic piston actuators or the transfer of power from electric actuators through an auxetic metamaterial to a deformation of the overall structure as is the case for \gls{HSA} robots, (2) or kinematic parametrizations that can selectively keep specific strains (e.g., bending, twist, axial) constant or piecewise constant over the backbone discretization.
% Therefore, we decided to dedicate a significant part of this thesis to advancing our understanding of soft robotic behavior by modeling these effects and dynamics using physics-based approaches, subsequently experimentally verifying them, and finally exploiting this gained model knowledge for improving the shape sensing and model-based control of soft robots.
% This content is contained in Part~\ref{part:physicsmodels}, where we present advanced physics-based models for shape sensing and model-based control.
At the outset of this thesis, our understanding of several fundamental aspects of soft robot behavior was still limited. For instance, we lacked detailed insights into actuation dynamics~\citep{della2023model}—such as the coupled behavior between soft robots and pneumatic piston actuators or how power is transmitted from electric actuators through auxetic metamaterials to induce deformations in \gls{HSA} robots—as well as in kinematic parameterizations that allow for an optimal tradeoff between dimensionality and expressiveness - as ones that maintain specific strains, such as bending, twist, axial, either constant or piecewise constant along the backbone. Consequently, we devoted a significant portion of this thesis to advancing our understanding by developing physics-based models, experimentally validating them, and leveraging the results to enhance shape sensing and model-based control. This work is presented in Part~\ref{part:physicsmodels}, where we introduce advanced physics-based models for these two applications.

% This better understanding of advanced soft robot characteristics gave us the necessary know-how to propose better-suited physical priors for integrating into learned models.
% Specifically, Part~\ref{part:learning} is dedicated to incorporating priors related to kinematics, stability characteristics, and physical structure into \gls{ML}-based methods, utilizing these hybrid models for shape sensing and control. The content of both parts is discussed in greater detail below. Finally, in Chapter~\ref{chp:conclusion}, we summarize our findings and suggest promising directions for future research.
This deeper insight into soft robot characteristics has equipped us to propose more suitable physical priors for integration into learned models. Specifically, Part~\ref{part:learning} focuses on incorporating priors related to kinematics, stability, and physical structure into \gls{ML}-based methods, thereby employing these hybrid models for improved shape sensing and control. 

Parts~\ref{part:physicsmodels} \& \ref{part:learning} are discussed in further detail in the following subsections.

\subsection*{Part I - Shape Sensing and Control with Advanced Physics-based Models}

% In Part~\ref{part:physicsmodels}, we derive advanced physics-based models of soft robots and exploit them for shape sensing and control.
% Specifically, we as i) leverage project pose estimates stemming from \gls{SLAM} algorithms onto a kinematic model to increase the performance of shape sensing based on visual sensors, ii) develop kinematic and dynamic models for \gls{HSA} robots, and subsequently exploit them for model-based control, iv) propose a motor-imagery-based \glsxtrfull{BCI} interface for guiding a low-level impedance controller with brain signals, and v) model the actuation dynamics of pneumatic piston-driven soft robots and devise a provably-stable backstepping controller for regulating the motion of the piston.
% This part contains Contributions~\ref{contrib:kinematic_models_shape_sensing}-\ref{contrib:model_based_control_with_learned_models} and is composed of the following chapters:
In Part~\ref{part:physicsmodels}, we develop advanced physics-based models for soft robots and apply them to shape sensing and control. Specifically, we: i) enhance shape sensing based on visual sensors by leveraging pose estimates from \gls{SLAM} algorithms and projecting them onto a kinematic model, ii) propose kinematic and dynamic models for \gls{HSA} robots and utilize them for model-based control, iii) introduce a motor-imagery-based \glsxtrfull{BCI} interface for guiding a low-level impedance controller with brain signals, and iv) model the actuation dynamics of pneumatic piston-driven soft robots, designing a provably stable backstepping controller for piston motion regulation.
%
This part includes Contributions~\ref{contrib:kinematic_models_shape_sensing}-\ref{contrib:model_based_control_with_learned_models} and comprises the following chapters:

\begin{itemize}
    % \item \textbf{Chapter~\ref{chp:srslam}} presents a methodology for combining low-cost monocular cameras with \gls{vSLAM} algorithms and a projection of pose estimates onto the kinematics of the soft robot to achieve shape sensing for soft robots. The approach is verified in both simulations with a \gls{CC} model and experimentally with a pneumatic soft robot segment moving in 3D space.
    \item \textbf{Chapter~\ref{chp:srslam}} describes a method for combining low-cost monocular cameras with \gls{vSLAM} algorithms and projecting pose estimates onto the soft robot’s kinematics to enable shape sensing. The approach is validated through simulations using a \gls{CC} model and experimental tests on a pneumatic soft robot segment moving in 3D space.
    % \item \textbf{Chapter~\ref{chp:hsamodel}} proposes kinematic, dynamic, and actuation models for \gls{HSA} robots consisting of parallel auxetic metamaterial rods. Firstly, we propose a new kinematic parametrization (\gls{SPCS}) based on the \gls{PCS}~\citep{renda2018discrete} model that is able to capture the shape of \gls{HSA} rods using the least possible configuration variables.  Secondly, we establish a formalism for incorporating the auxetic trajectory (e.g., \gls{HSA} rods changing length in the presence of twist strains) into Cosserat rod-based models, which enables a PyElastica~\citep{naughton2021elastica}-based simulator for \gls{HSA} robots. Subsequently, we devise a kinematic model for planar \gls{HSA} robots by approximating its virtual backbone with \gls{CS}. Finally, we devise for the planar case an underactuated dynamical model in Euler-Lagrangian form that considers the particular characteristics of \gls{HSA} robots, such as the auxetic trajectory of the \glspl{HSA}, which causes rest length and stiffness of the robot to vary as a function of the twist strain. All devised models are experimentally verified. 
    \item \textbf{Chapter~\ref{chp:hsamodel}} presents kinematic, dynamic, and actuation models for \gls{HSA} robots comprising parallel auxetic metamaterial rods. First, we introduce a novel kinematic parametrization (\gls{SPCS}) based on the \gls{PCS}~\citep{renda2018discrete} model, effectively capturing \gls{HSA} rod shapes with minimal configuration variables. Next, we formalize the inclusion of auxetic trajectories (e.g., \gls{HSA} rods changing length with twist strains) in Cosserat rod-based models, enabling a PyElastica~\citep{naughton2021elastica}-based simulator for \gls{HSA} robots. Additionally, we devise kinematic and underactuated dynamic models for planar \gls{HSA} robots, incorporating auxetic trajectories that influence rest length and stiffness. These models are experimentally validated.  
    % \item \textbf{Chapter~\ref{chp:hsacontrol}} builds upon Chapter~\ref{chp:hsamodel} by devising two model-based control approaches for setpoint regulation with planar \gls{HSA} robots: a) a configuration-space controller consists of an integral-saturated PID controller as the feedback term and a potential shaping feedforward term, b) a operational space impedance controller that allows shaping of the closed-loop end-effector stiffness in Cartesian space through a PD feedback term. The configuration-space controller in a) is enabled by a feedforward term that evaluates the potential forces at the desired end-effector position via static inversion and ii) a feedback term that is enabled by a combination of approaches, such as linearization and mapping into actuation coordinates, to make the actuation term more control friendly. The operational space impedance controller in b) relies upon canceling the existing soft robot dynamics using the model knowledge and solving the underactuated mapping from configuration-space torques onto control inputs with a least-squares optimizer. The control approaches are extensively verified experimentally and benchmarked against a model-free PID controller.
    \item \textbf{Chapter~\ref{chp:hsacontrol}} extends Chapter~\ref{chp:hsamodel} by proposing two model-based control strategies for setpoint regulation with planar \gls{HSA} robots: a) a configuration-space controller using an integral-saturated PID feedback term and potential shaping feedforward term, and b) an operational space impedance controller that enables end-effector stiffness shaping in Cartesian space through a PD feedback term. The configuration-space controller leverages a feedforward term evaluating potential forces at desired end-effector positions via static inversion and a feedback term enabled by linearization of the actuation term and mapping into collocated form. The operational space controller uses model knowledge to cancel the existing soft robot dynamics and applies a least-squares optimizer to solve the underactuated control problem (i.e., mapping configuration space torques to actuation inputs). Both approaches are extensively validated experimentally and benchmarked against a model-free PID controller.  
    % \item \textbf{Chapter~\ref{chp:braincontrol}} augments the operational space impedance control from Chapter~\ref{chp:hsacontrol} with a \gls{BMI} protocol that allows a user to operate a soft robot by imagining motor movements. The \gls{BMI} protocol transforms measurements by a wearable \gls{EEG} device into spatial movements of an end-effector attractor by first classifying them using \gls{LDA} and subsequently translating the motor imagery into an incremental movement of the attractor along the coordinate axes. The impedance controller subsequently tracks the setpoints. We experimentally, with a planar \gls{HSA} robot, compare the control performance of the proposed approach on a reference trajectory of nine-step functions against two baselines: a) computational control, where the impedance controller directly has access to the privileged goal information, and b) control using a much more established \gls{HRI} interface - a keyboard. Finally, we take a first foray into assisting humans with a simple \gls{ADL} - releasing hairspray from a bottle by guiding the soft robot using motor imagery to exert force on the button with its end-effector.
    \item \textbf{Chapter~\ref{chp:braincontrol}} enhances the operational space impedance controller from Chapter~\ref{chp:hsacontrol} with a \gls{BMI} protocol, enabling users to control a soft robot through motor imagery. The protocol processes wearable \gls{EEG} device measurements, classifies motor imagery using \gls{LDA}, and translates the output into incremental end-effector attractor movements. The low-level impedance controller then tracks these setpoints. Experiments with a planar \gls{HSA} robot compare this approach's performance on a reference trajectory against two baselines: a computational control with privileged access to goal positions and a keyboard-based \gls{HRI} interface. Finally, we demonstrate a simple \gls{ADL}—releasing hairspray by guiding the soft robot to press a button using motor imagery.  
    % \item \textbf{Chapter~\ref{chp:backstepping}} models the actuation dynamics of a pneumatic piston-driven soft robot and subsequently exploits the model knowledge to devise a nonlinear controller for the actuation of the piston by following a backstepping approach~\citep{kokotovic1992joy, lozano1992adaptive, khalil2002nonlinear}. The model is enabled by formulating the potential energy stored in the fluid as a function of the soft robot configuration and the position of the piston. Its partial derivative then represents the potential forces that fluid exerts on the piston and the soft robot, respectively. The backstepping controller integrates a configuration-space setpoint regulator into a feedback controller for the pistons. The proposed approach is tested in simulation on a three-segment \gls{PCC} soft robot.
    \item \textbf{Chapter~\ref{chp:backstepping}} develops dynamic models of the actuation dynamics for pneumatic piston-driven soft robots and designs a nonlinear backstepping controller~\citep{kokotovic1992joy, lozano1992adaptive, khalil2002nonlinear}. The model calculates the potential energy stored in the fluid as a function of the robot's configuration and piston position, with partial derivatives representing the potential forces exerted by the fluid. The backstepping controller integrates a configuration-space setpoint regulator into a piston feedback controller. This approach is tested through simulations on a three-segment \gls{PCC} soft robot.  
\end{itemize}

\subsection*{Part II - Incorporating Physical Structure and Stability Guarantees into Learned Models and Controllers}

% Part~\ref{part:learning} encompasses various pieces of research where we combine modern \gls{ML} approaches with physics models with the aim of simplifying the learning problem, enabling insight into the learned component by establishing a physical interpretation, adding stability guarantees, and/or exploiting a physical structure for model-based control with energy-shaping approaches. It touches Contributions~\ref{contrib:kinematic_models_shape_sensing},\ref{contrib:learned_models}-\ref{contrib:motion_behaviors} and contains the following chapters
Part~\ref{part:learning} combines several research efforts integrating modern \gls{ML} techniques with physics-based models. The goal is to simplify the learning process, provide insights into the learned components through physical interpretation, ensure stability guarantees, and/or leverage physical structures for model-based control using energy-shaping approaches. This part covers Contributions~\ref{contrib:kinematic_models_shape_sensing},\ref{contrib:learned_models}-\ref{contrib:motion_behaviors} and includes the following chapters:

\begin{itemize}
    % \item \textbf{Chapter~\ref{chp:promasens}} develops a methodology to achieve shape sensing for 3D soft robots by embedding multiple magnets and magnetic sensors into the body.
    % The main challenge connected with magnetic sensors is interpreting their measurements and mapping them onto a configuration estimate. 
    % We leverage the kinematic model of the soft robot to parameterize the spatial relation between a magnetic sensor and its surrounding magnets. We then learn the forward measurement model from this low-dimensional parametrization to predicted measurements of each sensor. Interestingly, as we reduced the size of the learning \emph{black-box} by exploiting kinematic model knowledge, this learned sensor measurement model generalizes across all embedded magnetic sensors.
    % Finally, to achieve proprioception, we formulate an optimization problem with the cost function being the squared error between the predicted and actual sensor measurements. As both the kinematics and the neural sensor measurement predictor are analytically differentiable, we can efficiently solve this optimization problem using gradient descent. We verify the proposed approach both in simulation with \gls{PCC} and \gls{PAC} soft robots and experimentally with a pneumatically actuated soft segment.
    \item \textbf{Chapter~\ref{chp:promasens}} presents a methodology for achieving 3D shape sensing in soft robots by embedding multiple magnets and magnetic sensors within their body. The primary challenge with magnetic sensors lies in interpreting their measurements and mapping them to a configuration estimate. To address this, we use the kinematic model of the soft robot to parameterize the spatial relationship between a magnetic sensor and its surrounding magnets. We then learn the forward measurement model based on this low-dimensional parameterization to predict the measurements for each sensor. Notably, by incorporating knowledge of the kinematic model and reducing the complexity of the learning \emph{black-box}, the resulting sensor measurement model generalizes effectively across all embedded magnetic sensors. Finally, to achieve proprioception, we formulate an optimization problem where the cost function minimizes the squared error between predicted and actual sensor measurements. With both the kinematic model and the neural sensor measurement predictor being analytically differentiable, this optimization problem is efficiently solved using gradient descent. The proposed approach is validated through simulations with \gls{PCC} and \gls{PAC} soft robots, as well as experiments with a pneumatically actuated soft segment.
    % \item \textbf{Chapter~\ref{chp:pcsregression}} presents an algorithmic approach for identifying low-dimensional strain models based on data describing the shape evolution of soft robots. The approach assumes as inputs pose measurements along the soft robot's backbone, and it consists of two steps: First, we identify a suitable low-dimensional \gls{PCS}-based parametrization that can approximate the backbone shape well using an iterative procedure. Subsequently, a dynamical model is derived based on the kinematic parametrization, and its parameters are regressed in closed form. Furthermore, the algorithm leverages heuristics such as the identified stiffness to decide if some strains can be neglected and, with that, if the \gls{DOF} of the dynamical model can be further reduced. We verify the proposed approach by identifying the dynamical model directly from videos of a simulated soft robot. We quantitatively benchmark the performance of the derived model against \gls{SOTA} \gls{ML} approaches such as \glspl{RNN}, \glspl{NODE}, etc., and demonstrate how learning this fully physics-based strain model offers much-improved extrapolation performance when tested on outside of the training set actuation sequences. Finally, we leverage the learned model for model-based control with the controller containing an integral-saturated PID feedback term and a potential shaping feedforward term - a similar control approach as applied in Chapter~\ref{chp:hsacontrol}.
    \item \textbf{Chapter~\ref{chp:pcsregression}} introduces an algorithmic framework for identifying low-dimensional strain models from data describing the shape evolution of soft robots. The approach takes as input pose measurements along the soft robot’s backbone and consists of two key steps: First, a suitable low-dimensional \gls{PCS}-based parameterization is identified through an iterative procedure to approximate the backbone shape accurately. Next, a dynamical model is derived from this kinematic parameterization, with its parameters regressed in closed form. Additionally, the algorithm employs heuristics, such as the identified stiffness, to determine whether certain strains can be neglected, potentially further reducing the \gls{DOF} of the dynamical model. The proposed method is validated by deriving the dynamical model of a simulated soft robot directly from video data. Its performance is quantitatively benchmarked against \gls{SOTA} \gls{ML} approaches, including \glspl{RNN} and \glspl{NODE}. Results demonstrate that this fully physics-based strain model significantly outperforms others in extrapolation settings, particularly for actuation sequences outside the training set. Finally, the learned model is employed for model-based control using a controller that incorporates an integral-saturated PID feedback term alongside a potential-shaping feedforward term, similar to the control approach outlined in Chapter~\ref{chp:hsacontrol}.
    % \item \textbf{Chapter~\ref{chp:con}} proposes to learn the latent dynamics of physical systems, such as soft robots, using Coupled Oscillator Networks (CONs). Specifically, this means that we map high-dimensional observations, such as images of the soft robot, into a low-dimensional latent space with a learned autoencoder. The latent space now possesses a mechanical interpretation as each latent variable corresponds to the position of a mechanical oscillator. One benefit of this strategy is that the presented formulation of \gls{CON} exhibits strong stability guarantees as the unactuated system is \glsxtrfull{GAS} and the actuated system \glsxtrfull{ISS}~\citep{khalil2002nonlinear}. We verify the learning approach, among others, on multiple datasets containing image sequences of the motion of simulated soft robots and benchmark the motion prediction accuracy against various \gls{SOTA} \gls{ML} approaches such as \gls{coRNN}, \gls{GRU}, etc.
    % Even more importantly, we can leverage the physical structure of the \gls{CON} dynamics for closed-form model-based feedback control in latent space. Again, we use here an integral-saturated PID as the feedback term and reshape the potential field of \gls{CON} to have its minimum at the desired latent goal. We verify the proposed control strategy on a simulated two-segment \gls{PCC} soft robot.
    \item \textbf{Chapter~\ref{chp:con}} introduces a method for learning the latent dynamics of physical systems, such as soft robots, using Coupled Oscillator Networks (CONs). Specifically, high-dimensional observations, like images of the soft robot, are mapped into a low-dimensional latent space using a learned autoencoder. Each latent variable in this space corresponds to the position of a mechanical oscillator, providing a clear mechanical interpretation. A key advantage of this approach is that the \gls{CON} formulation offers strong stability guarantees: the unactuated system exhibits \glsxtrfull{GAS}, while we prove for the actuated system \glsxtrfull{ISS}~\citep{khalil2002nonlinear}. The proposed learning method is validated on multiple datasets comprising image sequences of simulated soft robot motions, and its motion prediction accuracy is benchmarked against various \gls{SOTA} \gls{ML} approaches, such as \gls{coRNN} and \gls{GRU}. Furthermore, the physical structure of the \gls{CON} dynamics is leveraged for closed-form model-based feedback control in the latent space. This involves using an integral-saturated PID feedback term and reshaping the \gls{CON} potential field to exhibit its minimum at the desired latent goal. The proposed control strategy is demonstrated on a simulated two-segment \gls{PCC} soft robot.
    % \item \textbf{Chapter~\ref{chp:osmp}} puts forward a method for learning orbitally stable motion policies from demonstration that can be used as a reference to the low-level motor controller presented, for example in Chapters~\ref{chp:hsacontrol} \& \ref{chp:pcsregression} - \ref{chp:con}. Specifically, we build on the existing pioneering research on \glsxtrfull{SMP} that parametrizes motion policies with (latent) dynamical systems~\citep{ijspeert2013dynamical, rana2020euclideanizing}. In this chapter, we combine a learned bijective encoder implemented as a Euclideanizing flow~\citep{dinh2016density, rana2020euclideanizing} with an orbitally stable supercritical Hopf bifurcation as the latent dynamics to learn periodic motions from demonstrations with stability guarantees.
    % The inverse Jacobian of the bijective encoder allows the projection of predicted latent velocities back into the oracle space. We validate the approach experimentally on multiple robot embodiments, such as a tendon-driven helicoid soft robot~\citep{guan2023trimmed}, a swimming turtle robot, a UR5 robotic manipulator, and a Kuka \gls{Cobot}.
    \item \textbf{Chapter~\ref{chp:osmp}} proposes a method for learning orbitally stable motion policies from demonstration, which can serve as a reference for the low-level motor controllers introduced in Chapters~\ref{chp:hsacontrol} and \ref{chp:pcsregression}–\ref{chp:con}. Building on pioneering research in \glsxtrfull{SMP}, which parameterizes motion policies using (latent) dynamical systems~\citep{ijspeert2013dynamical, rana2020euclideanizing}, this chapter integrates a learned bijective encoder implemented as a Euclideanizing flow~\citep{dinh2016density, rana2020euclideanizing} with an orbitally stable supercritical Hopf bifurcation as the latent dynamics. This approach enables the learning of periodic motions from demonstrations with stability guarantees. The inverse Jacobian of the bijective encoder facilitates the projection of predicted latent velocities back into the oracle space. The proposed method is experimentally validated on various robot embodiments, including a tendon-driven helicoid soft robot~\citep{guan2023trimmed}, a swimming turtle robot, a UR5 robotic manipulator, and a Kuka \gls{Cobot}.
\end{itemize}

\subsection*{Conclusion and Appendices}
In the following, we will detail the content of Parts~\ref{part:conclusion} \& \ref{part:appendices} \ref{part:appendices}.
\begin{itemize}
    \item \textbf{Chapter~\ref{chp:conclusion}} summarizes our findings and suggests promising directions for future research.
    \item 
    % \textbf{Appendix~\ref{chp:apx:holisticcodesign}} proposes a framework for holistic co-design of soft robots that enables co-optimization of soft robot body (e.g., its morphology) and brain (e.g., perception and control systems)~\citep{spielberg2019learning, wang2024diffusebot, navez2024contributions}. Crucially, this framework considers a broad set of values beyond task-centric performance, such as safety, manufacturability, and regulatory considers, it explicitly accounts for realization of the design and how prototyping can decrease the uncertainty in the evaluation metrics used to assess designs computationally, and it proposes modifications and augmentations to the co-design process that have the potential to make it computationally significantly more efficient. In summary, this holistic co-design framework guides a way towards how the soft robot's structural shape, the actuator and sensing placement (see Chapters~\ref{chp:srslam} \& \ref{chp:promasens}), the reduced-order modeling (see Chapters~\ref{chp:hsamodel}, \ref{chp:pcsregression}, \& \ref{chp:con}), and the controller (see Chapters~\ref{chp:hsacontrol}, \ref{chp:backstepping}, \ref{chp:pcsregression}, \& \ref{chp:con}) could be jointly optimized in the future to optimize performance while ensuring that the closed-loop system remains sufficiently compliant and safe (see Chapter~\ref{chp:safetymetric}).
    \textbf{Appendix~\ref{chp:apx:holisticcodesign}} introduces a framework for the holistic co-design of soft robots, enabling the simultaneous optimization of both the robot’s body (e.g., its morphology) and its brain (e.g., the perception and control systems)~\citep{spielberg2019learning, wang2024diffusebot, navez2024contributions}. Importantly, the framework considers a broad array of criteria beyond task-centric performance, including safety, manufacturability, and regulatory considerations. It explicitly addresses the practical realization of designs by demonstrating how prototyping can reduce uncertainty in the computational evaluation metrics, while also proposing modifications to the co-design process that could greatly enhance computational efficiency. In summary, this holistic co-design framework provides guidance on how the soft robot’s structural shape, actuator and sensor placement (see Chapters~\ref{chp:srslam} \& \ref{chp:promasens}), reduced-order modeling (see Chapters~\ref{chp:hsamodel}, \ref{chp:pcsregression}, \& \ref{chp:con}), and controller (see Chapters~\ref{chp:hsacontrol}, \ref{chp:backstepping}, \ref{chp:pcsregression}, \& \ref{chp:con}) might be jointly optimized in the future to boost performance while ensuring that the closed-loop system remains sufficiently compliant and safe (see Chapter~\ref{chp:safetymetric}).
    \item \textbf{Appendix~\ref{chp:apx:con}} contains supplementary material, including details on the experimental setup and additional results, for Chapter~\ref{chp:con}.
    \item 
    % \textbf{Appendix~\ref{chp:apx:infrastructure}} details some of the lab infrastructure, such as a motion capture system and pneumatic pressure regulation, that was developed as part of this PhD to support and enable the research presented in this thesis. Furthermore, we present some of the software libraries that support many of the chapters contained in this thesis. For example, a JAX implementation of control-oriented soft robot models that enables simulation, model-based control, or in the future motion planning, or a ROS2 ecosystem for the control of \gls{HSA} robots.
    \textbf{Appendix~\ref{chp:apx:infrastructure}} outlines some of the laboratory infrastructure developed during this PhD—such as a motion capture system and pneumatic pressure regulation—to support the research presented in this thesis. Additionally, we introduce several software libraries that underpin many of the thesis chapters, including a JAX implementation of control-oriented soft robot models for simulation, model-based control, and motion planning, as well as a ROS2 ecosystem for model-based control of \gls{HSA} robots.
\end{itemize}
 

\subsection{Important Notes}
% We note that while we aimed to keep the notation similar throughout the chapters, the variety of problem settings and methods combined with the limited number of available symbols allowed us to develop tailored notations for each chapter.
We acknowledge that, although we strived to maintain consistent notation across chapters, the diversity of problem settings and methods, coupled with the limited pool of available symbols, necessitated the use of tailored notations for each chapter.
% Also, please note that in the following, if not explicitly otherwise noted, we refer with the expression \emph{soft robot} to \emph{continuum soft robots} (versus articulated soft robots)~\cite{della2020softencyclopedia}.
Also, please note that, unless explicitly stated otherwise, the term "soft robot" in this thesis refers to "continuum soft robots", as opposed to articulated soft robots~\citep{della2020softencyclopedia}.
The code and data associated with this thesis are made available as a dataset on the \emph{4TU.ResearchData} platform\footnote{\url{https://doi.org/10.4121/a9ee4280-4ef1-4c2b-bcef-526cd50292a9}}.
