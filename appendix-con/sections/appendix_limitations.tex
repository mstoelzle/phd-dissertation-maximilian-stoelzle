\section{Extended Discussion on Future Applications and Limitations}\label{apx:sec:limitations}

\subsection{Systems for which we would expect the proposed method to work}
\paragraph{Mechanical systems with continuous dynamics, dissipation, and a single, attractive equilibrium point.}
The proposed method is a very good fit for mechanical systems with continuous dynamics, dissipation, and a single, attractive equilibrium point. In this case, the real system and the latent dynamics share the energetic structure and stability guarantees. Examples of such systems include many soft robots, deformable objects with dominant elastic behavior, and other mechanical structures with elasticity.

\paragraph{Local modeling of (mechanical) systems that do not meet the global assumptions.}
Even if the global assumptions of the proposed method are not met, the method can still be applied to model the local behavior around a local asymptotic equilibrium point of the system (i.e., in the case of multi-stability). For example, the method could be used to model the behavior of a robotic leg locally in contact with the ground, a \gls{Cobot}'s interaction with its environment, etc.

\subsection{Systems for which we could envision the proposed method to work under (minor) modifications}
\paragraph{Mechanical systems without dissipation.}
The proposed method would currently not work well for mechanical systems without any dissipation, as (a) the original system will likely not have a globally asymptotically stable equilibrium point, and more importantly, (b) we currently force the damping learned in latent space to be positive definite. However, these systems are not common in practice as friction and other dissipation mechanisms are omnipresent, and the proposed method can learn very small damping values (e.g., the mass-spring+friction system). A possible remedy could be to relax the positive definiteness of the damping matrix in the latent space, allowing for zero damping. This would allow the method to work for systems without dissipation, such as conservative systems. Examples of such systems include a mass-spring system without damping, the n-body problem, etc.

\paragraph{Systems with discontinuous dynamics.}
The proposed method might underperform for systems with highly discontinuous dynamics, such as systems with impacts, friction, or other discontinuities. In these cases, the latent dynamics might not capture the real system's behavior accurately, and the control performance of feedforward + feedback will very likely be worse than pure feedback. Again, the method should be able to capture local behavior well. A possible remedy for learning global dynamics could be to augment the latent dynamics with additional terms that capture the discontinuities, such as contact and friction models (e.g., stick-slip friction).

\paragraph{Systems with multiple equilibrium points.}
The original system having multiple equilibria conflicts with the stability assumptions underlying the proposed CON latent dynamics. In this case, as, for example, seen on the pendulum+friction and double pendulum + friction results, the method might work locally but will not be able to capture the global behavior of the system. A possible remedy could be to relax the global stability assumptions of the CON network. For example, the latent dynamics could be learned in the original coordinates of CON while allowing $W$ also to be negative definite. This would allow the system to have multiple equilibria \& attractors. Examples of such systems include a robotic arm under gravity, pendula under gravity, etc.

\paragraph{Systems with periodic behavior.}
The proposed method will likely not work well for systems with periodic behavior, as they do not have a single, attractive equilibrium point. Examples of such systems include a mass-spring system with a periodic external force, a pendulum with a periodic external force, some chemical reactions, etc. Again, it is likely possible to apply the presented method to learning a local behavior (i.e., not completing the full orbit). A possible remedy could be to augment the latent dynamics with additional terms that capture the periodic behavior, such as substituting the harmonic oscillators with Van der Pol oscillators to establish a limit cycle or a supercritical Hopf bifurcation.

\subsection{Systems for which we would not expect the proposed method to work}
\paragraph{Nonholonomic systems.}
The proposed method likely would not work well for nonholonomic systems, as both structure (e.g., physical constraints) and stability characteristics would not be shared between the real system and the latent dynamics. Examples of such systems include vehicles, a ball rolling on a surface, and many mobile robots.

\paragraph{Partially observable and non-markovian systems.}
As the CON dynamics are evaluated based on the latent position and velocity encoded by the observation of the current time step and the observation-space velocity, we implicitly assume that the system is (a) fully observable and (b) fulfills the Markov property. This assumption might not hold for partially observable systems, such as systems with hidden states or systems with delayed observations. Examples of such cases include settings where the system is partially occluded or in situations without sufficient (camera) perspectives covering the system. Furthermore, time-dependent material properties, such as viscoelasticity or hysteresis, that are present and significant in some soft robots and deformable objects are not captured by the method in its current formulation.