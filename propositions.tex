\documentclass{propositions}

%% Turn off page numbering for the propositions and make the margins on both
%% sides equal and symmetrical.
\geometry{twoside=false}
\pagestyle{empty}

%% \RequirePackage{unicode-math}

% \setmainfont[Path = fonts/libertinus/, ItalicFont=libertinusserif-italic.otf, BoldFont=libertinusserif-bold.otf, BoldItalicFont=libertinusserif-bolditalic.otf]{libertinusserif-regular.otf}
% \setsansfont[Path = fonts/libertinus/, BoldFont=libertinussans-bold.otf, ItalicFont=libertinussans-italic.otf]{libertinussans-regular.otf}
% \setmathfont[Path = fonts/libertinus/]{libertinusmath-regular.otf}
% \setmonofont[Scale=MatchLowercase, Path=fonts/inconsolata/]{Inconsolata-Regular.ttf}
%% The default style for text is Tahoma (sans-serif).
%\renewcommand*\familydefault{\rmfamily}

% space modifications
\renewcommand{\baselinestretch}{.9} 

\begin{document}

%% Specify the title and author of the thesis. This information will be used on
%% both the English and Dutch side and in the metadata of the final PDF.
% \title{Towards Safe and Effective Model-based Control of Soft Robots: A Journey From First-Principle to Data-driven Models Incorporating Physical Structure}
% \title{Model-based Control of Soft Robots: Bridging Physics and Learned Models}
% \title[Synergizing Physics and Learned Models for Ensuring Precision and Safety]{Efficient Control of Soft Robots}
\title[Incorporating Physics into Learned Models for Control]{Safe yet Precise Soft Robots}
% synonyms of combining: fusing, incorporating, blending

\author{Maximilian}{Stölzle}

\begin{center}

{\Large\titlefont\bfseries Propositions}

\medskip

accompanying the dissertation

\medskip

%% Print the title.
{\makeatletter
\titlestyle\bfseries\large\@title
\makeatother}

%% Print the optional subtitle.
{\makeatletter
\ifx\@subtitle\undefined\else
    \titlefont\titleshape\@subtitle
\fi
\makeatother}

\medskip

by

\medskip

%% Print the full name of the author.
\makeatletter
{\large\titlefont\bfseries\@firstname\ {\titleshape\@lastname}}
\makeatother

\end{center}

% \bigskip

\begin{enumerate}
    % \item In the future, we will no longer learn robot models that lack a physical structure [This thesis, Core Contribution].
    \item % Incorporating physics into learned dynamical models unlocks computationally efficient and provably stable control while increasing expressiveness by learning the model.
    Integrating physical priors into learned dynamical models enables computationally efficient, provably stable control while also enhancing model expressiveness through data-driven learning.
    [Core Contribution]
    % \item We should not regard the material softness (i.e., the Elastic modulus) as the sole driver of safety in soft robots. [This thesis, Safety \& Co-design]
    % \item Contrary to the current soft robot design practices, the material softness is not the sole driver of safety in soft robots. [This thesis, Safety \& Co-design]
    \item % A major shortcoming in the existing soft robotics literature is that the performance vs. safety tradeoff is not thoroughly analyzed, quantified, and exploited. 
    A key limitation in the soft robotic literature is that the tradeoff between performance and safety remains insufficiently quantified and exploited. [Chapter~3]
    % \item Instead of fiddling around with fancy new shape-sensing approaches (e.g., magnetic or fluidic sensors), we should adopt for proprioception established, robust, and commercialized sensors (e.g., cameras, IMUs). [This thesis, SLAM]
    \item % Significant progress in soft robot proprioception can be achieved by a better hardware and software integration of existing commercialized sensing modalities such as cameras or IMUs rather than involving novel immature technologies. 
    Substantial progress in soft robot proprioception can be achieved by optimizing the hardware and software integration of established commercial sensing modalities—such as cameras or IMUs—instead of relying on novel, yet unproven, technologies.
    % Optimizing the integration of commercial sensors (e.g., cameras, IMUs) can significantly enhance soft robot proprioception over unproven novel technologies.
    [Chapter~4]
    % \item Even though the manufacturing of soft robots based on metamaterials such as HSAs requires advanced manufacturing techniques (e.g., high-precision 3D printing), serial production . [This thesis, Modeling \& Control of HSA Robots]
    % \item Operation-space control is a much more natural fit for continuum soft robots than configuration-space control. [This thesis, HSA Control]
    % \item Currently, it is too dangerous to control rigid robots with EEG signals. [This thesis, Brain Control]
    \item % We need compliance in both body and brain when operating robots close to humans. 
    We need to strive for compliance in both body and brain when operating robots close to humans.
    [Chapter~7]
    % \item We should not only strive for soft robots to consist of compliant structures but also be actuated by compliant control strategies (e.g., low feedback gain, without integral terms). [This thesis, Chapter 7]
    % \item Modeling and considering actuator dynamics in control is essential on our path towards dynamic, fast-moving soft robots. [This thesis, Chapter 8]
    % \item The key to improving the generalization performance of learned dynamical models is to integrate structural, geometrical, and physical priors as inductive biases. [This thesis, Chapter 10]
    % \item Instead of relying on auxiliary tasks such as image classification, the best representations of (soft) robotic systems from high-dimensional observations (e.g., images) are learned by minimizing dynamical prediction losses with the latent dynamics sharing the physical structure with the original system. [This thesis, Chapter 11]
    % \item Incorporating physics into latent dynamical models unlocks computationally efficient and provably stable latent space control. [This thesis, Chapter 11]
    % \item Stability guarantees and machine learning are not inherently contradictory concepts [This thesis, Chapter 11.]
    % \item Integrating motion policies parametrized by dynamical systems (i.e., stable motion primitives) with state-of-the-art large-scale policy learning approaches will drastically increase data efficiency, smoothness of motions, and policy robustness to unseen states. [This thesis, Chapter 12]
    \item % The lack of integrated benchmarks and baselines (i.e., combining soft robot design, sensing, actuation, and control) is the main issue holding back soft robotics research.
    A primary barrier to advancing soft robotics research is the absence of integrated benchmarks and baselines.
    % \item The research progress in the domain of soft robotics will drastically accelerate upon the commercial availability of well-designed continuum soft robotic manipulators.
    % \item The combination of (a) large-scale data and (b) incorporation of (physical) structure and stability guarantees into the learning models \& policies will (finally) allow autonomous robots to be both effective and robust.
    \item % Scaling data quantity and variety alone is insufficient for learning effective and robust robot models and autonomy policies if we cannot incorporate physical structure and stability guarantees.
    Simply increasing the quantity and diversity of data is not enough to develop effective, robust robot models and motion policies unless we also integrate physical structure and stability guarantees.
    % \item Cross-embodiment policy learning severely decreases the maximum achievable performance on highly dynamic motion tasks.
    \item % Learning motion policies jointly among various robot models (i.e., cross-embodiment policies) severely decreases the maximum achievable performance on highly dynamic motion tasks.
    % Jointly learning motion policies across different robot models (i.e., cross embodiment policies) (O’Neill \textit{et al.}, 2024) significantly reduces the maximum achievable performance on highly dynamic motion tasks.
    Although motion policies jointly learned across different robots can improve generalization, specialized policies for each robot are essential to fully exploit its characteristics and maximize performance on highly dynamic tasks.
    %
    % \item Large Language Models (LLMs) should not be used for robotic control but instead solely as a Human-Robot Interaction (HRI) interface
    % \item Without the establishment of international treaties providing guardrails to the deployment of robot within the next decade, the negative societal impact of robotics will outweigh its benefits.
    % \item Introductory leadership training must be a mandatory component of any Doctoral education program. % Alternatively: Incorporating leadership training as a mandatory component of Doctoral education programs is essential for the development of well-rounded Ph.D. students.
    % \item To maximize the impact of our published research, more research engineers should be hired even if that means that there is less funding for hiring Ph.D.s and PostDocs.
    % \item The impact of the research at the department of Cognitive Robotics would be increased by hiring more research engineers even if that means that less funding is available for academic positions.
    % \item There should exist career tracks at universities for researchers who do not want to become professors.
    % \item The number of positions and the research community recognition for researchers on a non-faculty career track should be increased.
    % \item The research community needs to increase the recognition for non-faculty researchers, for example, by inviting them for keynotes at major workshops and conferences or by announcing specialized awards and grants.
    % \item Expanding at universities non-managerial career paths that are solely dedicated to research would enhance research quality and its societal impact.
    \item Universities expanding non-managerial career paths solely dedicated to research would enhance research quality and its societal impact.
    % \item The availability of the \emph{Senaatszaal} should not be the bottleneck for holding Ph.D. defenses at TU Delft.
    \item % Structuring large tech companies as worker cooperatives would foster more balanced decision-making, promote inclusivity, and reduce (political) division among users.
    Restructuring large tech companies as worker cooperatives would promote balanced decision-making, enhance inclusivity, and reduce societal division.
    \item % If international treaties establishing guidelines and guardrails for the deployment of robots \& AI are not implemented within the current decade, the societal drawbacks of robotics will surpass its advantages.
    Unless international treaties that establish guidelines and guardrails for deploying robots and AI are implemented within this decade, the societal drawbacks of robotics are likely to outweigh its benefits. 
    % (G. Hinton, NYT 2023)
    % https://www.nytimes.com/2023/05/01/technology/ai-google-chatbot-engineer-quits-hinton.html
    % https://mitsloan.mit.edu/ideas-made-to-matter/why-neural-net-pioneer-geoffrey-hinton-sounding-alarm-ai
\end{enumerate}

% \bigskip
% \bigskip

%% Apart from the name and title of the supervisor, the following text is
%% dictated by the promotieregelement.
\begin{center}
These propositions are regarded as opposable and defendable and have been approved as such by the promotor Prof.\ Dr.\ R.\ Babu\v{s}ka and the copromotor Dr.\ C.\ Della Santina.
\end{center}

%% \clearpage
%% {\selectlanguage{dutch}

%% \begin{center}

%% {\Large\titlefont\bfseries Stellingen}

%% \bigskip

%% behorende bij het proefschrift

%% \bigskip

%% %% Print the title.
%% {\makeatletter
%% \titlestyle\bfseries\large\@title
%% \makeatother}

%% %% Print the optional subtitle.
%% {\makeatletter
%% \ifx\@subtitle\undefined\else
%%     \titlefont\titleshape\@subtitle
%% \fi
%% \makeatother}

%% \bigskip

%% door

%% \bigskip

%% %% Print the full name of the author.
%% \makeatletter
%% {\large\titlefont\bfseries\@firstname\ {\titleshape\@lastname}}
%% \makeatother

%% \end{center}

%% \bigskip
%% \bigskip

%% \begin{enumerate}

%% \item Stelling 1.
%% \item Stelling 2.
%% \item Stelling 3.
%% \item Stelling 4.
%% \item Stelling 5.
%% \item Stelling 6.
%% \item Stelling 7.
%% \item Stelling 8.
%% \item Stelling 9.
%% \item Stelling 10.

%% \end{enumerate}

%% \bigskip
%% \bigskip

%% %% Apart from the name and title of the supervisor, the following text is
%% %% dictated by the promotieregelement.
%% \begin{center}
%% Deze stellingen worden opponeerbaar en verdedigbaar geacht en zijn als zodanig goedgekeurd door de promotoren prof.\ dr.\ A.\ van Deursen and dr.\ A.\ Zaidman.
%% \end{center}

%% }

\end{document}

