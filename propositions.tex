\documentclass{propositions}

%% Turn off page numbering for the propositions and make the margins on both
%% sides equal and symmetrical.
\geometry{twoside=false}
\pagestyle{empty}

%% \RequirePackage{unicode-math}

% \setmainfont[Path = fonts/libertinus/, ItalicFont=libertinusserif-italic.otf, BoldFont=libertinusserif-bold.otf, BoldItalicFont=libertinusserif-bolditalic.otf]{libertinusserif-regular.otf}
% \setsansfont[Path = fonts/libertinus/, BoldFont=libertinussans-bold.otf, ItalicFont=libertinussans-italic.otf]{libertinussans-regular.otf}
% \setmathfont[Path = fonts/libertinus/]{libertinusmath-regular.otf}
% \setmonofont[Scale=MatchLowercase, Path=fonts/inconsolata/]{Inconsolata-Regular.ttf}
%% The default style for text is Tahoma (sans-serif).
%\renewcommand*\familydefault{\rmfamily}

\begin{document}

%% Specify the title and author of the thesis. This information will be used on
%% both the English and Dutch side and in the metadata of the final PDF.
% \title{Towards Safe and Effective Model-based Control of Soft Robots: A Journey From First-Principle to Data-driven Models Incorporating Physical Structure}
\title{Model-based Control of Soft Robots: Bridging Physics and Data-driven Models}
\author{Maximilian}{Stölzle}

\begin{center}

{\Large\titlefont\bfseries Propositions}

\medskip

accompanying the dissertation

\medskip

%% Print the title.
{\makeatletter
\titlestyle\bfseries\large\@title
\makeatother}

%% Print the optional subtitle.
{\makeatletter
\ifx\@subtitle\undefined\else
    \titlefont\titleshape\@subtitle
\fi
\makeatother}

\medskip

by

\medskip

%% Print the full name of the author.
\makeatletter
{\large\titlefont\bfseries\@firstname\ {\titleshape\@lastname}}
\makeatother

\end{center}

\bigskip

\begin{enumerate}
    % \item We should not regard the material softness (i.e., the Elastic modulus) as the sole driver of safety in soft robots. [This thesis, Safety \& Co-design]
    % \item Contrary to the current soft robot design practices, the material softness is not the sole driver of safety in soft robots. [This thesis, Safety \& Co-design]
    \item A major shortcoming in the existing soft robotics literature is that the performance vs. safety tradeoff is not thoroughly analyzed, quantified, and exploited. [This thesis, Safety \& Co-design]
    % \item Instead of fiddling around with fancy new shape-sensing approaches (e.g., magnetic or fluidic sensors), we should adopt for proprioception established, robust, and commercialized sensors (e.g., cameras, IMUs). [This thesis, SLAM]
    \item Significant progress in soft robot proprioception can be achieved by a better hardware and software integration of existing commercialized sensing modalities such as cameras or IMUs rather than involving novel immature technologies. [This thesis, SLAM]
    % \item Even though the manufacturing of soft robots based on metamaterials such as HSAs requires advanced manufacturing techniques (e.g., high-precision 3D printing), serial production . [This thesis, Modeling \& Control of HSA Robots]
    % \item Operation-space control is a much more natural fit for continuum soft robots than configuration-space control. [This thesis, HSA Control]
    % \item Currently, it is too dangerous to control rigid robots with EEG signals. [This thesis, Brain Control]
    \item We need compliance in both body and brain when operating robots close to humans. [This thesis, Brain Control]
    % \item We should not only strive for soft robots to consist of compliant structures but also be actuated by compliant control strategies (e.g., low feedback gain, without integral terms). [This thesis, EEG]
    % \item Modeling and considering actuator dynamics in control is essential on our path towards dynamic, fast-moving soft robots. [This thesis, Backstepping]
    % \item Instead of relying on auxiliary tasks such as image classification, the best representations of (soft) robotic systems from high-dimensional observations (e.g., images) are learned by minimizing dynamical prediction losses with the latent dynamics sharing the physical structure with the original system. [This thesis, CON]
    \item Incorporating a physics into latent dynamical models unlocks computationally efficient and provably stable latent space control. [This thesis, CON]
    % \item The key to improving the generalization performance of learned dynamical models is to integrate structural, geometrical, and physical priors as inductive biases. [This thesis, PCS Regression]
    \item Integrating motion policies parametrized by dynamical systems (i.e., stable motion primitives) with state-of-art large-scale policy learning approaches will drastically increase data efficiency, smoothness of motions, and policy robustness to unseen states. [This thesis, Orbitally Stable Motion Primitives]
    \item The lack of integrated benchmarks and baselines (i.e., combining soft robot design, sensing, actuation, and control) is the main issue holding back soft robotics research.
    % \item The research progress in the domain of soft robotics will drastically accelerate upon the commercial availability of well-designed continuum soft robotic manipulators.
    % \item The combination of (a) large scale data, and (b) incorporation of (physical) structure and stability guarantees into the learning models \& policies will (finally) allow autonomous robots to be both effective and robust.
    \item Scaling data quantity and variety alone is insufficient for learning effective and robust robot models and autonomy policies if we cannot incorporate physical structure and stability guarantees.
    % \item Cross-embodiment policy learning severely decrease the maximum achievable performance on highly dynamic motion tasks.
    \item Learning motion policies jointly among various robot models (i.e., cross-embodiment policies) severely decreases the maximum achievable performance on highly dynamic motion tasks.
    % \item Large Language Models (LLMs) should not be used for robotic control but instead solely as a Human-Robot Interaction (HRI) interface
    \item Leadership training must be a mandatory component of any Doctoral education program. % Alternatively: Incorporating leadership training as a mandatory component of Doctoral education programs is essential for the development of well-rounded Ph.D. students.
    % \item To maximize the impact of our published research, more research engineers should be hired even if that means that there is less funding for hiring Ph.D.s and PostDocs.
    \item The impact of the research at the department of Cognitive Robotics would be increased by hiring more research engineers even if that means that less funding is available for academic positions.
    % \item There should exist career tracks at universities for researchers who do not want to become professors.
    % \item The number of positions and the research community recognition for researchers on a non-faculty career track should be increased.
    \item The research community needs to increase the recognition for non-faculty researchers, for example, by inviting them for keynotes at major workshops and conferences or by announcing specialized awards and grants.
    % \item The availability of the \emph{Senaatszaal} should not be the bottleneck for holding Ph.D. defenses at TU Delft.
\end{enumerate}

\bigskip
\bigskip

%% Apart from the name and title of the supervisor, the following text is
%% dictated by the promotieregelement.
\begin{center}
These propositions are regarded as opposable and defendable and have been approved as such by the
promotor Prof.\ Dr.\ R.\ Babu\v{s}ka.
and the co-promotor Dr.\ C.\ Della Santina
\end{center}

%% \clearpage
%% {\selectlanguage{dutch}

%% \begin{center}

%% {\Large\titlefont\bfseries Stellingen}

%% \bigskip

%% behorende bij het proefschrift

%% \bigskip

%% %% Print the title.
%% {\makeatletter
%% \titlestyle\bfseries\large\@title
%% \makeatother}

%% %% Print the optional subtitle.
%% {\makeatletter
%% \ifx\@subtitle\undefined\else
%%     \titlefont\titleshape\@subtitle
%% \fi
%% \makeatother}

%% \bigskip

%% door

%% \bigskip

%% %% Print the full name of the author.
%% \makeatletter
%% {\large\titlefont\bfseries\@firstname\ {\titleshape\@lastname}}
%% \makeatother

%% \end{center}

%% \bigskip
%% \bigskip

%% \begin{enumerate}

%% \item Stelling 1.
%% \item Stelling 2.
%% \item Stelling 3.
%% \item Stelling 4.
%% \item Stelling 5.
%% \item Stelling 6.
%% \item Stelling 7.
%% \item Stelling 8.
%% \item Stelling 9.
%% \item Stelling 10.

%% \end{enumerate}

%% \bigskip
%% \bigskip

%% %% Apart from the name and title of the supervisor, the following text is
%% %% dictated by the promotieregelement.
%% \begin{center}
%% Deze stellingen worden opponeerbaar en verdedigbaar geacht en zijn als zodanig goedgekeurd door de promotoren prof.\ dr.\ A.\ van Deursen and dr.\ A.\ Zaidman.
%% \end{center}

%% }

\end{document}

