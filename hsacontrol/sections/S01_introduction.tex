\section{Introduction}
% The deformability, adaptiveness, and compliance of invertebrates serve as an inspiration for continuum soft robots.
While serial continuum soft robots have been intensively investigated in recent years~\cite{della2023model}, parallel soft robots~\cite{hughes2020extensible} % more references for parallel robots: zhang2020modeling
are less studied despite exhibiting exciting properties such as an improved stiffness-to-weight ratio. 
One recent development in this field is robots based on \glspl{HSA}~\cite{truby2021recipe, kaarthik2022motorized, stolzle2024guiding} in which multiple \gls{HSA} rods are connected at their distal end through a rigid platform. 
Twisting of the proximal end of an \gls{HSA} % with an electric actuator 
causes the rod to elongate 
% Differential electric actuation of the HSA robot 
and enables complex motion primitives in 3D space.
Recent work has investigated 
% proprioception~\cite{zhang2022vision}, 
the mechanical characterization~\cite{good2022expanding}, simulation~\cite{stolzle2023modelling}, and kinematic modeling~\cite{garg2022kinematic, stolzle2023modelling} of \gls{HSA} robots but control has yet to be tackled.
% Still, the task of control is still unsolved as \textcolor{orange}{prior work} and solutions need to be developed on how to deal with the challenges of underactuation and changing stiffness properties. 
In this work, we make a first step towards achieving task-space control by designing model-based regulators for planar motions. Our approach considers essential characteristics of \gls{HSA} robots, such as underactuation, shear strains, and varying stiffness. % and thus can serve as a building block for future research.

In Chapter~\ref{chp:hsamodel}, we derived a dynamic model for planar \gls{HSA} in Euler-Lagrange form and experimentally verified it.
We notice that the resulting planar dynamics are underactuated and that the actuation forces are non-affine with respect to the control inputs, which are the motor angles. The latter is a peculiarity of these systems, rarely observed in other robots.
Based on the model knowledge, we propose in this chapter two control strategies for planar \gls{HSA} robots capable of regulating the end-effector towards a desired position in task-space.
The first strategy, as shown in Fig.~\ref{fig:hsacontrol:configuration_space_regulation:block_scheme_closed_loop_control}, performs steady-state planning to identify an admittable configuration and steady-state control input matching the desired end-effector position and then subsequently applies a P-satI-D feedback controller~\cite{pustina2022p} on the collocated form~\cite{pustina2024input} of the system dynamics.
The second strategy, as shown in Fig.~\ref{fig:hsacontrol:task_space_impedance_control:block_scheme_closed_loop_control}, directly regulates the end-effector position using a Cartesian impedance controller that fully preserves the softness of the robot.

In summary, we state our contributions as (i) a provably stable model-based control strategy for guiding the end-effector of the robot towards a desired position in Cartesian space with a configuration-space controller that combines an integral-saturated PID with a potential shaping feedforward term, (ii) a Cartesian impedance controller that allows combining the passive compliance of the \gls{HSA} robot with active compliance in the control strategy and (iii) extensive experimental verification of both control strategies. 
A video accompanies this chapter explaining the methodology and displaying video recordings of the control experiments\footnote{\url{https://youtu.be/7PgKnE_MOsY}}.


%Structure
%\begin{enumerate}
%    \item Why soft robots
%    % \item Parallel soft robots have not been widely tested out,  have interesting characteristics such as higher bending stiffness with low weight
%    \item HSAs have this special mechanism in which the motors acts throught he stiffness of the hsa on the robot shape
%    \item Control has only been performed with PID, no model-based control. Problem of underactuation needs to be solved. Way to deal with changing stiffness characteristics
%\end{enumerate}
%
%Our contributions
%\begin{itemize}
%    \item Closed-form inverse kinematics for planar continuum robots modelled using PCS
%    \item An Euler-Lagrangian model for the dynamics of HSA robots, which is then also verified experimentally 
%    \item Proposal of a control strategy involving mapping from task-space to configuration-space and PID controller respecting the underactuation
%    \item Experiments involving Model-based regulation of HSA robots
%\end{itemize}