\section{Experimental insights}
This chapter presented two strategies for effective, model-based regulation of planar \gls{HSA} robots: the first proposes a configuration-space controller in conjunction with steady-state planning for mapping task-space setpoints to configuration-space setpoints, and the second introduces a task-space impedance controller.

\subsection{Configuration-space regulation}

First, we introduced a configuration-space controller that combines an integral-saturated PID controller with a potential-shaping feedforward term.
The excellent agreement of the model with the actual system behavior (as shown in Sec.~\ref{sec:hsamodel:planar_hsa_robot_model}) enables our model-based controllers to perform very well at the task of setpoint regulation.
For the model-based controllers, any mismatch between the dynamic model and the actual system (as analyzed in Section~\ref{sub:hsamodel:planar_hsa_robot_model:model_verification}) has two impacts: (i) the steady-state planning provides us with a desired configuration $q^\mathrm{d}$ which the underactuated robot cannot achieve. This then, in turn, causes a small steady-state error in the end-effector position as seen for the manual setpoints in Fig.~\ref{fig:hsacontrol:experimental_results:configuration_space_regulation:fpu:p_sati_d:pee} \& \ref{fig:hsacontrol:experimental_results:configuration_space_regulation:fpu:p_sati_d_plus_gc:pee} 
and for the continuous references in Fig.\ref{fig:hsacontrol:experimental_results:configuration_space_regulation:fpu:task_space_trajectories}. This steady-state error is absent in the baseline PID as its integral term acts directly in task space. We suggest that future work include an integral term directly on the end-effector position to remove the remaining steady-state error of the model-based controller. Secondly, as (ii), model errors will lead to an offset in the planned steady-state actuation $\phi^\mathrm{ss}$. Therefore, applying a constant $\phi^\mathrm{ss}$ will not move the robot exactly to $p_\mathrm{ee}^\mathrm{d}$. As shown in Fig.~\ref{fig:hsacontrol:experimental_results:configuration_space_regulation:fpu:p_sati_d:phi}, the P-satI-D feedback term can compensate for this effect through its proportional and integral terms applied in the collocated variables \& \ref{fig:hsacontrol:experimental_results:configuration_space_regulation:fpu:p_sati_d_plus_gc:phi} 
As clearly seen in Fig.~\ref{fig:hsacontrol:experimental_results:configuration_space_regulation:fpu:p_sati_d:phi}, the control input drifts away from the planned $\phi^\mathrm{ss}$ to counteract the modeling errors.
In particular, for small to medium bending and axial strains, we observe a very good agreement between the model and the actual system behavior and, therefore, excellent control performance. 
The story slightly changes for more significant bend and twist angles. While the system identification showed that the expressiveness of the model is sufficient, we saw for some of our control experiments significant errors in our feedforward terms, which needed to be compensated by the P-satI-D feedback controller. Precisely, Fig.~\ref{fig:hsacontrol:experimental_results:configuration_space_regulation:fpu:p_sati_d:q} displays errors in the planned shear strain, which subsequently also leads to minor steady-state deviations in the x-coordinate of the end-effector as it can be seen in Fig.~\ref{fig:hsacontrol:experimental_results:configuration_space_regulation:fpu:p_sati_d:pee}. We hypothesize that these errors are mainly caused by the hysteresis characteristics of the \glspl{HSA}~\cite{good2022expanding}. This hypothesis is corroborated by the fact that we had to re-calibrate the axial rest strain $\sigma_\mathrm{ax}^0$ before the start of each experiment.

\subsection{Task-space impedance control}
The Cartesian impedance controller, on the other hand, is designed to regulate the end-effector position directly and allows us to shape the impedance in task-space and, with that, fully preserve the softness of the robot.
It is therefore, in particular, suitable for applications where the robot interacts with the environment. For example, we will in Chapter~\ref{chp:braincontrol} demonstrate how the impedance controller can be combined with motor-imagery brain signals to guide the robot towards a desired position and how the impedance controller guarantees safety even when the setpoints are wrongly planned.
This impedance needs to be traded off with steady-state errors. We observed that the impedance controller is particularly sensitive to the model errors, as it does not have an integral term to compensate for them.