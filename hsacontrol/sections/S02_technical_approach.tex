\section{Technical approach}
In the following, we consider a parallel HSA robot moving in a plane. 
First, we derive the kinematic and dynamic models. Subsequently, we devise a planning and control strategy to move the end-effector (i.e., the platform) to a desired position in Cartesian space.

\textcolor{red}{
    TODO: connect to HSA model chapter.
    We stress that (a) the derived dynamical model is not affine in the control input and (b) the system is underactuated.
    We provide examples in Fig.~\ref{fig:hsacontrol:kinematics:workspace} of the operational workspace that can be achieved with this kinematic model.
}

\subsection{Control}\label{sub:hsacontrol:methodology:control}
Our goal is to control the end-effector, which is defined as the distal surface of the platform, to a desired position in Cartesian space $p_\mathrm{ee}^\mathrm{d} \in \mathbb{R}^2$. 
However, the mapping into configuration space is not trivial as we do not know which end-effector orientation $\theta_\mathrm{ee}$ is feasible at steady-state. 
To tackle this challenge, we perform steady-state planning identifying admittable configurations $q^\mathrm{d}$ and matching steady-state actuations $\phi^\mathrm{ss}$, which allow the robot's end-effector to statically remain at $p_\mathrm{ee}^\mathrm{d}$. More details on the used planning procedure can be found in Section~\ref{sub:hsacontrol:experiments:steady_state_planning}.

In principle, we can command $\phi = \phi^\mathrm{ss}$ to achieve regulation towards the desired end-effector position.
Nevertheless, we add a feedback controller to compensate for any errors in $\phi^\mathrm{ss}$ caused by unmodelled effects such as hysteresis. Unfortunately, the non-affine actuation $\alpha(q,\phi)$ would complicate the design of such a feedback controller.
% Now, we can regulate the robot in configuration space towards $q^\mathrm{d}$. However, we notice that our system is non-affine in the control input $\phi$. 
Therefore, we perform a first-order Taylor expansion of the actuation forces with respect to $\phi$ resulting in a configuration-dependent actuation matrix $A_{\phi^\mathrm{ss}}(q) = \frac{\partial \alpha}{\partial \phi} \big|_{\phi = \phi_\mathrm{ss}} \in \mathbb{R}^{3 \times 2}$. This allows us to re-write the right side of the \gls{EOM} as $\tau_q = \alpha(q^\mathrm{ss}, \phi^\mathrm{ss}) + A_{\phi^\mathrm{ss}}(q) \, u$ where $u = \phi - \phi^\mathrm{ss}$ is the new control input.
% We remark that $\alpha(q^\mathrm{ss}, \phi^\mathrm{ss})$ is already compensating for the gravitational and elastic forces at the desired configuration. 
To improve the robustness of the control loop, we compute $u$ with a P-satI-D control law~\cite{pustina2022p}. However, our system is underactuated and in a non-collocated form.
Therefore, we apply a coordinate transformation $h: q \rightarrow \varphi \in \mathbb{R}^3$ recently introduced by Pustina et al.~\cite{pustina2024input} which maps the \gls{EOM} into a form where $\phi$ applies direct forces on the actuated configuration variables. The map is given by { $h(q) = \begin{bmatrix}
    \int_0^t \dot{q}^\mathrm{T} A_{\phi^\mathrm{ss}}(q) \mathrm{d}\tau, & \sigma_\mathrm{sh}
\end{bmatrix}^\mathrm{T} = \begin{bmatrix}
    h_1(q), & h_2(q), & \sigma_\mathrm{sh}
\end{bmatrix}^\mathrm{T}$}
with
\begin{footnotesize}
\begin{multline}\footnotesize
    h_i(q) = 
    C_{\mathrm{S},\mathrm{ax}} \, \frac{h_i}{l^0} \, \Big [ 2 \, \varepsilon_i(\phi^\mathrm{ss}_i) \left ( \pm r_\mathrm{off} \kappa_\mathrm{be} + \sigma_\mathrm{ax} \right ) \mp r_\mathrm{off}^2 \frac{\kappa_\mathrm{be}^2}{2} \pm r_\mathrm{off} \, \sigma_\mathrm{ax}^0 \, \kappa_\mathrm{be} \mp r_\mathrm{off} \, \kappa_\mathrm{be} \, \sigma_\mathrm{ax} + \sigma_\mathrm{ax}^0 \, \sigma_\mathrm{ax}\\ - \frac{\sigma_\mathrm{ax}^2}{2} \Big ] 
    + C_{\mathrm{S},\mathrm{b}} \, \frac{h_i}{l^0} \, \Big [ \kappa_\mathrm{be}^0 \, \kappa_\mathrm{be} - \frac{\kappa_\mathrm{be}^2}{2} \Big ] 
    + C_{\mathrm{S},\mathrm{sh}} \, \frac{h_i}{l^0} \, \Big [\sigma_\mathrm{sh}^0 \, \sigma_\mathrm{sh} - \frac{\sigma_\mathrm{sh}^2}{2} \Big ]
    + \hat{S}_\mathrm{ax} \, \frac{h_i}{l^0} \, C_\varepsilon \Big [ \pm r_\mathrm{off} \, \kappa_\mathrm{be} + \sigma_\mathrm{ax} \Big ].
\end{multline}
\end{footnotesize}
% \begin{equation}\tiny
%     \begin{bmatrix}
%          \frac{h_{1} \cdot \left(2 C_{S a1} C_\varepsilon h_{1} \phi_{1} roff_{1} \kappa_\mathrm{be} + 2 C_{S a1} C_\varepsilon h_{1} \phi_{1} \sigma_\mathrm{ax} - \frac{C_{S a1} l_{1} roff_{1}^{2} \kappa_\mathrm{be}^{2}}{2} + C_{S a1} l_{1} roff_{1} \sigma_\mathrm{ax}^0 \kappa_\mathrm{be} - C_{S a1} l_{1} roff_{1} \kappa_\mathrm{be} \sigma_\mathrm{ax} + C_{S a1} l_{1} \sigma_\mathrm{ax}^0 \sigma_\mathrm{ax} - \frac{C_{S a1} l_{1} \sigma_\mathrm{ax}^{2}}{2} + C_{S b1} \kappa_{b eq1} l_{1} \kappa_\mathrm{be} - \frac{C_{S b1} l_{1} \kappa_\mathrm{be}^{2}}{2} + C_{S sh1} l_{1} \sigma_\mathrm{sh}^0 \sigma_\mathrm{sh} - \frac{C_{S sh1} l_{1} \sigma_\mathrm{sh}^{2}}{2} + C_\varepsilon S_{a hat1} l_{1} roff_{1} \kappa_\mathrm{be} + C_\varepsilon S_{a hat1} l_{1} \sigma_\mathrm{ax}\right)}{l_{1}^{2}}\\
%          \frac{h_{2} \cdot \left(2 C_{S a2} C_\varepsilon h_{2} \phi_{2} roff_{2} \kappa_\mathrm{be} + 2 C_{S a2} C_\varepsilon h_{2} \phi_{2} \sigma_\mathrm{ax} - \frac{C_{S a2} l_{1} roff_{2}^{2} \kappa_\mathrm{be}^{2}}{2} + C_{S a2} l_{1} roff_{2} \sigma_\mathrm{ax}^0 \kappa_\mathrm{be} - C_{S a2} l_{1} roff_{2} \kappa_\mathrm{be} \sigma_\mathrm{ax} + C_{S a2} l_{1} \sigma_\mathrm{ax}^0 \sigma_\mathrm{ax} - \frac{C_{S a2} l_{1} \sigma_\mathrm{ax}^{2}}{2} + C_{S b2} \kappa_{b eq2} l_{1} \kappa_\mathrm{be} - \frac{C_{S b2} l_{1} \kappa_\mathrm{be}^{2}}{2} + C_{S sh2} l_{1} \sigma_\mathrm{sh}^0 \sigma_\mathrm{sh} - \frac{C_{S sh2} l_{1} \sigma_\mathrm{sh}^{2}}{2} + C_\varepsilon S_{a hat2} l_{1} roff_{2} \kappa_\mathrm{be} + C_\varepsilon S_{a hat2} l_{1} \sigma_\mathrm{ax}\right)}{l_{1}^{2}}\\
%          \sigma_\mathrm{sh}
%     \end{bmatrix}
% \end{equation}
The Jacobian $J_\mathrm{h}(q) = \frac{\partial h}{\partial q}$ is used to formulate the dynamics $M_\varphi \ddot{\varphi} + \eta(\varphi, \dot{\varphi}) + G_\varphi + K_\varphi + D_\varphi \, \dot{\varphi} = J_\mathrm{h}^\mathrm{-T}(q) \, \alpha(q^\mathrm{ss},\phi^\mathrm{ss}) + A_\varphi \, u $ in the collocated variables~\cite{khatib1987unified}, where $A_\varphi^\mathrm{T} = \begin{bmatrix}
    \mathbb{I}^{2} & 0^\mathrm{2 \times 1}
\end{bmatrix}^\mathrm{T}$. In the following, we will denote with the subscript $a$ the first two actuated coordinates $\varphi_\mathrm{a}$.
% 
Finally, the full control law of the \emph{P-satI-D} is given in collocated form as
\begin{equation}\label{eq:hsacontrol:gravity_compensation_controller}
    \phi = \phi^\mathrm{ss} + K_\mathrm{p} (\varphi_\mathrm{a}^\mathrm{d} - \varphi) - K_\mathrm{d} \dot{\varphi}_\mathrm{a} + K_\mathrm{i} \int_0^t \tanh(\gamma \, ( \varphi_{\mathrm{a},t'}^\mathrm{d}-\varphi_{\mathrm{a},t'})) \: \mathrm{d} t',
\end{equation}
where $K_\mathrm{p}, K_\mathrm{d}, K_\mathrm{i} \in \mathbb{R}^{2 \times 2}$ are the proportional, derivative, and integral gains respectively, and $\gamma \in \mathbb{R}^{2 \times 2}$ horizontally compresses the hyperbolic tangent. While the proposed P-satI-D control law compensates gravity through $\phi^\mathrm{ss}$, we can extend the approach to include gravity cancellation (\emph{P-satI-D + GC}) by evaluating $G_{\varphi,\mathrm{a}}$ at the current configuration:
\begin{equation}\label{eq:hsacontrol:gravity_cancellation_controller}
    \phi = \phi^\mathrm{ss} - G_{\varphi,\mathrm{a}}(q^\mathrm{d}) + G_{\varphi,\mathrm{a}}(q) + K_\mathrm{p} (\varphi_\mathrm{a}^\mathrm{d} - \varphi) - K_\mathrm{d} \dot{\varphi}_\mathrm{a} + K_\mathrm{i} \int_0^t \tanh(\gamma \, ( \varphi_{\mathrm{a},t'}^\mathrm{d}-\varphi_{\mathrm{a},t'})) \: \mathrm{d} t'.
\end{equation}
The implementation of all control laws is available on GitHub\footnote{\url{https://github.com/tud-phi/hsa-planar-control}}.