\section{Conclusion}
% In this chapter, we first highlighted the absence of a safety metric for soft robots, a gap that prevents the community from quantifying their safety advantages over rigid robot counterparts. This shortfall often results in designs that are ineffective in real-world tasks because they are excessively soft and lack sufficient payload capacity. We then discussed how a quantitative safety metric could be exploited in both design and control applications, ensuring safety without unduly compromising system performance by making the robot overly soft or the motion behavior overly cautious. Finally, we outlined the criteria such a safety metric must meet.
In this chapter, we first underscored the absence of a safety metric for soft robots—a gap that hinders the community from quantifying their safety benefits compared to rigid robots. This shortfall often leads to designs that perform poorly in real-world tasks because they tend to be overly soft and lack sufficient payload capacity. We then explored how a quantitative safety metric could be applied in both design and control contexts, ensuring safety without forcing the robot to be too soft or its motion overly cautious. Finally, we defined the criteria that such a safety metric must satisfy.

% Subsequently, we derived a quantitative safety metric for soft robots while considering the maximum contact pressure experienced during the collision, analog to the existing ISO norms for \glspl{Cobot}~\citep{Isots_15066_2016}, as the safety criterion.
% For this the derivation of the safety metric, we rely on existing soft robot dynamic models, such as \gls{PCS}~\citep{renda2018discrete}\footnote{Please refer to Chapter~\ref{chp:background} for a discussion about alternative soft robot models that the safety metric could be based on.} and assume the human to be constrained, which represents the \emph{worst case}~\citep{haddadin2009requirements}.
% The proposed safety metric comes into two flavors: the \gls{ISC} can be evaluated in closed-form and determines the maximum contact pressure experienced during a soft robot - human collision for a given initial condition, consting of soft robot configuration, velocity and contact geometry, and a given constant (e.g., worst-case) actuation. The \gls{ISC} is in particular suitable for \emph{safety-aware control} applications.
% However, when designing soft robots, we need to ensure safety within a given operating envelope consisting of infinitely many initial conditions. For this case, we developed the \gls{DHC} which determines the maximum \gls{ISC} for a given specification of operating conditions (e.g., maximum deformation, actuator bounds, configuration-space velocities), in particular across all possible contact geometries.
Next, we derived a quantitative safety metric for soft robots by considering the maximum contact pressure during a collision, analogous to the existing ISO norms for \glspl{Cobot}~\citep{Isots_15066_2016}, as our safety criterion. For the derivation, we relied on established soft robot dynamic models, such as \gls{PCS}~\citep{renda2018discrete}\footnote{Please refer to Chapter~\ref{chp:background} for a discussion of alternative soft robot models that could serve as the basis for the safety metric.}, and assumed a constrained human model to represent the \emph{worst case} scenario~\citep{haddadin2009requirements}. The proposed safety metric is offered in two forms: the \gls{ISC}, which can be computed in closed-form and estimates the maximum contact pressure during a soft robot–human collision given a specific initial condition—comprising the robot’s configuration, velocity, contact geometry, and a predetermined (e.g., worst-case) actuation—and is particularly well-suited for \emph{safety-aware control} applications. In contrast, when designing soft robots, it is necessary to guarantee safety across an operating envelope that encompasses infinitely many initial conditions. For this purpose, we developed the \gls{DHC}, which determines the maximum \gls{ISC} for a given set of operating conditions (e.g., maximum deformation, actuator bounds, configuration-space velocities), especially considering all potential contact geometries.

% Crucially, the proposed safety metric meets all (compulsory) requirements specified in Sec.~\ref{sub:safetymetric:safety_metric_requirements}, as it, for example, accounts for collisions along the entire soft robot body, the assumptions and simplifications taken during the derivation lead to a conservative estimate of the true safety, the computation of the safety metric is tractable as (a) we do not need to the collision dynamics in time but rather have access to the maximum contact force in closed form, (b) part of the computation can be vectorized/parallelized. Furthermore, the safety metric is differentiable - even analytically as our implementation in JAX~\citep{jax2018github} provides gradients via autodifferentiation.
Notably, the proposed safety metric fulfills all mandatory requirements outlined in Sec.~\ref{sub:safetymetric:safety_metric_requirements}. For instance, it takes into account collisions along the entire soft robot body, and the assumptions and simplifications made during its derivation result in a conservative estimate of true safety. Additionally, the metric is computationally tractable because (a) it bypasses the need to simulate collision dynamics over time by providing the maximum contact pressure in closed form, and (b) portions of the computation can be vectorized or parallelized. Moreover, the metric is differentiable—even analytically—since our implementation in JAX~\citep{jax2018github} offers gradients through autodifferentiation.

% For future work, two critical aspects then emerge: (1) experimentally validating the safety criterion—i.e., measuring the discrepancy between the predicted and actual safety criterion—and (2) establishing appropriate thresholds~\citep{Isots_15066_2016, behrens2022statistical} for the safety criterion (i.e., acceptable injury risk) to ensure safety in human-centric environments. Here, analog the existing literature on rigid robots~\citep{yamada1997evaluation, muttray2014collaborative, behrens2022statistical}, empirical biomechanical studies could play a crucial role in this process.
% It is important to note that safety requirements may differ significantly depending on the specific robotic application and task. For instance, a soft robot designed as a children's toy must adhere to much stricter safety standards than one used for industrial refueling. Understanding such varying requirements is crucial to ensuring safe and effective performance. 
Looking ahead, two key aspects emerge for future work: (1) the experimental validation of the safety criterion—i.e., quantifying the discrepancy between predicted and actual safety—and (2) the establishment of appropriate thresholds~\citep{Isots_15066_2016, behrens2022statistical} for the safety criterion (in terms of acceptable injury risk) to ensure safety in human-centric environments. Drawing inspiration from existing literature on rigid robots~\citep{yamada1997evaluation, muttray2014collaborative, behrens2022statistical}, empirical biomechanical studies could play a crucial role in this process. It is also essential to recognize that safety requirements may vary considerably depending on the specific robotic application and task. For example, a soft robot designed as a children’s toy must comply with much stricter safety standards than one intended for industrial refueling. Understanding these differences is key to achieving safe and effective performance.

% Furthermore, the safety metric itself can be extended and improved. For example, specializing in safety metrics for different soft robot modeling techniques, such as FEM-based and geometric strain-based approaches, will enable more accurate risk assessments tailored to specific designs. Extending these metrics to closed-chain robots, locomotors, and wearable systems will further solidify safety evaluation frameworks across diverse robotic applications.
Furthermore, the safety metric itself can be extended and refined. By developing specialized metrics tailored to different soft robot modeling techniques—such as \gls{FEM}- or \gls{GVS}-based approaches—more accurate risk assessments can be achieved for specific designs. Extending these metrics to closed-chain robots, locomotors, and wearable systems will further strengthen safety evaluation frameworks across a diverse range of robotic applications.

% Finally moving to applications, we find it crucial to develop develop design and control concepts that can exploit the proposed safety metric.
% First considering \emph{safety-aware design}, a critical next step involves identifying which design parameters most significantly affect safety in different types of actuation, such as tendon-driven and pneumatically actuated soft robots. Mapping out these dependencies will provide actionable insights for designers seeking to enhance safety.
% Furthermore, we propose in Appendix~\ref{chp:apx:holisticcodesign} a strategy for co-optimizing the safety of soft robots via co-design~\citep{zardini2023co, spielberg2019learning, wang2024diffusebot, navez2024contributions}, which will allow to explicitly formalize and quantify the tradeoff between safety and performance in soft robotics.
Finally, with regard to practical applications, it is vital to develop design and control strategies that take full advantage of the proposed safety metric. In the context of\emph{safety-aware design}, a crucial next step is to identify which design parameters most significantly impact safety for various actuation types, such as tendon-driven versus pneumatically actuated soft robots. Mapping these dependencies will provide designers with actionable insights to enhance safety. Additionally, as outlined in Appendix~\ref{chp:apx:holisticcodesign}, we propose a holistic co-design strategy for optimizing the safety of soft robots~\citep{zardini2023co, spielberg2019learning, wang2024diffusebot, navez2024contributions}, which will enable the explicit formalization and quantification of the trade-off between safety and performance in soft robotics.
% On the other hand, \emph{safety-aware control} could be achieved via \gls{MPC} or \glspl{CBF}~\citep{ames2016control}, which would enable us to push the performance (e.g., speed~\citep{haggerty2023control}) as much as possible while still guaranteeing safety in all scenarios.
Alternatively, \emph{safety-aware control} can be implemented, for example, using \gls{MPC} or \glspl{CBF}~\citep{ames2016control}, enabling us to maximize performance, such as the motion speed~\citep{haggerty2023control}, while ensuring safety in all scenarios.