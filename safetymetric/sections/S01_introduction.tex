\section{Introduction}
% As already mentioned in Chapter~\ref{chp:introduction}, the soft robotics community frequently emphasizes the "intrinsic safety" and natural compliance of soft robots~\citep{abidi2017intrinsic} - particularly as a differentiating factor from traditional rigid robots and even \glspl{Cobot}~\citep{laschi2014soft, rus2015design, yasa2023overview}.
% However, against scientific best practices, the safety improvement of continuum soft robots compared to rigid manipulators have, to the best of our knowledge, never been quantified. 
% We think that it is crucial for the community to thoroughly validate intrinsic compliance as a very prominent motivation for soft robots in order to justify continued spending into research and development of soft robots~\citep{hawkes2021hard}.
\dropcap{A}s noted in Chapter~\ref{chp:introduction}, the soft robotics community often highlights the “intrinsic safety” and natural compliance of soft robots~\citep{abidi2017intrinsic} as a key advantage over conventional rigid robots and even \glspl{Cobot}~\citep{laschi2014soft, rus2015design, yasa2023overview}. Yet, in contrast to established scientific practices, the safety improvements of continuum soft robots compared to rigid manipulators have not been rigorously quantified. We believe it is essential for the community to validate intrinsic compliance as a fundamental benefit of soft robots, thereby justifying continued investments in their research and development~\citep{hawkes2021hard}.

% Furthermore, as designers currently have no clear guideline for the “right” level of safety, we find that currently they prioritizes an overly soft material for safety.
% Indeed, designers have treated this challenge as a balance between precision and softness, assuming that increasing material compliance inherently improves safety.
% However, \emph{too soft} material choices can compromise the effectiveness of the soft robot, leading to imprecise, oscillatory motions, limited payload capacity, and the inability to exert sufficient forces on the environment~\citep{iida2011soft, cianchetti2013stiff, mazzolai2022roadmap, majidi2014soft, hawkes2017soft}.
% Indeed, we argue that precision vs. softness is an oversimplification, because the safety of soft robots is a function of both body (e.g., morphology) and brain (e.g., control and perception systems).
% Formalizing and quantifying such safety vs. performance tradeoff would enable designers to identify a good balance between the two and maximizing the performance while guaranteeing the necessary safety.
% Indeed, we we anticipate that removing this road-block would lead to more optimal and performant soft robot designs while preserving safety, which is in particular crucial in human-centered environment.
Moreover, in the absence of clear guidelines for determining the optimal level of safety, designers currently tend to favor overly compliant materials. Many treat the challenge as a trade-off between performance/precision and softness, assuming that higher material compliance automatically results in enhanced safety. However, choosing materials that are too soft can undermine the robot’s effectiveness, leading to imprecise or oscillatory motions, reduced payload capacity, and an inability to apply sufficient force to the environment~\citep{iida2011soft, cianchetti2013stiff, mazzolai2022roadmap, majidi2014soft, hawkes2017soft}. We argue that framing the issue as merely a balance between performance and softness oversimplifies the problem, since the safety of soft robots depends on both their body (e.g., morphology) and their brain (e.g., control and perception systems). Formalizing and quantifying this safety–performance trade-off would enable designers to find an optimal balance that maximizes performance while ensuring the necessary safety, a consideration that is particularly critical in human-centered environments.

% Apart from \emph{safety-aware design}, we expect that a safety metric would also be very beneficial for other purposes such as \emph{safety certification} of a soft robotic design or \emph{safety-aware control} where the controller chooses its actions such as that the required safety levels are always met.
% After detailing such potential applications of a safety metric for soft robots in more detail, we devise requirements that a safety metric would need to meet in order to be suitable for the mentioned applications.
% Subsequently, we then propose a new model-based safety metric for soft robots by building an injury criterion on top of the existing ISO norms for collaborative robots~\citep{Isots_15066_2016} while modifying the underlying model to account for the essential characteristics of soft robots, such as the elastic structure that deforms under the influence of internal actuation and external forces, and the potential of collisions along its entire body.
% After modeling the contact model and the collision dynamics between soft robots and humans, we consider analog to ISO/TS 15066:2016~\citep{Isots_15066_2016}, the maximum contact pressure experienced during the collision as a proxy for the injury severity. Instead of requiring computationally expensive simulations, we devise a conservative approximation of the collision dynamics, which allows us to determine the maximum contact pressure in closed form.
% Importantly, the proposed safety metric considers the safety of the closed-loop system, which allows us to take the control policy into account when assessing the severity of the injury.
% The safety metric comes in two flavors: \gls{ISC} computes the injury severity for a known soft robot state, control input, and contact geometry (e.g., point along the soft robot body of contact and collision direction) and is particularly well suited for \emph{safety-aware control} applications. 
% When we want to judge the safety of a soft robot design, we can instead compute the \gls{DHC}, which considers all possible contact geometries, soft robot states, etc., within the given operating conditions. We envision the \gls{DHC} the metric of choice for \emph{safety-aware design} and \emph{design safety certification}.
Beyond \emph{safety-aware design}, we expect that a comprehensive safety metric will be invaluable for other applications, such as the \emph{safety certification} of soft robotic designs and \emph{safety-aware control}, where the controller adjusts its actions to continuously meet required safety levels. After discussing these potential applications in detail, we establish the requirements a safety metric must satisfy for these purposes. We then introduce a new model-based safety metric for soft robots by extending the injury criterion from the existing ISO norms for collaborative robots~\citep{Isots_15066_2016}. Our approach modifies the underlying model to capture key characteristics of soft robots, such as their elastic structures—which deform under internal actuation and external forces—and the possibility of collisions occurring along their entire body. By modeling the contact interactions and collision dynamics between soft robots and humans, and analogously to ISO/TS 15066:2016~\citep{Isots_15066_2016}, we use the maximum contact pressure during a collision as a proxy for injury severity. Instead of relying on computationally expensive simulations, we develop a conservative approximation of the collision dynamics that allows us to compute the maximum contact pressure in closed form. Importantly, our safety metric considers the closed-loop system dynamics, incorporating the control policy when assessing injury severity. It is offered in two variants: the \gls{ISC}, which calculates injury severity for a known soft robot state, control input, and contact geometry (e.g., the specific point of contact and collision direction), making it especially suited for \emph{safety-aware control} applications; and the \gls{DHC}, which evaluates safety across all possible contact geometries and robot states within a given operating envelope. We envision the \gls{DHC} as the metric of choice for both \emph{safety-aware design} and \emph{design safety certification}.