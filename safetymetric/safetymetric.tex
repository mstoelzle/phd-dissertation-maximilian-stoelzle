\chapter{Towards Ensuring Robotic Safety with Embodied and Computational Intelligence}
\label{chp:safetymetric}

\begin{foreword}
    % Traditionally, most effort in the robotics community has been on achieving safety through computational intelligence, i.e., by constraining motions to remain in a safe set of states, avoid obstacles, and compliant controllers such as impedance control. One of the most interesting and promising aspects of soft robots, on the other hand, is that their embodied intelligence contributes passive compliance to the system. However, in existing literature, the effect on the overall safety of the entire soft robotic system has not been thoroughly investigated or quantified. This lack of safety metric for soft robotics prevents us from considering it during the design process leading to either to badly performing and/or unsafe designs. This chapter proposes a safety metric for soft robots that captures both the embodied intelligence (i.e., the passive behavior) and the computational intelligence (i.e., the controller) of the system, allowing us, for the first time, to investigate quantitatively the effect on each on the overall safety and performance.
    Traditionally, efforts in the robotics community have focused on achieving safety through computational intelligence. This includes constraining motions to remain within safe state sets, avoiding obstacles, and employing compliant controllers such as impedance control. In contrast, one of the most intriguing and promising features of soft robots is their embodied intelligence, which provides passive compliance to the system. Despite this promise, existing literature has yet to thoroughly investigate or quantify the impact of embodied intelligence on the overall safety of soft robotic systems. The absence of a dedicated safety metric for soft robotics hampers the design process, often leading to underperforming or unsafe designs. This chapter introduces a safety metric for soft robots that captures both the embodied intelligence (i.e., the passive behavior) and computational intelligence (e.g., controller) of the system. For the first time, this allows for a quantitative evaluation of their combined effects on system safety and performance.
\end{foreword}

\blfootnote{This chapter is partly based on \faFileTextO~\emph{\textbf{M. Stölzle}*, N. Pagliarani*, F. Stella*, J. Hughes, C. Laschi, D. Rus, M. Cianchetti, C. Della Santina, and G. Zardini (2025). Safe yet Effective Soft Robots via Co-Design. In Nature Machine Intelligence, \textbf{\emph{In Preparation}}}.
}

\pagebreak

\begin{abstract}
    % Original version
    % As we work towards integrating robots into human-centric environments, we need to ensure safety while not overly restricting the robot design and/or behavior design. The quest for making robots safer has existed for a long time, with efforts mostly focused on making their control safer and more compliant. Soft robots offer a fundamentally different take on this topic by achieving safety through passive compliance of the entire body. While there has been tremendous research progress in the domain of soft robotics in recent years, ranging from novel actuators to effective model-based control approaches, one of the fundamental promises, safety, has rarely been studied in a principled fashion. In particular, we notice the lack of a quantitative metric to assess how \emph{safe} a soft robot really is. Adding on, we notice that most works take too much of a simplified approach to this topic, equaling safety to material softness. Instead, in reality, the safety of the closed-loop system is influenced by many aspects such as topology, controller, etc, leading to suboptimal soft robot designs that are, for example, too soft to carry practical payloads. Instead, a quantitative safety metric would allow designers to analyze and exploit the safety-vs-performance tradeoff and identify more performant designs while still meeting the necessary safety requirements. In this chapter, we devise requirements a safety metric for soft robots would need to meet. Subsequently, we propose a safety metric that measures the injury severity based on the maximum contact pressure experienced during a collision, as it is already the standard for collaborative robots. Finally, we analyze the safety of continuum soft robots modeled according to the \gls{PCS} parametrization and devise a few simple recommendations for safe soft robot design.
    % short version by ChatGPT
    % Integrating robots into human-centric environments requires ensuring safety without overly restricting the robot's behavior. While soft robotics inherently enhances safety by adding passive body compliance, its core promise—safety—has been largely understudied in a principled manner. Most research equates safety to material softness, overlooking the complex interplay of factors like topology and control systems, which often leads to suboptimal designs that may be too soft to carry practical payloads. A quantitative safety metric could address this issue, enabling designers to balance the trade-off between safety and performance, thus achieving more effective designs while meeting safety requirements. In this chapter, we identify the requirements for a meaningful safety metric for soft robots and propose one based on maximum contact pressure during collisions, consistent with standards for collaborative robots. Using this metric, we analyze the safety of continuum soft robots modeled with the \gls{PCS} parametrization. Finally, we provide recommendations for designing safer soft robots that optimize both performance and safety.
    % longer version by ChatGPT
    As we work toward integrating robots into human-centric environments, it is crucial to ensure safety without overly constraining robot design or behavior. Efforts to enhance robot safety have a long history, primarily focusing on improving control systems to make them safer and more compliant. Soft robots offer a fundamentally different approach to this challenge by achieving safety through the passive compliance of their entire structure. Despite significant advancements in soft robotics over recent years—spanning novel actuators to sophisticated model-based control methods—one of the field’s core promises, safety, has rarely been systematically studied. Notably, there is a lack of a quantitative metric to evaluate how safe a soft robot truly is.
    Moreover, most existing studies oversimplify the concept of safety, equating it to material softness. In reality, the safety of a closed-loop system depends on multiple factors, such as topology and control strategies. This narrow focus often results in suboptimal soft robot designs, such as robots that are too soft to handle practical payloads effectively. A quantitative safety metric would enable designers to evaluate and balance the tradeoff between safety and performance, leading to more optimal designs that meet safety requirements without sacrificing functionality.
    In this chapter, we outline the essential criteria for a safety metric tailored to soft robots. We then propose a metric that assesses injury severity based on the maximum contact pressure during a collision, aligning with established standards for collaborative robots. Finally, we evaluate the safety of continuum soft robots modeled using the \gls{PCS} parametrization and provide practical recommendations for designing safe soft robots.
\end{abstract}

%% Start the actual chapter on a new page.
\newpage

\section{Introduction}
Robots have long played a pivotal role in industrial applications, such as assembly and manufacturing, where precision and force are crucial~\cite{todd1996fundamentals}.
However, to address pressing societal challenges, there is a growing demand for robotic systems that are specifically designed for and adaptable to human-centered environments (e.g., homes and public spaces)~\cite{nahavandi2019industry, chibani2013ubiquitous, royakkers2015literature}.
To unlock their full potential, robots must be designed with inherent physical compliance to operate safely around humans in dynamic, unconstrained, unpredictable settings.
Such requirement aligns with Asimov's First Law--robots must never harm humans--making safety fundamental to their design and deployment~\cite{villani2018survey}.
%
In the context of autonomous vehicles (AVs) and mobile robots, safety is typically ensured through safety-aware autonomy stacks. 
Specifically, existing technologies leverage advanced control mechanisms such as safety filters, real-time collision detection, and predefined safety zones~\cite{zhao2024potential}. 
These systems rely on sophisticated sensors, algorithms, and extensive pre-programming to anticipate and avoid hazards \cite{fragapane2021planning}. 
\glspl{Cobot}~\cite{el2019cobot}, designed to safely interact with humans, achieve this through mechanical and control strategies such as decoupling actuators from links, reducing inertia, and using compliant controllers to detect and respond to external forces. 
Yet, their considerable link inertia and rigid structures still pose significant risks to nearby humans~\cite{haddadin2013towards}.
Collision detection further enhances safety by stopping or slowing the robot upon contact. While reducing the worst-case injury likelihood, this approach is inherently reactive and cannot fully prevent injuries.
Design standards such as ISO/TS 15066:2016~\cite{Isots_15066_2016} demand that \gls{Cobot} controllers be designed so that when in proximity to humans, they move slowly enough to stop in time to prevent collisions~\cite{ajoudani2018progress, lucci2020combining}. 
However, this strategy poses high requirements on perception systems and significantly reduces \gls{Cobot}'s efficiency and effectiveness in the context of human-centered environments.

Interestingly, soft robotics redefines safety from the ground up. 
In these robots~\cite{rus2015design, laschi2016soft}, safety is not an add-on or managed through autonomy stacks but is embedded in the material and structural properties of the robot itself. 
Their compliant nature allows soft robots to interact safely with humans and operate in sensitive environments where safety is essential, such as personal assistance, caregiving, and handling delicate objects and produce~\cite{abidi2017intrinsic}. 
Despite their potential to revolutionize human-robot interaction, soft robots face significant challenges hindering their full integration into practical applications. These obstacles include limited durability, complex manufacturing processes, inadequate stiffness for carrying substantial payloads, and imprecise, often oscillatory motion~\cite{mazzolai2022roadmap, majidi2014soft, hawkes2017soft}. 
Indeed, while much focus has traditionally centered on the trade-off between performance and material softness, this approach is somewhat misguided. 
Many designers have treated this challenge as a balance between precision and softness, assuming that increasing material compliance inherently improves safety.
However, we argue that this is an oversimplification, that the safety of soft robots is a function of both body (e.g., geometry, materials, etc.) and mind (e.g., control policy) and that it should be explicitly quantified, contrary to what has been done so far in the literature.
% Indeed, we find that material softness is not equivalent to safety, as closed-loop safety depends on various design parameters, not just the material's compliance.

Rigid robots, often employed in industrial settings, prioritize performance and precision but compromise safety, making them unsuitable for human-centered environments.
For instance, many soft robots are designed to be highly compliant, prioritizing safety but sacrificing performance \cite{cianchetti2013stiff}. 
In some cases, they tend to be too soft and flexible, lacking the structural integrity needed to perform tasks efficiently. 
While such compliance improves safety, it often makes these robots ineffective in applications that could tolerate a slightly stiffer design without compromising safety. 
As a result, overly soft robots may struggle to exert sufficient force or handle loads, limiting their utility \cite{iida2011soft}.
The key challenge is balancing safety with the ability to perform tasks efficiently and precisely~\cite{khanna2022human}.
In conclusion, we find that the lack of a quantitative safety metric for soft robots is currently a major road-block that prevents soft robotic researchers and designers from formalizing the trade-off between safety and performance, and we anticipate that removing this road-block would lead to more optimal and performant soft robot designs.

Apart from \emph{safety-aware design}, we expect that a safety metric would also be very beneficial for other purposes such as \emph{safety certification} of a soft robotic design or \emph{safety-aware control} where the controller chooses its actions such as that the required safety levels are always met.
After detailing such potential applications of a safety metric for soft robots in more detail, we devise requirements that a safety metric would need to meet in order to be suitable for the mentioned applications.
Subsequently, we then propose a new model-based safety metric for soft robots by building an injury criterion on top of the existing ISO norms for collaborative robots~\cite{Isots_15066_2016} while modifying the underlying model to account for the essential characteristics of soft robots, such as the elastic structure that deforms under the influence of internal actuation and external forces, and the potential of collisions along its entire body.
After modeling the contact model and the collision dynamics between soft robots and humans, we consider analog to ISO/TS 15066:2016~\cite{Isots_15066_2016}, the maximum contact pressure experienced during the collision as a proxy for the injury severity. Instead of requiring computationally expensive simulations, we devise a conservative approximation of the collision dynamics, which allows us to determine the maximum contact pressure in closed form.
Importantly, the proposed safety metric considers the safety of the closed-loop system, which allows us to take the control policy into account when assessing the severity of the injury.
The safety metric comes in two flavors: \gls{ISC} computes the injury severity for a known soft robot state, control input, and contact geometry (e.g., point along the soft robot body of contact and collision direction) and is particularly well suited for \emph{safety-aware control} applications. 
When we want to judge the safety of a soft robot design, we can instead compute the \gls{DHC}, which considers all possible contact geometries, soft robot states, etc., within the given operating conditions. We envision the \gls{DHC} the metric of choice for \emph{safety-aware design} and \emph{design safety certification}.

\section{Towards Quantifying the Safety of Soft Robots}
\subsection{Potential Applications for a Safety Metric}
\subsection{Requirements for a Soft Robotic Safety Metric}
\section{A Metric for Quantifying the Safety of Soft Robots}
\section{Recommendations for the Design of Safe Soft Robots}
