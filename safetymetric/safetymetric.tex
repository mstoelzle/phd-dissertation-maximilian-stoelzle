\chapter{Towards Quantifying the Safety of Soft Robots}
\label{chp:safetymetric}

\begin{foreword}
    % Traditionally, most effort in the robotics community has been on achieving safety through computational intelligence, i.e., by constraining motions to remain in a safe set of states, avoid obstacles, and compliant controllers such as impedance control. One of the most interesting and promising aspects of soft robots, on the other hand, is that their embodied intelligence contributes passive compliance to the system. However, in existing literature, the effect on the overall safety of the entire soft robotic system has not been thoroughly investigated or quantified. This lack of safety metric for soft robotics prevents us from considering it during the design process leading to either to badly performing and/or unsafe designs. This chapter proposes a safety metric for soft robots that captures both the embodied intelligence (i.e., the passive behavior) and the computational intelligence (i.e., the controller) of the system, allowing us, for the first time, to investigate quantitatively the effect on each on the overall safety and performance.
    Traditionally, efforts in the robotics community have focused on achieving safety through computational intelligence. This includes constraining motions to remain within safe state sets, avoiding obstacles, and employing compliant controllers such as impedance control. In contrast, one of the most intriguing and promising features of soft robots is their embodied intelligence, which provides passive compliance to the system. Despite this promise, existing literature has yet to thoroughly investigate or quantify the impact of embodied intelligence on the overall safety of soft robotic systems. The absence of a dedicated safety metric for soft robotics hampers the design process, often leading to underperforming or unsafe designs. This chapter introduces a safety metric for soft robots that captures both the embodied intelligence (i.e., the passive behavior) and computational intelligence (e.g., controller) of the system. For the first time, this allows for a quantitative evaluation of their combined effects on system safety and performance.
\end{foreword}

\blfootnote{This chapter is partly based on \faFileTextO~\emph{\textbf{M. Stölzle}*, N. Pagliarani*, F. Stella*, J. Hughes, C. Laschi, D. Rus, M. Cianchetti, C. Della Santina, and G. Zardini (2025). Safe yet Effective Soft Robots via Holistic Co-Design. In Nature Machine Intelligence, \textbf{\emph{In Preparation}}}.

\nth{1}-author contributions: M. Stölzle conceived and implemented the safety metric and all other content presented in this chapter, including the writing, except for Figures \ref{fig:safetymetric:safety_metric_applications} \& \ref{fig:safetymetric:injury_severity_criterion}. 
}

\pagebreak

\begin{abstract}
    % Original version
    % As we work towards integrating robots into human-centric environments, we need to ensure safety while not overly restricting the robot design and/or behavior design. The quest for making robots safer has existed for a long time, with efforts mostly focused on making their control safer and more compliant. Soft robots offer a fundamentally different take on this topic by achieving safety through passive compliance of the entire body. While there has been tremendous research progress in the domain of soft robotics in recent years, ranging from novel actuators to effective model-based control approaches, one of the fundamental promises, safety, has rarely been studied in a principled fashion. In particular, we notice the lack of a quantitative metric to assess how \emph{safe} a soft robot really is. Adding on, we notice that most works take too much of a simplified approach to this topic, equaling safety to material softness. Instead, in reality, the safety of the closed-loop system is influenced by many aspects such as topology, controller, etc, leading to suboptimal soft robot designs that are, for example, too soft to carry practical payloads. Instead, a quantitative safety metric would allow designers to analyze and exploit the safety-vs-performance tradeoff and identify more performant designs while still meeting the necessary safety requirements. In this chapter, we devise requirements a safety metric for soft robots would need to meet. Subsequently, we propose a safety metric that measures the injury severity based on the maximum contact pressure experienced during a collision, as it is already the standard for collaborative robots. Finally, we analyze the safety of continuum soft robots modeled according to the \gls{PCS} parametrization and devise a few simple recommendations for safe soft robot design.
    % short version by ChatGPT
    % Integrating robots into human-centric environments requires ensuring safety without overly restricting the robot's behavior. While soft robotics inherently enhances safety by adding passive body compliance, its core promise—safety—has been largely understudied in a principled manner. Most research equates safety to material softness, overlooking the complex interplay of factors like topology and control systems, which often leads to suboptimal designs that may be too soft to carry practical payloads. A quantitative safety metric could address this issue, enabling designers to balance the trade-off between safety and performance, thus achieving more effective designs while meeting safety requirements. In this chapter, we identify the requirements for a meaningful safety metric for soft robots and propose one based on maximum contact pressure during collisions, consistent with standards for collaborative robots. Using this metric, we analyze the safety of continuum soft robots modeled with the \gls{PCS} parametrization. Finally, we provide recommendations for designing safer soft robots that optimize both performance and safety.
    % longer version by ChatGPT
    As we work toward integrating robots into human-centric environments, it is crucial to ensure safety without overly constraining robot design or behavior. Efforts to enhance robot safety have a long history, primarily focusing on improving control systems to make them safer and more compliant. Soft robots offer a fundamentally different approach to this challenge by achieving safety through the passive compliance of their entire structure. Despite significant advancements in soft robotics over recent years—spanning novel actuators to sophisticated model-based control methods—one of the field’s core promises, safety, has rarely been systematically studied. Notably, there is a lack of a quantitative metric to evaluate how safe a soft robot truly is.
    Moreover, most existing studies oversimplify the concept of safety, equating it to material softness. In reality, the safety of a closed-loop system depends on multiple factors, such as topology and control strategies. This narrow focus often results in suboptimal soft robot designs, such as robots that are too soft to handle practical payloads effectively. A quantitative safety metric would enable designers to evaluate and balance the tradeoff between safety and performance, leading to more optimal designs that meet safety requirements without sacrificing functionality.
    In this chapter, we outline the essential criteria for a safety metric tailored to soft robots. We then propose a metric that assesses injury severity based on the maximum contact pressure during a collision, aligning with established standards for collaborative robots. Finally, we characterize the proposed safety metric on continuum soft robots modeled using the \gls{PCS} parametrization. % and provide practical recommendations for designing safe soft robots.
\end{abstract}

%% Start the actual chapter on a new page.
\newpage

\section{Introduction}
% As already mentioned in Chapter~\ref{chp:introduction}, the soft robotics community frequently emphasizes the "intrinsic safety" and natural compliance of soft robots~\citep{abidi2017intrinsic} - particularly as a differentiating factor from traditional rigid robots and even \glspl{Cobot}~\citep{laschi2014soft, rus2015design, yasa2023overview}.
% However, against scientific best practices, the safety improvement of continuum soft robots compared to rigid manipulators have, to the best of our knowledge, never been quantified. 
% We think that it is crucial for the community to thoroughly validate intrinsic compliance as a very prominent motivation for soft robots in order to justify continued spending into research and development of soft robots~\citep{hawkes2021hard}.
\dropcap{A}s noted in Chapter~\ref{chp:introduction}, the soft robotics community often highlights the “intrinsic safety” and natural compliance of soft robots~\citep{abidi2017intrinsic} as a key advantage over conventional rigid robots and even \glspl{Cobot}~\citep{laschi2014soft, rus2015design, yasa2023overview}. Yet, in contrast to established scientific practices, the safety improvements of continuum soft robots compared to rigid manipulators have not been rigorously quantified. We believe it is essential for the community to validate intrinsic compliance as a fundamental benefit of soft robots, thereby justifying continued investments in their research and development~\citep{hawkes2021hard}.

% Furthermore, as designers currently have no clear guideline for the “right” level of safety, we find that currently they prioritizes an overly soft material for safety.
% Indeed, designers have treated this challenge as a balance between precision and softness, assuming that increasing material compliance inherently improves safety.
% However, \emph{too soft} material choices can compromise the effectiveness of the soft robot, leading to imprecise, oscillatory motions, limited payload capacity, and the inability to exert sufficient forces on the environment~\citep{iida2011soft, cianchetti2013stiff, mazzolai2022roadmap, majidi2014soft, hawkes2017soft}.
% Indeed, we argue that precision vs. softness is an oversimplification, because the safety of soft robots is a function of both body (e.g., morphology) and brain (e.g., control and perception systems).
% Formalizing and quantifying such safety vs. performance tradeoff would enable designers to identify a good balance between the two and maximizing the performance while guaranteeing the necessary safety.
% Indeed, we we anticipate that removing this road-block would lead to more optimal and performant soft robot designs while preserving safety, which is in particular crucial in human-centered environment.
Moreover, in the absence of clear guidelines for determining the optimal level of safety, designers currently tend to favor overly compliant materials. Many treat the challenge as a trade-off between performance/precision and softness, assuming that higher material compliance automatically results in enhanced safety. 
% However, we find that this is an oversimplification and in the reality, the lightweightness and many \glspl{DOF} that soft robot exhibit significantly contribute to their safety.
However, we find that this oversimplifies the matter; in reality, the lightweight nature and numerous/infinite \glspl{DOF} exhibited by soft robots significantly enhance their safety, and the control approach also has a considerable influence.
Furthermore, choosing materials that are too soft can undermine the robot’s effectiveness, leading to imprecise or oscillatory motions, reduced payload capacity, and an inability to apply sufficient force to the environment~\citep{iida2011soft, cianchetti2013stiff, mazzolai2022roadmap, majidi2014soft, hawkes2017soft}. We argue that framing the issue as merely a balance between performance and softness oversimplifies the problem, since the safety of soft robots depends on both their body (e.g., morphology) and their brain (e.g., control and perception systems). Formalizing and quantifying this safety–performance trade-off would enable designers to find an optimal balance that maximizes performance while ensuring the necessary safety, a consideration that is particularly critical in human-centered environments.

% Apart from \emph{safety-aware design}, we expect that a safety metric would also be very beneficial for other purposes such as \emph{safety certification} of a soft robotic design or \emph{safety-aware control} where the controller chooses its actions such as that the required safety levels are always met.
% After detailing such potential applications of a safety metric for soft robots in more detail, we devise requirements that a safety metric would need to meet in order to be suitable for the mentioned applications.
% Subsequently, we then propose a new model-based safety metric for soft robots by building an injury criterion on top of the existing ISO norms for collaborative robots~\citep{Isots_15066_2016} while modifying the underlying model to account for the essential characteristics of soft robots, such as the elastic structure that deforms under the influence of internal actuation and external forces, and the potential of collisions along its entire body.
% After modeling the contact model and the collision dynamics between soft robots and humans, we consider analog to ISO/TS 15066:2016~\citep{iso2016collaborative}, the maximum contact pressure experienced during the collision as a proxy for the injury severity. Instead of requiring computationally expensive simulations, we devise a conservative approximation of the collision dynamics, which allows us to determine the maximum contact pressure in closed form.
% Importantly, the proposed safety metric considers the safety of the closed-loop system, which allows us to take the control policy into account when assessing the severity of the injury.
% The safety metric comes in two flavors: \gls{SRISC} computes the injury severity for a known soft robot state, control input, and contact geometry (e.g., point along the soft robot body of contact and collision direction) and is particularly well suited for \emph{safety-aware control} applications. 
% When we want to judge the safety of a soft robot design, we can instead compute the \gls{SRDHC}, which considers all possible contact geometries, soft robot states, etc., within the given operating conditions. We envision the \gls{SRDHC} the metric of choice for \emph{safety-aware design} and \emph{design safety certification}.
Beyond \emph{safety-aware design}, we expect that a comprehensive safety metric will be invaluable for other applications, such as the \emph{safety certification} of soft robotic designs and \emph{safety-aware control}, where the controller adjusts its actions to continuously meet required safety levels. After discussing these potential applications in detail, we establish the requirements a safety metric must satisfy for these purposes. We then introduce a new model-based safety metric for soft robots by extending the injury criterion from the existing ISO norms for collaborative robots~\citep{iso2016collaborative}. Our approach modifies the underlying model to capture key characteristics of soft robots, such as their elastic structures—which deform under internal actuation and external forces—and the possibility of collisions occurring along their entire body. By modeling the contact interactions and collision dynamics between soft robots and humans, and analogously to ISO/TS 15066:2016~\citep{iso2016collaborative}, we use the maximum contact pressure during a collision as a proxy for injury severity. Instead of relying on computationally expensive simulations, we develop a conservative approximation of the collision dynamics that allows us to compute the maximum contact pressure in closed form. Importantly, our safety metric considers the closed-loop system dynamics, incorporating the control policy when assessing injury severity. It is proposed in two variants: the \glsxtrfull{SRISC}, which calculates injury severity for a known soft robot state, control input, and contact geometry (e.g., the specific point of contact and collision direction), making it especially suited for \emph{safety-aware control} applications; and the \glsxtrfull{SRDHC}, which evaluates safety across all possible contact geometries and robot states within a given operating envelope. We envision the \gls{SRDHC} as the metric of choice for both \emph{safety-aware design} and \emph{design safety certification}.
\section{Towards Quantifying the Safety of Soft Robots}
In the following, we will motivate the need for a quantitative safety metric by showcasing future applications that such a metric would enable.
This will then allow us to define a list of requirements for characteristics that a safety metric needs to exhibit.
Subsequently, we contextualize the topic by reviewing the literature on how safety has been assessed and quantified in the realm of robotics before, where almost all prior work is on the safety of industrial and collaborative rigid robotic manipulators.

\begin{figure}
    \centering
    \includegraphics[width=0.5\linewidth]{safetymetric/figures/safety_metric_applications.pdf}
    \caption{Future applications that a quantitative safety metric for soft robots would enable. \textbf{Top:} Here, we showcase how the safety metric could be used for \emph{safety-aware design optimization} and, particularly, for analyzing and exploiting the performance vs. safety tradeoff. Subsequently, the safety metric could be used for certifying a given design as \emph{safe} for the respective application. \textbf{Bottom:} The safety metric could also be used for \emph{safety-aware control} - either by filtering the outputs of the controller without safety guarantees or by explicitly including the safety requirements as constraints during the optimization of the control input sequence (e.g., MPC, Control Barrier Functions).}
    \label{fig:safetymetric:safety_metric_applications}
\end{figure}

\subsection{Potential Applications for a Safety Metric}\label{sub:safetymetric:safety_metric_applications}
% We envision a safety metric for soft robots to unlock a variety of applications that we showcase in Fig.~\ref{fig:safety_metric_applications}, and that can be grouped into the domains of \emph{safety-aware control} and \emph{safety-aware design}.
We envision a safety metric for soft robots that will enable a variety of applications, as illustrated in Fig.~\ref{fig:safetymetric:safety_metric_applications}. These applications can be categorized into two main domains: \emph{safety-aware control} and \emph{safety-aware design}.

% In the domain of \emph{safety-aware control}, we assume the soft robotic design to be given, and we strive to control the actions of the soft robot in such a way that the achieved safety level remains within an acceptable range. We remark that this does not exclude contact or even collision with the environment. Instead, we strive to guarantee that these collisions do not cause any (significant) injuries. Examples include \emph{safety filters}~\cite {bertino2023prescribed} that allow us to use performant control policies that do not explicitly consider the safety constraints (e.g., RL) and still guarantee safety by filtering/saturating the control input. Alternative techniques such as MPC~\citep{hewing2020learning} or Control Barrier Functions~\citep{ames2016control}, which we refer to as \emph{safety-constrained control}, allow us to take the safety constraints directly into account when devising the control input.
In the realm of \emph{safety-aware control}, we assume the soft robotic design is already established and focus on controlling its actions to maintain an acceptable safety level. This approach does not preclude contact or collisions with the environment; rather, it ensures that such interactions do not result in significant injuries. Examples include \emph{safety filters}~\citep{bertino2023prescribed}, which allow the use of high-performance control policies (such as \gls{RL}) that do not explicitly account for safety constraints while still guaranteeing safety by filtering or saturating the control input. Alternative methods like \gls{MPC}~\citep{hewing2020learning} or Control Barrier Functions~\citep{ames2016control}—referred to as \emph{safety-constrained control}—integrate safety constraints directly into the control strategy.

% Another avenue that would be unlocked by a safety metric is \emph{safety-aware design}, which would include assessing an integrated soft robot design for its safety. Specifically, we consider here two subcategories: \emph{safety-aware design optimization} would consider safety when developing/optimizing the soft robot design either by means of inequality constraints (i.e., a minimum safety needs to be guaranteed) or by maximizing safety through the cost function.
% After a design is finalized, \emph{safety certification} would allow manufacturers to certify their product as being sufficiently safe for the respective applications (e.g., healthcare, agri-food, manufacturing, etc.). This is well-aligned with established safety standards for collaborative rigid robots as defined in ISO/TS 15066:2016~\citep{Isots_15066_2016}.
Another promising direction unlocked by a safety metric is \emph{safety-aware design}, which involves evaluating an integrated soft robot design for its safety. We see this as comprising two subcategories: \emph{safety-aware design optimization}, which incorporates safety into the design process either through inequality constraints (ensuring a minimum safety level) or by maximizing safety via the cost function; and \emph{safety certification}, where, after a design is finalized, manufacturers can certify that their product meets the necessary safety standards for specific applications (e.g., healthcare, agri-food, manufacturing, etc.). This approach aligns well with established safety standards for collaborative rigid robots as defined in ISO/TS 15066:2016~\citep{Isots_15066_2016}.

\subsection{Requirements for a Soft Robotic Safety Metric}\label{sub:safetymetric:safety_metric_requirements}
% In the following, we will list some requirements that, in our opinion, a safety metric needs to meet in order to be well suited for the applications listed in Sec.~\ref{sub:safetymetric:safety_metric_applications}.
% First (1), the metric shall consider the dynamics inherent to continuum soft robots and their particular characteristics. For example, one of the main differences between rigid and soft manipulators is that free-moving joints with integrated motors are replaced by an elastic structure that deforms under the influence of internal actuation and external forces. Therefore, any safety metric needs to crucially consider the elastic and inertial characteristics of the soft robot that are generated by the distributed material along its backbone.
% Secondly (2), the safety metric shall consider not just collision at the end-effector but instead anywhere along the body of the soft robot. This is a significant difference to the existing safety metrics for rigid/collaborative robots, which, for simplicity, usually only consider collisions at the end-effector as we would expect there the largest motion velocities~\citep{haddadin2011safe, Isots_15066_2016}. Instead, soft robots exhibit large Cartesian stiffnesses close to their proximal end, thus requiring us to consider the safety of collisions anywhere along the backbone.
% Next, (3) any simplifying assumptions shall lead to a conservative estimate of the achieved safety. For example, soft robot models underlying the safety metric would likely use a finite-dimensional approximation of the continuum shape; instead, in reality, flexible structures such as soft robots exhibit infinite degrees of freedom~\citep{della2023model, armanini2023soft}. Therefore, we would like any safety metric making use of finite-dimensional approximations to underestimate instead of overestimate the safety of the design.
% Fourthly (4), the computation of the safety metric shall be computationally tractable, which is essential for safety-aware control and design applications, where evaluations on the scale of sub-seconds and seconds are necessary.
% Finally, we end with a few desirable characteristics: (5) some applications, such as safety-aware control or safety-aware design, might benefit from the differentiability of the safety metric with respect to design parameters, robot states, and control inputs; (6) ideally, the safety metric shall also consider how \emph{safe} the soft robot's design (e.g., friendly-looking) and behavior (e.g., smooth and predictable movements) is perceived by the user. In addition to user studies, the evaluation of this metric might be assisted by VLMs.
In the following, we outline several requirements that, in our opinion, a safety metric must satisfy to be well-suited for the applications described in Sec.~\ref{sub:safetymetric:safety_metric_applications}. First (1), the metric should account for the dynamics inherent to continuum soft robots and their unique characteristics. For example, a primary difference between rigid and soft manipulators is that free-moving joints with integrated motors are replaced by a compliant structure that deforms under internal actuation and external forces. Consequently, any safety metric must consider the elastic and inertial properties generated by the distributed material along the robot’s backbone.
%
Secondly (2), the safety metric must evaluate collisions occurring anywhere along the soft robot’s body—not just at the end-effector. This represents a significant departure from existing safety metrics for rigid or collaborative robots, which often focus solely on the end-effector due to its typically higher motion velocities~\citep{haddadin2009requirements, haddadin2011safe, Isots_15066_2016}. In contrast, soft robots exhibit high Cartesian stiffness near their proximal end, making it necessary to assess safety along the entire structure.
% 
Next (3), any simplifying assumptions should yield a conservative safety estimate. For instance, while soft robot models used in the metric might employ a finite-dimensional approximation of the continuum shape, in reality, these flexible structures possess infinite degrees of freedom~\citep{della2023model, armanini2023soft}. Therefore, a safety metric based on such approximations should tend to underestimate rather than overestimate the design’s safety.
% 
Fourthly (4), the computation of the safety metric must be tractable, which is essential for safety-aware control and design applications that require evaluations on sub-second to second timescales, respectively.
% 
Finally, we highlight a few desirable characteristics: (5) in applications such as safety-aware control or design, it is advantageous if the safety metric is differentiable with respect to design parameters, robot states, and control inputs; and (6) ideally, the metric should also reflect how “safe” the soft robot’s design (e.g., its friendly appearance) and behavior (e.g., smooth and predictable movements) are perceived by users. In addition to user studies, the evaluation of this metric might be enhanced by leveraging \glspl{VLM}~\citep{touvron2023llama, grattafiori2024llama}.

\subsection{Background on Injury Risk Criteria for Robotic Manipulators}

\begin{itemize}
    \item Give a motivation for why the robotics community strives to quantify safety/injury risk (e.g., optimizing designs and controllers to make them more human-friendly, safety-aware control, etc.).
    \item Give an overview of the trailblazing literature on measuring the injury risk of rigid robots.
    \item Mention the ISO norm that establishes the standards for safe, collaborative robots.
    \item List the existing attempts to assess the safety of soft robots and why they are not sufficient (e.g., too simplistic models, not taking into account the control, lacking a contact model, etc.)
    \item List some of the initial attempts to give designers an idea of which factors influence the safety of soft robots (e.g.,~\citep{abidi2017intrinsic}).
\end{itemize}

\textcolor{red}{Stucture: \begin{enumerate}
    \item Motivation for measuring safety: selecting suitable design, defining constraints that guarantee safety (i.e., ISO norms), and safety-aware control and motion planning
    \item Stress that this is a well-established line of research in (rigid) robotics involving both conceptual and experimental analysis
    \item Modes of impact and injury: constrained vs. unconstrained (human), static vs. dynamic, sharp surfaces, different body parts, etc.~\citep{haddadin2009requirements}
    \item Injurity severity criterias: the first were based on automobile crash models (HIC), but found to be unsuitable as they are calibrated for higher velocities and inertias that are not even relevant for rigid robots. In particular, taking the head acceleration as a injury criteria is not suitable, as not in practice reached, even when robots are moving at 2m/s. Furthermore, it mostly focuses on the question of fatale impacts. Instead, other injury modes, even if not fatal, become relevant as (rigid) robots can still cause bones to break and other significant injuries at their speeds. There exist specialized injury severity criteria inspired by biomechanics for the various body parts. For example, body parts have different contact stiffnesses and injury is (most likely) caused when different thresholds are exceeded (e.g., penetration depth, maximum force, energy density, acceleration). However, it seems that most of them are correlated with the the maximum force experienced during impact and that the maximum force generalizes the best across body parts.
    \item Mention ISO/TS 15066:2016 (Collaborative robots) and ISO/PAS 5672:2023
    \item Safety metrics for soft robots: basically not existing, only~\citep{abidi2017intrinsic}, but only super-simplified beam model. A safety metric is especially important for soft robots as we continuously stress the inherent safety of soft robots, so we also need to be able to quantify it.
\end{enumerate}}

Various aspects have motivated the robotic community to try to assess the safety of our robots~\citep{de2008atlas, van2018spatial}: 
First of all, understanding the important factors influencing safety allows us to make robotic designs and control algorithms safer~\citep{bicchi2004fast, zinn2004new}. 
Secondly, quantifying the injury risk stemming from a robot allows the establishment of minimal safety standards and specifically constraints on the design and actuation that guarantee safe deployment of the robots, as done, for example, in ISO 10218-1:2011~\citep{iso2011robots} for industrial robots and ISO/TS 15066:2016~\citep{Isots_15066_2016} for collaborative robots. This allows the designers and manufacturers of robots to certify that their design is \emph{safe}, which is, in turn, crucial for successful adoption by industrial customers and consumers.
Thirdly, modeling the injury risk of robots and explicitly setting operation constraints that guarantee safety enables safety-aware control~\citep{lacevic2011safety, zanchettin2015safety, mansfeld2018safety}, motion planning~\citep{lacevic2022safe, pupa2024efficient} and also the deployment of safety filters~\citep{hewing2020learning, bertino2023prescribed}.

% Before improving the safety of (collaborative) robots, we first need to understand the important factors influencing the injury risk and be able to compare different robot design w.r.t. to their achieved safety level~\citep{de2008atlas}.
For the reasons mentioned above, quantifying the safety and associated injury risk has been a longstanding research topic, with most effort centered on determining the safety of collaborative rigid robotic manipulators~\citep{zinn2004playing, bicchi2004fast, haddadin2009requirements, mansfeld2018safety}.
\section{A Metric for Quantifying the Safety of Soft Robots}
Thereafter, we propose a quantitative safety metric for \emph{blunt} contacts~\citep{haddadin2011safe} between soft robots and humans that captures the particular characteristics that continuum soft robots exhibit (e.g., elasticity, actuation through their structure, Cosserat rod dynamics, etc.).
Importantly, the presented safety metric fulfills all the requirements that we laid out previously. % (e.g., based on a soft robotic dynamical model, computationally tractable, differentiable, etc.).
First, we state the necessary background on soft robotic dynamics and contact models. Subsequently, we derive the dynamics of a collision between the continuum soft robot and the human (body part).
Next, we propose two flavors of the safety metric: (i) the \glsxtrfull{SRISC} captures the injury risk for a \emph{given} contact geometry, soft robot state, and actuation sequence. We envision this criterion to be useful for control applications with safety guarantees.
The second flavor, (ii) the \glsxtrfull{SRDHC}, captures the inherent safety of an integrated soft robot design (e.g., also considering the control policy) and leverages the \gls{SRISC} for estimating the maximum injury risk over all possible contact geometries, feasible robot states and actuation sequences.
Apart from the procedure for formulating this safety metric, one of the key innovations here is that we identify a closed-form solution to the collision dynamics, which renders the computation of the \gls{SRISC} to be computationally tractable.

\subsection{Background on Soft Robot Dynamics and Contact Model}
Following the Cosserat rod theory, we can capture the kinematic behavior of slender structures such as continuum soft robots by considering the deformations of the robot's backbone. As the 1D spatial deformations of this backbone are still an infinite-dimensional problem, the field has developed many methods (e.g., \gls{PCC}~\citep{webster2010design}, \gls{PCS}~\citep{renda2018discrete}, \gls{GVS}~\citep{renda2020geometric}, etc.) to describe such deformations with a finite number of finite vector of configuration variables $q \in \mathbb{R}^n$. The associated forward kinematic model then allows us to define the geometric positional Jacobian $J_\mathrm{p}(q, s) \in \mathbb{R}^{3 \times n}$, where $s \in [0,L]$ is the backbone abscissa/coordinate and $L$ is the length of the entire continuum structure.
Independent of the specific chosen kinematic model, the \gls{EOM} of a continuum soft robot can often be stated as~\citep{armanini2023soft, della2023model}
\begin{equation}\label{eq:safetymetric:soft_robot_configuration_space_dynamics}
    B(q) \, \ddot{q} + C(q, \dot{q}) \, \dot{q} + \partial_{q} \, \mathcal{U}(q) + D \, \dot{q} = A(q) \, u + \tau_\mathrm{c},
\end{equation}
$B(q) \in \mathbb{R}^{n \times n}$ and $C(q, \dot{q}) \in \mathbb{R}^{n \times n}$ considers the inertial and Coriolis effects of the soft robot system, respectively.
$\partial_{q} \, \mathcal{U}(q) \in \mathbb{R}^n$ captures the forces stemming from the potential $\mathcal{U}(q): \mathbb{R}^n \to \mathbb{R}$.
Often times, we the potential forces simplify to $\partial_{q} \, \mathcal{U}(q) =  G(q) + K q$, where $G(q) \in \mathbb{R}^{n}$ describes the gravitational forces, and $K \succ 0 \in \mathbb{R}^{n \times n}$ is the stiffness matrix.
Dissipation is integrated through the damping matrix $D \succ 0 \in \mathbb{R}^{n \times n}$.
$u(t,q,\dot{q}) \in \mathbb{R}^{m}$ contributes the actuation (determined by a control policy) that acts through the linear map $A(q) \in \mathbb{R}^{n \times m}$ on the generalized coordinates.

The term $\tau_\mathrm{c} \in \mathbb{R}^n$ collects all contributions by external contact forces on the generalized coordinates.
In the following, we will assume that the soft robot is only in contact with the human at one discrete point and that only pure forces are reflected between the bodies during the contact (i.e., no Cartesian torques).
Specifically, we assume that the contact occurs at a specific point $s_\mathrm{c}$ and that the contact exhibits a constant surface normal of $n_\mathrm{c} \in \mathcal{S}^3$ which is a unit vector and, with that, $\mathcal{S}^3 = \{ n_\mathrm{c} \in \mathbb{R}^3: \lVert n_\mathrm{c} \rVert_2 = 1 \}$.
We now describe with $\delta_\mathrm{c} > 0$ a penetration between the soft robot and the soft tissue of the human.
Then, the generalized torque acting on the soft robot as a consequence of the contact is given by $\tau_\mathrm{c} = -J^\top(q,s_\mathrm{c}) \, n_\mathrm{c} \, f_\mathrm{c}(\delta_\mathrm{c}, \dot{\delta}_\mathrm{c}) = -J_\mathrm{c}^\top(q) \, f_\mathrm{c}(\delta_\mathrm{c}, \dot{\delta}_\mathrm{c})$.
While the formulation that we use in this perspective for formulating the safety metric is compatible with many of the contact models that have been studied in the literature, such as Hunt-Crossley~\citep{hunt1975coefficient, aouaj2021predicting}, Hertz~\citep{johnson1987contact, park2011designing, she2020comparative}, etc., we will mainly focus in the following on a linear spring-damper contact model~\citep{iso2016collaborative, haddadin2009requirements} given by
\begin{equation}
    f_\mathrm{c}(\delta_\mathrm{c}, \dot{\delta}_\mathrm{c}) = \begin{cases}
        0 & \delta_\mathrm{c} \leq 0,\\
        k_\mathrm{c} \, \delta_\mathrm{c} + d_\mathrm{c} \, \dot{\delta}_\mathrm{c} & \delta_\mathrm{c} > 0,\\
\end{cases}
\end{equation}
where $k_\mathrm{c} \in \mathbb{R}_{>0}$ is the contact stiffness and $d_\mathrm{c} \in \mathbb{R}_{\geq 0}$ is the contact damping coefficient.
If we assume the soft robot surface material and the human soft tissue to have spring constants and damping coefficients of $k_\mathrm{R,surf}$, $k_\mathrm{H,st}$ and $d_\mathrm{R}$, $d_\mathrm{H}$, respectively, then we can connect the spring-dampers in series
\begin{equation}
    k_\mathrm{c} = \left (\frac{1}{k_\mathrm{R,surf}} + \frac{1}{k_\mathrm{H,st}} \right )^{-1},
    \qquad
    d_\mathrm{c} = \left (\frac{1}{d_\mathrm{R}} + \frac{1}{d_\mathrm{H}} \right )^{-1}.
\end{equation}
Please note that effective spring constant of many human body parts are reported in ISO/TS 15066:2016~\citep{iso2016collaborative}.

\subsection{Collision Dynamics}
We now progress towards a formulation of the collision dynamics as motions of the soft robot and the human body part along the contact surface normal $n_\mathrm{c}$.

First, we describe the motion of the contact point of the soft robot with position and velocity $x_\mathrm{R}, \dot{x}_\mathrm{R} \in \mathbb{R}$.
We can project the dynamics of \eqref{eq:safetymetric:soft_robot_configuration_space_dynamics} into this 1D motion through the expression $\dot{x}_\mathrm{R} = J_\mathrm{c} \, \dot{q}$ yielding the form~\citep{khatib1987unified, della2019exact, della2020model, stolzle2024guiding}
\begin{equation}
    \Lambda_\mathrm{c}(q) \, \Ddot{x}_\mathrm{R} + \mu_\mathrm{c}(q,\dot{q}) \, \dot{x}_\mathrm{R} + J_\mathrm{c,B}^{+\top}(q) ( \partial_{q} \, \mathcal{U}(q) + D \dot{q} ) = J_\mathrm{c,B}^{+\top}(q) \, A(q) \, u - f_{\mathrm{c}}(\delta_\mathrm{c}, \dot{\delta}_\mathrm{c}),
\end{equation}
where $J_\mathrm{c,B}^{+\top}(q) = J_\mathrm{B}^+(q,s_\mathrm{c},n_\mathrm{c}) = B^{-1}J_\mathrm{c}^\top(J_\mathrm{c} B^{-1} J_\mathrm{c}^\top)^{-1} \in \mathbb{R}^{n \times 1}$ is the dynamically consistent pseudo-inverse, $\Lambda_\mathrm{c}(q) = \Lambda(q,s_\mathrm{c},n_\mathrm{c}) = (J_\mathrm{c} \, B^{-1} J_\mathrm{c}^\top)^{-1} \in \mathbb{R}^{1 \times 1}$ is the reflected inertia of the soft robot at the contact point~\citep{haddadin2009requirements, iso2016collaborative}, and $\mu_\mathrm{c}(q,\dot{q}) = \mu(q, \dot{q},s_\mathrm{c},n_\mathrm{c}) = \Lambda(q) \, (J_\mathrm{c} B^{-1} C - \dot{J}_\mathrm{c}) \in \mathbb{R}^{1 \times n}$ collects the Cartesian Coriolis and centrifugal terms~\citep{khatib1987unified}.
If not explicitly stated otherwise, we will in the following, to simplify the notation, drop the specific dependency on the contact geometry $(s_\mathrm{c}, n_\mathrm{c})$: $J_\mathrm{c,B}^{+\top}(q) = J_\mathrm{B}^{+\top}(q,s_\mathrm{c},n_\mathrm{c})$, $\Lambda_\mathrm{c}(q) = \Lambda(q,s_\mathrm{c},n_\mathrm{c})$, etc.

Next, we move towards modeling the behavior of the human body part. In literature, the human body part is usually modeled as a point mass $m_\mathrm{H}$~\citep{haddadin2011safe, iso2016collaborative} that moves in 1D along the surface normal of the contact with state $(x_\mathrm{H},\dot{x}_\mathrm{H})$\footnote{Please note that the effective mass of various human body parts is reported in ISO/TS 15066:2016~\citep{iso2016collaborative}.}. Instead, we take here a conservative approach and assume that the human body is constrained in its motion with velocity $v_\mathrm{H} \in \mathbb{R}$ towards the soft robots (i.e., $m_\mathrm{H} \gg \Lambda(q) \: \forall q$). This represents the \emph{worst case}.
After the coordinate change $\delta_\mathrm{c}(t) = x_\mathrm{R}(t) - x_\mathrm{H}$, $\dot{\delta}_\mathrm{c} = \dot{x}_\mathrm{R}(t) + v_\mathrm{H}$, % where $x^{\mathrm{c}0} \in \mathbb{R}$ is the position of the initial contact, 
where $x_\mathrm{H} \in \mathbb{R}$ is the position of the soft tissue surface, and while only considering the case of contact (i.e., $\delta_\mathrm{c} \geq 0$), the collision dynamics are given by
\begin{equation}
    \Lambda_\mathrm{c}(q) \, \Ddot{\delta}_\mathrm{c} + \mu_\mathrm{c}(q,\dot{q}) \, \dot{\delta}_\mathrm{c} + J_\mathrm{c,B}^{+\top}(q) ( \partial_{q} \, \mathcal{U}(q) + D \dot{q} ) = J_\mathrm{c,B}^{+\top}(q) \, A(q) \, u - k_\mathrm{c} \, \delta_\mathrm{c} - d_\mathrm{c} \, \dot{\delta}_\mathrm{c}.
\end{equation}
We are now interested in identifying the maximum force $f_\mathrm{c}(t)$ that occurs during the entire collision of the contact.
Therefore, we can neglect any damping forces as they dissipate energy.
Furthermore, we assume that the Coriolis effects are sufficiently small and can be neglected as well.
Finally, we assume that the change of configuration during the collision is sufficiently small such that the dynamic matrices can be approximated as constant: $m_\mathrm{R} \approx \Lambda_\mathrm{c}(q^{\mathrm{c}0})$, 
$A_\mathrm{c} \approx J_\mathrm{c,B}^{+\top}(q^{\mathrm{c}0}) \, A(q^{\mathrm{c}0})$, where the $q^{\mathrm{c}0}$ is the configuration of the robot at the beginning of the contact.
The same assumption also allows us to linearize the potential forces of the soft robot with 
\begin{equation}
    f_{\mathcal{U}} = J_\mathrm{c,B}^{+\top}(q) \, \partial_{q} \, \mathcal{U}(q) \approx \underbrace{J_\mathrm{c,B}^{+\top}(q^{\mathrm{c}0}) \, \partial_{q} \, \mathcal{U}(q^{\mathrm{c}0})}_{f_{\mathcal{U}}^{\mathrm{c}0}} + \underbrace{\frac{\partial}{\partial q} J_\mathrm{c,B}^{+\top}(q) \, \partial_{q} \, \mathcal{U}(q) \Big |_{q=q^{\mathrm{c}0}} \,  J_\mathrm{c,B}^{+}(q^{\mathrm{c}0})}_{k_\mathrm{R}}  \, \delta_\mathrm{c},
\end{equation}
where $k_\mathrm{R} \in \mathbb{R}$ is the local stiffness of the system against small perturbations and $f_{\mathcal{U}}^{\mathrm{c}0}$ are the potential forces present at the start of the contact.
Integrating the stated assumptions results in the approximated collision dynamics
\begin{equation}\label{eq:safetymetric:simplified_collision_dynamics}
    m_\mathrm{R} \, \Ddot{\delta}_\mathrm{c} + (k_\mathrm{R} + k_\mathrm{c}) \, \delta_\mathrm{c} = f_u - f_{\mathcal{U}}^{\mathrm{c}0}.
\end{equation}
To avoid computationally expensive simulations of the collision, we identify a closed-form solution to the collision dynamics
\begin{equation}\small\label{eq:safetymetric:collision_dynamics_cfs}
\begin{split}
    \delta_\mathrm{c}(t) =& \: \left (\delta_\mathrm{c}^0-\frac{f_u-f_{\mathcal{U}}^{\mathrm{c}0}}{k_\mathrm{R} + k_\mathrm{c}} \right ) \cos \left ( \sqrt{\frac{k_\mathrm{R} + k_\mathrm{c}}{m_\mathrm{R}}} \, t \right ) + \dot{\delta}_\mathrm{c}^0 \sqrt{\frac{m_\mathrm{R}}{k_\mathrm{R} + k_\mathrm{c}}} \, \sin \left ( \sqrt{\frac{k_\mathrm{R} + k_\mathrm{c}}{m_\mathrm{R}}} \, t \right ) + \frac{f_u-f_{\mathcal{U}}^{\mathrm{c}0}}{k_\mathrm{R} + k_\mathrm{c}},\\
    % \dot{\delta}_\mathrm{c}(t) =& \: -\sqrt{\frac{k_\mathrm{R} + k_\mathrm{c}}{m_\mathrm{R}}} \left (\delta_\mathrm{c}^0 - \frac{f_u-f_{\mathcal{U}}^{\mathrm{c}0}}{k_\mathrm{R} + k_\mathrm{c}} \right ) \, \sin \left ( \sqrt{\frac{k_\mathrm{R} + k_\mathrm{c}}{m_\mathrm{R}}} \, t \right ) + \dot{\delta}_\mathrm{c}^0 \, \cos \left ( \sqrt{\frac{k_\mathrm{R} + k_\mathrm{c}}{m_\mathrm{R}}} \, t \right ),
\end{split}
\end{equation}
where we assume without loss of generality that $t=0$ at the start of the collision, and $\delta_\mathrm{c}^0$ is the initial penetration depth.
The initial penetration velocity can be computed as a function of the configuration-space velocity as $\dot{\delta}_\mathrm{c}^0 = J_\mathrm{c}(q^{\mathrm{c}0}) \, \dot{q}^{\mathrm{c}0} + v_\mathrm{H}$.
Furthermore, we assume the actuation force to be constant, which can be easily accomplished by conservatively considering the maximum actuation force $f_u = \max_t A_\mathrm{c} \, u(t)$ that the robot experiences during the collision.

\begin{figure}[h!]
    \centering
    \includegraphics[width=0.9\linewidth]{safetymetric/figures/injury_severity_criterion.pdf}
    \caption{Proposed \glsxtrfull{SRISC} as a safety criterion for soft robots: The maximum contact pressure $\max_t p_\mathrm{c}(t) = \max_t \frac{f_\mathrm{c}(t)}{A_\mathrm{c}}$ experienced during the (potential) collision acts as a proxy for the expected injury risk~\citep{iso2016collaborative}, where $f_\mathrm{c}(t)$ denotes the contact force and $A_\mathrm{c}$ the contact area. For computing $f_\mathrm{c}(t)$, we derive the dynamics of the collision (i.e., the time evolution of the penetration depth $\delta_\mathrm{c}(t))$ by projecting the dynamics of the soft robot onto a 1D Cartesian motion along the contact surface normal. In order to get a conservative estimate of the injury risk, we assume the human body part to be constrained in its motion (i.e., that the inertia of the human body part dominates the reflected inertia of the soft robot $m_\mathrm{R}$).}
    \label{fig:safetymetric:injury_severity_criterion}
\end{figure}

\subsection{Injury Severity Criterion}
Following the standards established in ISO/TS 15066:2016~\citep{iso2016collaborative}, we consider the maximum contact pressure, also sometimes referred to as stress~\citep{haddadin2009requirements}, experienced during the collision as a proxy for the injury risk. Therefore, we define the \gls{SRISC} for a given tuple $(q^{\mathrm{c}0},s_\mathrm{c}, n_\mathrm{c})$ capturing the contact geometry as
\begin{equation}
    \mathrm{SRISC}(q^{\mathrm{c}0},\dot{\delta}_\mathrm{c}^0,u,s_\mathrm{c},n_\mathrm{c}) = \max_t p_\mathrm{c} = \max_t \frac{f_\mathrm{c}(t)}{A_\mathrm{c}(t)} \leq \frac{\max_t f_\mathrm{c}(t)}{\min_t A_\mathrm{c}(t)} =  \frac{k_\mathrm{c} \max_t \delta_\mathrm{c}(t)}{\min_t A_\mathrm{c}(t)},
\end{equation}
where $p_\mathrm{c}(t)$ is the contact pressure, and $A_\mathrm{c}$ is the contact area.

The closed-form solution to the collision dynamics of \eqref{eq:safetymetric:collision_dynamics_cfs} allows us to upper-bound the maximum contact force $\max_t f_\mathrm{c}(t)$ that is encountered during the collision as
\begin{equation}
     \max_{t}f_\mathrm{c}(t) = k_\mathrm{c} \, \left ( \frac{f_u-f_{\mathcal{U}}^{\mathrm{c}0}}{k_\mathrm{R} + k_\mathrm{c}} + \sqrt{\left ( \delta_\mathrm{c}^0 - \frac{f_u-f_{\mathcal{U}}^{\mathrm{c}0}}{k_\mathrm{R} + k_\mathrm{c}} \right )^2 + \left (\dot{\delta}_\mathrm{c}^0 \right )^2 \frac{m_\mathrm{R}}{k_\mathrm{R} + k_\mathrm{c}} } \right ).
\end{equation}

% \subsubsection{Example: Mass-Spring Robot}
% First, we consider the most \emph{basic} soft robot - a damped mass-spring system with dynamics $m_\mathrm{R} \, \ddot{q} + k_\mathrm{R} \, (q-q^0) + d_\mathrm{R} \, \dot{q} = u + \tau_\mathrm{c}$, where $q \in \mathbb{R}$ and $q^0$ is the equilibrium extension. The \emph{Injury Severity Criterion} is then given by
% \begin{equation}
%      \mathrm{SRISC} = \frac{k_\mathrm{c}}{A_\mathrm{c}} \, \left (\frac{u - k_\mathrm{R} (q^{\mathrm{c}0} - q^0)}{k_\mathrm{R} + k_\mathrm{c}} + \sqrt{\left ( \delta_\mathrm{c}^0 - \frac{u - k_\mathrm{R} (q^{\mathrm{c}0} - q^0)}{k_\mathrm{R} + k_\mathrm{c}} \right )^2 + \left (\dot{\delta}_\mathrm{c}^0 \right )^2 \frac{m_\mathrm{R}}{k_\mathrm{R} + k_\mathrm{c}} } \right ),
% \end{equation}
% with the limit $\lim_{k_\mathrm{R} \to \infty} \mathrm{SRISC} = 2 \, \frac{k_\mathrm{c}}{A_\mathrm{c}} \left ( q^{\mathrm{c}0} - q^0 \right )$ for $\delta_\mathrm{c}^0 = 0$. 

\subsubsection{Example: Planar Piecewise Constant Strain Robot}
1 Figure giving an example how the contact location influences the injury severity.
\begin{itemize}
    \item Variation of injury severity for various contact geometries (e.g., various locations along the backbone, various contact surface normal) and configurations.
    \item Variation across model discretization granularity.
\end{itemize}

\subsubsection{Example: Integrating a Control Policy}
% alternative heading: \subsection{Influence of Control Policy}
An important question when analyzing the safety of a closed-loop soft robotic system is what influence the control policy has on the \gls{SRISC}.
First, we consider a case where the behavior of the control policy $u = \phi(q,\dot{q})$ cannot be bounded or even inspected, such as it would be the case for controllers that contain integral terms or for many RL-based control policies.
In this case, access to actuation bounds $[u_\mathrm{min}, u_\mathrm{max}]$ provides us with the injury risk in the \emph{worst case scenario}.

Next, we consider the example of a \emph{PD+Feedforward}-like control structure that is relevant for many control policies that involve feedforward and/or feedback terms.
Specifically, we consider a fully-actuated setting (i.e., $n=m$) with an identity actuation matrix $A(q) = \mathbb{I}^n$.
Then, a regulator $\phi(q,\dot{q}) = \partial_{q} \, \mathcal{U}( q^\mathrm{d}) + K_\mathrm{p} \, (q^\mathrm{d}-q) - K_\mathrm{d} \, \dot{q}$ drives the system towards the setpoint $q^\mathrm{d}$~\citep{della2023model} and establishes the closed-loop dynamics
\begin{equation}
    B(q) \, \ddot{q} + C(q, \dot{q}) \, \dot{q} + \partial_{q} \, \mathcal{U}(q) + K_\mathrm{p} \, q + (D+K_\mathrm{d}) \, \dot{q} = \partial_{q} \, \mathcal{U}( q^\mathrm{d}) + K_\mathrm{p} \, q^\mathrm{d} + \tau_\mathrm{c},
\end{equation}
where $K_\mathrm{p}, K_\mathrm{d} \in \mathbb{R}^{n \times n}$ are the proportional and derivative feedback gains, respectively.
When re-formulating the simplified collision dynamics of \eqref{eq:safetymetric:simplified_collision_dynamics}, 
\begin{equation}
    m_\mathrm{R} \, \Ddot{\delta}_\mathrm{c} + (k_\mathrm{R} + k_\mathrm{c}) \, \delta_\mathrm{c} = J_\mathrm{c,B}^{+}(q^{\mathrm{c}0}) \, (\partial_{q} \, \mathcal{U}(q^\mathrm{d}) +  K_\mathrm{p} \, q^\mathrm{d} - \partial_{q} \, \mathcal{U}(q^{\mathrm{c}0})).
\end{equation}
We notice that the feedforward control term acts through a constant force on the oscillatory system.
The proportional feedback term increases the local stiffness of the robot: $k_\mathrm{R} = \frac{\partial}{\partial q} J_\mathrm{c,B}^{+\top}(q) \, \left ( \partial_{q} \, \mathcal{U}(q) + K_\mathrm{p} \, q \right )\Big |_{q=q^{\mathrm{c}0}} \,  J_\mathrm{c,B}^{+}(q^{\mathrm{c}0})$.
This analysis agrees with similar results known in literature~\citep{della2017controlling}.

\subsection{Design Hazardousness Criterion}
As presented in Fig.~\ref{fig:safetymetric:safety_metric_applications}, one application of a soft robotic safety metric would be to answer the question "\emph{How safe is this proposed soft robot design?}". The \gls{SRISC} on its own is not sufficient to answer this question as it relies on a knowledge of the soft robot state at the beginning of the collision $(q^{\mathrm{c}0}, \dot{q}^{\mathrm{c}0})$, and the contact geometry $(s_\mathrm{c},n_\mathrm{c})$.
Therefore, we define the \gls{SRDHC} as the maximum injury severity that can be imposed by the soft robot over all feasible soft robot states, all possible contact geometries, and actuation sequences~\citep{wassink2007towards}
\begin{equation}
\begin{split}
    \mathrm{SRDHC} =& \: \max_{q \in \mathcal{Q}} \max_{\dot{\delta}_\mathrm{c}^0 \in [0,\dot{\delta}_\mathrm{c}^\mathrm{max}]} \max_{u \in [u_\mathrm{min}, u_\mathrm{max}]} \max_{s_\mathrm{c} \in [0,L]} \max_{n_\mathrm{c} \in \mathcal{S}^3} \mathrm{SRISC}(q^{\mathrm{c}0},\dot{\delta}_\mathrm{c}^0,u,s_\mathrm{c},n_\mathrm{c}),\\
    \leq& \: \max_{q \in \mathcal{Q}} \max_{s_\mathrm{c} \in [0,L]} \max_{n_\mathrm{c} \in \mathcal{S}^3} \mathrm{SRISC}(q^{\mathrm{c}0},\dot{\delta}_\mathrm{c}^\mathrm{max},\lVert u_\mathrm{max} \rVert_2,s_\mathrm{c},n_\mathrm{c}),
\end{split}
\end{equation}
where $\mathcal{Q}$ is the set of feasible soft robot configurations.
To make this optimization more tractable, we can leverage the stated upper bound with $\dot{\delta}_\mathrm{c}^\mathrm{max} = \lVert J_\mathrm{c} \rVert_2 \, \lVert \dot{q}_\mathrm{max} \rVert_2 + v_\mathrm{H}$
and $f_u^\mathrm{max} = \lVert A_\mathrm{c} \rVert_2 \, \lVert \max(|u_\mathrm{min}|,|u_\mathrm{max}|) \rVert_2$.
We note that the maximum value of $\dot{q}_\mathrm{max}$ that the robot can achieve autonomously is, in practice, often given by certain actuator characteristics (e.g., maximum servo velocity for tendon-driven actuation).
\section{Conclusion}
% In this chapter, we first highlighted the absence of a safety metric for soft robots, a gap that prevents the community from quantifying their safety advantages over rigid robot counterparts. This shortfall often results in designs that are ineffective in real-world tasks because they are excessively soft and lack sufficient payload capacity. We then discussed how a quantitative safety metric could be exploited in both design and control applications, ensuring safety without unduly compromising system performance by making the robot overly soft or the motion behavior overly cautious. Finally, we outlined the criteria such a safety metric must meet.
In this chapter, we first underscored the absence of a safety metric for soft robots—a gap that hinders the community from quantifying their safety benefits compared to rigid robots. This shortfall often leads to designs that perform poorly in real-world tasks because they tend to be overly soft and lack sufficient payload capacity. We then explored how a quantitative safety metric could be applied in both design and control contexts, ensuring safety without forcing the robot to be too soft or its motion overly cautious. Finally, we defined the criteria that such a safety metric must satisfy.

% Subsequently, we derived a quantitative safety metric for soft robots while considering the maximum contact pressure experienced during the collision, analog to the existing ISO norms for \glspl{Cobot}~\citep{iso2016collaborative}, as the safety criterion.
% For this the derivation of the safety metric, we rely on existing soft robot dynamic models, such as \gls{PCS}~\citep{renda2018discrete}\footnote{Please refer to Chapter~\ref{chp:background} for a discussion about alternative soft robot models that the safety metric could be based on.} and assume the human to be constrained, which represents the \emph{worst case}~\citep{haddadin2009requirements}.
% The proposed safety metric comes into two flavors: the \gls{SRISC} can be evaluated in closed-form and determines the maximum contact pressure experienced during a soft robot - human collision for a given initial condition, consting of soft robot configuration, velocity and contact geometry, and a given constant (e.g., worst-case) actuation. The \gls{SRISC} is in particular suitable for \emph{safety-aware control} applications.
% However, when designing soft robots, we need to ensure safety within a given operating envelope consisting of infinitely many initial conditions. For this case, we developed the \gls{SRDHC} which determines the maximum \gls{SRISC} for a given specification of operating conditions (e.g., maximum deformation, actuator bounds, configuration-space velocities), in particular across all possible contact geometries.
Next, we derived a quantitative safety metric for soft robots by considering the maximum contact pressure during a collision, analogous to the existing ISO norms for \glspl{Cobot}~\citep{iso2016collaborative}, as our safety criterion. For the derivation, we relied on established soft robot dynamic models, such as \gls{PCS}~\citep{renda2018discrete}\footnote{Please refer to Chapter~\ref{chp:background} for a discussion of alternative soft robot models that could serve as the basis for the safety metric.}, and assumed a constrained human model to represent the \emph{worst case} scenario~\citep{haddadin2009requirements}. The proposed safety metric is offered in two forms: the \gls{SRISC}, which can be computed in closed-form and estimates the maximum contact pressure during a soft robot–human collision given a specific initial condition—comprising the robot’s configuration, velocity, contact geometry, and a predetermined (e.g., worst-case) actuation—and is particularly well-suited for \emph{safety-aware control} applications. In contrast, when designing soft robots, it is necessary to guarantee safety across an operating envelope that encompasses infinitely many initial conditions. For this purpose, we developed the \gls{SRDHC}, which determines the maximum \gls{SRISC} for a given set of operating conditions (e.g., maximum deformation, actuator bounds, configuration-space velocities), especially considering all potential contact geometries.

% Crucially, the proposed safety metric meets all (compulsory) requirements specified in Sec.~\ref{sub:safetymetric:safety_metric_requirements}, as it, for example, accounts for collisions along the entire soft robot body, the assumptions and simplifications taken during the derivation lead to a conservative estimate of the true safety, the computation of the safety metric is tractable as (a) we do not need to the collision dynamics in time but rather have access to the maximum contact force in closed form, (b) part of the computation can be vectorized/parallelized. Furthermore, the safety metric is differentiable - even analytically as our implementation in JAX~\citep{jax2018github} provides gradients via autodifferentiation.
Notably, the proposed safety metric fulfills all mandatory requirements outlined in Sec.~\ref{sub:safetymetric:safety_metric_requirements}. For instance, it takes into account collisions along the entire soft robot body, and the assumptions and simplifications made during its derivation result in a conservative estimate of true safety. Additionally, the metric is computationally tractable because (a) it bypasses the need to simulate collision dynamics over time by providing the maximum contact pressure in closed form, and (b) portions of the computation can be vectorized or parallelized. Moreover, the metric is differentiable—even analytically—since our implementation in JAX~\citep{jax2018github} offers gradients through autodifferentiation.

% For future work, two critical aspects then emerge: (1) experimentally validating the safety criterion—i.e., measuring the discrepancy between the predicted and actual safety criterion—and (2) establishing appropriate thresholds~\citep{iso2016collaborative, behrens2022statistical} for the safety criterion (i.e., acceptable injury risk) to ensure safety in human-centric environments. Here, analog the existing literature on rigid robots~\citep{yamada1997evaluation, muttray2014collaborative, behrens2022statistical}, empirical biomechanical studies could play a crucial role in this process.
% It is important to note that safety requirements may differ significantly depending on the specific robotic application and task. For instance, a soft robot designed as a children's toy must adhere to much stricter safety standards than one used for industrial refueling. Understanding such varying requirements is crucial to ensuring safe and effective performance. 
Looking ahead, two key aspects emerge for future work: (1) the experimental validation of the safety criterion—i.e., quantifying the discrepancy between predicted and actual safety—and (2) the establishment of appropriate thresholds~\citep{iso2016collaborative, behrens2022statistical} for the safety criterion (in terms of acceptable injury risk) to ensure safety in human-centric environments. Drawing inspiration from existing literature on rigid robots~\citep{yamada1997evaluation, muttray2014collaborative, behrens2022statistical}, empirical biomechanical studies could play a crucial role in this process. It is also essential to recognize that safety requirements may vary considerably depending on the specific robotic application and task. For example, a soft robot designed as a children’s toy must comply with much stricter safety standards than one intended for industrial refueling. Understanding these differences is key to achieving safe and effective performance.

% Furthermore, the safety metric itself can be extended and improved. For example, specializing in safety metrics for different soft robot modeling techniques, such as FEM-based and geometric strain-based approaches, will enable more accurate risk assessments tailored to specific designs. Extending these metrics to closed-chain robots, locomotors, and wearable systems will further solidify safety evaluation frameworks across diverse robotic applications.
Furthermore, the safety metric itself can be extended and refined. By developing specialized metrics tailored to different soft robot modeling techniques—such as \gls{FEM}- or \gls{GVS}-based approaches—more accurate risk assessments can be achieved for specific designs. Extending these metrics to closed-chain robots, locomotors, and wearable systems will further strengthen safety evaluation frameworks across a diverse range of robotic applications.

% Finally moving to applications, we find it crucial to develop develop design and control concepts that can exploit the proposed safety metric.
% First considering \emph{safety-aware design}, a critical next step involves identifying which design parameters most significantly affect safety in different types of actuation, such as tendon-driven and pneumatically actuated soft robots. Mapping out these dependencies will provide actionable insights for designers seeking to enhance safety.
% Furthermore, we propose in Appendix~\ref{chp:apx:holisticcodesign} a strategy for co-optimizing the safety of soft robots via co-design~\citep{zardini2023co, spielberg2019learning, wang2024diffusebot, navez2024contributions}, which will allow to explicitly formalize and quantify the tradeoff between safety and performance in soft robotics.
Finally, with regard to practical applications, it is vital to develop design and control strategies that take full advantage of the proposed safety metric. In the context of\emph{safety-aware design}, a crucial next step is to identify which design parameters most significantly impact safety for various actuation types, such as tendon-driven versus pneumatically actuated soft robots. Mapping these dependencies will provide designers with actionable insights to enhance safety. Additionally, as outlined in Appendix~\ref{chp:apx:holisticcodesign}, we propose a holistic co-design strategy for optimizing the safety of soft robots~\citep{zardini2023co, spielberg2019learning, wang2024diffusebot, navez2024contributions}, which will enable the explicit formalization and quantification of the trade-off between safety and performance in soft robotics.
% On the other hand, \emph{safety-aware control} could be achieved via \gls{MPC} or \glspl{CBF}~\citep{ames2016control, ferraguti2020control}, which would enable us to push the performance (e.g., speed~\citep{haggerty2023control}) as much as possible while still guaranteeing safety in all scenarios.
Alternatively, \emph{safety-aware control} can be implemented, for example, using \gls{MPC} or \glspl{CBF}~\citep{ames2016control, ferraguti2020control}, enabling us to maximize performance, such as the motion speed~\citep{haggerty2023control}, while ensuring safety in all scenarios.

% \section{Design of Safe Soft Robots}
% \begin{itemize}
%     \item Based on the insights gained in Section~\ref{sec:safety_metric}, give \emph{simple} insights how safe designs can be achieved.
%     \item Stiffness of the soft robot design (i.e., not just of the material) should be reduced, particularly close to the base/proximal end.
%     \item Reflected mass (e.g., material density \& volume) should be reduced, in particular close to the tip/distal end.
%     \item Guidelines for establishing safe controllers.
%     \item Specifically, explain how the combination of robot stiffness and the proportional control gains impacts the design hazardousness.
%     \item However, we stress that \emph{simple} guidelines that guarantee a specific safety level cannot be (easily) expressed. Therefore, we require a co-design methodology that takes into account the safety metric.
% \end{itemize}
