\section{Conclusion}
% In this chapter, we presented an approach for learning periodic/cyclical motion from demonstration using \glspl{OSMP}, where the motion policy is, analog to \glspl{DMP}~\cite{ijspeert2002learning, ijspeert2013dynamical}, parametrized by a dynamical system.
% Specifically, we combined the strengths of \gls{ML} approaches, such as expressiveness and the lack of requirements for heuristics, with nonlinear system theory to guarantee that the periodic is tracked via a limit cycle behavior that exhibits orbital stability.
% We achieve this by combining a learned bijective encoder based on Euclideanizing flows~\citep{dinh2016density, rana2020euclideanizing} that establishes a diffeomorphism between the oracle and the latent space and inspectable and fixed dynamics in latent space given by supercritical Hopf birfurcation~~\cite{strogatz2018nonlinear}.
% Compared to existing work~\cite{zhi2024teaching}, we significantly improve the accuracy of the matching between periodic demonstration and the generated limit cycle.
% Additionally, we benchmark the \glspl{OSMP} against standard \gls{ML} methods which reveals that they guarnatee convergence while standard \gls{ML} methods such as \glspl{MLP} or \glspl{RNN} often exhibit spurious attractors.
% Also, we extensively validate the proposed method in the real-world on various robot embodiments, including robotic manipulators and \glspl{Cobot}, soft robots, and bioinspired hybrid soft-rigid underwater robots.
In this chapter, we introduced an approach for learning periodic/cyclical motions from demonstrations using \glspl{OSMP}, where—similar to \glspl{DMP}~\citep{ijspeert2002learning, ijspeert2013dynamical}—the motion policy is parametrized by a dynamical system. We combined the advantages of \gls{ML} methods, such as their high expressiveness, with nonlinear system theory to ensure that the periodic motion is tracked via a limit cycle that exhibits \glsxtrfull{OS}. This is achieved by integrating a learned bijective encoder based on Euclideanizing flows~\citep{dinh2016density, rana2020euclideanizing}, which establishes a diffeomorphism between the oracle and latent space, with fixed, inspectable dynamics in latent space provided by a supercritical Hopf bifurcation~\citep{strogatz2018nonlinear}. Compared to previous work~\citep{zhi2024teaching}, our approach significantly improves the accuracy of matching the periodic demonstration to the generated limit cycle. Furthermore, benchmarking against standard \gls{ML} techniques shows that \glspl{OSMP} guarantee convergence, whereas traditional methods such as \glspl{RNN} and \glspl{NODE}~\citep{zhi2024teaching} often exhibit spurious attractors. We also validate our method extensively in real-world settings across various robotic platforms, including robotic manipulators and \glspl{Cobot}, soft robots, and bioinspired hybrid soft-rigid underwater robots.

% Moreover, we devise several techniques that increase the capability of dynamic motion primitives, including online reshaping of the velocity field that adjust the convergence characteristics as needed without requiring retraining, the synchronization of multiple \glspl{OSMP} in their phase by leveraging a error-based feedback term, and the capability to learn several distinct motion behaviors with the same motion policy by conditioning the encoder on the task/motion behavior. After adding a loss term, we are even able to smoothly interpolate between two motion behaviors. 
Moreover, we developed several techniques to enhance the capabilities of dynamic motion primitives. These include online reshaping of the velocity field to adjust convergence characteristics without retraining, synchronizing multiple \glspl{OSMP} in phase via an error-based feedback term, and enabling the same motion policy to learn multiple distinct behaviors by conditioning the encoder on the task. With the addition of a tailored loss term, we can even achieve smooth interpolation between two motion behaviors.

% For future work, it would be interesting to allow the same motion primitive to exhibit multiple classes of attractors~\citep{strogatz2018nonlinear}, including \gls{GAS} (for point-to-point motions), \gls{MS} (for multiple goals with equal value), and \gls{OS} (for periodic motions).
% A first step would be to build on top of the formulation employed in this paper, as the supercritical Hopf bifurcation can capture both exhibit a single, isolated equilibrium and limit cycle behavior if we were to add an additional parameter that would allow us to control that~\citep{strogatz2018nonlinear}.
% Furthermore, it seems timely and obvious, to condition the encoder on the outputs/embeddings of modern \glspl{VLM}~\citep{o2024open, touvron2023llama, grattafiori2024llama}, which would allow to significantly increase the generalization, reasoning, and planning capabilities of the motion policies while preserving the insight, robustness, compliance, and convergence guarantees that \glspl{DMP} exhibit.
For future work, it would be intriguing to enable a single motion primitive to exhibit multiple classes of attractors~\citep{strogatz2018nonlinear}, such as \gls{GAS} for point-to-point motions, \gls{MS} for multiple equally valued goals, and \gls{OS} for periodic motions. An initial step could build on the formulation presented here, as the supercritical Hopf bifurcation can capture both a single isolated equilibrium and limit cycle behavior if an additional \emph{attractor type} parameter is introduced. Furthermore, it appears timely to condition the encoder on the outputs or embeddings of modern \glspl{VLM}~\citep{o2024open, touvron2023llama, grattafiori2024llama}, which would significantly enhance the generalization, reasoning, and planning capabilities of the motion policies while preserving the insight, robustness, compliance, and convergence guarantees inherent to \glspl{DMP}.
