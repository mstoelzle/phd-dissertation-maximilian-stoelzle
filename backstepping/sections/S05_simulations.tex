\section{Simulations}
%
% \begin{figure*}[ht]
%   \centering
%   \subfigure[Configuration $q$]{\includegraphics[width=0.33\textwidth]{figures/time_series_plot_m_p-0.5_untuned_configuration_v2.eps}\label{fig:backstepping:time_series_plot_m_p_0.5_configuration}}
%   \subfigure[Piston position $\mu_\mathrm{p}$]{\includegraphics[width=0.33\textwidth]{figures/time_series_plot_m_p-0.5_untuned_piston_position_v2.eps}\label{fig:backstepping:time_series_plot_m_p_0.5_piston_position}}
%   \subfigure[Actuation force $f_\mathrm{p}$]{\includegraphics[width=0.33\textwidth]{figures/time_series_plot_m_p-0.5_untuned_actuation_force_v2.eps}\label{fig:backstepping:time_series_plot_m_p_0.5_actuation_force}}\\
%   \caption{Simulation of posture regulation under \gls{PCC} approximation for an actuation system with increased inertia ($m_\mathrm{p} = \SI{0.5}{kg}$) comparing the performance of the nonlinear backstepping controller (dashed lines) with a PID baseline controller (dotted lines). The set-point reference configuration is shown with solid lines. All gains remain unchanged and are tuned for the original system with $m_\mathrm{p} = \SI{0.19}{kg}$.}
% \end{figure*}
\begin{figure}[ht]
  \centering
  % End-to-end PID
  \subfigure{\includegraphics[width=0.49\textwidth, trim={0.75cm 0.85cm 0.75cm 0cm}]{backstepping/figures/time_series_plot_m_p-0.5_untuned_configuration_full_system_PID_v2.eps}}
  \subfigure{\includegraphics[width=0.49\textwidth, trim={0.75cm 0.85cm 0.75cm 0cm}]{backstepping/figures/time_series_plot_m_p-0.5_untuned_piston_position_full_system_PID_v2.eps}}
  % Coupling-aware PID
  \subfigure{\includegraphics[width=0.49\textwidth, trim={0.75cm 0.85cm 0.75cm 0cm}]{backstepping/figures/time_series_plot_m_p-0.5_untuned_configuration_coupling_aware_PID_v2.eps}}
  \subfigure{\includegraphics[width=0.49\textwidth, trim={0.75cm 0.85cm 0.75cm 0cm}]{backstepping/figures/time_series_plot_m_p-0.5_untuned_piston_position_coupling_aware_PID_v2.eps}}
  % Backstepping
  \setcounter{subfigure}{0}
  \subfigure[Configuration $q$]{\includegraphics[width=0.49\textwidth, trim={0.75cm 0cm 0.75cm 0cm}]{backstepping/figures/time_series_plot_m_p-0.5_untuned_configuration_backstepping_v2.eps}}
  \subfigure[Piston position $\mu_\mathrm{p}$]{\includegraphics[width=0.49\textwidth, trim={0.75cm 0cm 0.75cm 0cm}]{backstepping/figures/time_series_plot_m_p-0.5_untuned_piston_position_backstepping_v2.eps}}
  \caption{Simulation of posture regulation under \gls{PCC} approximation for an actuation system with increased inertia ($m_\mathrm{p} = \SI{0.5}{kg}$) comparing the performance of an end-to-end PID baseline controller (1st row), with a coupling-aware PID controller (2nd row) and the nonlinear backstepping controller (3rd row). The set-point reference configuration is shown with solid lines.}
  \label{fig:backstepping:time_series_plots_m_p-0.5_untuned}
\end{figure}
\begin{figure}[ht]
  \centering
  % End-to-end PID
  \subfigure[End-to-end PID (baseline)]{\includegraphics[width=0.49\textwidth, trim={0.75cm 0.0cm 0.75cm 0cm}]{backstepping/figures/time_series_plot_m_p-0.5_untuned_actuation_force_full_system_PID_v2.eps}}
  % Coupling-aware PID
  \subfigure[Coupling-aware PID]{\includegraphics[width=0.49\textwidth, trim={0.75cm 0.0cm 0.75cm 0cm}]{backstepping/figures/time_series_plot_m_p-0.5_untuned_actuation_force_coupling_aware_PID_v2.eps}}\\
  % Backstepping
  \subfigure[Backstepping controller] {\includegraphics[width=0.49\textwidth, trim={0.75cm 0cm 0.75cm 0cm}]{backstepping/figures/time_series_plot_m_p-0.5_untuned_actuation_force_backstepping_v2.eps}}\\
  \caption{Simulation of posture regulation under \gls{PCC} approximation for an actuation system with increased inertia ($m_\mathrm{p} = \SI{0.5}{kg}$) comparing the control input (i.e., the piston actuation force $f_\mathrm{p}$]) of an end-to-end PID baseline controller, of a coupling-aware PID controller and the nonlinear backstepping controller.}
  \label{fig:backstepping:time_series_plots_m_p-0.5_untuned_actuation_force}
\end{figure}
%
\begin{figure}[ht]
  \centering
  \subfigure[End-to-end PID]{\includegraphics[width=0.32\columnwidth, trim={0.5cm 0 0.5cm 0}]{backstepping/figures/cartesian_evolution_plot_m_p-0.5_untuned_full_system_pid_v1.eps}\label{fig:backstepping:cartesian_evolution_plot_m_p-0.5_untuned_full_system_pid}}
  \subfigure[Coupling-aware PID]{\includegraphics[width=0.32\columnwidth, trim={0.5cm 0 0.5cm 0}]{backstepping/figures/cartesian_evolution_plot_m_p-0.5_untuned_coupling_aware_pid_v1.eps}\label{fig:backstepping:cartesian_evolution_plot_m_p-0.5_untuned_coupling_aware_pid}}
  \subfigure[Backstepping]{\includegraphics[width=0.32\columnwidth, trim={0.5cm 0 0.5cm 0}]{backstepping/figures/cartesian_evolution_plot_m_p-0.5_untuned_backstepping_v2.eps}\label{fig:backstepping:cartesian_evolution_plot_m_p-0.5_untuned_backstepping}}
  \caption{Cartesian evolution of the soft robot for an actuation system with increased inertia ($m_\mathrm{p} = \SI{0.5}{kg}$). All gains remain unchanged and are tuned for the original system with $m_\mathrm{p} = \SI{0.19}{kg}$. The dotted lines mark the evolution of the tip of the segments. The soft robot consists of three segments (blue, orange, and yellow). The reference configuration at the three set points is marked with a thick black line.}\label{fig:backstepping:cartesian_evolution_plot_m_p-0.5_untuned}
\end{figure}
%
\subsection{System}
%
We consider a planar soft robot arm consisting of three independently actuated CC segments, modeled upon the second half of the robot in \citep{della2020model}. % in Simulink. It is inspired by recent designs and models of similar pneumatically actuated continuum arms in literature~\citep{della2020model, marchese2014design}. 
%We model the 
Segments have equal length $l_{0} = \SI{11}{cm}$, uniform mass density $\rho = \SI{0.99}{kg \per m}$ concentrated on the central axis. % - resulting in \SI{109}{g} per segment. %We use the \gls{PCC} assumption~\citep{jones2006kinematics} to compute the Jacobian for any point along the center-line of the segment. The \gls{EoM} are derived using the Euler-Lagrange equation. We add an elastic term $K q$ with $K$ defined as a diagonal matrix with the elastic constant \SI{0.01}{N \per rad}. Natural damping is considered with $D \dot{q}$ and the diagonal damping constant \SI{0.01}{Ns \per rad}.
%
The stiffness $K$ and damping $D$ matrices are diagonal with constants \SI{0.01}{N \per rad} and \SI{0.01}{Ns \per rad}. The segment has a diameter of \SI{44.5}{mm}. Based on CAD analyses of a real system, we take $d_{\mathrm{C},\mathrm{a}} = \SI{7.14}{mm}$, $d_{\mathrm{C},\mathrm{b}} = \SI{20.19}{mm}$, and $b_\mathrm{C} = \SI{8.07}{mm}$.
%We set the parameters $d_{\mathrm{C},\mathrm{a}}$ and $d_{\mathrm{C},\mathrm{b}}$ as the inner and outer air chamber walls are placed at a radial distance of \SI{7.14}{mm} and \SI{20.19}{mm}. We analysed the chamber volume in CAD to be \SI{46.2}{cm^3}, which results in a modelled planar chamber depth of $b_\mathrm{C} = \SI{8.07}{mm}$.
%
%The base of the first segment is aligned with the x-axis and gravity is acting along the negative y-axis. 
A positive curvature and positive configuration $q_i$ correspond to bending counter-clockwise. 
% The base of the robot is oriented perpendicularly to gravity, so that it tends to induce clock-wise bending.
The straight configuration of the robot along the x-axis is perpendicular to gravity acting in negative y-direction as shown in Figure~\ref{fig:backstepping:pcc_case_overview}, so that gravity tends to induce clockwise bending.
%
%We were inspired by the fluidic drive cylinder designed by Marchese et al.~\citep{marchese2014design} in our choice of parameters for the pistons. Accordingly, 
Moving to the pistons, $A_\mathrm{p} = \SI{7.9}{cm^2}$, $m_\mathrm{p} = \SI{0.19}{kg}$, $l_\mathrm{p} = \SI{0.5}{m}$ are chosen.
% We model the cross-sectional area of the piston $A_\mathrm{p}$ as \SI{7.9}{cm^2}, the mass of the piston $m_\mathrm{p}$ with \SI{0.19}{kg} resulting in a diagonal mass matrix $M_\mathrm{p}$, the length of the piston $l_\mathrm{p}$ as \SI{0.5}{m} and 
We consider a damping matrix $D_\mathrm{p}$ with damping constants $d_\mathrm{p} = \SI{10}{\kilo \newton \second \per \meter}$ along the diagonal. The pistons are filled with air at $\mu_{\mathrm{p}, 0} = l_\mathrm{p}$ and $p_\mathrm{atm} = \SI{1}{bar}$ and subsequently pre-loaded to $\mu_{\mathrm{p}, \mathrm{t}0} = 0.25 \, l_\mathrm{p}$. % before the experiment is started.
We set the backstepping gains to $K_1 = \SI{6000}{\per \second}$ and $K_2 = \SI{4.5}{\kilo \newton \per \meter}$.

\subsection{End-to-end PID}
We first introduce an end-to-end PID controller, which will serve as a baseline
%
\begin{equation}
    \Delta f_\mathrm{p} = K_\mathrm{p} (\bar{q}-q) + K_\mathrm{i} \int_0^t (\bar{q}-q) \, \mathrm{d}t' - K_\mathrm{d} \, \dot{q}
\end{equation}
%
where $K_\mathrm{p},K_\mathrm{i},K_\mathrm{d} \geq 0$ are scalar gains. $\Delta f_\mathrm{p} \in \mathbb{R}^{n_q}$ is the scalar offset from the actuation force $f_{\mathrm{p},\mathrm{t}0}$ corresponding to the pre-loaded pressure $p_{\mathrm{t}0}$.
%
Analog to \eqref{eq:backstepping:dist_G_p_q_planar_pcc}, $\Delta f_\mathrm{p}$ can be equally distributed on both chambers within a segment.
% \begin{equation}
% \begin{split}
%     f_{\mathrm{p},\mathrm{L},i} = f_{\mathrm{p},\mathrm{t}0} + 0.5 \, \Delta f_{\mathrm{p},i}, \quad f_{\mathrm{p},\mathrm{R},i} = f_{\mathrm{p},\mathrm{t}0} - 0.5 \, \Delta f_{\mathrm{p},i}.
% \end{split}
% \end{equation}
The PID gains have been selected so to achieve a similar transient behavior as for the backstepping controller % required lengthy heuristic tuning and the selection was guided by the Ziegler-Nichols method,
and are equal to $K_\mathrm{p} = \SI{200}{\newton \per \radian}$, $K_\mathrm{i} = \SI{7}{\newton \per \radian \per \second}$, and $K_\mathrm{d} = \SI{200}{\newton \second \per \radian}$. 
% Segment 1, 2 and 3 are weighted with gain multipliers of $1.5$, $1$, and $0.5$.

%
\subsection{Coupling-aware PID}
%
% We attempted implementing a standard low-level PID with static compensation as in \citep{della2020model} for fair comparison, but we could not find a set of gains that was stable when using \eqref{eq:backstepping:high_level_regulation}.
Next, we implement a control strategy that takes advantage of the understanding of the potential coupling and uses a PID for low-level control of the pistons
%
\begin{equation}
    f_{\mathrm{p}} = K_\mathrm{p} \left (\Gamma(q, \bar{q})-\mu_\mathrm{p} \right ) +  K_\mathrm{i} \int_{0}^t \left ( \Gamma(q, \bar{q})-\mu_\mathrm{p} \right ) \mathrm{d}t' - K_\mathrm{d} \, \dot{\mu}_\mathrm{p}.
\end{equation}
%
Here, $K_\mathrm{p},K_\mathrm{i},K_\mathrm{d} \geq 0$ are scalar gains, and $\Gamma(q, \bar{q})$ is the correction on \eqref{eq:backstepping:high_level_regulation} which takes the coupling defined in \eqref{eq:backstepping:gamma} in account. 
%
The PID gains are tuned similarly to the coupling-aware PID
and are equal to $K_\mathrm{p} = \SI{150 000}{\newton \per \meter}$, $K_\mathrm{i} = \SI{15 000}{\newton \per \meter \per \second}$, and $K_\mathrm{d} = \SI{100}{\newton \second \per \meter}$. 
% Segment 1, 2 and 3 are weighted with gain multipliers of $1.5$, $1$, and $0.5$.  %~\citep{ziegler1942optimum}.

\subsection{Results}

We simulate the response of the closed-loop generated by all three controllers to a sequence of step references. The segments are initialized at the equilibrium configuration. % of the system $\begin{bmatrix} \SI{-1.8850}{\radian} & \SI{0.3752}{\radian} & \SI{-0.0593}{\radian} \end{bmatrix}^\mathrm{T}$, where they are for \SI{10}{s}. 
At \SI{10}{s}, the reference is moved to the straight configuration $\bar{q} =  0$. After another \SI{30}{s}, we change it again to $\bar{q} = \begin{bmatrix} \SI{0.6981}{\radian} & \SI{-0.5236}{\radian} & \SI{0.1745}{\radian} \end{bmatrix}^\mathrm{T}$.
% We choose an \emph{ode45} with variable step size and a max step size of $0.001$ for our Simulink simulation. %To avoid numerical instabilities of the PCC formulation close to a straight configuration, we use $\tilde{q}$  with $\lvert \tilde{q}_i \rvert = \max (q_i, \SI{3}{\degree})$ as the input into the \gls{EoM} and the controller formulations.

Figures~\ref{fig:backstepping:time_series_plots}-\ref{fig:backstepping:time_series_plots_actuation_force} shows that the backstepping controller is approaching the set-point reference with no oscillations nor overshooting. These are instead visible for coupling-are PID controller after the second change in reference configuration. 
The end-to-end PID controller does not converge to the desired configuration within \SI{60}{s} as it does not take into account gravity.

Next, we increase the inertia of the actuation system by setting the piston mass $m_\mathrm{p}$ to \SI{0.5}{kg}. We leave both the backstepping and the PID gains unchanged. Figures~\ref{fig:backstepping:time_series_plots_m_p-0.5_untuned}-\ref{fig:backstepping:cartesian_evolution_plot_m_p-0.5_untuned} demonstrate that the backstepping-based approach is able to adapt to the new system, while the end-to-end PID shows large oscillations at \SI{50}{s} and the coupling-aware PID displays significantly overshoot in curvatures and piston positions. 
Note that the latter are especially dangerous in real experiments since they may signify that pistons reach their limits.

% \begin{table*}
% \centering
% \caption{System parameters used in simulations}
% \begin{tabular}{c c c c c c c c c c}\toprule
% $l$ & $\rho$ & $k$ & $d$ & $b_\mathrm{C}$ & $d_{\mathrm{C},a}$ & $d_{\mathrm{C},b}$ & $A_\mathrm{p}$ & $l_\mathrm{p}$ & $m_\mathrm{p}$\\
% \midrule
% \SI{110}{mm} & \SI{0.99}{g \per mm} & \SI{0.01}{N \per \radian} & \SI{0.01}{Ns \per \radian} & \SI{8.07}{mm} & \SI{7.14}{mm} & \SI{20.19}{mm} & \SI{7.9}{cm^2}~\citep{marchese2014design} & \SI{0.5}{m} & \SI{0.19}{kg}~\citep{marchese2014design}\\
% \bottomrule
% \end{tabular}
% \label{tab:system_parameters}
% \end{table*}