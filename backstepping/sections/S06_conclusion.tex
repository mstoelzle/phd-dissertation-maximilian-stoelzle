\section{Conclusion}\label{sec:backstepping:conclusion}
%
This chapter proposed a model for soft robots actuated using pneumatic fluidic drive cylinders and introduced a model-based controller to take actuators' dynamics into account. The stability of this backstepping-based control strategy was proven using a Lyapunov argument.  As an example of application, model and control strategy have been specialized for the planar \gls{PCC}-case. We also proposed a coupling-aware extension of the standard hierarchical PID strategy as a middle-ground solution. %Our simulations show the feasibility and convergence properties of the nonlinear backstepping controller for the set-point control. % of a \gls{PCC}-modeled soft robotic arm consisting of three segments. 
% While the backstepping controller is agnostic to any changes in the actuation system parameters, the PID controller would require an extensive re-tuning of its gains.
%
Future work will focus on applying this strategy to more sophisticated models and controllers and on experimental validation in a lab environment. %Also, we will consider different kind of actuation dynamics.

\section{Afterword}
In this chapter, we have presented a dynamical model that, in addition to the standard soft robot dynamics, also considers the dynamics of a piston-driven pneumatic actuation system. Subsequently, we exploited the actuation dynamics within a backstepping controller.
We believe that similar approaches could also be applied to other soft robotic actuation methods, such as tendons pulled by electrical motors or Shape Memory Alloys actuators~\cite{zaidi2021actuation}.
For example, recently, a similar backstepping-based control approach has been successfully applied to soft robots actuated by valve-driven pressure regulators~\cite{franco2024model}.
Sometimes, modeling the dynamics of soft robots from first principles can be challenging, making it difficult to effectively control their behavior. In other cases, such as pneumatic systems, hysteresis effects are prevalent~\cite{vo2010new}, and direct measurement of the hysteretic displacement is not possible using available sensor data.
In such situations, employing learning approaches (e.g., machine learning with neural networks) can be beneficial. Still, we would like to preserve the interpretability, structure, and guarantees provided by fully physics-based models. For that reason, we explore integrating physical structures and stability guarantees into learning-based models and controllers in Part~\ref{part:learning} of this thesis.