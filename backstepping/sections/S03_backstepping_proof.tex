\section{Backstepping Control of Piston-Driven\\ Pneumatically-Actuated Soft Robots}\label{sec:backstepping:backstepping_proof}

This section discusses the main contribution of this paper, a backstepping-based approach to generalize controllers $\Gamma(q,\dot{q})$ designed in the directly actuated case, to systems that can be modeled through \eqref{eq:backstepping:complete_dyn}.
We suppose that we have access to $\Gamma(q, \dot{q})$ controlling the piston position $\mu_\mathrm{p}$. Next, we perform backstepping twice to the controllers of the piston velocity $\dot{\mu}_\mathrm{p}$ and the piston actuation force $f_\mathrm{p}$ and prove the stability of each controller with Lyapunov arguments.
The derived model-based control approach assumes that all model parameters are known, and all states are measurable (namely the configuration $q$, its time derivative $\dot{q}$, the piston position $\mu_\mathrm{p}$ and the piston velocity $\dot{\mu}_\mathrm{p}$), and that there are no disturbances or model uncertainties. 
We first introduce a Lemma, which will be instrumental to the proof of the main theorem. It allows to relate an offset in the actuation-space to a change in acceleration in configuration-space that is proportional to the offset.
%
\begin{Lemma}\label{lemma:f_g_S}%{\color{red} I have the feeling that words are missing here}
The input field defined in \eqref{eq:backstepping:gf} verifies
%
\begin{equation*}\small
    g(q,\mu_{\mathrm{p},\mathrm{a}}) - g(q,\mu_{\mathrm{p},\mathrm{b}}) \!=\!  -B^{-1}\!(q)S(q,\mu_{\mathrm{p},\mathrm{a}},\mu_{\mathrm{p},\mathrm{b}})(\mu_{\mathrm{p},\mathrm{a}} \!- \mu_{\mathrm{p},\mathrm{b}}),    
\end{equation*}
%
$\forall \mu_{\mathrm{p},\mathrm{a}},\mu_{\mathrm{p},\mathrm{b}} \in \mathbb{R}^{n_{\mu_\mathrm{p}}}$ and $q \in \mathbb{R}^{n_{\mathrm{q}}}$
	with $S \in \mathbb{R}^{n_{\mathrm{q}} \times n_{\mu_\mathrm{p}}} $ so defined
	%
	\begin{equation*}\small
	    S_{i,j} = \frac{ A_{\mathrm{p},j} \alpha_{\mathrm{air},j} \partial_{q_i}V_{\mathrm{C},j}}{(V_{\mathrm{C},j}(q_i) + A_{\mathrm{p},j} \mu_{\mathrm{p},\mathrm{a},j})(V_{\mathrm{C},j}(q_i) + A_{\mathrm{p},j} \mu_{\mathrm{p},\mathrm{b},j})}.
	\end{equation*}
\end{Lemma}
%
\begin{proof}
	We express the left term of the equality using \eqref{eq:backstepping:gf}
	%
	\begin{equation*}\small%\footnotesize
	\begin{split}
	g(q,\mu_{\mathrm{p},\mathrm{a}}) - g(q,\mu_{\mathrm{p},\mathrm{b}}) 
	=-B^{-1}(q) (G_{\mathrm{P}}^{\mathrm{q}}(q,\mu_{\mathrm{p},\mathrm{a}}) - G_{\mathrm{P}}^{\mathrm{q}}(q,\mu_{\mathrm{p},\mathrm{b}})),
	\end{split}
	\end{equation*}
	%
	where we recognize the term $B^{-1}(q)$ appearing in the Lemma. The term between brackets can be adjusted by using \eqref{eq:backstepping:GPq}
	%
	\begin{equation}\small
	\begin{split}
		&G_{\mathrm{P},i}^{\mathrm{q}}(q,\mu_{\mathrm{p},\mathrm{a}}) - G_{\mathrm{P},i}^{\mathrm{q}}(q,\mu_{\mathrm{p},\mathrm{b}})  \\
	= &-\left(\sum_{j = 1}^{n_{\mu_\mathrm{p}}}  \frac{\alpha_{\mathrm{air},j} \partial_{q_i}V_{\mathrm{C},j}}{V_{\mathrm{C},j}(q_i) + A_{\mathrm{p},j} \mu_{\mathrm{p},\mathrm{a},i}} \!-\! \sum_{j = 1}^{n_{\mu_\mathrm{p}}}  \frac{\alpha_{\mathrm{air},j} \partial_{q_i}V_{\mathrm{C},j}}{V_{\mathrm{C},j}(q_i) + A_{\mathrm{p},j} \mu_{\mathrm{p},\mathrm{b},j}}\right)  \\
	= &\sum_{j = 1}^{n_{\mu_\mathrm{p}}} \frac{ (\mu_{\mathrm{p},\mathrm{a},j} - \mu_{\mathrm{p},\mathrm{b},j})  A_{\mathrm{p},j} \alpha_{\mathrm{air},j} \partial_{q_i}V_{\mathrm{C},j}}{(V_{\mathrm{C},j}(q_i) + A_{\mathrm{p},j} \mu_{\mathrm{p},\mathrm{a},j})(V_{\mathrm{C},j}(q_i) + A_{\mathrm{p},j} \mu_{\mathrm{p},\mathrm{b},j})}.
	\end{split}
	\end{equation}
	%
	The Lemma follows by simple factorization of the latter term.
\end{proof}

Thus, even if the robot side of the dynamics \eqref{eq:backstepping:complete_dyn} is not affine in control when taking $\mu_\mathrm{p}$ as input, still Lemma \ref{lemma:f_g_S} provides some structure that we leverage in the next theorem.

\begin{Theorem}
	Suppose that a $\Gamma(q,\dot{q})$ exists s.t. the reduced system
	%
	\begin{equation}\small\label{eq:backstepping:reduced_sys}
		\ddot{q} = f(q,\dot{q}) + g(q,\Gamma(q,\dot{q}))
	\end{equation}
	%
	converges to a desired trajectory $\bar{q}(t)$, $\forall (q(0),\dot{q}(0)) \in \mathbb{R}^{2 n_{\mathrm{q}}}$. Suppose that the convergence is proven by Lyapunov arguments through the function $H(q,\dot{q})$. Then the closed loop of the full system \eqref{eq:backstepping:complete_dyn} and the controller
	%
%	\begin{equation}\small\small
%		\begin{split}
%%			\tau &= G_{\mathrm{P}}^{\mathrm{l}} + J  \ddot{\Gamma} + (JK_1 + K_2) (\dot{l} - \dot{\Gamma}) + (K_2 K_1 - I) (l - \Gamma)\\
%%			&- J\frac{\mathrm{d}}{\mathrm{d} t}\left(M^{\mathrm{T}}(q,l,\Gamma)B^{-\mathrm{T}}(q)\partial_{\dot{q}} H^{\mathrm{T}}\right) \\
%%			&- K_2 M^{\mathrm{T}}(q,l,\Gamma)B^{-\mathrm{T}}(q)\partial_{\dot{q}} H^{\mathrm{T}}) \\
%			\tau &= G_{\mathrm{P}}^{\mathrm{l}} + J \ddot{\Gamma} + (J + I) (\dot{l} - \dot{\Gamma})  \\
%			&- \frac{\mathrm{d}}{\mathrm{d} t}\left(M^{\mathrm{T}}(q,l,\Gamma)B^{-\mathrm{T}}(q)\partial_{\dot{q}} H^{\mathrm{T}}\right) \\
%			&- M^{\mathrm{T}}(q,l,\Gamma)B^{-\mathrm{T}}(q)\partial_{\dot{q}} H^{\mathrm{T}}
%		\end{split}
%	\end{equation}
	\begin{equation}\small\label{eq:backstepping:pi_psi}
		\begin{split}
			f_\mathrm{p} &= \Psi = G_{\mathrm{P}}^{\mu_\mathrm{p}} + D_\mathrm{p} \Pi + M_\mathrm{p} \dot{\Pi} - K_2 (\dot{\mu}_\mathrm{p} - \Pi) - (\mu_\mathrm{p} - \Gamma),\\
			\Pi &= \dot{\Gamma} - K_1 (\mu_\mathrm{p} - \Gamma) 
			+ S^{\mathrm{T}}(q,\mu_\mathrm{p},\Gamma)B^{-1}(q)\partial_{\dot{q}} H^{\mathrm{T}},
		\end{split}
	\end{equation}
	%
	with $K_1,K_2 \succ 0$, is such that $q \rightarrow \bar{q}$ and $\mu_\mathrm{p} \rightarrow \Gamma(\bar q,\dot{\bar q})$, $\forall (\mu_\mathrm{p}(0),\dot{\mu}_\mathrm{p}(0)) \in \mathbb{R}^{2n_{\mu_\mathrm{p}}}$, and $\forall (q(0),\dot{q}(0)) \in \mathbb{R}^{2 n_{\mathrm{q}}}$.
\end{Theorem}
\begin{proof}
	
	We first consider the problem of deriving a controller under the assumption that the velocity of the piston $v_\mathrm{p}$ is set by a controller. This serves as a first step toward the general solution of the problem. System \eqref{eq:backstepping:complete_dyn} is thus reduced into
	%
	\begin{equation}\small\label{eq:backstepping:intermediate}\small
	\begin{split}
	B(q)\ddot{q} \! + \! C(q,\dot{q})\dot{q} \! + \! G(q) \! + \! K(q) \! + \! D(q,\dot{q}) \! + \! G_{\mathrm{P}}^{\mathrm{q}}(q,\mu_\mathrm{p}) &= 0, \\
	\dot{\mu}_\mathrm{p} &= v_\mathrm{p}. 
	\end{split}
	\end{equation}
	%
	We introduce the following control Lyapunov candidate
	%
	\begin{equation}\small
		W(q,\dot{q},\mu_\mathrm{p}) = H(q,\dot{q}) + \frac{1}{2}(\mu_\mathrm{p} - \Gamma)^{\mathrm{T}}(\mu_\mathrm{p} - \Gamma)\;,
	\end{equation}
	%
	which can thus be differentiated obtaining
	%
	\begin{equation}\small
		\begin{split}
			&\dot{W}(q,\dot{q},\mu_\mathrm{p}) = 
			\dot{H} + (\mu_\mathrm{p} - \Gamma)^{\mathrm{T}}(v_\mathrm{p} - \dot\Gamma) \\
			& =  \partial_{q} H \dot{q} + \partial_{\dot{q}} H (f(q,\dot{q}) + g(q,\mu_\mathrm{p})) + (\mu_\mathrm{p} - \Gamma)^{\mathrm{T}}(v_\mathrm{p} - \dot\Gamma) \\
			& = \partial_{q} H \dot{q} + \partial_{\dot{q}} H (f(q,\dot{q}) + g(q,\Gamma(q,\dot{q}))) \\
			&+ \partial_{\dot{q}} H (g(q,\mu_\mathrm{p}) - g(q,\Gamma(q,\dot{q}))) + (\mu_\mathrm{p} - \Gamma)^{\mathrm{T}} (v_\mathrm{p} - \dot\Gamma)\;,
		\end{split}
	\end{equation}
	%
	where we first used the chain rule on $\dot{H}$ and then we added and subtracted $\partial_{\dot{q}} H g(q,\Gamma(q,\dot{q}))$.
	%
	We now propose the controller $v_\mathrm{p} = \Pi(q,\dot{q},\mu_\mathrm{p})$ for stabilizing this system, with
	%
	\begin{equation*}\small
		\begin{split}
			\Pi(q,\dot{q},\mu_\mathrm{p}) &= \dot{\Gamma} - K_1 (\mu_\mathrm{p} - \Gamma) + S^{\mathrm{T}}(q,\mu_\mathrm{p},\Gamma)B^{-\mathrm{T}}(q)\partial_{\dot{q}} H^{\mathrm{T}}.
		\end{split}
	\end{equation*}
	%
	The derivative of the Lyapunov candidate for the closed loop system is thus
	%
	\begin{equation*}\small
	\begin{split}
	&\dot{W}(q,\dot{q},\mu_\mathrm{p}) =  \partial_{q} H \dot{q} + \partial_{\dot{q}} H (f(q,\dot{q}) + g(q,\Gamma(q,\dot{q}))) \\
	&\quad + \partial_{\dot{q}} H (g(q,\mu_\mathrm{p}) - g(q,\Gamma(q,\dot{q}))) - (\mu_\mathrm{p} - \Gamma)^{\mathrm{T}}K_1(\mu_\mathrm{p} - \Gamma) \\
	&\quad +  \partial_{\dot{q}} H B^{-1}(q) S(q,\mu_\mathrm{p},\Gamma) (\mu_\mathrm{p} - \Gamma)\;,
	\end{split}
	\end{equation*}
	%
	where we exploited that all terms are scalar to extract the transpose of the last one. This equation can be  simplified by invoking Lemma 1 into
	%
	\begin{equation}\small
	\begin{split}
	\dot{W}(q,\dot{q},\mu_\mathrm{p}) &=  \partial_{q} H \dot{q} + \partial_{\dot{q}} H (f(q,\dot{q}) + g(q,\Gamma(q,\dot{q}))) \\
	&\quad - (\mu_\mathrm{p} - \Gamma)^{\mathrm{T}}K_1(\mu_\mathrm{p} - \Gamma).
	\end{split}
	\end{equation}
	%
	Consider now that $H$ is a Lyapunov function for \eqref{eq:backstepping:reduced_sys} under the control action $\Gamma$. This assures that
	%
	\begin{equation}\small
		0 > \dot{H} = \partial_{q} H \dot{q} + \partial_{\dot{q}} H (f(q,\dot{q}) + g(q,\Gamma(q,\dot{q}))).
	\end{equation}
	%
	Note that we are considering here the case of strict sign definiteness of $\dot{H}$. However, the same results can be achieved in the case of semi\--definiteness.
	%
	We can now conclude that $\dot{W} < 0$, thus proving that the controller $\Pi$ stabilizes \eqref{eq:backstepping:intermediate}. This conclude the first step of the proof.
	
	We now reiterate this sequence of operations, to generalize the controller $\Pi$ to work on the actual system \eqref{eq:backstepping:complete_dyn}.
	%
	The complete Lyapunov candidate that we propose is
	%
	\begin{equation}\small
		Q = W + \frac{1}{2}(\dot{\mu}_\mathrm{p} - \Pi)^{\mathrm{T}} M_\mathrm{p} (\dot{\mu}_\mathrm{p} - \Pi),
	\end{equation}
	%
	with time derivative
	%
	\begin{equation*}\small
		\begin{split}
			\dot{Q} &= \dot{W} + (\dot{\mu}_\mathrm{p} - \Pi)^{\mathrm{T}} (\Psi - G_{\mathrm{P}}^{\mu_\mathrm{p}} - M_\mathrm{p} \dot{\Pi}) \\
			 &= \partial_{q} W \dot{q} + \partial_{\dot{q}} W \ddot{q} + \partial_{\mu_\mathrm{p}} W \dot{\mu}_p\\
			 & + (\dot{\mu}_\mathrm{p} - \Pi)^{\mathrm{T}} (\Psi - G_{\mathrm{P}}^{\mu_\mathrm{p}} - D_\mathrm{p} \dot{\mu}_\mathrm{p} - M_\mathrm{p} \dot{\Pi}) .
		\end{split}
	\end{equation*}
	%
	We therefore propose the controller
	%
	\begin{equation}\small\label{eq:backstepping:final_controller_1}
	%\begin{split}
	    \Psi(q, \dot{q}, \mu_\mathrm{p}, \dot{\mu}_\mathrm{p}) = G_{\mathrm{P}}^{\mu_\mathrm{p}} + D_\mathrm{p} \Pi + M_\mathrm{p} \dot{\Pi} - K_2 (\dot{\mu}_\mathrm{p} - \Pi) 
	    - \partial_{\mu_\mathrm{p}} W^{\mathrm{T}}\!,
	    %(\dot{\mu}_\mathrm{p}^\mathrm{T} – \Pi^\mathrm{T})^{-1} \Pi^\mathrm{T} D_\mathrm{p} \dot{\mu}_\mathrm{p},
	%\end{split}
	\end{equation}
	%
	which generates the following closed loop Lyapunov candidate
	%
	\begin{equation}\small
	    \begin{split}
			\dot{Q} &= \partial_{q} W \dot{q} + \partial_{\dot{q}} W \ddot{q} + \partial_{\mu_\mathrm{p}} W \Pi \\
			&- (\dot{\mu}_\mathrm{p} - \Pi)^{\mathrm{T}} (K_2 + D_\mathrm{p}) (\dot{\mu}_\mathrm{p} - \Pi)
			%& \quad -\dot{\mu}_\mathrm{p}^\mathrm{T} M_\mathrm{p} D_\mathrm{p} \dot{\mu}_\mathrm{p}  
			< 0,
	    \end{split}
	\end{equation}
	%
	where we exploit that $W$ is a Lyapunov function for the previous system when $\dot{\mu}_\mathrm{p} \equiv \Pi$. This assures the asymptotic stability of the closed loop system, when \eqref{eq:backstepping:final_controller_1} is used. The Theorem follows considering that $\partial_{\mu_\mathrm{p}} W^{\mathrm{T}} = (\mu_\mathrm{p} - \Gamma)$.
	
%	The Theorem follows considering that
%	%
%	%\begin{equation}\small
%	%	\begin{split}
%			$\dot{\Pi} =  \ddot{\Gamma} + P^{-1} K_1 (\dot{l} - \dot{\Gamma}) - P^{-1}\frac{\mathrm{d}}{\mathrm{d} t}\left(M^{\mathrm{T}}(q,l,\Gamma)B^{-\mathrm{T}}(q)\partial_{\dot{q}} H^{\mathrm{T}}\right)$, 
%			$\partial_{l} W^{\mathrm{T}} = (l - \Gamma)$.
%	%	\end{split}
%	%\end{equation}
%	%
%	\begin{equation}\small\small
%	\begin{split}
%	%			\tau &= G_{\mathrm{P}}^{\mathrm{l}} + J  \ddot{\Gamma} + (JK_1 + K_2) (\dot{l} - \dot{\Gamma}) + (K_2 K_1 - I) (l - \Gamma)\\
%	%			&- J\frac{\mathrm{d}}{\mathrm{d} t}\left(M^{\mathrm{T}}(q,l,\Gamma)B^{-\mathrm{T}}(q)\partial_{\dot{q}} H^{\mathrm{T}}\right) \\
%	%			&- K_2 M^{\mathrm{T}}(q,l,\Gamma)B^{-\mathrm{T}}(q)\partial_{\dot{q}} H^{\mathrm{T}}) \\
%	\tau &= G_{\mathrm{P}}^{\mathrm{l}} + J \ddot{\Gamma} \\
%	&+ (J P^{-1} K_1 + K_2) (\dot{l} - \dot{\Gamma})  + (K_2 P^{-1} K_1 - I) (l - \Gamma) \\
%	&- J P^{-1}\frac{\mathrm{d}}{\mathrm{d} t}\left(M^{\mathrm{T}}(q,l,\Gamma)B^{-\mathrm{T}}(q)\partial_{\dot{q}} H^{\mathrm{T}}\right) \\
%	&- K_2 P^{-1}M^{\mathrm{T}}(q,l,\Gamma)B^{-\mathrm{T}}(q)\partial_{\dot{q}} H^{\mathrm{T}}
%	\end{split}
%	\end{equation}
%	
%	which for $K_2 = J$, $P = J$, $K_1 = I$ becomes the controller used in the thesis 
\end{proof}